\documentclass[11pt]{article}
\maxdeadcycles=200
\usepackage{graphicx}
\usepackage{morefloats}
\extrafloats{100}
\usepackage{float}
\usepackage{subfig}
\usepackage{amsmath,amsfonts,amssymb}
\usepackage{fancyhdr}
\usepackage{cite}
\usepackage{mathtools}
\usepackage{booktabs,topcapt}
\usepackage[dvipsnames]{xcolor}
\usepackage[spaces,hyphens]{url}
\usepackage[margin=1in,headheight=14pt]{geometry}
\usepackage{pgfplots}
\pgfplotsset{width=4cm,compat=1.9}
\usepackage[labelfont=bf]{caption}
\captionsetup[figure]{font=small}
\captionsetup[subfigure]{subrefformat=simple,labelformat=simple,listofformat=subsimple,font=small}
% \renewcommand\thesubfigure{(\alph{subfigure})}
\pagestyle{fancy}
\lhead{\textsc{Game Theory and U.S. Senate Campaigns}}
\chead{}
\rhead{\textsc{Cal Poly SLO Math Dept}}
\lfoot{\textsc{Wetzel}}
%%%%%%%%%%%%%%%%%%%%%%%%%%%%%%%%%%%%%%%%%%%%%%%%%

% Grayscale colors
\definecolor{gr30}{gray}{.00}
\definecolor{gr29}{gray}{.02}
\definecolor{gr28}{gray}{.04}
\definecolor{gr27}{gray}{.06}
\definecolor{gr26}{gray}{.08}
\definecolor{gr25}{gray}{.10}
\definecolor{gr24}{gray}{.12}
\definecolor{gr23}{gray}{.14}
\definecolor{gr22}{gray}{.16}
\definecolor{gr21}{gray}{.18}
\definecolor{gr20}{gray}{.20}
\definecolor{gr19}{gray}{.22}
\definecolor{gr18}{gray}{.24}
\definecolor{gr17}{gray}{.26}
\definecolor{gr16}{gray}{.28}
\definecolor{gr15}{gray}{.30}
\definecolor{gr14}{gray}{.32}
\definecolor{gr13}{gray}{.34}
\definecolor{gr12}{gray}{.36}
\definecolor{gr11}{gray}{.38}
\definecolor{gr10}{gray}{.40}
\definecolor{gr9}{gray}{.42}
\definecolor{gr8}{gray}{.44}
\definecolor{gr7}{gray}{.46}
\definecolor{gr6}{gray}{.48}
\definecolor{gr5}{gray}{.50}
\definecolor{gr4}{gray}{.52}
\definecolor{gr3}{gray}{.54}
\definecolor{gr2}{gray}{.56}
\definecolor{gr1}{gray}{.58}
\definecolor{purp}{rgb}{1.0,0,1.0}

\begin{document}

%\pagenumbering{gobble}

\title{Game Theory and U.S. Senate Campaigns}

\author{
Tim Wetzel \\[30eM]
Mathematics Department \\
California Polytechnic State University\\
San Luis Obispo, California\\
}

\date{\today}

\maketitle

\newpage

\mbox{}

\vfill

\begin{center}
\textbf{Approval Page}
\end{center}

\vfill

\begin{flushleft}
\textbf{Title:}
Game Theory and U.S. Senate Campaigns
\\[10ex]

\textbf{Author:} Tim Wetzel \\[10ex]

\textbf{Date:} \today \\[20ex]
\end{flushleft}

\vfill

\begin{flushleft}
\begin{minipage}{37.5ex}
\begin{flushright}
\rule{37.5ex}{.1ex} \\
\hspace{5ex} Senior Project Advisor
\end{flushright}
\end{minipage}
\hfill
\begin{minipage}{37.5ex}
\begin{flushright}
\rule{37.5ex}{.1ex} \\
\hspace{5ex} Department Chair
\end{flushright}
\end{minipage}

\vspace{10ex}

\begin{minipage}{37.5ex}
\begin{flushright}
\rule{37.5ex}{.1ex} \\
\hspace{5ex} Advisor Signature
\end{flushright}
\end{minipage}
\hfill
\begin{minipage}{37.5ex}
\begin{flushright}
\rule{37.5ex}{.1ex} \\
\hspace{5ex} Chair Signature
\end{flushright}
\end{minipage}
\end{flushleft}

\vfill

\newpage

% -------------------------------------Abstract------------------------------
\begin{abstract}
A mathematical model is developed for deciding when a candidate running for a
United States Senate seat should spend campaign funds. The Senate race is modeled as a zero-sum game. This requires finding player strategies and generating player utilities based on those strategies.

Applying least-squares regression to FEC spending and fundraising data from
the 2016 U.S. Senate race, a fundraising return function is defined which is
used to generate possible strategies. Several different utility functions
are then calculated, using FEC spending data from the 2018 U.S. Senate race
and straw poll data collected by the website FiveThirtyEight.com.

The strategy space turns out to be quite large, and it was not possible to
solve the largest games directly using the standard methods from linear
programming. Therefore the model was tested using a combination of linear programming and Monte-Carlo methods. These simulations yield strategy solutions for both players.
\end{abstract}

% ---------------------------Intro----------------------------------------
\section{Introduction}

When determining optimal campaign spending strategies, most strategists focus
on location, demographics, and medium; much time and paper has been spent
researching these aspects of spending. An often overlooked aspect of a
campaign, and the main focus of this project, is \textit{when} in the campaign
to spend. A model is created using FEC senate campaign fundraising and
expenditure receipts, straw poll data from FiveThirtyEight, game theory, and
linear regression to create a weighted Blotto model of campaign spending,
focusing on which quarters to spend.
% INCLUDE FEC FORM


\section{Model Discussion}

% -------------------------------Definitions------------------------------------
\subsection{Game Theory}

The following definitions are used throughout the report:
\begin{itemize}
\item \textbf{Pure Strategy} --- A pure strategy is a strategy that requires
no randomization or deviation from a single strategy. A pure strategy in
poker, for example, is to fold every hand and never bet. This strategy
might look like $[1.0,0.0]$, with $1.0$ referring to the proportion of
hands a player should fold and $0.0$ referring to the proportion of hands
a player should bet on.
\item \textbf{Mixed Strategy} --- A mixed strategy is one that calls for the
employment of more than one strategy, and specifies how often each
strategy should be used. One mixed strategy in poker, for example, is to
fold 30$\%$ of hands while betting on the other $70\%$. This strategy
might look like $[0.3,0.7]$, with $0.3$ referring to the percentage of
hands a player should fold and $0.7$ referring to the percentage of hands
a player should bet on. The sum of a mixed strategy's elements should be
1.
\item \textbf{Strategy Vector} --- A strategy vector is a vector in which
\textit{each element} corresponds to one strategy a player can employ;
these may be any combination of \textbf{mixed strategies} and
\textbf{pure strategies}. To use the above examples, a strategy vector
for poker containing the two vectors above would look like
\begin{align*}
\left[ {\begin{array}{cc}
[1.0,0.0] & [0.3,0.7]
\end{array} } \right].
\end{align*}

In this report, each strategy in the strategy vector will contain four
elements. These refer to the amount a player will spend in each quarter
of a simulated campaign season. The strategy vectors, then, will be of
size $n \times 4$, where $n$ refers to the number of strategies, and
$4$ refers to the four quarters of the simulated election season. Each
player will have a corresponding strategy vector.
\item \textbf{Zero-Sum Game} --- This is a game in which the sum of all
payoffs for all players is zero. In a two player zero-sum game, the
payoffs can be represented by a single number. This single number $x$
will mean that the first player receives a payoff of $x$ and the second
player receives a payoff of $-x$.
\item \textbf{Value of a Game} --- The value of a two player zero sum game
is the expected payoff of the game when both players play optimal
strategies.
\item \textbf{Payout Matrix} --- A payout matrix the matrix with rows
indexed by the first players strategies, columns indexed by the second
players strategies, and (row, column) entry equal to the payoff when
(row,column) is played by the first and second players.

Consider a payout matrix $M$, corresponding to a game between players
P1 and P2:
\begin{align*}
\setlength\arraycolsep{8pt}
\left[ {\begin{array}{ccc}
t_1 & t_2 & t_3
\end{array} } \right]\\
M= \left[ {\begin{array}{c}
s_1 \\
s_2 \\
s_3 \\
\end{array} } \right]
\left[ {\begin{array}{ccc}
-1 & \textcolor{white}{-}2 & -1\\
\textcolor{white}{-}1 & \textcolor{white}{-}0 & \textcolor{white}{-}0\\
-1 & \textcolor{white}{-}1 & \textcolor{white}{-}1\\
\end{array} } \right].
\end{align*}
This matrix is 3$\times$3, since each player has 3 possible
strategies. Rows of the matrix correspond to possible payouts given
any strategy of P1 ($s_1$, $s_2$, or $s_3$ in this case) while
columns correspond to possible payouts given any strategy of P2
($t_1$, $t_2$, and $t_3$ in this case). Therefore, if $m_{j,k}$
refers to an element of $M$, then $m_{j,k}$ refers to the value of
playing these current strategies. A positive number is good for P1
and bad for P2.
\item \textbf{Optimal strategies} A strategy $S$ is optimal if a
player who uses $S$ will earn a least the value of the game,
regardless of the actions taken by his opponent. Optimal solutions
can be found using \textbf{linear programming}.
\end{itemize}

The next section outlines the basics of linear programming, and shows how it can be used to find optimal strategies.

% ---------------------------Linear Programming---------------------------------

\subsection{Finding a Game Value}
Linear programming can be used to find a solution to a matrix game.

For example, consider the following strategy vectors and associated
payout matrix $M$, of a zero-sum game between P1 and P2:
\begin{equation*}
\setlength\arraycolsep{9pt}
\textcolor{white}{+++}\left[ {\begin{array}{ccccc}
t_1 & t_2 & ... & t_j & ... \\
\end{array} } \right]
\end{equation*}
\begin{equation*}
\left[ {\begin{array}{c}
s_1 \\
s_2 \\
... \\
s_i \\
... \\
\end{array} } \right]
\left[ {\begin{array}{ccccc}
m_{1,1} & m_{1,2}   & ... & m_{1,j} & ... \\
m_{2,1} & m_{2,2}   & ... & m_{2,j} & ... \\
...    & ...      & ... & ...    & ... \\
m_{i,1} & m_{i,2}   & ... & m_{i,j} & ... \\
...    & ...      & ... & ...    & ... \\
\end{array} } \right]
\end{equation*}
The $\mathbf{s}$ vector denotes every possible strategy P1 can employ,
while the $\mathbf{t}$ vector denotes every possible strategy P2 can
employ. Row $i$ gives the payoffs when P1 plays strategy $i$ and column
$j$ gives the payoffs when P2 plays strategy $j$.

If P1 plays the pure strategy $s_i$ and P2 plays the pure strategy
$t_j$, then the value of the game is simply the corresponding matrix
element, $m_{i,j}$. These pure strategies can be represented as
vectors with a $1$ in the played strategy and $0$'s elsewhere. The
value of the game can now be solved using matrix multiplication, by
finding ${s^*}^T M t^*$. In this example, the resulting equation would
look like

\begin{equation*}
v = \left[ {\begin{array}{ccccc}
0.00 & 0.00 & ... & 1.00 & ... \\
\end{array} } \right]
\left[ {\begin{array}{ccccc}
m_{1,1} & m_{1,2}   & ... & m_{1,j} & ... \\
m_{2,1} & m_{2,2}   & ... & m_{2,j} & ... \\
...    & ...      & ... & ...    & ... \\
m_{i,1} & m_{i,2}   & ... & m_{i,j} & ... \\
...    & ...      & ... & ...    & ... \\
\end{array} } \right]
\left[ {\begin{array}{c}
0.00 \\
0.00 \\
... \\
1.00 \\
... \\
\end{array} } \right] = m_{i,j}
\end{equation*}

The value of a game in which players employ mixed strategies can be
calculated in a similar way. Suppose, instead of only playing strategy
$s_i$, P1 played each of their strategies in different
proportions. Then $s^*$ would be populated with decimals denoting,
proportionally, how often P1 chose to employ each strategy. For each
strategy $s_i$, denote the proportion of turns P1 uses strategy $s_i$
as $p_i$. Similarly, denote the proportion of turns P2 uses strategy
$t_j$ as $q_j$. The value of the game is
\begin{equation*}
v = \left[ {\begin{array}{ccccc}
p_1 & p_2 & ... & p_i & ... \\
\end{array} } \right]
\left[ {\begin{array}{ccccc}
m_{1,1} & m_{1,2}   & ... & m_{1,j} & ... \\
m_{2,1} & m_{2,2}   & ... & m_{2,j} & ... \\
...    & ...      & ... & ...    & ... \\
m_{i,1} & m_{i,2}   & ... & m_{i,j} & ... \\
...    & ...      & ... & ...    & ... \\
\end{array} } \right]
\left[ {\begin{array}{c}
q_1 \\
q_2 \\
... \\
q_j \\
... \\
\end{array} } \right]
\end{equation*}

\subsection{Linear Programming}

Linear programming can be used to find optimal strategies and the value
of a matrix game $M$. Consider, as an example, the following payout
matrix:
\begin{equation*}
\setlength\arraycolsep{8pt}
\textcolor{white}{++++++}\left[ {\begin{array}{ccc}
t_1 & t_2 & t_3
\end{array} } \right]
\end{equation*}
\begin{equation*}
M= \left[ {\begin{array}{c}
s_1 \\
s_2 \\
s_3 \\
\end{array} } \right]
\left[ {\begin{array}{ccc}
-1 & \textcolor{white}{-}2 & -1\\
\textcolor{white}{-}1 & \textcolor{white}{-}0 & \textcolor{white}{-}0\\
-1 & \textcolor{white}{-}1 & \textcolor{white}{-}1\\
\end{array} } \right]
\end{equation*}
In this matrix, P1 employs strategies corresponding to rows of $M$, P2
employs strategies corresponding to columns of $M$. P1's goal is to
maximize the game's value, while P2's goal is to minimize it. The matrix is written as follows:
\begin{equation*}
\begin{tabular}{ |p{0.5cm}|p{0.5cm} p{0.5cm} p{0.5cm}|  }
\hline
\multicolumn{4}{|c|}{M} \\
\hline
\textcolor{white}{.}& $t_1$ & $t_2$ & $t_3$\\
\hline
$s_1$ & $-1$ & $\textcolor{white}{-}$2 & $-1$\\
$s_2$ & $\textcolor{white}{-}$1 & $\textcolor{white}{-}$0 & $\textcolor{white}{-}$0\\
$s_3$ & $-1$ & $\textcolor{white}{-}$1 & $\textcolor{white}{-}$1\\
\hline
\end{tabular}
\end{equation*}

To find the solution to the matrix game, the Pivot
method may be used. To begin the algorithm, since there are no initial constraints
on the strategies, arbitrary constraints will be assigned. The purpose
of this is to track what happens to each row and column, which will
ultimately solve for how often to use each strategy. To do this, create
a column of 1's on the far right side of M, and a row of $-1$'s on the
bottom of $M$. Additionally, add a zero in the new bottom corner of the
matrix. This creates the initial tableau:

\begin{equation*}
\begin{tabular}{ |p{0.5cm}|p{0.5cm} p{0.5cm} p{0.5cm}|p{0.5cm}| }
\hline
\multicolumn{5}{|c|}{M} \\
\hline
\textcolor{white}{.}& $t_1$ & $t_2$ & $t_3$ & $\textcolor{white}{-}$\\
\hline
$s_1$ & $-1$ & $\textcolor{white}{-}$2 & $-1$ & $\textcolor{white}{-}$1\\
$s_2$ & $\textcolor{white}{-}$1 & $\textcolor{white}{-}$0 & $\textcolor{white}{-}$0 & $\textcolor{white}{-}$1\\
$s_3$ & $-1$ & $\textcolor{white}{-}$1 & $\textcolor{white}{-}$1 & $\textcolor{white}{-}$1\\
\hline
$\textcolor{white}{-}$ & $-1$ & $-1$ & $-1$ & $\textcolor{white}{-}$0\\
\hline
\end{tabular}
\end{equation*}
In order to use the simplex algorithm to solve the game, the value of
the game must be positive. Therefore, add a constant to each element to
ensure this. In this case, add 2, so that every value is
positive. Don't forget this step was taken; it's crucial to fix it at
the end.
\begin{equation*}
\begin{tabular}{ |p{0.5cm}|p{0.5cm} p{0.5cm} p{0.5cm}|p{0.5cm}| }
\hline
\multicolumn{5}{|c|}{M} \\
\hline
\textcolor{white}{.}& $t_1$ & $t_2$ & $t_3$ & \textcolor{white}{-}\\
\hline
$s_1$ & \textcolor{white}{-}1 & \textcolor{white}{-}4 & \textcolor{white}{-}1 & \textcolor{white}{-}1\\
$s_2$ & \textcolor{white}{-}3 & \textcolor{white}{-}2 & \textcolor{white}{-}2 & \textcolor{white}{-}1\\
$s_3$ & \textcolor{white}{-}1 & \textcolor{white}{-}3 & \textcolor{white}{-}3 & \textcolor{white}{-}1\\
\hline
\textcolor{white}{-} & $-1$ & $-1$ & $-1$ & $\textcolor{white}{-}$0\\
\hline
\end{tabular}
\end{equation*}

The next step is called a pivot operation. First, a pivot value must be
picked. Any number in $M$ (the original game matrix, not along the
added edges) may be picked, as long as it satisfies the following
constraints:
\begin{enumerate}
\item The value in the added bottom row corresponding to the pivot's
column MUST be negative. (This is trivial in the first step, since
all the numbers on the bottom are $-1$.)
\item The pivot itself must be positive.
\item The pivot row must be chosen to minimize the ratio:
$\frac{\text{Number in added column corresponding to pivot's row}}{\text{Pivot}}$
among the positive pivots in that column
\end{enumerate}

For example, suppose a pivot is being picked for the $t_1$ column (this
is possible because the value at the bottom of the $t_1$ column is
$-1$, which is negative). It may be any of 1, 3, or 1, since these are
the positive values in this column. Since the values on the very right
column for all these numbers are all 1, compare the ratios
$\frac{1}{1}$, $\frac{1}{3}$, and $\frac{1}{1}$. The smallest of these
is $\frac{1}{3}$; therefore, 3 is the first pivot. Circle the number 3
to indicate this.

\begin{equation*}
\begin{tabular}{ |p{0.5cm}|p{0.5cm} p{0.5cm} p{0.5cm}|p{0.5cm}| }
\hline
\multicolumn{5}{|c|}{M} \\
\hline
\textcolor{white}{.}& $t_1$ & $t_2$ & $t_3$ & $\textcolor{white}{-}$\\
\hline
$s_1$ & $\textcolor{white}{-}$1 & $\textcolor{white}{-}$4 & $\textcolor{white}{-}$1 & $\textcolor{white}{-}$1\\
$s_2$ & $\textcolor{white}{-}$\textcircled{3} & $\textcolor{white}{-}$2 & $\textcolor{white}{-}$2 & $\textcolor{white}{-}$1\\
$s_3$ & $\textcolor{white}{-}$1 & $\textcolor{white}{-}$3 & $\textcolor{white}{-}$3 & $\textcolor{white}{-}$1\\
\hline
$\textcolor{white}{-}$ & $-1$ & $-1$ & $-1$ & $\textcolor{white}{-}$0\\
\hline
\end{tabular}
\end{equation*}

Denote each element of $M$ as $m_{i,j}$, where $i$ refers to the row
and $j$ refers to the column. Let $i_p$ and $j_p$ denote the row and
column of the pivot; in this case, $i=1$ and $j=1$. First, for each
$m_{i,j}$ were $i\neq i_p$ and $j\neq j_p$ (all elements not in the same row
or column as the pivot), replace $m_{i,j}$ with
$m_{i,j}-\frac{m_{i,j_p}m_{i_p,j}}{p}$. This would look like:
\begin{equation*}
\renewcommand{\arraystretch}{1.5}
\begin{tabular}{ |p{0.5cm}|p{0.5cm} p{0.5cm} p{0.5cm}|p{0.5cm}| }
\hline
\multicolumn{5}{|c|}{M} \\
\hline
\textcolor{white}{.}& $t_1$ & $t_2$ & $t_3$ & $\textcolor{white}{-}$\\
\hline
$s_1$ & $\textcolor{white}{-}$1 & \textcolor{red}{$\frac{10}{3}$} & $\textcolor{white}{-}$\textcolor{red}{$\frac{1}{3}$} & $\textcolor{white}{-}$\textcolor{red}{$\frac{2}{3}$}\\
$s_2$ & \textcircled{3} & $\textcolor{white}{-}$2 & $\textcolor{white}{-}$2 & $\textcolor{white}{-}$1\\
$s_3$ & $\textcolor{white}{-}$1 & $\textcolor{white}{-}$\textcolor{red}{$\frac{7}{3}$} & $\textcolor{white}{-}$\textcolor{red}{$\frac{7}{3}$} & $\textcolor{white}{-}$\textcolor{red}{$\frac{2}{3}$}\\
\hline
$\textcolor{white}{-}$ & $-1$ & \textcolor{red}{$-\frac{1}{3}$} & \textcolor{red}{$-\frac{1}{3}$} & $\textcolor{white}{-}$\textcolor{red}{$\frac{1}{3}$}\\
\hline
\end{tabular}
\end{equation*}
Then replace all elements $m_{i,j_p}$ in the pivot column (except the
pivot itself) with $-\frac{m_{i_p,j}}{p}$:
\begin{equation*}
\renewcommand{\arraystretch}{1.5}
\begin{tabular}{ |p{0.5cm}|p{0.5cm} p{0.5cm} p{0.5cm}|p{0.5cm}| }
\hline
\multicolumn{5}{|c|}{M} \\
\hline
\textcolor{white}{.}& $t_1$ & $t_2$ & $t_3$ & $\textcolor{white}{-}$\\
\hline
$s_1$ & \textcolor{red}{$-\frac{1}{3}$} & $\frac{10}{3}$ & $\textcolor{white}{-}$$\frac{1}{3}$ & $\textcolor{white}{-}$$\frac{2}{3}$\\
$s_2$ & \textcircled{3} & $\textcolor{white}{-}$2 & $\textcolor{white}{-}$2 & $\textcolor{white}{-}$1\\
$s_3$ & \textcolor{red}{$-\frac{1}{3}$} & $\textcolor{white}{-}$$\frac{7}{3}$ & $\textcolor{white}{-}$$\frac{7}{3}$ & $\textcolor{white}{-}$$\frac{2}{3}$\\
\hline
$\textcolor{white}{-}$ & $\textcolor{white}{-}$\textcolor{red}{$\frac{1}{3}$} & $-\frac{1}{3}$ & $-\frac{1}{3}$ & $\textcolor{white}{-}$$\frac{1}{3}$\\
\hline
\end{tabular}
\end{equation*}
Next, replace all elements $m_{i_p,j}$ in the pivot row (except the
pivot itself) with $\frac{m_{i_p,j}}{p}$:
\begin{equation*}
\renewcommand{\arraystretch}{1.5}
\begin{tabular}{ |p{0.5cm}|p{0.5cm} p{0.5cm} p{0.5cm}|p{0.5cm}| }
\hline
\multicolumn{5}{|c|}{M} \\
\hline
\textcolor{white}{.}& $t_1$ & $t_2$ & $t_3$ & $\textcolor{white}{-}$\\
\hline
$s_1$ & $-\frac{1}{3}$ & $\frac{10}{3}$ & $\textcolor{white}{-}$$\frac{1}{3}$ & $\textcolor{white}{-}$$\frac{2}{3}$\\
$s_2$ & \textcircled{3} & $\textcolor{white}{-}$\textcolor{red}{$\frac{2}{3}$} & $\textcolor{white}{-}$\textcolor{red}{$\frac{2}{3}$} & $\textcolor{white}{-}$\textcolor{red}{$\frac{1}{3}$}\\
$s_3$ & $-\frac{1}{3}$ & $\textcolor{white}{-}$$\frac{7}{3}$ & $\textcolor{white}{-}$$\frac{7}{3}$ & $\textcolor{white}{-}$$\frac{2}{3}$\\
\hline
$\textcolor{white}{-}$ & $\textcolor{white}{-}$$\frac{1}{3}$ & $-\frac{1}{3}$ & $-\frac{1}{3}$ & $\textcolor{white}{-}$$\frac{1}{3}$\\
\hline
\end{tabular}
\end{equation*}
Then replace the pivot $p$ with $\frac{1}{p}$:
\begin{equation*}
\renewcommand{\arraystretch}{1.5}
\begin{tabular}{ |p{0.5cm}|p{0.5cm} p{0.5cm} p{0.5cm}|p{0.5cm}| }
\hline
\multicolumn{5}{|c|}{M} \\
\hline
\textcolor{white}{.}& $t_1$ & $t_2$ & $t_3$ & $\textcolor{white}{-}$\\
\hline
$s_1$ & $-\frac{1}{3}$ & $\frac{10}{3}$ & $\textcolor{white}{-}$$\frac{1}{3}$ & $\textcolor{white}{-}$$\frac{2}{3}$\\
$s_2$ & $\textcolor{white}{-}$\textcolor{red}{$\frac{1}{3}$} & $\textcolor{white}{-}$$\frac{2}{3}$ & $\textcolor{white}{-}$$\frac{2}{3}$ & $\textcolor{white}{-}$$\frac{1}{3}$\\
$s_3$ & $-\frac{1}{3}$ & $\textcolor{white}{-}$$\frac{7}{3}$ & $\textcolor{white}{-}$$\frac{7}{3}$ & $\textcolor{white}{-}$$\frac{2}{3}$\\
\hline
$\textcolor{white}{-}$ & $\textcolor{white}{-}$$\frac{1}{3}$ & $-\frac{1}{3}$ & $-\frac{1}{3}$ & $\textcolor{white}{-}$$\frac{1}{3}$\\
\hline
\end{tabular}
\end{equation*}
Last, switch the labels in the same row and column as the pivot (in this case, $s_1$ and $t_1$):
\begin{equation*}
\renewcommand{\arraystretch}{1.5}
\begin{tabular}{ |p{0.5cm}|p{0.5cm} p{0.5cm} p{0.5cm}|p{0.5cm}| }
\hline
\multicolumn{5}{|c|}{M} \\
\hline
\textcolor{white}{.}& \textcolor{red}{$s_2$} & $t_2$ & $t_3$ & $\textcolor{white}{-}$\\
\hline
$s_1$ & $-\frac{1}{3}$ & $\frac{10}{3}$ & $\textcolor{white}{-}$$\frac{1}{3}$ & $\textcolor{white}{-}$$\frac{2}{3}$\\
\textcolor{red}{$t_1$} & $\textcolor{white}{-}$$\frac{1}{3}$ & $\textcolor{white}{-}$$\frac{2}{3}$ & $\textcolor{white}{-}$$\frac{2}{3}$ & $\textcolor{white}{-}$$\frac{1}{3}$\\
$s_3$ & $-\frac{1}{3}$ & $\textcolor{white}{-}$$\frac{7}{3}$ & $\textcolor{white}{-}$$\frac{7}{3}$ & $\textcolor{white}{-}$$\frac{2}{3}$\\
\hline
$\textcolor{white}{-}$ & $\textcolor{white}{-}$$\frac{1}{3}$ & $-\frac{1}{3}$ & $-\frac{1}{3}$ & $\textcolor{white}{-}$$\frac{1}{3}$\\
\hline
\end{tabular}
\end{equation*}

Step 1 is now complete! Repeat this process, picking a new pivot, until
there are no more negative values on the added bottom row. Once this is
complete, the only thing that matters is the added row and column,
which indicate optimal strategies for each player. When this matrix is
solved, it looks like:
\begin{equation*}
\renewcommand{\arraystretch}{1.5}
\begin{tabular}{ |p{0.5cm}|p{0.7cm} p{0.7cm} p{0.7cm}|p{0.7cm}| }
\hline
\multicolumn{5}{|c|}{M} \\
\hline
$\textcolor{white}{-}$ & $s_2$ & $s_1$ & $s_3$ & $\textcolor{white}{-}$\\
\hline
$t_2$ & $-\frac{2}{21}$ & $\textcolor{white}{-}$$\frac{1}{3}$ & $-\frac{1}{21}$ & $\textcolor{white}{-}$$\frac{4}{21}$\\
$t_1$ & \textcolor{white}{--}$\frac{3}{7}$ & $\textcolor{white}{-}$0 & $\textcolor{white}{-}$$-\frac{3}{7}$ & \textcolor{white}{--}$\frac{1}{7}$\\
$t_3$ & $-\frac{1}{21}$ & $-\frac{1}{3}$ & $\textcolor{white}{-}$$\frac{10}{21}$ & $\textcolor{white}{-}$$\frac{2}{21}$\\
\hline
$\textcolor{white}{-}$ & \textcolor{white}{--}$\frac{2}{7}$ & $\textcolor{white}{-}$0 & \textcolor{white}{--}$\frac{1}{7}$ & \textcolor{white}{--}$\frac{3}{7}$\\
\hline
\end{tabular}
\end{equation*}
To simplify the solution, here's a more straightforward version of the
matrix:
\begin{equation*}
\renewcommand{\arraystretch}{1.5}
\begin{tabular}{ |p{0.5cm}|p{0.7cm} p{0.7cm} p{0.7cm}|p{0.7cm}| }
\hline
\multicolumn{5}{|c|}{M} \\
\hline
$\textcolor{white}{-}$ & $s_2$ & $s_1$ & $s_3$ & $\textcolor{white}{-}$\\
\hline
$t_2$ & * & * & * & $\textcolor{white}{-}$$\frac{4}{21}$\\
$t_1$ & * & * & * & \textcolor{white}{--}$\frac{1}{7}$\\
$t_3$ & * & * & * & $\textcolor{white}{-}$$\frac{2}{21}$\\
\hline
$\textcolor{white}{-}$ & \textcolor{white}{--}$\frac{2}{7}$ & $\textcolor{white}{-}$0 & \textcolor{white}{--}$\frac{1}{7}$ & \textcolor{white}{--}$\frac{3}{7}$\\
\hline
\end{tabular}
\end{equation*}

The only numbers needed from this matrix are those on the added bottom
row corresponding to
$s$ values, those on the added far right column corresponding to
$t$ values, and the value of the game on the bottom right. Any
$s$ or
$t$ strategy that did not switch axes would now be set to zero (there
were none here, but that's not always the case).

But what do these numbers mean? To find out, divide each number by the
value at the very bottom right ($\frac{3}{7}$, in this case). After
this, subtract the 2 that was added to the value at the beginning of
the game. This gives
\begin{equation*}
\renewcommand{\arraystretch}{1.5}
\begin{tabular}{ |p{0.5cm}|p{0.7cm} p{0.7cm} p{0.7cm}|p{0.7cm}| }
\hline
\multicolumn{5}{|c|}{M} \\
\hline
$\textcolor{white}{-}$ & $s_2$ & $s_1$ & $s_3$ & $\textcolor{white}{-}$\\
\hline
$t_2$ & * & * & * & \textcolor{white}{--}$\frac{4}{9}$\\
$t_1$ & * & * & * & \textcolor{white}{--}$\frac{1}{3}$\\
$t_3$ & * & * & * & \textcolor{white}{--}$\frac{2}{9}$\\
\hline
$\textcolor{white}{-}$ & $\frac{2}{3}$ & 0 & $\frac{1}{3}$ & $-\frac{11}{7}$\\
\hline
\end{tabular}
\end{equation*}
The number across from each strategy now denotes what percentage of
turns a player should employ that strategy. The value at the bottom
right of the matrix now denotes the average value of the game, if each
player plays optimally. Therefore:
\begin{itemize}
\item To play optimally, P1 should play $s_1$m 0\% of the time, $s_2$
66.67\% of the time, and $s_3$ 33.33\% of the time.
\item To play optimally, P2 should play $t_1$m 33.33\% of the time,
$t_2$ 44.44\% of the time, and $t_3$ 22.22\% of the time.
\item If both players employ these strategies over an infinite number
of games, the average value of these games will be $-\frac{11}{7}$
(in favor of P2).
\end{itemize}
For more information on why this algorithm works, see
\cite{gametheory}.

\subsection{Colonel Blotto Problem}

The campaign trail is often compared to a battlefield because of its
zero-sum nature and the tactics involved in winning. It makes sense,
then, that games used to model battle strategies can also be used to
model campaign strategies.

One model is given by the Colonel Blotto Problem. There lives an
acclaimed strategist named Colonel Blotto (actually there doesn't, but
for some reason that name has stuck in the math world). Colonel Blotto
is in the final dregs of a battle against his rival, Commodore
Hopper\cite{hopper} (the name of his opponent varies by author). To win the battle, Colonel Blotto has to defeat
his opponent in the last 2 remaining battlefields, $f_1$ and $f_2$. To
do this, he had to find a way to allot his 4 armies between the two
battlefields. His opponent, Commodore Hopper, has a similar charge ---
however, Commodore Hopper only has 3 armies.

The utility function was defined as follows. Let $b_1$ be the number of
armies Colonel Blotto allots to $f_1$, and $b_2$ be the number of
armies he allots to $f_2$. Similarly, let $h_1$ be the number of armies
Commodore Hopper allots to $f_1$ and $h_2$ be the number of armies she
allots to $f_2$. This yields different strategy vectors for each
player,
\begin{equation}\label{eq:bstrats1}
b = \{(4,0), (3,1), (2,2), (1,3), (0,4)\},
\end{equation}
\begin{equation}\label{eq:bstrats2}
h = \{(3,0), (2,1), (1,2), (0,3)\},
\end{equation}
where $b$ is the strategy vector for Colonel Blotto and $h$ is the
strategy vector for Commodore Hopper. Each element of the strategy
vector is a \textbf{tuple} (or finite, ordered list) of size 2,
consisting of $(b_1, b_2)$ for Colonel Blotto and $(h_1,h_2)$ for
Commodore Hopper. In the most basic Blotto model, the utility function
$m_{i,j}$ is defined as $1$ if Blotto wins the most battles, $-1$ if Hopper
wins the most battles, and 0 if they win the same number of
battles. Informally, this can be written:
\begin{equation*}
m_{i,j} = \begin{cases}
\hphantom{-}1, & \text{if Blotto wins more battles than Hopper}\\
\hphantom{-}0, & \text{if Blotto and Hopper win the same number of battles}\\
-1, & \text{if Hopper wins more battles than Blotto}
\end{cases}
\end{equation*}
This can be written more mathematically as well, as follows:
\begin{equation*}
m_{i,j} = \begin{cases}
\hphantom{-}1, & \text{if } (b_1-h_1)+(b_2-h_2)>0\\
\hphantom{-}0, & \text{if } (b_1-h_1)+(b_2-h_2)=0\\
-1, & \text{if } (b_1-h_1)+(b_2-h_2)<0
\end{cases}
\end{equation*}
This yields the following payout matrix, with rows denoting Blotto's
strategies and columns denoting Hopper's:
\begin{align*}
\setlength\arraycolsep{9pt}
\hphantom{++++++++++++}\left[{\begin{array}{cccc}
(3,0), (2,1), (1,2), (0,3) \\
\end{array} } \right]
\end{align*}
\begin{equation*}\hphantom{++++++++}
\left[ {\begin{array}{c}
(4,0) \\
(3,1) \\
(2,2) \\
(1,3) \\
(0,4) \\
\end{array} } \right]
\left[ {\begin{array}{cccc}
1\hphantom{(,0)} & 0\hphantom{(,0)} & 0\hphantom{(,0)} & 0 \\
1\hphantom{(,0)} & 1\hphantom{(,0)} & 0\hphantom{(,0)} & 0 \\
0\hphantom{(,0)} & 1\hphantom{(,0)} & 1\hphantom{(,0)} & 0 \\
0\hphantom{(,0)} & 0\hphantom{(,0)} & 1\hphantom{(,0)} & 1 \\
0\hphantom{(,0)} & 0\hphantom{(,0)} & 0\hphantom{(,0)} & 1 \\
\end{array} } \right].
\end{equation*}

In this case, Hopper does not win in any case; however, even with fewer
troops, she can tie Blotto overall by winning one of the two
battles. While this specific payout matrix has no negative values, this
is not necessarily true in a more generalized game.

This game can be written in a similar but more nuanced way. Rather than
using the utility function as above, payouts can be weighted with the
differential by which a battle was won. This gives more points for a
strong win, rather than a single value for all battles won. This problem
is called a \textbf{Weighted Blotto Model}, and its utility function is
\begin{equation*}
m_{i,j} = (b_1-h_1)+(b_2-h_2)
\end{equation*}

If \eqref{eq:bstrats1} and \eqref{eq:bstrats2} are used, $m_i,j$ is equal
to 1 for every element of the matrix (since
$(b_1-h_1)+(b_2-h_2)=(b_1+b_2)-(h_1-h_2)=b-h=5-4=1$), giving a relatively
uninteresting game. However, suppose the two leaders are given more
constraints in the ways they arrange their armies - perhaps Blotto can
send at most 2 armies to $f_1$ due to the difficulty of moving troops,
and Hopper can gain two extra armies from an ally if she sends all of her
troops to $f_2$. These new strategy vectors would be
\begin{equation*}
b = \{(2,2), (1,3), (0,4)\}
\end{equation*}
\begin{equation*}
h = \{(3,0), (2,1), (1,2), (0,5)\}
\end{equation*}
This yields the payout matrix
\begin{align*}
\setlength\arraycolsep{9pt}
\hphantom{++++++++++++}\left[{\begin{array}{cccc}
(3,0), (2,1), (1,2), (0,5) \\
\end{array} } \right]
\end{align*}
\begin{equation*}\hphantom{++++++++}
\left[ {\begin{array}{c}
(2,2) \\
(1,3) \\
(0,4) \\
\end{array} } \right]
\left[ {\begin{array}{cccc}
1\hphantom{(,0)} & 1\hphantom{(,0)} & 1\hphantom{(,)} & -1 \\
1\hphantom{(,0)} & 1\hphantom{(,0)} & 1\hphantom{(,)} & -1 \\
1\hphantom{(,0)} & 1\hphantom{(,0)} & 1\hphantom{(,)} & -1 \\
\end{array} } \right].
\end{equation*}
As the strategy vectors become more complex, the payout matrix has the
potential to become more interesting as well.

While all these examples divide armies into different locations, this
is not the only way to create strategies. In political campaigns,
dollars spent can replace armies allocated, and fiscal quarters can
replace battlefields. The utility function, then, outputs the
percentage of votes a candidate wins over the course of the campaign,
and is a sum of differentials in poll numbers over the course of the
campaign. This will be explained in more detail in Section 4:
\textbf{Methods}.
% -----------------------------Monte-Carlo-----------------------------------

\subsection{Monte-Carlo Method}

Some matrices, though finite and solvable through linear programming,
are so large that they can't be solved in any reasonable amount of
time. In these cases, a very close estimate of the correct answer is
acceptable, and much more achievable. These answers can be solved using
subsets of the original strategy vectors and many simulations. This
method, named after the casino in Monaco\cite{montecarlo}, is called
the Monte-Carlo method. This employs the following process:
\begin{itemize}
\item For each simulation:
\begin{enumerate}
\item Select a new subset of strategies for each player
\item Create a payout matrix using only these strategies
\item Solve this small payout matrix using linear programming
\item Record optimal strategies on this matrix
\item Add each of these proportions to an array corresponding to every
possible strategy for each player, where each proportion goes in its
corresponding spot
\end{enumerate}
\item Repeat the above steps many, many times
\item There now exists an array, with high values for strategies that won
a lot and low numbers for strategies that did not. Divide each of these
numbers by the number of repetitions to find the approximate proportion
of times each strategy should be used in optimal play.
\end{itemize}
As an example, consider the following payout matrix:
\begin{equation*}
\begin{tabular}{ |p{0.5cm}|p{0.5cm} p{0.5cm} p{0.5cm}|  }
\hline
\multicolumn{4}{|c|}{M} \\
\hline
\textcolor{white}{.}& $t_1$ & $t_2$ & $t_3$\\
\hline
$s_1$ & $-1$ & $\textcolor{white}{-}$2 & $-1$\\
$s_2$ & $\textcolor{white}{-}$1 & $\textcolor{white}{-}$0 & $\textcolor{white}{-}$0\\
$s_3$ & $-1$ & $\textcolor{white}{-}$1 & $\textcolor{white}{-}$1\\
\hline
\end{tabular}
\end{equation*}
One way to solve this is direct linear programming, outlined in Section
2.3: \textbf{Linear Programming}. Another way, however, is to test many
different subsets of $s$ and $t$ against each other, and see which
strategy wins the most. For this example, use 2 as a subset
size. First, create empty arrays $s$ and $t$, whose elements correspond
to how often each strategy should be played:
\begin{equation*}
\label{montecarlos}
s = [s_1\:wins, s_2\:wins, s_3\:wins] = [0,0,0]
\end{equation*}
\begin{equation*}
\label{montecarlot}
t = [t_1\:wins, t_2\:wins, t_3\:wins] = [0,0,0]
\end{equation*}
Next, pick two strategies randomly for each player, and construct a payout
matrix using only these strategies.
\begin{equation*}
\begin{tabular}{ |p{0.5cm}|p{0.5cm} p{0.5cm}|  }
\hline
\multicolumn{3}{|c|}{M} \\
\hline
\textcolor{white}{.}& $t_2$ & $t_3$\\
\hline
$s_1$ & $\textcolor{white}{-}$2 & $-1$\\
$s_3$ & $\textcolor{white}{-}$1 & $\textcolor{white}{-}$1\\
\hline
\end{tabular}
\end{equation*}
Next, solve this matrix using linear programming. This yields the
following optimal strategies:
\begin{itemize}
\item P1: Never play $s_1$, always play $s_3$
\item P2: Play $t_2$ on $\frac{2}{3}$ of turns, play $t_3$ on $\frac{1}{3}$ of turns
\end{itemize}
Add these proportions to the corresponding spots in \eqref{montecarlos} and \eqref{montecarlot}:
\begin{equation*}
s = [0,0,1]
\end{equation*}
\begin{equation*}
t = [0,0.6667,0.3333]
\end{equation*}
Now repeat the process. Pick another two strategies randomly for each player, and construct and solve a payout matrix using these strategies:
\begin{equation*}
\begin{tabular}{ |p{0.5cm}|p{0.5cm} p{0.5cm}|  }
\hline
\multicolumn{3}{|c|}{M} \\
\hline
\textcolor{white}{.}& $t_1$ & $t_3$\\
\hline
$s_1$ & $-1$ & $-1$\\
$s_2$ & $\textcolor{white}{-}$1 & $\textcolor{white}{-}$0\\
\hline
\end{tabular}
\end{equation*}
This yields the following optimum strategies:
\begin{itemize}
\item P1: Never play $s_1$, always play $s_2$
\item P2: Never play $t_1$, always play $t_3$
\end{itemize}
Add these to $s$ and $t$:
\begin{equation*}
s = [0,1,1]
\end{equation*}
\begin{equation*}
t = [0,0.6667,1.3333]
\end{equation*}
After 1,000,000 simulations like this, $s$ and $t$ will look something like
this:
\begin{equation*}
s = [247.3980,666698.3446,333054.2574]
\end{equation*}
\begin{equation*}
t = [333498.3152,444330.4690,222171.2158]
\end{equation*}
Dividing each element of $s$ and $t$ by 1,000,000 yields
\begin{equation*}
s = [0,0.667,0.333]
\end{equation*}
\begin{equation*}
t = [0.333,0.444,0.222]
\end{equation*}
Do these look familiar? They should! They're the same answers derived from
direct linear programming of the entire matrix (see Section 2.3:
\textbf{Linear Programming}). Sometimes these numbers are slightly
different than the correct ones, but increasing the number of simulations
almost always takes care of this.

% ----------------------------Linear Regression---------------------------------

\subsection{Linear Regression}

Linear regression is a method used mostly in statistics to determine
linear trends in a discrete dataset. Consider a dataset $D$, populated
by 2-dimensional data (\textbf{Figure \ref{fakedata}}. The goal of
linear regression is to find a single linear function to describe the
trend of the data over the $x$-axis in terms of the $y$-axis. In
simpler terms, it's a line that shows a trend in the data. Doing this
requires a simple optimization equation.

First, consider the set of data points in \textbf{Figure
\ref{fakedata}}. Call these points
$\{(x_i,y_i)|i\in [1,16] \cap \mathbb{N} \}$ (this is just a way to index
the 16 points in the set). The goal is to find a line that follows the
trend of the data as closely as possible. Plotting any line will create
an error between the line and the data; therefore, to get the most
accurate line, one must minimize that error equation.

\pgfplotsset{width=7cm,compat=1.9}
\begin{figure}
\centering
\begin{tikzpicture}
\begin{axis}[
title={Data Without Regresssion Line},
xlabel={$x$-axis},
ylabel={$y$-axis},
xmin=0, xmax=1.0,
ymin=0, ymax=1.0,
xtick={0,0.25,0.5,0.75,1.0},
ytick={0,0.25,0.5,0.75,1.0},
legend pos=north west,
ymajorgrids=true,
grid style=dashed,
]
\addplot[
color=blue,
mark=*,
draw=none
]
coordinates {
(0.1,0.14)(0.15,0.21)(0.22,0.25)(0.23,0.49)(0.29,0.19)(0.34,0.35)(0.46,0.43)(0.5,0.44)(0.53,0.31)(0.54,0.66)(0.61,0.54)(0.69,0.59)(0.75,0.76)(0.81,0.69)(0.87,0.81)(0.94,0.8)
};
\end{axis}
\end{tikzpicture}
\begin{tikzpicture}
\begin{axis}[
title={Data With Regression Line},
xlabel={$x$-axis},
ylabel={$y$-axis},
xmin=0, xmax=1.0,
ymin=0, ymax=1.0,
xtick={0,0.25,0.5,0.75,1.0},
ytick={0,0.25,0.5,0.75,1.0},
legend pos=north west,
ymajorgrids=true,
grid style=dashed,
]
\addplot[
color=blue,
mark=*,
draw=none
]
coordinates {
(0.1,0.14)(0.15,0.21)(0.22,0.25)(0.23,0.49)(0.29,0.19)(0.34,0.35)(0.46,0.43)(0.5,0.44)(0.53,0.31)(0.54,0.66)(0.61,0.54)(0.69,0.59)(0.75,0.76)(0.81,0.69)(0.87,0.81)(0.94,0.8)
};
\addplot[
color=black,
]
coordinates {
(0,.093)(1,0.862)
};
\end{axis}
\end{tikzpicture}
\caption{Theoretical data set, with and without regression line.}
\label{fakedata}
\end{figure}

To find the regression line, the first step is to calculate a simple mean
of the $x_i$ and $y_i$ values. The equations for the means of this dataset,
$\bar{x}$ and $\bar{y}$ respectively, are
\begin{equation*}
\bar{x} = \frac{\sum_{i=1}^{n}x_i}{n}\hphantom{++++}
\bar{y} = \frac{\sum_{i=1}^{n}y_i}{n}
\end{equation*}
where $n$ is the number of points in the data set. Next, to calculate
the difference between each $x$ and $y$ value and their respective
means, use the equations
\begin{equation*}
a_i = x_i - \bar{x}\hphantom{++++}
b_i = y_i - \bar{y}
\end{equation*}
From these variables, a slope $m$ can be calculated. This value
indicates how strongly the data is correlated - an $m$ close to 1
indicates a strong direct correlation between the variables, while an
$m$ close to -1 indicates a strong inverse correlation. $m$ values
close to 0 indicate extremely weak correlation between the variables,
or none at all. This $m$ value is calculated by
\begin{equation*}
m = \frac{\sum_{i=1}^{n}(x_i-\bar{x})(y_i-\bar{y})}{\sum_{i=1}^{n}(x-\bar{x})^2}
\end{equation*}
There are other ways to calculate this $m$ value, which won't be
covered in this paper.

For the data in \textbf{Figure \ref{fakedata}}, this $r_{xy}$ value
turns out to be 0.769. This implies that a line with slope 0.769 is the
the simple linear function which most closely models the data. This
value can be calculated much more easily on Microsoft Excel, using the
\texttt{SLOPE()} funciton \cite{slope}. Normally when using regression,
a $y$-intecept value is also calculated. However, in this model, only
the slope was used.

% ----------------------------------Game----------------------------------------
\section{The Game}

% --------------------------------- Initial Assumptions ---------------------------------

\subsection{Initial Assumptions}

Before even beginning to model the election, some initial assumptions are
required. These serve as the basis for the rules of the game, since they allow
abstract concepts to be quanitfied as variables or equations. These assumptions
are as follows:
\begin{enumerate}
\item A typical U.S. Senate campaign starts as early as January $1^{st}$ of the
election year.
\item Trends in the spending and fundraising data from the 2016 and 2018 United
States Senate races are representative of trends in spending and fundraising
data for all modern U.S. Senate races
\item Because spending and fundraising reports for U.S. Senate races are due on
the first day of January, April, July, and October \cite{fecform}, these
dates can be reasonably chosen to denote the end of a fiscal
quarter. Furthermore, spending and fundraising strategies can be categorized
into these three-month increments:
\begin{itemize}
\item Quarter 1 (Q1) of a given year is January $1^{st}$ - March $31^{st}$ of
that year
\item Quarter 2 (Q2) of a given year is April $1^{st}$ - June $30^{th}$ of
that year
\item Quarter 3 (Q3) of a given year is July $1^{st}$ - September $30^{th}$
of that year
\item Quarter 4 (Q4) of a given year is October $1^{st}$ - December $31^{st}$
of that year
\end{itemize}
\item The spending and fundraising strategies of a U.S. Senate campaign are
comprehensively reported on the FEC website. Any PAC or committee not listed
as an approved committee on the FEC website is not considered in spending and
fundraising strategies, nor are they considered in calculating the
fundraising return function.
\item U.S. Senate candidates in the 2016 election who also ran for President
collected votes and money using strategies fundamentally different than other
U.S. Senate candidates, and these races were therefore not considered in any
calculation.
\item How much a candidate will fundraise in any given quarter may be
reasonably predicted as a linear function of past spending and fundraising.
\item Trends in a candidate's popularity in any given quarter may be reasonably
predicted as a function of that candidate's entire last quarter of spending,
their opponent's entire last quarter of spending, and differences in the
results of independently collected straw polls.
\item U.S. Senate candidates spend their entire budget by the end of the last quarter of their campaign.
\end{enumerate}
Obviously, many of these assumptions are \textit{at least} somewhat removed from
reality. Assumption 1, for example, wrongly assumes that candidates do not
begin campaigning before January 1 of an election year. Assumptions 5, 6, 7, and 8
are also extremely presumptive. To calculate any sort of reasonable
model in the time allotted, however, these assumptions had to be made. As further
outlined in Section 6: \textbf{Critiques}, the model has immense room for
improvement in many aspects.

Throughout the report, the utility vectors take the form 
\begin{align*}
[a, b, c, d] \cdot 2\times 10^{-5}.
\end{align*}
In the code, spending amounts were calculated in increments of \$200,000, rather than in dollar amounts. Therefore, utility vectors which in the report which would have appeared as 
\begin{align*}
[0.00001242, 0.00000143, -0.00001119, 0.00003519] 
\end{align*}
would be written in the code as 
\begin{align*}
[2.484, 0.286, -2.238, 7.038].
\end{align*}
In other words, every element of the utility vectors in the code are multiplied by 200,000 (or $2 \times 10^{5}$). As it turns out, these vectors are much easier to read and interpret. Therefore, rather than writing them in the report in their original form, each number is left multiplied by 200,000, and the scalar $2 \times 10^{-5}$ is left outside the matrix to indicate this change.

With these assumptions in place, a proper game may be constructed. Like any
good game, this model utilizes a nice set of rules, outlined below.

\subsection{The Rules}
\begin{enumerate}
\item The Race for U.S. Senate is a zero-sum game consisting of two players, A
and B.
\item Each player has an initial budget, \(y_A\) and \(y_B\). A player may
never spend more in any quarter than their budget holds at the beginning of that quarter, after fundraising is accounted for.
\item There are 4 rounds in the game, corresponding to 4 quarters (3-month
blocks) outlined in Section 3.1: \textbf{Assumptions}. Candidates play each
round simultaneously.
\item Each round consists of two phases: Players receive fundraising, and
players choose an amount to spend in the current month.
\begin{itemize}
\item Fundraised money is earned in phase 1 as a function of a player's last
fundraised amount and most recently spent amount. This money is added to
their budget before they pick another spending strategy.
\item On the first turn, a player receives no fundraising in addition to
their initial budget. Therefore, fundraising added in round two is given as a function of that player's Q1 spending and their initial budget.
\item After fundraising is added to each player's budget, they may choose to
spend any amount of money between zero and their current budget, as long as
the amount is a whole number.
\end{itemize}
\item The fundraising return function for each player in quarter $i$ is given
by
\begin{equation*}
f_i(f_{i-1}, s_i) = 0.1124s_i + 1.1057f_{i-1}
\end{equation*}
where $s_i$ denotes the last player's amount spent in quarter $i$, and
$f_{i-1}$ denotes their amount fundraised in the previous quarter. This
function is not implemented in the first turn. At the beginning of the
second turn, $f_{i-1}$ is replaced by the player's initial budget
$y_0$.
\item The game also includes an unspecified, finite, and constant number
of voters. Each player's objective for the game is to maximize the
\textit{percentage} of these voters who will vote for them at the end
of the last round.
\item Players must only spend money in multiples of \$200,000 each round,
except in the last round.
\item Players must spend their entire remaining budget in the last round
of the game.
\item The \textbf{cost of votes} is defined by how much \textbf{more} a
player must spend than their opponent in that quarter to move 1\% of
votes (for P1, this move is positive; for P2, this move is
negative). The utility function for each quarter, then, is defined by
percentage of votes a player gains by spending \$1 more than
their opponent in that round.
\item The utility function is unique to each round.
\begin{itemize}
\item In Round 1, the utility function is defined by
\begin{equation*}
u(s_1,t_1) = (0.611974882 \cdot 2 \times 10^5) \cdot(s_1-t_1)
\end{equation*}
where $s_1$ is the amount P1 spends in Round 1, $t_1$ is the
amount P2 spends in Round 1.
\item In Round 2, the utility function is defined by
\begin{equation*}
u(s_2,t_2) = (0.214168992 \cdot 2 \times 10^5) \cdot(s_2-t_2) 
\end{equation*}
where $s_2$ is the amount P1 spends in Round 2, $t_2$ is the
amount P2 spends in Round 2.
\item In Round 3, the utility function is defined by
\begin{equation*}
u(s_3,t_3) = (0.234584564 \cdot 2 \times 10^5) \cdot (s_3-t_3)
\end{equation*}
where $s_3$ is the amount P1 spends in Round 3, $t_3$ is the
amount P3 spends in Round 3.
\item In Round 4, the utility function is defined by
\begin{equation*}
u(s_4,t_4) = (2.111203038 \cdot 2 \times 10^5) \cdot(s_4-t_4)
\end{equation*}
where $s_4$ is the amount P1 spends in Round 4, $t_4$ is the
amount P2 spends in Round 4.
\end{itemize}
\item A negative value at the end of the game denotes a win for P1. A
positive value at the end of the game denotes a win for P2. The absolute value of the game value itself denotes
\begin{equation*}
\mid v \mid = \mid v_1 - v_2 \mid,
\end{equation*}
where $v_1$ denotes the percentage of voters who voted for P1, and $v_2$ denotes the percentage of voters who voted for P2. In other words, $\mid v \mid$ denotes the point differential of the winner over the loser.
\end{enumerate}

\newpage
\section{Methods}
% -----------------------------Methods------------------------------------

Construction of the game matrix consisted of three discrete parts: a
\textbf{fundraising return function}, a \textbf{weighted payout vector}, and
each player's \textbf{strategy vector}.
\begin{itemize}
\item The \textbf{fundraising return function} models how candidates can make
money later in a campaign by spending their money on fundraising events and
advertising
\item The \textbf{weighted payout vector} essentially models how expensive it
is to gain votes in different quarters of a campaign
\item The \textbf{strategy vectors} model all possible spending strategies a
player can employ, given a specified budget and number of quarters
\end{itemize}

\subsection{Fundraising Return Function}
Building the fundraising function required, first of all, lots of data on how
past candidates raised and spent their funds. This data can be found on the FEC
website \cite{fec}. Data was downloaded for the two most popular Senate
candidates in each 2016 race; this was determined by simple comparison of final
poll numbers posted on the FEC website. The data was then grouped into
different time increments of spending and fundraising, including months and
quarters.

The next step was finding a the fundraising return function, using spending and
fundraising as parameters. The output was the amount of money a candidate would
gain from fundraising in a current time increment, dependent on their past
spending and fundraising. This function was found using least-squares
regression, and was calculated using both month-long and quarter-long time
increments, with quarters of each year being defined as the following:
\begin{itemize}
\item Q1 defined as January $1^{st}$ - March $31^{st}$ of a single year
\item Q2 defined as April $1^{st}$ - June $30^{th}$ of a single year
\item Q3 defined as July $1^{st}$ - September $30^{th}$ of a single year
\item Q4 defined as October $1^{st}$ - December $31^{st}$ of a single year
\end{itemize}
Note that in Q4 of an election year, candidates generally only spend money up
until November $6^{th}$, when the election takes place.

The challenge was finding which parameters most accurately predicted a
candidate's fundraising. After consultation with subject matter experts (Max Hokit, Staff Assistant to California Congressman Salud Carbajal\cite{max} and Jacqueline Wetzel, Financial Director to Texas State Senator Beverly Powell\cite{jackie}), it was determined that any number of factors could
affect campaign fundraising for a given quarter; however, the most influential
contributing factors would be the most recent fundraising amounts the most recent spending
amounts. Using FEC data on the 2016 Senate race, multiple possible formulas
were explored for creating this return function. This function has immense room
for improvement, but in order to simplify the model, it was solved as a
function of the past quarter's spending and the past quarter's fundraising. To
solve for the final function, data was gathered for each Senate candidate on
how much they spent and fundraised each quarter of 2017 and 2018. These amounts
were determined by accessing the FEC Campaign Finance Database \cite{fec}, and
including the fundraising and expenditure receipts from the candidate's FEC
profile, as well as all authorized committees associated with that profile. As
will be expanded upon in Section 7: \textbf{Critiques}, the greatest
confounding variables in this function are likely:
\begin{itemize}
\item Committees not authorized by the candidate who spent and fundraised
independently for the campaign, which creates neither expenditure nor
fundraising receipts
\item Volunteer work done for the campaign, which creates no expenditure
receipts
\item Candidates whose fundraising was based more on social media messaging
(and other free forms of messaging) than it was on tactics which create
expenditure receipts
\item News stories in the media that either curbed or increased fundraising for
a given quarter, which may have indirectly increased or decreased spending in
the campaign to make up for lost points
\item Campaigns that incorrectly report expenditures
\end{itemize}
None of these possibilities could be accounted for in this simplistic model of
fundraising, creating one of the major weaknesses in the model. This weakness
is touched upon again later, in Section 6: \textbf{Critiques}.

Using least-squares approximation in MATLAB, many possible solutions to this
fundraising equation were calculated. These solutions used many different
combinations of past spending and past fundraising; however, with little hard
evidence backing a specific solution, the simplest equation was picked. This
equation, solving for the fundraising $f_i$ garnered in month $i$, was
ultimately calculated as a function of the previous quarter's fundraising
($f_{i-1}$) and the total amount spent in month $i$ ($s_i$). Least-squares
approximation yielded the following equation:

\begin{equation*}
f_i(f_{i-1}, s_i) = 0.1124s_i + 1.1057f_{i-1}
\end{equation*}

In this equation, the $f_{i-1}$ mandates that a candidate fundraise more each
consecutive quarter, which is typically (but not universally) true. In
addition, spending more also allows a candidate to fundraise more effectively
that quarter, which is also typically true.

% Note that most of these examples contain negative coefficients. While
% positive coefficients indicate a positive correlation between spending money
% and improving performance in straw polls, negative numbers indicate a
% negative correlation between spending money and improving performance in
% straw polls. After consulting experts in the field, [WHICHONE] was chosen to
% be used in order to simplify the model. It should be duly noted, however,
% than a refinement of this fundraising return function could be an entire
% research endeavor on its own.

\subsection{Weighted Utility Vector}
The next step was creating some sort of payout vector, using techniques from
Weighted Blotto Models. First, spending data was gathered for the 2018 Senate
race, using the same methods used in creating the fundraising return
function. The vector still needed some sort of weighted payout per quarter; it
was determined that straw polls taken throughout the race were the only way to
calculate this payout. The main challenge was finding straw polls that were
both comprehensive and accurate. Two sites were considered for these polls:
Real Clear Politics \cite{realclearpolitics} (a political news and poll
conglomeration website), and FiveThirtyEight \cite{538} (a news company
focusing on statistical analysis). To check the biases of each polling site,
Media Bias Fact Check \cite{mediabias} was consulted to give comprehensive
bias-check reviews of them. The website gives the following ratings for each:
\begin{itemize}
\item \textbf{Real Clear Politics: Right-Center Bias} ``In review, the website
features selected political news stories and op-eds from various news
publications in addition to commentary from its own contributors. Though
their own political views lean conservative, the site’s founders say their
goal is to give readers “ideological diversity” in its commentary section. In
reviewing their political news and opinions there are slightly more that are
published from right leaning sources, however both sides are
represented. Real Clear Politics is perhaps best known for their RCP Polling
Average, which combines all polling data to create a statistical
average. Most of the news content on Real Clear Politics is aggregated from
other sources such as: Washington Post, New York Post, Salon, Fox News, The
Federalist and National Review. Several of the sources used by Real Clear
Politics are listed as Mixed factual due to failed fact checks. In reviewing
original Real Clear Politics articles, there is a right leaning bias in
wording and story selection such as this: Gov. Jerry ‘Moonbeam’ Brown’s
Warning to Fellow Democrats. Although this is an opinion piece, it completely
lacks sourcing of any kind. In general, the majority of stories from Real
Clear Politics comes from a right leaning perspective. Overall, we rate RCP
as Right-Center biased based on source selection that leans right and Mixed
for factual reporting due to use of multiple sources who have failed fact
checks.''
\item \textbf{FiveThirtyEight: Left-Center Bias} ``FiveThirtyEight, sometimes
referred to as 538, is a website that focuses on opinion poll analysis,
politics, economics, and sports blogging. The website, which takes its name
from the number of electors in the United States electoral college, was
founded on March 7, 2008, as a polling aggregation website with a blog
created by analyst Nate Silver. FiveThirtyEight typically relies on its
methodology and not opinion for its reports, however they do publish new
stories that have a slight left-center bias in coverage.''
\end{itemize}
Due to a higher number of polls reaching further back in the race, transferable
data organization, and Tim Wetzel's undying love and respect for Nate Silver,
FiveThirtyEight was chosen as the source for this polling data. The data was
next organized by state, and then by the first day of data collection. The
first day was chosen because any other date used would require manual entry to
assign a date to each poll. Polls were then categorized into quarters, using
the first day of data collection.

Next, each race was analyzed individually. For each quarter, poll numbers were
arranged in a 2-dimensional array, with the x-axis denoting which day in the
quarter the poll's data collection began, and the y-axis denoting the
differential between the two candidates' poll ratings. After the data was
organized in this way for each race and quarter, a simple regression was
calculated in Microsoft Excel, using the SLOPE function \cite{slope}. This
regression was designed to give a number approximating the number of percentage
points the polling differential changed between the two candidates, with an
upward trend denoting a change toward the ``positively-designated'' candidate
and a downward trend denoting a change toward the ``negatively-designated''
candidate.

It should be noted that these regression lines generally had many more data
points at the end of the election than at the beginning, due to the nature of
election news coverage. This likely created the largest errors in the first two
quarters of 2018. In addition, Q4 2018 consists almost exclusively of data from
October 1, 2018 - November 6, 2018, since November 6 is election day.

In order to determine the best utility function, a variety of options were
explored for the model:
\begin{enumerate}
\item Function calculated directly from polling numbers (Unweighted)
\item Functions where polls with higher sample sizes were weighted more heavily
(Weighted by Sample)
\item Functions where polls were weighted more heavily for having a higher
sample size-to-state population ratio (Weighted by State)
\item Functions where individual negative values that composed a race's utility
function were set to zero, and positive values were left unedited (Floored)
\end{enumerate}
This yielded six different utility functions, and in all likelihood, none of
them are even close to correct. However, they do offer interesting ideas of what a utility
function over the last year of a Senate election might look like. Each vector
has four values, each corresponding to a different fiscal quarter of an
election year ([Q1, Q2, Q3, Q4] respectively). The values themselves represent
how many points the polls would move in a player's favor if that player were to
spend \$1. For example, a value of 0.00000113 indicates that spending
\$1 creates a .0.00000113\% change in voters' favor of that player; a value of
-0.00000009 indicates a 0.00000009\% change in favor of that player's opponent. However, as previously mentioned, a constant $2 \times 10^{-5}$ is placed outside the vector, to make the numbers more readable. Without this constant, the numbers would indicate the effect in poll points a \$200,000 spend would create (see Section \textbf{3.1: Initial Assumptions}). The equations are as follows:\\

\begin{tabular}{|p{4.5cm}|p{2.2cm}|p{2.2cm}|p{2.2cm}|p{2.2cm}|}
\hline
\multicolumn{5}{|c|}{Utility Vectors Considered} \\
\hline
& & & &\\
\textbf{Name of Utility Vector} & \textbf{Q1} & \textbf{Q2} & \textbf{Q3} & \textbf{Q4}\\
& & & &\\
\hline
\tiny{Unweighted, Unfloored} & $6.30872096 \cdot 2\times 10^{-5}$ & $2.549945091 \cdot 2\times 10^{-5}$ & $0.401012477 \cdot 2\times 10^{-5}$ & $-1.653903518 \cdot 2\times 10^{-5}$\\
\hline
\tiny{Weighted by Sample, Unfloored} & $0.26559308 \cdot 2\times 10^{-5}$ & $0.242170061\cdot 2\times 10^{-5}$ & $0.124087002\cdot 2\times 10^{-5}$ & $-0.419994208\cdot 2\times 10^{-5}$\\
\hline
\tiny{Weighted by State, Unfloored} & $0.353833059\cdot 2\times 10^{-5}$ & $-0.460278059\cdot 2\times 10^{-5}$ & $0.204454969\cdot 2\times 10^{-5}$ & $-0.480740837\cdot 2\times 10^{-5}$\\
\hline
\tiny{Unweighted, Floored} & $0.611974882\cdot 2\times 10^{-5}$ & $0.214168992\cdot 2\times 10^{-5}$ & $0.234584564\cdot 2\times 10^{-5}$ & $2.111203038\cdot 2\times 10^{-5}$\\
\hline
\tiny{Weighted by Sample, Floored} & $0.167306442\cdot 2\times 10^{-5}$ & $0.089473901\cdot 2\times 10^{-5}$ & $0.11420262\cdot 2\times 10^{-5}$ & $2.841187593\cdot 2\times 10^{-5}$\\
\hline
\tiny{Weighted by State, Floored} & $0.243860933\cdot 2\times 10^{-5}$ & $0.282414783\cdot 2\times 10^{-5}$ & $0.14780057\cdot 2\times 10^{-5}$ & $3.216185277\cdot 2\times 10^{-5}$\\
\hline
\end{tabular}
 \\
Some omissions were made in this part of the model. Specifically, for any
fiscal quarter for a race, if there were not at least two polls taken at
different dates in that quarter, a regression line could not be
calculated. This eliminated many different quarters for many states, and it
eliminated Maryland, Nebraska, Washington, and Wyoming's races
altogether. Additionally, any poll made using SurveyMonkey was not considered,
due to poor collection methods and the overwhelming presence of this data in
several races.

\subsection{Generating Strategies}
With a fundraising function and payout vector in place, the next step was
generating strategies two generalized players P1 and P2, given two initial
budgets. This was more complicated than a simple iteration, since the
fundraising return function created extra strategies in Q2, Q3, and
Q4. Therefore, strategies had to be created iteratively, one quarter at a
time. For each candidate, Q1 strategies were simply a list of numbers from zero
to the maximum whole number that was less than or equal to the candidate's
current budget. These data points were listed in conjunction with the
candidate's budget after spending that amount.

Next, fundraising from the previous quarter had to be determined. Recall from
equation [CITE] that this function uses parameters (previous quarter's
fundraising return) and (previous quarter's spending). However, due to the
recursive nature of this function, initial budget $y_0$ was substituted into
the function in lieu of previous fundraising return. This yielded
\begin{equation*}
f_1(y_0, s_1) = 0.1124s_1 + 1.1057y_0
\end{equation*}
as the function for the first quarter's fundraising.

At this point, the strategy array contained strategies only for the first
quarter, with a remaining budget $y_i$ for each candidate in quarter $i$
calculated by
\begin{equation*}
y_i = y_{i-1} - s_i + f(f_{i-1}, s_i)
\end{equation*}
The process was then repeated, this time iterating through each number from 0
to the highest integer less than or equal to the strategy's remaining
budget. This was repeated once more for the third quarter, then each remaining
budget was spent in the fourth quarter. P1's strategy vector will be denoted
$s$, while t's strategy vector will be denoted $t$.

After these strategies were generated for both players, dominated strategies
were eliminated in order to make solving the payout matrix more efficient. To
do this, a function was written to compare elements of each strategy vector to
each other (recall that elements of these strategy arrays are vectors of length
4). If every element of one vector was greater than or equal to every element
of another vector, the latter vector is considered a \textbf{dominated
strategy} and was eliminated from its strategy vector.
\subsection{Solving the Payout Matrix}
With all dominated strategies eliminated, a payout matrix $M$ could be
constructed. Like in \textbf{Section 2.2}, positive values were assigned to P1,
and negative values were assigned to P2. A matrix was constructed with rows
corresponding to each element of $s$ and columns corresponding to each element
of $t$. Each element of $M$ was calcuated using the utility function
constructed in \textbf{Section 6.2}.

It would make the most sense, from a mathematical point of view, to solve this
matrix constructed from every possible strategy of each player. However, in
practice, this was nearly impossible to do. As initial budgets increased, the
size of the matrix increased exponentially, and the code would not run in an
even remotely reasonable amount of time. To address this, a Monte-Carlo
approach to solving the matrix was required.

First, each player's strategies were generated the same as before. Then, 100
random subsets were generated from $s$ and $t$. 100,000 matrices were created
in this way, pitting pairs of strategies against each other. This generated
optimal strategies for 100,000 games between subsets of $s$ and $t$. Each
strategy belonging to an optimized strategy vector was added to a ``winners list,'' then each instance of the
strategy was counted. This yielded a vector consisting of tuples, where each tuple contained indices of every strategy that was used in an optimized strategy vector, and the number of times it was used. Last, the latter of these numbers was divided by 100,000 to give the proportion of games that strategy should be used. 
% ------------------------------Results----------------------------------
\newpage
\section{Results}
The first step in getting actual results was determining which initial budgets
to assign to players. There are infinitely many possible initial budgets to
assign to each player, creating infinite possible games. Therefore, reasonable
budgets had to be picked. To do this, real-world examples of ``initial
budgets'' calculated for January of election years 2016 and 2018 were collected
as reference (see \textbf{Figure \ref{examplebudgets}}. These ``initial
budgets'' were calculated by summing each candidate's amount fundraised in Q4
of the year before the election (Q4 2015 and Q4 2017, respectively) and
subtracting the amount that candidate spent that same quarter. However, many of
these races had huge discrepancies in budgets, which would lead to very
uninteresting races if plugged into the model. Therefore, the example budgets
in \textbf{Figure \ref{initialbudgets}} were instead used. 
\begin{figure}
\tiny{
\begin{tabular}{ |p{.3cm} p{.4cm}|p{2.7cm} p{.4cm} p{1.6cm}|p{2.7cm} p{.4cm} p{1.6cm}|p{1.3cm}|}
\hline
\multicolumn{9}{|c|}{Example Initial Budgets} \\
\hline
Year & State & P1 Name & Party & Initial Budget & P2 Name & Party & Initial Budget & Budget Differential\\
\hline
2018 & CA & Dianne Feinstein & D & \$7,540,299 & Kevin DeLeon & D & \$1,189,460 & \$6,350,838\\
2018 & PA & Bob Casey & D & \$5,857,855 & Lou Barletta & R & \$83,772 & \$5,774,083\\
2018 & MO & Claire McCaskill & D & \$5,491,965 & Josh Hawley & R & \$1,156,672 & \$4,335,292\\
2018 & MT & Jon Tester & D & \$2,601,442 & Matthew Rosendale & R & \$315,399 & \$2,286,043\\
2018 & IN & Mike Braun & R & \$2,845,151 & Joe Donnelly & D & \$1,214,857 & \$1,630,293\\
2018 & VA & Tim Kaine & D & \$1,652,252 & Corey Stewart & R & \$39,334 & \$1,612,917\\
2018 & AZ & Kyrsten Sinema & D & \$2,139,420 &Martha McSally & R & \$630,040 & \$1,509,380\\
2018 & WI & Tammy Baldwin & D & \$1,712,842 & Leah Vukmir & R & \$210,657 & \$1,502,184\\
2018 & NV & Jacky Rosen & D & \$1,621,476 & Dean Heller & R & \$317,237 & \$1,304,238\\
2018 & MI & Debbie Stabenow & D & \$1,637,430 & John James & R & \$405,181 & \$1,232,249\\
2018 & ND & Heidi Heitkamp & D & \$1,375,695 & Kevin Cramer & R & \$213,441 & \$1,162,254\\
2018 & RI & Sheldon Whitehouse & D & \$781,534 & Robert Flanders & R & \$84,280 & \$697,253\\
2018 & TN & Marsha Blackburn & R & \$1,485,492 & Phil Bredesen & D & \$808,984 & \$676,507\\
2018 & ME & Angus King & I & \$614,868 & Eric Brakey & R & \$47,671 & \$567,197\\
2018 & NM & Martin Heinrich & D & \$572,439 & Mick Rich & R & \$19,943 & \$552,496\\
2018 & WV & Joe Manchin & D & \$814,799 & Patrick Morrisey & R & \$641,753 & \$173,045\\
2018 & WY & John Barrasso & R & \$424,225 & Gary Trauner & D & \$254,031 & \$170,194\\
2018 & CT & Chris Murphy & D & \$18,350 & Matthey Corey & R & \$3,001 & \$15,349\\
2018 & NE & Debbie Fischer & R & \$306,930 & Jane Raybould & D & \$302,889 & \$4,041\\
\hline
2016 & WI & Russ Feingold & D & \$3,433,425 & Ron Johnson & R & \$888,443 & \$2,544,981\\
2016 & CA & Loretta Sanchez & D & \$2,060,321 & Kamala Harris & D & \$66,0586 & \$1,399,734\\
2016 & OH & Rob Portman & R & \$1,434,749 & Ted Strickland & D & \$879,287 & \$555,462\\
2016 & NH & Margaret Hassan & D & \$2,086,756 & Kelly Ayotte & R & \$1,536,590 & \$550,165\\
2016 & MO & Roy Blunt & R & \$1,146,922 & Jason Kander & D & \$837,161 & \$309,761\\
2016 & PA & Pat Toomey & R & \$1,359,343 & Katie McGinty & D & \$1,079,492 & \$279,851\\
2016 & NV & Catherine Cortez Masto & D & \$765,080 & Joe Heck & R & \$633,667 & \$131,413\\
2016 & NC & Richard Burr & R & \$639,766 & Deborah Ross & D & \$583,201 & \$56,564\\
2016 & AZ & John McCain & R & \$368,348 & Ann Kirkpatrick & D & \$332,292 & \$36,055\\
\hline
\end{tabular}
}
\caption{Real-world examples of starting budgets of top U.S. Senate candidates. Initial budgets calculated by subtracting money spent in Q4 the year before the race from money fundraised in Q4 the year before the race. Initial budgets less than or equal to zero omitted.}
\label{examplebudgets}
\end{figure}
\begin{figure}
\centering
\begin{tikzpicture}
\begin{axis}[
width=15.0cm,
height = 7cm,
title={Example Initial Budgets},
xlabel={\tiny{Higher-Budget Candidate (Mill. \$)}},
ylabel={\tiny{Lower-Budget Candidate (Mill. \$)}},
xmin=0, xmax=8,
ymin=0, ymax=3.0,
xtick={0.0,2.0,4.0,6.0,8.0},
ytick={0.0,0.5,1.0,1.5,2.0,2.5,3.0},
legend pos=north east,
ymajorgrids=true,
grid style=dashed,
]
\addplot[
color=blue,
mark=*,
draw=none
]
coordinates {
(7.540299,1.189460)(5.857855,.083772)(5.491965,1.156672)(2.601442,.315399)(2.845151,1.214857)(1.652252,.039334)(2.139420,.630040)(1.712842,.210657)(1.621476,.317237)(1.637430,.405181)(1.375695,.213441)(.781534,.084280)(1.485492,.808984)(.614868,.047671)(.572439,.019943)(.814799,.641753)(.424225,.254031)(.018350,.003001)(.306930,.302889)
};
]
\addplot[
color=red,
mark=*,
draw=none
]
coordinates {(3.433425,.888443)(2.060321,.660586)(1.434749,.879287)(2.086756,1.536590)(1.146922,.837161)(1.359343,1.079492)(.765080,.633667)(.639766,.583201)(.368888,.332292)
};
\addplot[
color=black,
mark=o,
draw=none
]
coordinates {(.8,.8)(.9,.8)(1,.8)(1.1,.8)(1.2,.8)(1.3,.8)(1.4,.8)(1.6,.8)};
\addplot [gr29, line width = 1, dotted, domain=0:3] {x};
\legend{2018 Races, 2016 Races, Test Races, Equal Budget Races}
\end{axis}
\end{tikzpicture}
\caption{Discrepancies between initial budgets of 2016 and 2016 top Senate candidates listed in \textbf{Figure \ref{examplebudgets}}. Blue dots indicate races in 2018, and red dots indicate races in 2016. Note that, in most races, one candidate has a significantly higher initial budget than the other. No points occur over the Equal Budget Races line (though some occur exactly on the line), since the higher-budget candidate was always plotted on the $x$-axis.}
\label{budgetgraph}
\end{figure}
For testing, \$800,000 was picked as the budget for each lower-budget player
(P1). This number was picked to ensure that each lower-budget player had at
least 4 different strategy options in Q1 of the race, since \$200,000 was
picked as the increment in which players could spend. To test qualitatively
different races, the budget of the higher-budget player was determined by
multiplying the budget of P1 by 1.00, 1.125, 1.25, 1.375, 1.5, 1.675, 1.75, and
2.00 (see \textbf{Figure \ref{initialbudgets}}). This covered many theoretical
races, ranging from equally-matched races to those in which one player had
double the budget of the other.

\begin{figure}
\tiny
\begin{tabular}{ |p{2.0cm}|p{2.0cm} p{2.0cm}|p{3.0cm}|p{2.0cm} p{2.0cm}|}
\hline
\multicolumn{6}{|c|}{Tested Initial Budgets} \\
\hline
&&&&&\\
Example & P1 Initial Budget & P2 Initial Budget & Initial Budget Differential & $\frac{P1\;Initial\;Budget}{P2\;Initial\;Budget}$ & $\frac{P2\;Initial\;Budget}{P1\;Initial\;Budget}$\\
&&&&&\\
\hline
Budget Test 1 & \$800,000 & \$800,000 & \$0 & 1.0000 & 1.000\\
\hline
Budget Test 2 & \$800,000 & \$900,000 & \$100,000 & 0.8889 & 1.1250\\
\hline
Budget Test 3 & \$800,000 & \$1,000,000 & \$200,000 & 0.8000 & 1.2500\\
\hline
Budget Test 4 & \$800,000 & \$1,100,000 & \$300,000 & 0.7273 & 1.3750\\
\hline
Budget Test 5 & \$800,000 & \$1,200,000 & \$400,000 & 0.6667 & 1.5000\\
\hline
Budget Test 6 & \$800,000 & \$1,300,000 & \$500,000 & 0.6154 & 1.6750\\
\hline
Budget Test 7 & \$800,000 & \$1,400,000 & \$600,000 & 0.5714 & 1.7500\\
\hline
Budget Test 8 & \$800,000 & \$1,600,000 & \$800,000 & 0.5000 & 2.0000\\
\hline
\end{tabular}
\caption{All sets of theoretical budgets, tested against each other in pairs.}
\label{initialbudgets}
\end{figure}
Next, this set of budgets were tested on the utility vector calculated in Section 4.2: \textbf{Weighted Utility Vector}.
\subsection{Unweighted, Floored Utility Vector}
Recall that the Unweighted, Floored Utility Vector was calculated by setting
negative regression slopes in individual races to zero, then taking a simple
average of these slopes for each fiscal quarter. This yielded the vector
\begin{equation}
[ 0.611974882, 0.214168992, 0.234584564, 2.111203038 ]\cdot 2\times 10^{-5}.
\end{equation}
 Using this utility vector yielded the
following results. In every case, the top strategy for each player was to save
all their money until the last quarter, then spend it all. This makes sense,
since the last quarter was weighted so heavily in the utility vector.\\

In fact, as lesser-used strategies are examined, they reveal a common trend: spending as little as possible before Q4, then dumping the rest of the budget. The Q4 strategy column is arranged in almost exact descending order, denoting the importance of this quarter; the few exceptions to this rule are strategies which favor Q1, which has the next-highest voter return. It's likely this trend continues deep into the least-used strategies; however, in order to reduce runtime, strategies used less than 1\% of the time were not calculated.

Another interesting trend in this model is the tendency for higher-budget
candidates to have a wider range of strategies, and the tendency of
lower-budget candidates to have more concentrated strategies. This is likely
due to the way in which the matrices were solved. Because Monte-Carlo only
plays random subsets of strategies against each other, small and large wins are
counted equally. If the strongest strategy a player may employ is not present
in this subset, the algorithm solves the matrix with only the strategies it
has. While this may result in a lower-valued game, it still counts as a win for
the game winner. As the budget ratio between players increases, the
higher-budget player has many more strategies that beat their opponent. This
takes what may be a concentrated strategy and dilutes it (see \textbf{Figure
\ref{budgetratios.1}}).

Lastly, note that P2 wins in every case. While the game value in \textbf{Figure \ref{8v8table.1}} should be zero, since P1 and P2 have identical starting budgets, the errors caused in the Play \% column create a small error in the game value, resulting in a win for P2 in every case. This makes sense because players are never punished for spending more when using this vector, so a higher-budget player will always have a clear advantage. 

\begin{figure}
\tiny
\begin{tabular}{ |p{1.0cm}p{1.0cm}p{1.0cm}p{2.0cm}|p{1.0cm}||p{1.0cm}p{1.0cm}p{1.0cm}p{2.0cm}|p{1.0cm}|}
\hline
\multicolumn{10}{|c|}{P1 \$800,000 vs. P2 \$800,000} \\
\hline
Player 1 & (P1) & & & & Player 2 & (P2) & & & \\
\hline
Q1 Strategy & Q2 Strategy & Q3 Strategy & Q4 Strategy  &  Play \% & Q1 Strategy & Q2 Strategy & Q3 Strategy & Q4 Strategy  &  Play \%\\
\hline
\$0 & \$0 & \$0 & \$3,640,676 & 29.245\%            & \$0 & \$0 & \$0 & \$3,640,676 & 29.620\%\\
\$200,000 & \$0 & \$0 & \$3,512,868 & 20.670\%      & \$200,000 & \$0 & \$0 & \$3,512,868 & 20.822\%\\
\$0 & \$200,000 & \$0 & \$3,485,636 & 14.810\%      & \$0 & \$200,000 & \$0 & \$3,485,636 & 14.721\%\\
\$400,000 & \$0 & \$0 & \$3,385,061 & 10.478\%      & \$400,000 & \$0 & \$0 & \$3,385,061 & 10.365\%\\
\$0 & \$0 & \$200,000 & \$3,440,676 & 7.340\%       & \$0 & \$0 & \$200,000 & \$3,440,676 & 7.346\%\\
\$200,000 & \$200,000 & \$0 & \$3,357,828 & 5.210\% & \$200,000 & \$200,000 & \$0 & \$3,357,828 & 5.125\%\\
\$600,000 & \$0 & \$0 & \$3,257,253 & 3.651\%       & \$600,000 & \$0 & \$0 & \$3,257,253 & 3.588\%\\
\$200,000 & \$0 & \$200,000 & \$3,312,868 & 2.571\% & \$200,000 & \$0 & \$200,000 & \$3,312,868 & 2.556\%\\
\$0 & \$400,000 & \$0 & \$3,330,596 & 1.854\%       & \$0 & \$400,000 & \$0 & \$3,330,596 & 1.748\%\\
\$400,000 & \$200,000 & \$0 & \$3,230,021 & 1.304\% & \$400,000 & \$200,000 & \$0 & \$3,230,021 & 1.243\%\\
\hline
Other & & & & 2.867\%                               & Other & & & & 2.866\% \\
\hline
\small \textbf{Game Value:} &&& \small \textbf{-0.01} && \small \textbf{Winner:} &&& \small \textbf{P2}&\\
\hline
\end{tabular}
\caption{Mixed strategy output, when P1 and P2 are both given initial budgets of \$800,000 and the Unweighted, Floored Utility Vector is used.}
\label{8v8table.1}
\end{figure}

\begin{figure}
\centering
\captionsetup{justification=raggedright}
\subfloat[P1: \$800,000 Initial Budget]{
\begin{tikzpicture}
\begin{axis}[
width=7.0cm,
ylabel={\tiny{Amount to Spend (\$1,000s)}},
xlabel={\tiny{Quarter}},
xmin=1, xmax=4,
ymin=-50, ymax=4000,
xtick={1,2,3,4},
ytick={0,1000,2000,3000,4000},
legend pos=north west,
ymajorgrids=true,
grid style=dashed,
]
\addplot[draw=gr21]
coordinates {(1,200)(2,0)(3,0)(4,3512)};
\addplot[draw=gr15]
coordinates {(1,0)(2,200)(3,0)(4,3485)};
\addplot[draw=gr10]
coordinates {(1,400)(2,200)(3,0)(4,3385)};
\addplot[draw=gr7]
coordinates {(1,0)(2,0)(3,200)(4,3440)};
\addplot[draw=gr5]
coordinates {(1,200)(2,200)(3,0)(4,3357)};
\addplot[draw=gr4]
coordinates {(1,600)(2,0)(3,0)(4,3257)};
\addplot[draw=gr3]
coordinates {(1,200)(2,0)(3,200)(4,3312)};
\addplot[draw=gr2]
coordinates {(1,0)(2,400)(3,200)(4,3330)};
\addplot[draw=gr1]
coordinates {(1,400)(2,200)(3,0)(4,3230)};
\addplot[draw=red]
coordinates {(1,0)(2,0)(3,0)(4,3641)};
\end{axis}
\end{tikzpicture}
}
\qquad
\subfloat[P2: \$800,000 Initial Budget]{
\begin{tikzpicture}
\begin{axis}[
width=7.0cm,
ylabel={\tiny{Amount to Spend (\$1,000s)}},
xlabel={\tiny{Quarter}},
xmin=1, xmax=4,
ymin=-50, ymax=4000,
xtick={1,2,3,4},
ytick={0,1000,2000,3000,4000},
legend pos=north west,
ymajorgrids=true,
grid style=dashed,
]
\addplot[draw=gr21]
coordinates {(1,200)(2,0)(3,0)(4,3512)};
\addplot[draw=gr15]
coordinates {(1,0)(2,200)(3,0)(4,3485)};
\addplot[draw=gr10]
coordinates {(1,400)(2,200)(3,0)(4,3385)};
\addplot[draw=gr7]
coordinates {(1,0)(2,0)(3,200)(4,3440)};
\addplot[draw=gr5]
coordinates {(1,200)(2,200)(3,0)(4,3357)};
\addplot[draw=gr4]
coordinates {(1,600)(2,0)(3,0)(4,3257)};
\addplot[draw=gr3]
coordinates {(1,200)(2,0)(3,200)(4,3312)};
\addplot[draw=gr2]
coordinates {(1,0)(2,400)(3,200)(4,3330)};
\addplot[draw=gr1]
coordinates {(1,400)(2,200)(3,0)(4,3230)};
\addplot[draw=blue]
coordinates {(1,0)(2,0)(3,0)(4,3641)};
\end{axis}
\end{tikzpicture}
}
\\
\caption{Overlaid strategies for P1 and P2. Colored line indicates top strategy. Darker gray lines indicate heavily used strategies, while lighter gray lines indicated lightly used strategies.}
\label{8v8.1}
\end{figure}

\begin{figure}
\tiny
\begin{tabular}{ |p{1.0cm}p{1.0cm}p{1.0cm}p{2.0cm}|p{1.0cm}||p{1.0cm}p{1.0cm}p{1.0cm}p{2.0cm}|p{1.0cm}|}
\hline
\multicolumn{10}{|c|}{P1 \$800,000 vs. P2 \$900,000} \\
\hline
Player 1 & (P1) & & & & Player 2 & (P2) & & & \\
\hline
Q1 Strategy & Q2 Strategy & Q3 Strategy & Q4 Strategy  &  Play \% & Q1 Strategy & Q2 Strategy & Q3 Strategy & Q4 Strategy  &  Play \%\\
\hline
\$0 & \$0 & \$0 & \$3,640,676 & 29.451\%            & \$0 & \$0 & \$0 & \$4,095,760 & 22.894\%\\
\$200,000 & \$0 & \$0 & \$3,512,868 & 20.936\%      & \$200,000 & \$0 & \$0 & \$3,967,953 & 17.618\%\\
\$0 & \$200,000 & \$0 & \$3,485,636 & 14.660\%      & \$0 & \$200,000 & \$0 & \$3,940,720 & 13.742\%\\
\$400,000 & \$0 & \$0 & \$3,385,061 & 10.457\%      & \$400,000 & \$0 & \$0 & \$3,840,145 & 10.448\%\\
\$0 & \$0 & \$200,000 & \$3,440,676 & 7.413\%       & \$0 & \$0 & \$200,000 & \$3,895,760 & 8.264\%\\
\$200,000 & \$200,000 & \$0 & \$3,357,828 & 5.133\% & \$200,000 & \$200,000 & \$0 & \$3,812,913 & 6.258\%\\
\$600,000 & \$0 & \$0 & \$3,257,253 & 3.549\%       & \$600,000 & \$0 & \$0 & \$3,712,337 & 4.717\%\\
\$200,000 & \$0 & \$200,000 & \$3,312,868 & 2.570\% & \$200,000 & \$0 & \$200,000 & \$3,767,953 & 3.700\%\\
\$0 & \$400,000 & \$0 & \$3,330,596 & 1.765\%       & \$0 & \$400,000 & \$0 & \$3,785,680 & 2.867\%\\
\$400,000 & \$200,000 & \$0 & \$3,230,021 & 1.264\% & \$400,000 & \$200,000 & \$0 & \$3,685,105 & 2.222\%\\
& & & &                                            & \$800,000 & \$0 & \$0 & \$3,584,530 & 1.717\%\\
& & & &                                            & \$0 & \$200,000 & \$200,000 & \$3,740,720 & 1.288\%\\
& & & &                                            & \$400,000 & \$0 & \$200,000 & \$3,640,145 & 1.035\%\\
\hline
Other & & & & 2.802\%                               & Other & & & & 3.230\% \\
\hline
\small \textbf{Game Value:} &&& \small \textbf{-4.55} && \small \textbf{Winner:} &&& \small \textbf{P2}&\\
\hline
\end{tabular}
\caption{Mixed strategy output, when P1 is given an initial budget of \$800,000, P2 is given an initial budget of \$900,000, and the Unweighted, Floored Utility Vector is used.}
\label{8v9table.1}
\end{figure}

\begin{figure}
\centering
\captionsetup{justification=raggedright}
\subfloat[P1: \$800,000 Initial Budget]{
\begin{tikzpicture}
\begin{axis}[
width=7.0cm,
ylabel={\tiny{Amount to Spend (\$1,000s)}},
xlabel={\tiny{Quarter}},
xmin=1, xmax=4,
ymin=-50, ymax=5000,
xtick={1,2,3,4},
ytick={0,1000,2000,3000,4000,5000},
legend pos=north west,
ymajorgrids=true,
grid style=dashed,
]
\addplot[draw=gr21]
coordinates {(1,200)(2,0)(3,0)(4,3512)};
\addplot[draw=gr15]
coordinates {(1,0)(2,200)(3,0)(4,3485)};
\addplot[draw=gr10]
coordinates {(1,400)(2,200)(3,0)(4,3385)};
\addplot[draw=gr7]
coordinates {(1,0)(2,0)(3,200)(4,3440)};
\addplot[draw=gr5]
coordinates {(1,200)(2,200)(3,0)(4,3357)};
\addplot[draw=gr4]
coordinates {(1,600)(2,0)(3,0)(4,3257)};
\addplot[draw=gr3]
coordinates {(1,200)(2,0)(3,200)(4,3312)};
\addplot[draw=gr2]
coordinates {(1,0)(2,400)(3,200)(4,3330)};
\addplot[draw=gr1]
coordinates {(1,400)(2,200)(3,0)(4,3230)};
\addplot[draw=red]
coordinates {(1,0)(2,0)(3,0)(4,3641)};
\end{axis}
\end{tikzpicture}
}
\qquad
\subfloat[P2: \$900,000 Initial Budget]{
\begin{tikzpicture}
\begin{axis}[
width=7.0cm,
ylabel={\tiny{Amount to Spend (\$1,000s)}},
xlabel={\tiny{Quarter}},
xmin=1, xmax=4,
ymin=-50, ymax=5000,
xtick={1,2,3,4},
ytick={0,1000,2000,3000,4000,5000},
legend pos=north west,
ymajorgrids=true,
grid style=dashed,
]
\addplot[draw=gr18]
coordinates {(1,200)(2,0)(3,0)(4,3967)};
\addplot[draw=gr14]
coordinates {(1,0)(2,200)(3,0)(4,3940)};
\addplot[draw=gr10]
coordinates {(1,400)(2,0)(3,0)(4,3840)};
\addplot[draw=gr8]
coordinates {(1,0)(2,0)(3,200)(4,3895)};
\addplot[draw=gr6]
coordinates {(1,200)(2,200)(3,0)(4,3812)};
\addplot[draw=gr5]
coordinates {(1,600)(2,0)(3,0)(4,3712)};
\addplot[draw=gr4]
coordinates {(1,200)(2,0)(3,200)(4,3767)};
\addplot[draw=gr3]
coordinates {(1,0)(2,400)(3,0)(4,3785)};
\addplot[draw=gr2]
coordinates {(1,400)(2,200)(3,0)(4,3685)};
\addplot[draw=gr2]
coordinates {(1,800)(2,0)(3,0)(4,3584)};
\addplot[draw=gr1]
coordinates {(1,0)(2,200)(3,200)(4,3740)};
\addplot[draw=gr1]
coordinates {(1,400)(2,0)(3,200)(4,3640)};
\addplot[draw=blue]
coordinates {(1,0)(2,0)(3,0)(4,4095)};
\end{axis}
\end{tikzpicture}
}
\\
\caption{Overlaid strategies for P1 and P2. Colored line indicates top strategy. Darker gray lines indicate heavily used strategies, while lighter gray lines indicated lightly used strategies.}
\label{8v9.1}
\end{figure}

\begin{figure}
\tiny
\begin{tabular}{ |p{1.0cm}p{1.0cm}p{1.0cm}p{2.0cm}|p{1.0cm}||p{1.0cm}p{1.0cm}p{1.0cm}p{2.0cm}|p{1.0cm}|}
\hline
\multicolumn{10}{|c|}{P1 \$800,000 vs. P2 \$1,000,000} \\
\hline
Player 1 & (P1) & & & & Player 2 & (P2) & & & \\
\hline
Q1 Strategy & Q2 Strategy & Q3 Strategy & Q4 Strategy  &  Play \% & Q1 Strategy & Q2 Strategy & Q3 Strategy & Q4 Strategy  &  Play \%\\
\hline
\$0 & \$0 & \$0 & \$3,640,676 & 29.451\%            & \$0 & \$0 & \$0 & \$4,550,845 & 16.364\% \\
\$200,000 & \$0 & \$0 & \$3,512,868 & 20.936\%      & \$200,000 & \$0 & \$0 & \$4,423,037 & 13.499\% \\
\$0 & \$200,000 & \$0 & \$3,485,636 & 14.660\%      & \$0 & \$200,000 & \$0 & \$4,395,805 & 11.347\%\\
\$400,000 & \$0 & \$0 & \$3,385,061 & 10.457\%      & \$400,000 & \$0 & \$0 & \$4,295,230 & 9.589\% \\
\$0 & \$0 & \$200,000 & \$3,440,676 & 7.413\%       & \$0 & \$0 & \$200,000 & \$4,350,845 & 8.102\% \\
\$200,000 & \$200,000 & \$0 & \$3,357,828 & 5.133\% & \$200,000 & \$200,000 & \$0 & \$4,267,997 & 6.565\% \\
\$600,000 & \$0 & \$0 & \$3,257,253 & 3.549\%       & \$600,000 & \$0 & \$0 & \$4,167,422 & 5.659\% \\
\$200,000 & \$0 & \$200,000 & \$3,312,868 & 2.570\% & \$200,000 & \$0 & \$200,000 & \$4,223,037 & 4.658\% \\
\$0 & \$400,000 & \$0 & \$3,330,596 & 1.765\%       & \$0 & \$400,000 & \$0 & \$4,240,765 & 3.994\% \\
\$400,000 & \$200,000 & \$0 & \$3,230,021 & 1.264\% & \$400,000 & \$200,000 & \$0 & \$4,140,190 & 3.324\% \\
& & & &                                            & \$800,000 & \$0 & \$0 & \$4,039,614 & 2.674\% \\
& & & &                                            & \$0 & \$200,000 & \$200,000 & \$4,195,805 & 2.371\% \\
& & & &                                            & \$400,000 & \$0 & \$200,000 & \$4,095,230 & 1.906\% \\
& & & &                                            & \$200,000 & \$400,000 & \$0 & \$4,112,957 & 1.579\% \\
& & & &                                            & \$600,000 & \$200,000 & \$0 & \$4,012,382 & 1.406\% \\
& & & &                                            & \$1,000,000 & \$0 & \$0 & \$3,911,806 & 1.117\% \\
\hline
Other & & & & 3.002\%                               & Other & & & & 5.846\% \\
\hline
\small \textbf{Game Value:} &&& \small \textbf{-9.02} && \small \textbf{Winner:} &&& \small \textbf{P2}&\\
\hline
\end{tabular}
\caption{Mixed strategy output, when P1 is given an initial budget of \$800,000, P2 is given an initial budget of \$1,00,000, and the Unweighted, Floored Utility Vector is used.}
\label{8v10table.1}
\end{figure}

\begin{figure}
\centering
\captionsetup{justification=raggedright}
\subfloat[P1: \$800,000 Initial Budget]{
\begin{tikzpicture}
\begin{axis}[
width=7.0cm,
ylabel={\tiny{Amount to Spend (\$1,000s)}},
xlabel={\tiny{Quarter}},
xmin=1, xmax=4,
ymin=-50, ymax=5000,
xtick={1,2,3,4},
ytick={0,1000,2000,3000,4000,5000},
legend pos=north west,
ymajorgrids=true,
grid style=dashed,
]
\addplot[draw=gr21]
coordinates {(1,200)(2,0)(3,0)(4,3512)};
\addplot[draw=gr15]
coordinates {(1,0)(2,200)(3,0)(4,3485)};
\addplot[draw=gr10]
coordinates {(1,400)(2,200)(3,0)(4,3385)};
\addplot[draw=gr7]
coordinates {(1,0)(2,0)(3,200)(4,3440)};
\addplot[draw=gr5]
coordinates {(1,200)(2,200)(3,0)(4,3357)};
\addplot[draw=gr4]
coordinates {(1,600)(2,0)(3,0)(4,3257)};
\addplot[draw=gr3]
coordinates {(1,200)(2,0)(3,200)(4,3312)};
\addplot[draw=gr2]
coordinates {(1,0)(2,400)(3,200)(4,3330)};
\addplot[draw=gr1]
coordinates {(1,400)(2,200)(3,0)(4,3230)};
\addplot[draw=red]
coordinates {(1,0)(2,0)(3,0)(4,3641)};
\end{axis}
\end{tikzpicture}
}
\qquad
\subfloat[P2: \$1,000,000 Initial Budget]{
\begin{tikzpicture}
\begin{axis}[
width=7.0cm,
ylabel={\tiny{Amount to Spend (\$1,000s)}},
xlabel={\tiny{Quarter}},
xmin=1, xmax=4,
ymin=-50, ymax=5000,
xtick={1,2,3,4},
ytick={0,1000,2000,3000,4000,5000},
legend pos=north west,
ymajorgrids=true,
grid style=dashed,
]
\addplot[draw=gr13]
coordinates {(1,200)(2,0)(3,0)(4,4423)};
\addplot[draw=gr11]
coordinates {(1,0)(2,200)(3,0)(4,4395)};
\addplot[draw=gr10]
coordinates {(1,400)(2,0)(3,0)(4,4295)};
\addplot[draw=gr8]
coordinates {(1,0)(2,0)(3,200)(4,4350)};
\addplot[draw=gr7]
coordinates {(1,200)(2,200)(3,0)(4,4267)};
\addplot[draw=gr6]
coordinates {(1,600)(2,0)(3,0)(4,4167)};
\addplot[draw=gr5]
coordinates {(1,200)(2,0)(3,200)(4,4223)};
\addplot[draw=gr4]
coordinates {(1,0)(2,400)(3,0)(4,4240)};
\addplot[draw=gr3]
coordinates {(1,400)(2,200)(3,0)(4,4140)};
\addplot[draw=gr3]
coordinates {(1,800)(2,0)(3,0)(4,4039)};
\addplot[draw=gr2]
coordinates {(1,0)(2,200)(3,200)(4,4195)};
\addplot[draw=gr2]
coordinates {(1,400)(2,0)(3,200)(4,4095)};
\addplot[draw=gr2]
coordinates {(1,200)(2,400)(3,0)(4,4112)};
\addplot[draw=gr1]
coordinates {(1,600)(2,200)(3,0)(4,4012)};
\addplot[draw=gr1]
coordinates {(1,1000)(2,0)(3,0)(4,3911)};
\addplot[draw=blue]
coordinates {(1,0)(2,0)(3,0)(4,4550)};
\end{axis}
\end{tikzpicture}
}
\\
\caption{Overlaid strategies for P1 and P2. Colored line indicates top strategy. Darker gray lines indicate heavily used strategies, while lighter gray lines indicated lightly used strategies.}
\label{8v10.1}
\end{figure}

\begin{figure}
\tiny
\begin{tabular}{ |p{1.0cm}p{1.0cm}p{1.0cm}p{2.0cm}|p{1.0cm}||p{1.0cm}p{1.0cm}p{1.0cm}p{2.0cm}|p{1.0cm}|}
\hline
\multicolumn{10}{|c|}{P1 \$800,000 vs. P2 \$1,100,000} \\
\hline
Player 1 & (P1) & & & & Player 2 & (P2) & & & \\
\hline
Q1 Strategy & Q2 Strategy & Q3 Strategy & Q4 Strategy  &  Play \% & Q1 Strategy & Q2 Strategy & Q3 Strategy & Q4 Strategy  &  Play \%\\
\hline
\$0 & \$0 & \$0 & \$3,640,676 & 29.349\%                    & \$0 & \$0 & \$0 & \$5,005,929 & 12.834\% \\
\$200,000 & \$0 & \$0 & \$3,512,868 & 20.920\%              & \$200,000 & \$0 & \$0 & \$4,878,122 & 11.530\% \\
\$0 & \$200,000 & \$0 & \$3,485,636 & 14.729\%              & \$0 & \$200,000 & \$0 & \$4,850,889 & 9.950\% \\
\$400,000 & \$0 & \$0 & \$3,385,061 & 10.269\%              & \$400,000 & \$0 & \$0 & \$4,750,314 & 8.451\% \\
\$0 & \$0 & \$200,000 & \$3,440,676 & 7.297\%               & \$0 & \$0 & \$200,000 & \$4,805,929 & 7.421\% \\
\$200,000 & \$200,000 & \$0 & \$3,357,828 & 5.368\%         & \$200,000 & \$200,000 & \$0 & \$4,723,082 & 6.453\% \\
\$600,000 & \$0 & \$0 & \$3,257,253 & 3.580\%               & \$600,000 & \$0 & \$0 & \$4,622,506 & 5.599\% \\
\$200,000 & \$0 & \$200,000 & \$3,312,868 & 2.582\%         & \$200,000 & \$0 & \$200,000 & \$4,678,122 & 4.941\% \\
\$0 & \$400,000 & \$0 & \$3,330,596 & 1.822\%               & \$0 & \$400,000 & \$0 & \$4,695,849 & 4.211\% \\
\$400,000 & \$200,000 & \$0 & \$3,230,021 & 1.247\%         & \$400,000 & \$200,000 & \$0 & \$4,595,274 & 3.711\% \\
& & & &                                                     & \$800,000 & \$0 & \$0 & \$4,494,699 & 3.293\% \\
& & & &                                                     & \$0 & \$200,000 & \$200,000 & \$4,650,889 & 2.863\% \\
& & & &                                                     & \$400,000 & \$0 & \$200,000 & \$4,550,314 & 2.424\% \\
& & & &                                                     & \$200,000 & \$400,000 & \$0 & \$4,568,042 & 2.139\% \\
& & & &                                                     & \$600,000 & \$200,000 & \$0 & \$4,467,466 & 1.879\% \\
& & & &                                                     & \$1,000,000 & \$0 & \$0 & \$4,366,891 & 1.580\% \\
& & & &                                                     & \$0 & \$0 & \$400,000 & \$4,605,929 & 1.400\% \\
& & & &                                                     & \$200,000 & \$200,000 & \$200,000 & \$4,523,082 & 1.185\% \\
& & & &                                                     & \$600,000 & \$0 & \$200,000 & \$4,422,506 & 1.111\% \\
\hline
Other & & & & 2.837\%                                       & Other & & & & 7.025\% \\
\hline
\small \textbf{Game Value:} &&& \small \textbf{-13.58} && \small \textbf{Winner:} &&& \small \textbf{P2}&\\
\hline
\end{tabular}
\caption{Mixed strategy output, when P1 is given an initial budget of \$800,000, P2 is given an initial budget of \$1,100,000, and the Unweighted, Floored Utility Vector is used.}
\label{8v11table.1}
\end{figure}

\begin{figure}
\centering
\captionsetup{justification=raggedright}
\subfloat[P1: \$800,000 Initial Budget]{
\begin{tikzpicture}
\begin{axis}[
width=7.0cm,
ylabel={\tiny{Amount to Spend (\$1,000s)}},
xlabel={\tiny{Quarter}},
xmin=1, xmax=4,
ymin=-50, ymax=6000,
xtick={1,2,3,4},
ytick={0,1000,2000,3000,4000,5000,6000},
legend pos=north west,
ymajorgrids=true,
grid style=dashed,
]
\addplot[draw=gr21]
coordinates {(1,200)(2,0)(3,0)(4,3512)};
\addplot[draw=gr15]
coordinates {(1,0)(2,200)(3,0)(4,3485)};
\addplot[draw=gr10]
coordinates {(1,400)(2,200)(3,0)(4,3385)};
\addplot[draw=gr7]
coordinates {(1,0)(2,0)(3,200)(4,3440)};
\addplot[draw=gr5]
coordinates {(1,200)(2,200)(3,0)(4,3357)};
\addplot[draw=gr4]
coordinates {(1,600)(2,0)(3,0)(4,3257)};
\addplot[draw=gr3]
coordinates {(1,200)(2,0)(3,200)(4,3312)};
\addplot[draw=gr2]
coordinates {(1,0)(2,400)(3,200)(4,3330)};
\addplot[draw=gr1]
coordinates {(1,400)(2,200)(3,0)(4,3230)};
\addplot[draw=red]
coordinates {(1,0)(2,0)(3,0)(4,3641)};
\end{axis}
\end{tikzpicture}
}
\qquad
\subfloat[P2: \$1,100,000 Initial Budget]{
\begin{tikzpicture}
\begin{axis}[
width=7.0cm,
ylabel={\tiny{Amount to Spend (\$1,000s)}},
xlabel={\tiny{Quarter}},
xmin=1, xmax=4,
ymin=-50, ymax=6000,
xtick={1,2,3,4},
ytick={0,1000,2000,3000,4000,5000,6000},
legend pos=north west,
ymajorgrids=true,
grid style=dashed,
]
\addplot[draw=gr12]
coordinates{(1,200)(2,0)(3,0)(4,4878)};
\addplot[draw=gr10]
coordinates{(1,0)(2,200)(3,0)(4,4850)};
\addplot[draw=gr8]
coordinates{(1,400)(2,0)(3,0)(4,4750)};
\addplot[draw=gr7]
coordinates{(1,0)(2,0)(3,200)(4,4805)};
\addplot[draw=gr6]
coordinates{(1,200)(2,200)(3,0)(4,4723)};
\addplot[draw=gr6]
coordinates{(1,600)(2,0)(3,0)(4,4622)};
\addplot[draw=gr5]
coordinates{(1,200)(2,0)(3,200)(4,4678)};
\addplot[draw=gr4]
coordinates{(1,0)(2,400)(3,0)(4,4695)};
\addplot[draw=gr4]
coordinates{(1,400)(2,200)(3,0)(4,4595)};
\addplot[draw=gr3]
coordinates{(1,800)(2,0)(3,0)(4,4494)};
\addplot[draw=gr3]
coordinates{(1,0)(2,200)(3,200)(4,4650)};
\addplot[draw=gr2]
coordinates{(1,400)(2,0)(3,200)(4,4550)};
\addplot[draw=gr2]
coordinates{(1,200)(2,400)(3,0)(4,4568)};
\addplot[draw=gr2]
coordinates{(1,600)(2,200)(3,0)(4,4467)};
\addplot[draw=gr2]
coordinates{(1,1000)(2,0)(3,0)(4,4366)};
\addplot[draw=gr1]
coordinates{(1,0)(2,0)(3,400)(4,4605)};
\addplot[draw=gr1]
coordinates{(1,200)(2,200)(3,200)(4,4523)};
\addplot[draw=gr1]
coordinates{(1,600)(2,0)(3,200)(4,4422)};
\addplot[draw=blue]
coordinates {(1,0)(2,0)(3,0)(4,5005)};
\end{axis}
\end{tikzpicture}
}
\\
\caption{Overlaid strategies for P1 and P2. Colored line indicates top strategy. Darker gray lines indicate heavily used strategies, while lighter gray lines indicated lightly used strategies.}
\label{8v11.1}
\end{figure}

\begin{figure}
\tiny
\begin{tabular}{ |p{1.0cm}p{1.0cm}p{1.0cm}p{2.0cm}|p{1.0cm}||p{1.0cm}p{1.0cm}p{1.0cm}p{2.0cm}|p{1.0cm}|}
\hline
\multicolumn{10}{|c|}{P1 \$800,000 vs. P2 \$1,200,000} \\
\hline
Player 1 & (P1) & & & & Player 2 & (P2) & & & \\
\hline
Q1 Strategy & Q2 Strategy & Q3 Strategy & Q4 Strategy  &  Play \% & Q1 Strategy & Q2 Strategy & Q3 Strategy & Q4 Strategy  &  Play \%\\
\hline
\$0 & \$0 & \$0 & \$3,640,676 & 29.539\%                & \$0 & \$0 & \$0 & \$5,461,014 & 9.641\% \\
\$200,000 & \$0 & \$0 & \$3,512,868 & 20.915\%          & \$200,000 & \$0 & \$0 & \$5,333,206 & 8.965\% \\
\$0 & \$200,000 & \$0 & \$3,485,636 & 14.501\%          & \$0 & \$200,000 & \$0 & \$5,305,974 & 7.997\% \\
\$400,000 & \$0 & \$0 & \$3,385,061 & 10.416\%          & \$400,000 & \$0 & \$0 & \$5,205,399 & 7.071\% \\
\$0 & \$0 & \$200,000 & \$3,440,676 & 7.322\%           & \$0 & \$0 & \$200,000 & \$5,261,014 & 6.313\% \\
\$200,000 & \$200,000 & \$0 & \$3,357,828 & 5.270\%     & \$200,000 & \$200,000 & \$0 & \$5,178,166 & 5.938\% \\
\$600,000 & \$0 & \$0 & \$3,257,253 & 3.714\%           & \$600,000 & \$0 & \$0 & \$5,077,591 & 5.259\% \\
\$200,000 & \$0 & \$200,000 & \$3,312,868 & 2.558\%     & \$200,000 & \$0 & \$200,000 & \$5,133,206 & 4.830\% \\
\$0 & \$400,000 & \$0 & \$3,330,596 & 1.733\%           & \$0 & \$400,000 & \$0 & \$5,150,934 & 4.203\% \\
\$400,000 & \$200,000 & \$0 & \$3,230,021 & 1.231\%     & \$400,000 & \$200,000 & \$0 & \$5,050,359 & 3.854\% \\
& & & &                                                 & \$800,000 & \$0 & \$0 & \$4,949,783 & 3.526\% \\
& & & &                                                 & \$0 & \$200,000 & \$200,000 & \$5,105,974 & 3.169\% \\
& & & &                                                 & \$400,000 & \$0 & \$200,000 & \$5,005,399 & 2.788\% \\
& & & &                                                 & \$200,000 & \$4,000,00 & \$0 & \$5,023,126 & 2.601\% \\
& & & &                                                 & \$600,000 & \$200,000 & \$0 & \$4,922,551 & 2.365\% \\
& & & &                                                 & \$1,000,000 & \$0 & \$0 & \$4,821,975 & 2.180\% \\
& & & &                                                 & \$0 & \$0 & \$400,000 & \$5,061,014 & 1.890\% \\
& & & &                                                 & \$200,000 & \$200,000 & \$200,000 & \$4,978,166 & 1.718\% \\
& & & &                                                 & \$600,000 & \$0 & \$200,000 & \$4,877,591 & 1.583\% \\
& & & &                                                 & \$0 & \$600,000 & \$0 & \$4,995,894 & 1.382\% \\
& & & &                                                 & \$400,000 & \$400,000 & \$0 & \$4,895,319 & 1.253\% \\
& & & &                                                 & \$800,000 & \$200,000 & \$0 & \$4,794,743 & 1.179\% \\
& & & &                                                 & \$1,200,000 & \$0 & \$0 & \$4,694,168 & 1.036\% \\
\hline
Other & & & & 2.801\%                                   & Other & & & & 9.259\% \\
\hline
\small \textbf{Game Value:} &&& \small \textbf{-18.06} && \small \textbf{Winner:} &&& \small \textbf{P2}&\\
\hline
\end{tabular}
\caption{Mixed strategy output, when P1 is given an initial budget of \$800,000, P2 is given an initial budget of \$1,200,000, and the Unweighted, Floored Utility Vector is used.}
\label{8v12table.1}
\end{figure}

\begin{figure}
\centering
\captionsetup{justification=raggedright}
\subfloat[P1: \$800,000 Initial Budget]{
\begin{tikzpicture}
\begin{axis}[
width=7.0cm,
ylabel={\tiny{Amount to Spend (\$1,000s)}},
xlabel={\tiny{Quarter}},
xmin=1, xmax=4,
ymin=-50, ymax=6000,
xtick={1,2,3,4},
ytick={0,1000,2000,3000,4000,5000,6000},
legend pos=north west,
ymajorgrids=true,
grid style=dashed,
]
\addplot[draw=gr21]
coordinates {(1,200)(2,0)(3,0)(4,3512)};
\addplot[draw=gr15]
coordinates {(1,0)(2,200)(3,0)(4,3485)};
\addplot[draw=gr10]
coordinates {(1,400)(2,200)(3,0)(4,3385)};
\addplot[draw=gr7]
coordinates {(1,0)(2,0)(3,200)(4,3440)};
\addplot[draw=gr5]
coordinates {(1,200)(2,200)(3,0)(4,3357)};
\addplot[draw=gr4]
coordinates {(1,600)(2,0)(3,0)(4,3257)};
\addplot[draw=gr3]
coordinates {(1,200)(2,0)(3,200)(4,3312)};
\addplot[draw=gr2]
coordinates {(1,0)(2,400)(3,200)(4,3330)};
\addplot[draw=gr1]
coordinates {(1,400)(2,200)(3,0)(4,3230)};
\addplot[draw=red]
coordinates {(1,0)(2,0)(3,0)(4,3641)};
\end{axis}
\end{tikzpicture}
}
\qquad
\subfloat[P2: \$1,200,000 Initial Budget]{
\begin{tikzpicture}
\begin{axis}[
width=7.0cm,
ylabel={\tiny{Amount to Spend (\$1,000s)}},
xlabel={\tiny{Quarter}},
xmin=1, xmax=4,
ymin=-50, ymax=6000,
xtick={1,2,3,4},
ytick={0,1000,2000,3000,4000,5000,6000},
legend pos=north west,
ymajorgrids=true,
grid style=dashed,
]
\addplot[draw=gr9]
coordinates {(1,200)(2,0)(3,0)(4,5333)};
\addplot[draw=gr8]
coordinates {(1,0)(2,200)(3,0)(4,5305)};
\addplot[draw=gr7]
coordinates {(1,400)(2,0)(3,0)(4,5205)};
\addplot[draw=gr6]
coordinates {(1,0)(2,0)(3,200)(4,5261)};
\addplot[draw=gr6]
coordinates {(1,200)(2,200)(3,0)(4,5178)};
\addplot[draw=gr5]
coordinates {(1,600)(2,0)(3,0)(4,5077)};
\addplot[draw=gr5]
coordinates {(1,200)(2,0)(3,200)(4,5133)};
\addplot[draw=gr4]
coordinates {(1,0)(2,400)(3,0)(4,5150)};
\addplot[draw=gr4]
coordinates {(1,400)(2,200)(3,0)(4,5050)};
\addplot[draw=gr3]
coordinates {(1,800)(2,0)(3,0)(4,4949)};
\addplot[draw=gr3]
coordinates {(1,0)(2,200)(3,200)(4,5105)};
\addplot[draw=gr3]
coordinates {(1,400)(2,0)(3,200)(4,5005)};
\addplot[draw=gr3]
coordinates {(1,200)(2,400)(3,0)(4,5023)};
\addplot[draw=gr2]
coordinates {(1,600)(2,200)(3,0)(4,4922)};
\addplot[draw=gr2]
coordinates {(1,1000)(2,0)(3,0)(4,4821)};
\addplot[draw=gr2]
coordinates {(1,0)(2,0)(3,400)(4,5061)};
\addplot[draw=gr2]
coordinates {(1,200)(2,200)(3,200)(4,4978)};
\addplot[draw=gr2]
coordinates {(1,600)(2,0)(3,200)(4,4877)};
\addplot[draw=gr1]
coordinates {(1,0)(2,600)(3,0)(4,4995)};
\addplot[draw=gr1]
coordinates {(1,400)(2,400)(3,0)(4,4895)};
\addplot[draw=gr1]
coordinates {(1,800)(2,200)(3,0)(4,4794)};
\addplot[draw=gr1]
coordinates {(1,1200)(2,0)(3,0)(4,4694)};
\addplot[draw=blue]
coordinates {(1,0)(2,0)(3,0)(4,5461)};
\end{axis}
\end{tikzpicture}
}
\caption{Overlaid strategies for P1 and P2. Colored line indicates top strategy. Darker gray lines indicate heavily used strategies, while lighter gray lines indicated lightly used strategies.}
\label{8v12.1}
\end{figure}

\begin{figure}
\tiny
\begin{tabular}{ |p{1.0cm}p{1.0cm}p{1.0cm}p{2.0cm}|p{1.0cm}||p{1.0cm}p{1.0cm}p{1.0cm}p{2.0cm}|p{1.0cm}|}
\hline
\multicolumn{10}{|c|}{P1 \$800,000 vs. P2 \$1,300,000} \\
\hline
Player 1 & (P1) & & & & Player 2 & (P2) & & & \\
\hline
Q1 Strategy & Q2 Strategy & Q3 Strategy & Q4 Strategy  &  Play \% & Q1 Strategy & Q2 Strategy & Q3 Strategy & Q4 Strategy  &  Play \%\\
\hline
\$0 & \$0 & \$0 & \$3,640,676 & 29.363\%            & \$0 & \$0 & \$0 & \$5,916,098 & 8.105\% \\
\$200,000 & \$0 & \$0 & \$3,512,868 & 20.734\%      & \$200,000 & \$0 & \$0 & \$5,788,291 & 7.406\% \\
\$0 & \$200,000 & \$0 & \$3,485,636 & 14.613\%      & \$0 & \$200,000 & \$0 & \$5,761,058 & 6.841\% \\
\$400,000 & \$0 & \$0 & \$3,385,061 & 10.410\%      & \$400,000 & \$0 & \$0 & \$5,660,483 & 6.268\% \\
\$0 & \$0 & \$200,000 & \$3,440,676 & 7.476\%       & \$0 & \$0 & \$200,000 & \$5,716,098 & 5.739\% \\
\$200,000 & \$200,000 & \$0 & \$3,357,828 & 5.097\% & \$200,000 & \$200,000 & \$0 & \$5,633,251 & 5.248\% \\
\$600,000 & \$0 & \$0 & \$3,257,253 & 3.657\%       & \$600,000 & \$0 & \$0 & \$5,532,675 & 4.777\% \\
\$200,000 & \$0 & \$200,000 & \$3,312,868 & 2.555\% & \$200,000 & \$0 & \$200,000 & \$5,588,291 & 4.494\% \\
\$0 & \$400,000 & \$0 & \$3,330,596 & 1.777\%       & \$0 & \$400,000 & \$0 & \$5,606,018 & 4.167\% \\
\$400,000 & \$200,000 & \$0 & \$3,230,021 & 1.257\% & \$400,000 & \$200,000 & \$0 & \$5,505,443 & 3.826\% \\
& & & &                                            & \$800,000 & \$0 & \$0 & \$5,404,868 & 3.458\% \\
& & & &                                            & \$0 & \$200,000 & \$200,000 & \$5,561,058 & 3.346\% \\
& & & &                                            & \$400,000 & \$0 & \$200,000 & \$5,460,483 & 2.987\% \\
& & & &                                            & \$200,000 & \$400,000 & \$0 & \$5,478,211 & 2.781\% \\
& & & &                                            & \$600,000 & \$200,000 & \$0 & \$5,377,635 & 2.532\% \\
& & & &                                            & \$1,000,000 & \$0 & \$0 & \$5,277,060 & 2.297\% \\
& & & &                                            & \$0 & \$0 & \$400,000 & \$5,516,098 & 2.113\% \\
& & & &                                            & \$200,000 & \$200,000 & \$200,000 & \$5,433,251 & 1.943\% \\
& & & &                                            & \$600,000 & \$0 & \$200,000 & \$5,332,675 & 1.822\% \\
& & & &                                            & \$0 & \$600,000 & \$0 & \$5,450,978 & 1.621\% \\
& & & &                                            & \$400,000 & \$400,000 & \$0 & \$5,350,403 & 1.452\% \\
& & & &                                            & \$800,000 & \$200,000 & \$0 & \$5,249,828 & 1.412\% \\
& & & &                                            & \$1,200,000 & \$0 & \$0 & \$5,149,252 & 1.282\% \\
& & & &                                            & \$200,000 & \$0 & \$400,000 & \$5,388,291 & 1.170\% \\
& & & &                                            & \$0 & \$400,000 & \$200,000 & \$5,406,018 & 1.065\% \\
\hline
Other & & & & 3.061\%                               & Other & & & & 11.848\% \\
\hline
\small \textbf{Game Value:} &&& \small \textbf{-22.68} && \small \textbf{Winner:} &&& \small \textbf{P2}&\\
\hline
\end{tabular}
\caption{Mixed strategy output, when P1 is given an initial budget of \$800,000, P2 is given an initial budget of \$1,300,000, and the Unweighted, Floored Utility Vector is used.}
\label{8v13table.1}
\end{figure}

\begin{figure}
\centering
\captionsetup{justification=raggedright}
\subfloat[P1: \$800,000 Initial Budget]{
\begin{tikzpicture}
\begin{axis}[
width=7.0cm,
ylabel={\tiny{Amount to Spend (\$1,000s)}},
xlabel={\tiny{Quarter}},
xmin=1, xmax=4,
ymin=-50, ymax=6000,
xtick={1,2,3,4},
ytick={0,1000,2000,3000,4000,5000,6000},
legend pos=north west,
ymajorgrids=true,
grid style=dashed,
]
\addplot[draw=gr21]
coordinates {(1,200)(2,0)(3,0)(4,3512)};
\addplot[draw=gr15]
coordinates {(1,0)(2,200)(3,0)(4,3485)};
\addplot[draw=gr10]
coordinates {(1,400)(2,200)(3,0)(4,3385)};
\addplot[draw=gr7]
coordinates {(1,0)(2,0)(3,200)(4,3440)};
\addplot[draw=gr5]
coordinates {(1,200)(2,200)(3,0)(4,3357)};
\addplot[draw=gr4]
coordinates {(1,600)(2,0)(3,0)(4,3257)};
\addplot[draw=gr3]
coordinates {(1,200)(2,0)(3,200)(4,3312)};
\addplot[draw=gr2]
coordinates {(1,0)(2,400)(3,200)(4,3330)};
\addplot[draw=gr1]
coordinates {(1,400)(2,200)(3,0)(4,3230)};
\addplot[draw=red]
coordinates {(1,0)(2,0)(3,0)(4,3641)};
\end{axis}
\end{tikzpicture}
}
\qquad
\subfloat[P2: \$1,300,000 Initial Budget]{
\begin{tikzpicture}
\begin{axis}[
width=7.0cm,
ylabel={\tiny{Amount to Spend (\$1,000s)}},
xlabel={\tiny{Quarter}},
xmin=1, xmax=4,
ymin=-50, ymax=6000,
xtick={1,2,3,4},
ytick={0,1000,2000,3000,4000,5000,6000},
legend pos=north west,
ymajorgrids=true,
grid style=dashed,
]
\addplot[draw=gr7]
coordinates {(1,200)(2,0)(3,0)(4,5788)};
\addplot[draw=gr7]
coordinates {(1,0)(2,200)(3,0)(4,5761)};
\addplot[draw=gr6]
coordinates {(1,400)(2,0)(3,0)(4,5660)};
\addplot[draw=gr6]
coordinates {(1,0)(2,0)(3,200)(4,5716)};
\addplot[draw=gr5]
coordinates {(1,200)(2,200)(3,0)(4,5633)};
\addplot[draw=gr5]
coordinates {(1,600)(2,0)(3,0)(4,5532)};
\addplot[draw=gr4]
coordinates {(1,200)(2,0)(3,200)(4,5588)};
\addplot[draw=gr4]
coordinates {(1,0)(2,400)(3,0)(4,5606)};
\addplot[draw=gr4]
coordinates {(1,400)(2,200)(3,0)(4,5505)};
\addplot[draw=gr3]
coordinates {(1,800)(2,0)(3,0)(4,5404)};
\addplot[draw=gr3]
coordinates {(1,0)(2,200)(3,200)(4,5561)};
\addplot[draw=gr3]
coordinates {(1,400)(2,0)(3,300)(4,5460)};
\addplot[draw=gr3]
coordinates {(1,200)(2,400)(3,0)(4,5478)};
\addplot[draw=gr3]
coordinates {(1,600)(2,200)(3,0)(4,5377)};
\addplot[draw=gr2]
coordinates {(1,1000)(2,0)(3,0)(4,5277)};
\addplot[draw=gr2]
coordinates {(1,0)(2,0)(3,400)(4,5516)};
\addplot[draw=gr2]
coordinates {(1,200)(2,200)(3,200)(4,5433)};
\addplot[draw=gr2]
coordinates {(1,600)(2,0)(3,200)(4,5332)};
\addplot[draw=gr2]
coordinates {(1,0)(2,600)(3,0)(4,5450)};
\addplot[draw=gr1]
coordinates {(1,400)(2,400)(3,0)(4,5350)};
\addplot[draw=gr1]
coordinates {(1,800)(2,200)(3,0)(4,5249)};
\addplot[draw=gr1]
coordinates {(1,1200)(2,0)(3,0)(4,5149)};
\addplot[draw=gr1]
coordinates {(1,200)(2,0)(3,400)(4,5388)};
\addplot[draw=gr1]
coordinates {(1,0)(2,400)(3,200)(4,5406)};
\addplot[draw=blue]
coordinates {(1,0)(2,0)(3,0)(4,5916)};
\end{axis}
\end{tikzpicture}
}
\caption{Overlaid strategies for P1 and P2. Colored line indicates top strategy. Darker gray lines indicate heavily used strategies, while lighter gray lines indicated lightly used strategies.}
\label{8v13.1}
\end{figure}

\begin{figure}
\tiny
\begin{tabular}{ |p{1.0cm}p{1.0cm}p{1.0cm}p{2.0cm}|p{1.0cm}||p{1.0cm}p{1.0cm}p{1.0cm}p{2.0cm}|p{1.0cm}|}
\hline
\multicolumn{10}{|c|}{P1 \$800,000 vs. P2 \$1,400,000} \\
\hline
Player 1 & (P1) & & & & Player 2 & (P2) & & & \\
\hline
Q1 Strategy & Q2 Strategy & Q3 Strategy & Q4 Strategy  &  Play \% & Q1 Strategy & Q2 Strategy & Q3 Strategy & Q4 Strategy  &  Play \%\\
\hline
\$0 & \$0 & \$0 & \$3,640,676 & 29.308\%            & \$0 & \$0 & \$0 & \$6,371,183 & 6.379\% \\
\$200,000 & \$0 & \$0 & \$3,512,868 & 20.763\%      & \$200,000 & \$0 & \$0 & \$6,243,375 & 5.883\% \\
\$0 & \$200,000 & \$0 & \$3,485,636 & 14.774\%      & \$0 & \$200,000 & \$0 & \$6,216,143 & 5.563\% \\
\$400,000 & \$0 & \$0 & \$3,385,061 & 10.474\%      & \$400,000 & \$0 & \$0 & \$6,115,568 & 5.112\% \\
\$0 & \$0 & \$200,000 & \$3,440,676 & 7.391\%       & \$0 & \$0 & \$200,000 & \$6,171,183 & 4.854\% \\
\$200,000 & \$200,000 & \$0 & \$3,357,828 & 5.150\%  & \$200,000 & \$200,000 & \$0 & \$6,088,335 & 4.597\% \\
\$600,000 & \$0 & \$0 & \$3,257,253 & 3.692\%       & \$600,000 & \$0 & \$0 & \$5,987,760 & 4.299\% \\
\$200,000 & \$0 & \$200,000 & \$3,312,868 & 2.565\% & \$200,000 & \$0 & \$200,000 & \$6,043,375 & 4.007\% \\
\$0 & \$400,000 & \$0 & \$3,330,596 & 1.725\%       & \$0 & \$400,000 & \$0 & \$6,061,103 & 3.779\% \\
\$400,000 & \$200,000 & \$0 & \$3,230,021 & 1.249\% & \$400,000 & \$200,000 & \$0 & \$5,960,528 & 3.593\% \\
& & & &                                            & \$800,000 & \$0 & \$0 & \$5,859,952 & 3.269\% \\
& & & &                                            & \$0 & \$200,000 & \$200,000 & \$6,016,143 & 3.028\% \\
& & & &                                            & \$400,000 & \$0 & \$200,000 & \$5,915,568 & 2.895\% \\
& & & &                                            & \$200,000 & \$400,000 & \$0 & \$5,933,295 & 2.620\% \\
& & & &                                            & \$600,000 & \$200,000 & \$0 & \$5,832,720 & 2.553\% \\
& & & &                                            & \$1,000,000 & \$0 & \$0 & \$5,732,144 & 2.355\% \\
& & & &                                            & \$0 & \$0 & \$400,000 & \$5,971,183 & 2.231\% \\
& & & &                                            & \$200,000 & \$200,000 & \$200,000 & \$5,888,335 & 2.171\% \\
& & & &                                            & \$600,000 & \$0 & \$200,000 & \$5,787,760 & 2.022\% \\
& & & &                                            & \$0 & \$600,000 & \$0 & \$5,906,063 & 1.811\% \\
& & & &                                            & \$400,000 & \$400,000 & \$0 & \$5,805,488 & 1.719\% \\
& & & &                                            & \$800,000 & \$200,000 & \$0 & \$5,704,912 & 1.585\% \\
& & & &                                            & \$1,200,000 & \$0 & \$0 & \$5,604,337 & 1.478\% \\
& & & &                                            & \$200,000 & \$0 & \$400,000 & \$5,843,375 & 1.407\% \\
& & & &                                            & \$0 & \$400,000 & \$200,000 & \$5,861,103 & 1.353\% \\
& & & &                                            & \$400,000 & \$200,000 & \$200,000 & \$5,760,528 & 1.228\% \\
& & & &                                            & \$800,000 & \$0 & \$200,000 & \$5,659,952 & 1.116\% \\
& & & &                                            & \$200,000 & \$600,000 & \$0 & \$5,778,255 & 1.104\% \\
& & & &                                            & \$600,000 & \$400,000 & \$0 & \$5,677,680 & 1.049\% \\
\hline
Other & & & & 2.909\%                               & Other & & & & 14.940\% \\
\hline
\small \textbf{Game Value:} &&& \small \textbf{-27.20} && \small \textbf{Winner:} &&& \small \textbf{P2}&\\
\hline
\end{tabular}
\caption{Mixed strategy output, when P1 is given an initial budget of \$800,000, P2 is given an initial budget of \$1,400,000, and the Unweighted, Floored Utility Vector is used.}
\label{8v14table.1}
\end{figure}

\begin{figure}
\centering
\captionsetup{justification=raggedright}
\subfloat[P1: \$800,000 Initial Budget]{
\begin{tikzpicture}
\begin{axis}[
width=7.0cm,
ylabel={\tiny{Amount to Spend (\$1,000s)}},
xlabel={\tiny{Quarter}},
xmin=1, xmax=4,
ymin=-50, ymax=7000,
xtick={1,2,3,4},
ytick={0,1000,2000,3000,4000,5000,6000,7000},
legend pos=north west,
ymajorgrids=true,
grid style=dashed,
]
\addplot[draw=gr21]
coordinates {(1,200)(2,0)(3,0)(4,3512)};
\addplot[draw=gr15]
coordinates {(1,0)(2,200)(3,0)(4,3485)};
\addplot[draw=gr10]
coordinates {(1,400)(2,200)(3,0)(4,3385)};
\addplot[draw=gr7]
coordinates {(1,0)(2,0)(3,200)(4,3440)};
\addplot[draw=gr5]
coordinates {(1,200)(2,200)(3,0)(4,3357)};
\addplot[draw=gr4]
coordinates {(1,600)(2,0)(3,0)(4,3257)};
\addplot[draw=gr3]
coordinates {(1,200)(2,0)(3,200)(4,3312)};
\addplot[draw=gr2]
coordinates {(1,0)(2,400)(3,200)(4,3330)};
\addplot[draw=gr1]
coordinates {(1,400)(2,200)(3,0)(4,3230)};
\addplot[draw=red]
coordinates {(1,0)(2,0)(3,0)(4,3641)};
\end{axis}
\end{tikzpicture}
}
\qquad
\subfloat[P2: \$1,400,000 Initial Budget]{
\begin{tikzpicture}
\begin{axis}[
width=7.0cm,
ylabel={\tiny{Amount to Spend (\$1,000s)}},
xlabel={\tiny{Quarter}},
xmin=1, xmax=4,
ymin=-50, ymax=7000,
xtick={1,2,3,4},
ytick={0,1000,2000,3000,4000,5000,6000,7000},
legend pos=north west,
ymajorgrids=true,
grid style=dashed,
]
\addplot[draw=gr6]
coordinates {(1,200)(2,0)(3,0)(4,6243)};
\addplot[draw=gr6]
coordinates {(1,0)(2,200)(3,0)(4,6216)};
\addplot[draw=gr5]
coordinates {(1,400)(2,0)(3,0)(4,6115)};
\addplot[draw=gr5]
coordinates {(1,0)(2,0)(3,200)(4,6171)};
\addplot[draw=gr5]
coordinates {(1,200)(2,200)(3,0)(4,6088)};
\addplot[draw=gr4]
coordinates {(1,600)(2,0)(3,0)(4,5987)};
\addplot[draw=gr4]
coordinates {(1,200)(2,0)(3,200)(4,6043)};
\addplot[draw=gr4]
coordinates {(1,0)(2,400)(3,0)(4,6061)};
\addplot[draw=gr4]
coordinates {(1,400)(2,200)(3,0)(4,5960)};
\addplot[draw=gr3]
coordinates {(1,800)(2,0)(3,0)(4,5859)};
\addplot[draw=gr3]
coordinates {(1,0)(2,200)(3,200)(4,6016)};
\addplot[draw=gr3]
coordinates {(1,400)(2,0)(3,200)(4,5915)};
\addplot[draw=gr3]
coordinates {(1,200)(2,400)(3,0)(4,5933)};
\addplot[draw=gr3]
coordinates {(1,600)(2,200)(3,0)(4,5832)};
\addplot[draw=gr2]
coordinates {(1,1000)(2,0)(3,0)(4,5732)};
\addplot[draw=gr2]
coordinates {(1,0)(2,0)(3,400)(4,5971)};
\addplot[draw=gr2]
coordinates {(1,200)(2,200)(3,200)(4,5888)};
\addplot[draw=gr2]
coordinates {(1,600)(2,0)(3,200)(4,5787)};
\addplot[draw=gr2]
coordinates {(1,0)(2,600)(3,0)(4,5906)};
\addplot[draw=gr2]
coordinates {(1,400)(2,400)(3,0)(4,5805)};
\addplot[draw=gr2]
coordinates {(1,800)(2,200)(3,0)(4,5704)};
\addplot[draw=gr1]
coordinates {(1,1200)(2,0)(3,0)(4,5604)};
\addplot[draw=gr1]
coordinates {(1,200)(2,0)(3,400)(4,5843)};
\addplot[draw=gr1]
coordinates {(1,0)(2,400)(3,200)(4,5861)};
\addplot[draw=gr1]
coordinates {(1,400)(2,200)(3,200)(4,5760)};
\addplot[draw=gr1]
coordinates {(1,800)(2,0)(3,200)(4,5659)};
\addplot[draw=gr1]
coordinates {(1,200)(2,600)(3,0)(4,5778)};
\addplot[draw=gr1]
coordinates {(1,600)(2,400)(3,0)(4,5677)};
\addplot[draw=blue]
coordinates {(1,0)(2,0)(3,0)(4,6371)};
\end{axis}
\end{tikzpicture}
}
\caption{Overlaid strategies for P1 and P2. Colored line indicates top strategy. Darker gray lines indicate heavily used strategies, while lighter gray lines indicated lightly used strategies.}
\label{8v14.1}
\end{figure}

\begin{figure}
\tiny
\begin{tabular}{ |p{1.0cm}p{1.0cm}p{1.0cm}p{2.0cm}|p{1.0cm}||p{1.0cm}p{1.0cm}p{1.0cm}p{2.0cm}|p{1.0cm}|}
\hline
\multicolumn{10}{|c|}{P1 \$800,000 vs. P2 \$1,600,000} \\
\hline
Player 1 & (P1) & & & & Player 2 & (P2) & & & \\
\hline
Q1 Strategy & Q2 Strategy & Q3 Strategy & Q4 Strategy  &  Play \% & Q1 Strategy & Q2 Strategy & Q3 Strategy & Q4 Strategy  &  Play \%\\
\hline
\$0 & \$0 & \$0 & \$3,640,676 & 29.447\%            & \$0 & \$0 & \$0 & \$7,281,352 & 4.358\% \\
\$200,000 & \$0 & \$0 & \$3,512,868 & 20.949\%      & \$200,000 & \$0 & \$0 & \$7,153,544 & 4.223\% \\
\$0 & \$200,000 & \$0 & \$3,485,636 & 14.781\%      & \$0 & \$200,000 & \$0 & \$7,126,312 & 3.993\% \\
\$400,000 & \$0 & \$0 & \$3,385,061 & 10.64\%       & \$400,000 & \$0 & \$0 & \$7,025,737 & 3.813\% \\
\$0 & \$0 & \$200,000 & \$3,440,676 & 7.162\%       & \$0 & \$0 & \$200,000 & \$7,081,352 & 3.608\% \\
\$200,000 & \$200,000 & \$0 & \$3,357,828 & 5.038\% & \$200,000 & \$200,000 & \$0 & \$6,998,504 & 3.401\% \\
\$600,000 & \$0 & \$0 & \$3,257,253 & 3.58\%        & \$600,000 & \$0 & \$0 & \$6,897,929 & 3.259\% \\
\$200,000 & \$0 & \$200,000 & \$3,312,868 & 2.561\% & \$200,000 & \$0 & \$200,000 & \$6,953,544 & 3.159\% \\
\$0 & \$400,000 & \$0 & \$3,330,596 & 1.753\%       & \$0 & \$400,000 & \$0 & \$6,971,272 & 3.069\% \\
\$400,000 & \$200,000 & \$0 & \$3,230,021 & 1.226\% & \$400,000 & \$200,000 & \$0 & \$6,870,697 & 2.893\% \\
& & & &                                            & \$800,000 & \$0 & \$0 & \$6,770,121 & 2.852\% \\
& & & &                                            & \$0 & \$200,000 & \$200,000 & \$6,926,312 & 2.591\% \\
& & & &                                            & \$400,000 & \$0 & \$200,000 & \$6,825,737 & 2.578\% \\
& & & &                                            & \$200,000 & \$400,000 & \$0 & \$6,843,464 & 2.458\% \\
& & & &                                            & \$1,000,000 & \$0 & \$0 & \$6,642,313 & 2.315\% \\
& & & &                                            & \$600,000 & \$200,000 & \$0 & \$6,742,889 & 2.306\% \\
& & & &                                            & \$0 & \$0 & \$400,000 & \$6,881,352 & 2.123\% \\
& & & &                                            & \$200,000 & \$200,000 & \$200,000 & \$6,798,504 & 2.064\% \\
& & & &                                            & \$0 & \$600,000 & \$0 & \$6,816,232 & 2.010\% \\
& & & &                                            & \$600,000 & \$0 & \$200,000 & \$6,697,929 & 1.854\% \\
& & & &                                            & \$400,000 & \$400,000 & \$0 & \$6,715,657 & 1.826\% \\
& & & &                                            & \$800,000 & \$200,000 & \$0 & \$6,615,081 & 1.693\% \\
& & & &                                            & \$1,200,000 & \$0 & \$0 & \$6,514,506 & 1.691\% \\
& & & &                                            & \$200,000 & \$0 & \$400,000 & \$6,753,544 & 1.592\% \\
& & & &                                            & \$0 & \$400,000 & \$200,000 & \$6,771,272 & 1.540\% \\
& & & &                                            & \$400,000 & \$200,000 & \$200,000 & \$6,670,697 & 1.423\% \\
& & & &                                            & \$200,000 & \$600,000 & \$0 & \$6,688,424 & 1.368\% \\
& & & &                                            & \$800,000 & \$0 & \$200,000 & \$6,570,121 & 1.363\% \\
& & & &                                            & \$1,000,000 & \$200,000 & \$0 & \$6,487,273 & 1.274\% \\
& & & &                                            & \$600,000 & \$400,000 & \$0 & \$6,587,849 & 1.236\% \\
& & & &                                            & \$0 & \$200,000 & \$400,000 & \$6,726,312 & 1.136\% \\
& & & &                                            & \$1,400,000 & \$0 & \$0 & \$6,386,698 & 1.101\% \\
& & & &                                            & \$400,000 & \$0 & \$400,000 & \$6,625,737 & 1.043\% \\
\hline
Other & & & & 2.863\%                               & Other & & & & 22.787\% \\
\hline
\small \textbf{Game Value:} &&& \small \textbf{-36.45} && \small \textbf{Winner:} &&& \small \textbf{P2}&\\
\hline
\end{tabular}
\caption{Mixed strategy output, when P1 is given an initial budget of \$800,000, P2 is given an initial budget of \$1,600,000, and the Unweighted, Floored Utility Vector is used.}
\label{8v16table.1}
\end{figure}

\begin{figure}
\centering
\captionsetup{justification=raggedright}
\subfloat[P1: \$800,000 Initial Budget]{
\begin{tikzpicture}
\begin{axis}[
width=7.0cm,
ylabel={\tiny{Amount to Spend (\$1,000s)}},
xlabel={\tiny{Quarter}},
xmin=1, xmax=4,
ymin=-50, ymax=8000,
xtick={1,2,3,4},
ytick={0,1000,2000,3000,4000,5000,6000,7000,8000},
legend pos=north west,
ymajorgrids=true,
grid style=dashed,
]
\addplot[draw=gr21]
coordinates {(1,200)(2,0)(3,0)(4,3512)};
\addplot[draw=gr15]
coordinates {(1,0)(2,200)(3,0)(4,3485)};
\addplot[draw=gr10]
coordinates {(1,400)(2,200)(3,0)(4,3385)};
\addplot[draw=gr7]
coordinates {(1,0)(2,0)(3,200)(4,3440)};
\addplot[draw=gr5]
coordinates {(1,200)(2,200)(3,0)(4,3357)};
\addplot[draw=gr4]
coordinates {(1,600)(2,0)(3,0)(4,3257)};
\addplot[draw=gr3]
coordinates {(1,200)(2,0)(3,200)(4,3312)};
\addplot[draw=gr2]
coordinates {(1,0)(2,400)(3,200)(4,3330)};
\addplot[draw=gr1]
coordinates {(1,400)(2,200)(3,0)(4,3230)};
\addplot[draw=red]
coordinates {(1,0)(2,0)(3,0)(4,3641)};
\end{axis}
\end{tikzpicture}
}
\qquad
\subfloat[P2: \$1,400,000 Initial Budget]{
\begin{tikzpicture}
\begin{axis}[
width=7.0cm,
ylabel={\tiny{Amount to Spend (\$1,000s)}},
xlabel={\tiny{Quarter}},
xmin=1, xmax=4,
ymin=-50, ymax=8000,
xtick={1,2,3,4},
ytick={0,1000,2000,3000,4000,5000,6000,7000,8000},
legend pos=north west,
ymajorgrids=true,
grid style=dashed,
]
\addplot[draw=gr4]
coordinates {(1,200)(2,0)(3,0)(4,7153)};
\addplot[draw=gr4]
coordinates {(1,0)(2,200)(3,0)(4,7126)};
\addplot[draw=gr4]
coordinates {(1,400)(2,0)(3,0)(4,7025)};
\addplot[draw=gr4]
coordinates {(1,0)(2,0)(3,200)(4,7081)};
\addplot[draw=gr3]
coordinates {(1,200)(2,200)(3,0)(4,6998)};
\addplot[draw=gr3]
coordinates {(1,600)(2,0)(3,0)(4,6897)};
\addplot[draw=gr3]
coordinates {(1,200)(2,0)(3,200)(4,6953)};
\addplot[draw=gr3]
coordinates {(1,0)(2,400)(3,0)(4,6971)};
\addplot[draw=gr3]
coordinates {(1,400)(2,200)(3,0)(4,6870)};
\addplot[draw=gr3]
coordinates {(1,800)(2,0)(3,0)(4,6770)};
\addplot[draw=gr3]
coordinates {(1,0)(2,200)(3,200)(4,6926)};
\addplot[draw=gr3]
coordinates {(1,400)(2,0)(3,200)(4,6825)};
\addplot[draw=gr2]
coordinates {(1,200)(2,400)(3,0)(4,6843)};
\addplot[draw=gr2]
coordinates {(1,1000)(2,0)(3,0)(4,6642)};
\addplot[draw=gr2]
coordinates {(1,600)(2,200)(3,0)(4,6742)};
\addplot[draw=gr2]
coordinates {(1,0)(2,0)(3,400)(4,6881)};
\addplot[draw=gr2]
coordinates {(1,200)(2,200)(3,200)(4,6798)};
\addplot[draw=gr2]
coordinates {(1,0)(2,600)(3,0)(4,6816)};
\addplot[draw=gr2]
coordinates {(1,600)(2,0)(3,200)(4,6697)};
\addplot[draw=gr2]
coordinates {(1,400)(2,400)(3,0)(4,6715)};
\addplot[draw=gr2]
coordinates {(1,800)(2,200)(3,0)(4,6615)};
\addplot[draw=gr2]
coordinates {(1,1200)(2,0)(3,0)(4,6514)};
\addplot[draw=gr2]
coordinates {(1,200)(2,0)(3,400)(4,6753)};
\addplot[draw=gr2]
coordinates {(1,0)(2,400)(3,200)(4,6771)};
\addplot[draw=gr1]
coordinates {(1,400)(2,200)(3,200)(4,6670)};
\addplot[draw=gr1]
coordinates {(1,200)(2,600)(3,0)(4,6688)};
\addplot[draw=gr1]
coordinates {(1,800)(2,0)(3,200)(4,6570)};
\addplot[draw=gr1]
coordinates {(1,1000)(2,200)(3,0)(4,6487)};
\addplot[draw=gr1]
coordinates {(1,600)(2,400)(3,0)(4,6587)};
\addplot[draw=gr1]
coordinates {(1,0)(2,200)(3,400)(4,6726)};
\addplot[draw=gr1]
coordinates {(1,1400)(2,0)(3,0)(4,6386)};
\addplot[draw=gr1]
coordinates {(1,400)(2,0)(3,400)(4,6625)};
\addplot[draw=blue]
coordinates {(1,0)(2,0)(3,0)(4,7281)};
\end{axis}
\end{tikzpicture}
}
\caption{Overlaid strategies for P1 and P2. Colored line indicates top strategy. Darker gray lines indicate heavily used strategies, while lighter gray lines indicated lightly used strategies.}
\label{8v16.1}
\end{figure}

\begin{figure}
\centering
\begin{tikzpicture}
\begin{axis}[
width=10.0cm,
title={Mixed Strategies for Different Budget Ratios},
xlabel={\tiny{Ratio of P2 Initial Budget to P1 Initial Budget}},
ylabel={\tiny{Percent of Strategies Played in $<$1\% of Games}},
xmin=1, xmax=2,
ymin=0, ymax=25,
xtick={1,1.125,1.25,1.375,1.5,1.625,1.75,2},
ytick={0,5,10,15,20,25},
legend pos=outer north east,
ymajorgrids=true,
grid style=dashed,
]
\addplot[%weighted by sample unfloored
draw=red
]
coordinates {(1.0,2.921)(1.125,3.283)(1.25,5.727)(1.375,6.922)(1.5,9.335)(1.625,10.780)(1.75,14.716)(2.0,22.961)};
]
\addplot[%weighted by state unfloored
draw=red,
dashed
]
coordinates {(1.0,2.955)(1.125,3.215)(1.25,5.723)(1.375,6.839)(1.5,10.376)(1.625,11.890)(1.75,15.842)(2.0,21.957)};
]
\addplot[%unweighted unfloored
draw=red,
dotted
]
coordinates {(1.0,2.919)(1.125,3.233)(1.25,5.762)(1.375,6.664)(1.5,9.254)(1.625,10.753)(1.75,15.769)(2.0,21.682)};
]
\addplot[%weighted by sample floored
draw=blue
]
coordinates {(1.0,2.874)(1.125,4.145)(1.25,5.644)(1.375,6.784)(1.5,10.157)(1.625,11.821)(1.75,14.814)(2.0,22.646)};
]
\addplot[%weighted by state floored
draw=blue,
dashed
]
coordinates {(1.0,2.924)(1.125,3.293)(1.25,5.672)(1.375,6.966)(1.5,9.264)(1.625,11.693)(1.75,14.696)(2.0,21.869)};
]
\addplot[%unweighted floored
draw=blue,
dotted
]
coordinates {(1.0,2.866)(1.125,3.230)(1.25,5.846)(1.375,7.025)(1.5,9.259)(1.625,11.848)(1.75,14.940)(2.0,22.787)};
]
\legend{Weighted (sample),Weighted (state),Unweighted, Weighted (sample) Floored, Weighted (state) Floored, Unweighted Floored}
\end{axis}
\end{tikzpicture}
\caption{Games in which a player uses strategies designated for $<$1\% of iterations, tracked as the ratio between budgets of players increases. As one player gains a clear budget advantage, that player's strategy set becomes more uniform.}
\label{budgetratios.1}
\end{figure}

\begin{figure}
\centering
\label{fig:onepercent1}
\begin{tikzpicture}
\begin{axis}[
width=10.0cm,
title={Values for Different Budget Pairings},
xlabel={\tiny{P2 Budget in \$100,000s, played against \$800,000 P1 Budget}},
ylabel={\tiny{Game Value}},
xmin=8, xmax=16,
ymin=-60, ymax=5,
xtick={8, 9, 10, 11, 12, 13, 14, 16},
ytick={5, 0, -10, -20, -30, -40, -50, -60},
legend pos=outer north east,
ymajorgrids=true,
grid style=dashed,
]
\addplot[%weighted by sample unfloored
draw=red
]
coordinates {(8,0)(9,-0.10875)(10,-0.16125)(11,-0.1125)(12,-0.133125)(13,-0.176875)(14,-0.2675)(16,-0.3625)};
]
\addplot[%weighted by state unfloored
draw=red,
dashed
]
coordinates {(8,0.004375)(9,0.528125)(10,0.82375)(11,0.901875)(12,1.00875)(13,0.9975)(14,1.404375)(16,1.75875)};
]
\addplot[%unweighted unfloored
draw=red,
dotted
]
coordinates {(8,0.001875)(9,-1.750625)(10,-8.38875)(11,-10.254375)(12,-14.863125)(13,-15.258125)(14,-20.545625)(16,-27.573125)};
]
\addplot[%weighted by sample floored
draw=blue
]
coordinates {(8,-0.003125)(9,-6.069375)(10,-11.92)(11,-17.9625)(12,-23.934375)(13,-30.041875)(14,-36.033125)(16,-48.36)};
]
\addplot[%weighted by state floored
draw=blue,
dashed
]
coordinates {(8,-0.014375)(9,-6.84625)(10,-13.525625)(11,-20.404375)(12,-26.380625)(13,-34.080625)(14,-40.8975)(16,-54.82875)};
]
\addplot[%unweighted floored
draw=blue,
dotted
]
coordinates {(8,-0.010625)(9,-4.54625)(10,-9.01875)(11,-13.58125)(12,-18.06)(13,-22.676875)(14,-27.204375)(16,-36.45125)};
]
\legend{Weighted (sample),Weighted (state),Unweighted, Weighted (sample) Floored, Weighted (state) Floored, Unweighted Floored}
\end{axis}
\end{tikzpicture}
\caption{Game values using different utility vectors, using different budgets for P2 against an \$800,000 P1 budget. The fairest games are closest to the $x$-axis, while unfair games have average slopes with high absolute values. The fairest game in this graph is created by the Weighted by Sample, Unfloored Utility Vector, while the most unfair is created by the Weighted by State, Floored Utility Vector.}
\label{gamevalues}
\end{figure}


\newpage
\section{Exploration}
While the Unweighted, Floored Utility Vector was chosen as the most realistic, it's important to explore different results given by different utility vectors. It's especially important to explore different vectors in this report, since all the utility vectors are probably very inaccurate. This section contains theoretical results, had different utility vectors been chosen. 

\newpage
\subsection{Weighted by State, Floored Utility Vector}
This utility vector was calculated very similarly to the original vector used in the game rules (Unweighted, Floored Utility Vector). This is because the vector was calculated using methods very similar to that used to calculate the original vector. In fact, for each individual race, the regression slope was calculated exactly the same, and floored in the same way. The main difference in this calculation was finding the final value for each quarter. While the original was a simple average of the regression slopes calculated for each race in a quarter, the Weighted by State, Floored Utility Vector was a weighted average.

To calculate weights, the state population for each state with enough data for each quarter was determined using U.S. Census estimates. Note that, since not all states have a Senate race in 2018, and some were not covered closely enough to yield straw polls, each quarter used data from a different subset of states. After state population was determined for each state, the sum of populations for all state races used that quarter was calculated. No state in 2018 had more than one Senate race, so no state population was counted twice. The weight given to each race was calculated by 
\begin{equation}
w_{state} = \frac{pop_{state}}{pop_{sum}},
\end{equation}
where $w_{state}$ denotes the weight assigned to each state's senate race, $pop_{state}$ denotes the population of any state with a 2018 Senate race, and $pop_{sum}$ denotes the sum of all $pop_{state}$ values that were calculated. The final value for each quarter, then, was given by 
\begin{equation}
v_{q_k} = \sum w_{state} \mid state \in K,
\end{equation}
where $q_k$ denotes the value of the vector in quarter $k$, and $K$ denotes the set of states with enough data to calculate an individual regression slope in quarter K. 

Recall that the original game vector took the form 
\begin{equation}
[ 0.611974882, 0.214168992, 0.234584564, 2.111203038 ]\cdot 2\times 10^{-5}.
\end{equation}
The Weighted by State, Floored Utility Vector, by contrast, takes the form 
\begin{equation}
[ 0.243860933, 0.282414783, 0.14780057, 3.216185277 ]\cdot 2\times 10^{-5}.
\end{equation}
Note that when compared to the original utility vector, the Weighted by State, Floored Utility Vector grants a higher reward for spending in Q2 and Q4, and a lower reward for spending in Q1 and Q3. As a result, the top strategy for this vector is the same as for the original vector --- each player spends no money until the final quarter, and then spends their entire budget. In fact, the first difference in strategies between these two vectors is the fourth-most-used strategy. In the original vector, this strategy --- for either player, regardless of budget --- is to spend \$400,000 in Q1, then the rest in Q4. When using this alternative vector, however, the fourth-most-used strategy is to spend \$200,000 in Q3. This is the result of a much lower voter return in Q1; while in the original vector the return was about 9.79, in this alternative vector, the return is 3.901.

Similar to the original utility vector, this vector necessitates optimal strategies that spend the majority of their budget in Q4. The strategies are ordered, almost without exception, by the amount spent in Q4. The first exception to this rule in any pair of budgets occurs in the player with a larger budget, and always occurs in a strategy used less than 5\% of the time.

Note that, like the original utility vector, this vector yields only wins for P2. This happens for the same reason as before --- because players are never punished for spending more, a higher-budget candidate always has a clear advantage. 

\begin{figure}
\tiny
\begin{tabular}{ |p{1.0cm}p{1.0cm}p{1.0cm}p{2.0cm}|p{1.0cm}||p{1.0cm}p{1.0cm}p{1.0cm}p{2.0cm}|p{1.0cm}|}
\hline
\multicolumn{10}{|c|}{P1 \$800,000 vs. P2 \$800,000} \\
\hline
Player 1 & (P1) & & & & Player 2 & (P2) & & & \\
\hline
Q1 Strategy & Q2 Strategy & Q3 Strategy & Q4 Strategy  &  Play \% & Q1 Strategy & Q2 Strategy & Q3 Strategy & Q4 Strategy  &  Play \%\\
\hline
\$0 & \$0 & \$0 & \$3,640,675  & 29.234\%               & \$0 & \$0 & \$0 & \$3,640,675  & 29.591\% \\
\$200,000 & \$0 & \$0 & \$3,512,868  & 20.754\%         & \$200,000 & \$0 & \$0 & \$3,512,868  & 20.791\% \\
\$0& \$200,000 & \$0 & \$3,485,635  & 14.811\%          & \$0& \$200,000 & \$0 & \$3,485,635  & 14.686\% \\
\$0 & \$0& \$200,000 & \$3,440,675  & 10.323\%          & \$0 & \$0& \$200,000 & \$3,440,675  & 10.404\% \\
\$400,000 & \$0 & \$0 & \$3,385,060  & 7.451\%          & \$400,000 & \$0 & \$0 & \$3,385,060  & 7.319\% \\
\$200,000& \$200,000 & \$0 & \$3,357,828  & 5.240\%     & \$200,000& \$200,000 & \$0 & \$3,357,828  & 5.085\% \\
\$0& \$400,000 & \$0 & \$3,330,595  & 3.570\%           & \$0& \$400,000 & \$0 & \$3,330,595  & 3.651\% \\
\$200,000 & \$0& \$200,000 & \$3,312,868  & 2.611\%      & \$200,000 & \$0& \$200,000 & \$3,312,868  & 2.566\% \\
\$0& \$200,000& \$200,000 & \$3,285,635  & 1.811\%      & \$0& \$200,000& \$200,000 & \$3,285,635  & 1.773\% \\
\$600,000 & \$0 & \$0 & \$3,257,252  & 1.260\%           & \$600,000 & \$0 & \$0 & \$3,257,252  & 1.210\% \\
\hline
Other &&&& 2.935\% & Other &&&& 2.924\% \\
\hline
\small \textbf{Game Value:} &&& \small \textbf{-0.01} && \small \textbf{Winner:} &&& \small \textbf{P2}&\\
\hline
\end{tabular}
\caption{Mixed strategy output, when P1 is given an initial budget of \$800,000, P2 is given an initial budget of \$800,000, and the Weighted by State, Floored Utility Vector is used.}
\label{8v8table.2}
\end{figure}

\begin{figure}
\centering
\captionsetup{justification=raggedright}
\subfloat[P1: \$800,000 Initial Budget]{
\begin{tikzpicture}
\begin{axis}[
width=7.0cm,
ylabel={\tiny{Amount to Spend (\$1,000s)}},
xlabel={\tiny{Quarter}},
xmin=1, xmax=4,
ymin=-50, ymax=4000,
xtick={1,2,3,4},
ytick={0,1000,2000,3000,4000},
legend pos=north west,
ymajorgrids=true,
grid style=dashed,
]
\addplot[draw=gr10]
coordinates {(1,0)(2,0)(3,200)(4,3440)};
\addplot[draw=gr15]
coordinates {(1,0)(2,200)(3,0)(4,3485)};
\addplot[draw=gr2]
coordinates {(1,0)(2,200)(3,200)(4,3285)};
\addplot[draw=gr4]
coordinates {(1,0)(2,400)(3,0)(4,3330)};
\addplot[draw=gr21]
coordinates {(1,200)(2,0)(3,0)(4,3512)};
\addplot[draw=gr3]
coordinates {(1,200)(2,0)(3,200)(4,3312)};
\addplot[draw=gr5]
coordinates {(1,200)(2,200)(3,0)(4,3357)};
\addplot[draw=gr7]
coordinates {(1,400)(2,0)(3,0)(4,3385)};
\addplot[draw=gr1]
coordinates {(1,600)(2,0)(3,0)(4,3257)};
\addplot[draw=red]
coordinates {(1,0)(2,0)(3,0)(4,3640)};
\end{axis}
\end{tikzpicture}
}
\qquad
\subfloat[P2: \$800,000 Initial Budget]{
\begin{tikzpicture}
\begin{axis}[
width=7.0cm,
ylabel={\tiny{Amount to Spend (\$1,000s)}},
xlabel={\tiny{Quarter}},
xmin=1, xmax=4,
ymin=-50, ymax=4000,
xtick={1,2,3,4},
ytick={0,1000,2000,3000,4000},
legend pos=north west,
ymajorgrids=true,
grid style=dashed,
]
\addplot[draw=gr10]
coordinates {(1,0)(2,0)(3,200)(4,3440)};
\addplot[draw=gr15]
coordinates {(1,0)(2,200)(3,0)(4,3485)};
\addplot[draw=gr2]
coordinates {(1,0)(2,200)(3,200)(4,3285)};
\addplot[draw=gr4]
coordinates {(1,0)(2,400)(3,0)(4,3330)};
\addplot[draw=gr21]
coordinates {(1,200)(2,0)(3,0)(4,3512)};
\addplot[draw=gr3]
coordinates {(1,200)(2,0)(3,200)(4,3312)};
\addplot[draw=gr5]
coordinates {(1,200)(2,200)(3,0)(4,3357)};
\addplot[draw=gr7]
coordinates {(1,400)(2,0)(3,0)(4,3385)};
\addplot[draw=gr1]
coordinates {(1,600)(2,0)(3,0)(4,3257)};
\addplot[draw=blue]
coordinates {(1,0)(2,0)(3,0)(4,3640)};
\end{axis}
\end{tikzpicture}
}
\caption{Overlaid strategies for P1 and P2. Colored line indicates top strategy. Darker gray lines indicate heavily used strategies, while lighter gray lines indicated lightly used strategies.}
\label{8v8.2}
\end{figure}

\begin{figure}
\tiny
\begin{tabular}{ |p{1.0cm}p{1.0cm}p{1.0cm}p{2.0cm}|p{1.0cm}||p{1.0cm}p{1.0cm}p{1.0cm}p{2.0cm}|p{1.0cm}|}
\hline
\multicolumn{10}{|c|}{P1 \$800,000 vs. P2 \$900,000} \\
\hline
Player 1 & (P1) & & & & Player 2 & (P2) & & & \\
\hline
Q1 Strategy & Q2 Strategy & Q3 Strategy & Q4 Strategy  &  Play \% & Q1 Strategy & Q2 Strategy & Q3 Strategy & Q4 Strategy  &  Play \%\\
\hline
\$0 & \$0 & \$0 & \$3,640,675  & 29.371\%           & \$0 & \$0 & \$0 & \$4,095,760  & 23.150\% \\
\$200,000 & \$0 & \$0 & \$3,512,868  & 20.968\%     & \$200,000 & \$0 & \$0 & \$3,967,952  & 17.473\% \\
\$0& \$200,000 & \$0 & \$3,485,635  & 14.613\%      & \$0& \$200,000 & \$0 & \$3,940,720  & 13.577\% \\
\$0 & \$0& \$200,000 & \$3,440,675  & 10.487\%      & \$0 & \$0& \$200,000 & \$3,895,760  & 10.488\% \\
\$400,000 & \$0 & \$0 & \$3,385,060  & 7.308\%      & \$400,000 & \$0 & \$0 & \$3,840,145  & 8.122\% \\
\$200,000& \$200,000 & \$0 & \$3,357,828  & 5.157\% & \$200,000& \$200,000 & \$0 & \$3,812,912  & 6.270\% \\
\$0& \$400,000 & \$0 & \$3,330,595  & 3.671\%       & \$0& \$400,000 & \$0 & \$3,785,680  & 4.791\% \\
\$200,000 & \$0& \$200,000 & \$3,312,868  & 2.467\% & \$200,000 & \$0& \$200,000 & \$3,767,952  & 3.654\% \\
\$0& \$200,000& \$200,000 & \$3,285,635  & 1.802\%  & \$0& \$200,000& \$200,000 & \$3,740,720  & 2.915\% \\
\$600,000 & \$0 & \$0 & \$3,257,252  & 1.310\%       & \$600,000 & \$0 & \$0 & \$3,712,337  & 2.239\% \\
&&&&                                                & \$400,000& \$200,000 & \$0 & \$3,685,105  & 1.743\% \\
&&&&                                                & \$0 & \$0& \$400,000 & \$3,695,760  & 1.263\% \\
&&&&                                                & \$200,000& \$400,000 & \$0 & \$3,657,872  & 1.022\% \\
\hline
Other &&&& 2.846\% & Other &&&& 3.293\% \\
\hline
\small \textbf{Game Value:} &&& \small \textbf{-6.85} && \small \textbf{Winner:} &&& \small \textbf{P2}&\\
\hline
\end{tabular}
\caption{Mixed strategy output, when P1 is given an initial budget of \$800,000, P2 is given an initial budget of \$900,000, and the Weighted by State, Floored Utility Vector is used.}
\label{8v9table.2}
\end{figure}

\begin{figure}
\centering
\captionsetup{justification=raggedright}
\subfloat[P1: \$800,000 Initial Budget]{
\begin{tikzpicture}
\begin{axis}[
width=7.0cm,
ylabel={\tiny{Amount to Spend (\$1,000s)}},
xlabel={\tiny{Quarter}},
xmin=1, xmax=4,
ymin=-50, ymax=5000,
xtick={1,2,3,4},
ytick={0,1000,2000,3000,4000,5000},
legend pos=north west,
ymajorgrids=true,
grid style=dashed,
]
\addplot[draw=gr10]
coordinates {(1,0)(2,0)(3,200)(4,3440)};
\addplot[draw=gr15]
coordinates {(1,0)(2,200)(3,0)(4,3485)};
\addplot[draw=gr2]
coordinates {(1,0)(2,200)(3,200)(4,3285)};
\addplot[draw=gr4]
coordinates {(1,0)(2,400)(3,0)(4,3330)};
\addplot[draw=gr21]
coordinates {(1,200)(2,0)(3,0)(4,3512)};
\addplot[draw=gr2]
coordinates {(1,200)(2,0)(3,200)(4,3312)};
\addplot[draw=gr5]
coordinates {(1,200)(2,200)(3,0)(4,3357)};
\addplot[draw=gr7]
coordinates {(1,400)(2,0)(3,0)(4,3385)};
\addplot[draw=gr1]
coordinates {(1,600)(2,0)(3,0)(4,3257)};
\addplot[draw=red]
coordinates {(1,0)(2,0)(3,0)(4,3640)};
\end{axis}
\end{tikzpicture}
}
\qquad
\subfloat[P2: \$900,000 Initial Budget]{
\begin{tikzpicture}
\begin{axis}[
width=7.0cm,
ylabel={\tiny{Amount to Spend (\$1,000s)}},
xlabel={\tiny{Quarter}},
xmin=1, xmax=4,
ymin=-50, ymax=5000,
xtick={1,2,3,4},
ytick={0,1000,2000,3000,4000,5000},
legend pos=north west,
ymajorgrids=true,
grid style=dashed,
]
\addplot[draw=gr10]
coordinates {(1,0)(2,0)(3,200)(4,3895)};
\addplot[draw=gr1]
coordinates {(1,0)(2,0)(3,400)(4,3695)};
\addplot[draw=gr14]
coordinates {(1,0)(2,200)(3,0)(4,3940)};
\addplot[draw=gr3]
coordinates {(1,0)(2,200)(3,200)(4,3740)};
\addplot[draw=gr5]
coordinates {(1,0)(2,400)(3,0)(4,3785)};
\addplot[draw=gr17]
coordinates {(1,200)(2,0)(3,0)(4,3967)};
\addplot[draw=gr4]
coordinates {(1,200)(2,0)(3,200)(4,3767)};
\addplot[draw=gr6]
coordinates {(1,200)(2,200)(3,0)(4,3812)};
\addplot[draw=gr1]
coordinates {(1,200)(2,400)(3,0)(4,3657)};
\addplot[draw=gr8]
coordinates {(1,400)(2,0)(3,0)(4,3840)};
\addplot[draw=gr2]
coordinates {(1,400)(2,200)(3,0)(4,3685)};
\addplot[draw=gr2]
coordinates {(1,600)(2,0)(3,0)(4,3712)};
\addplot[draw=blue]
coordinates {(1,0)(2,0)(3,0)(4,4095)};
\end{axis}
\end{tikzpicture}
}
\caption{Overlaid strategies for P1 and P2. Colored line indicates top strategy. Darker gray lines indicate heavily used strategies, while lighter gray lines indicated lightly used strategies.}
\label{8v9.2}
\end{figure}

\begin{figure}
\tiny
\begin{tabular}{ |p{1.0cm}p{1.0cm}p{1.0cm}p{2.0cm}|p{1.0cm}||p{1.0cm}p{1.0cm}p{1.0cm}p{2.0cm}|p{1.0cm}|}
\hline
\multicolumn{10}{|c|}{P1 \$800,000 vs. P2 \$1,000,000} \\
\hline
Player 1 & (P1) & & & & Player 2 & (P2) & & & \\
\hline
Q1 Strategy & Q2 Strategy & Q3 Strategy & Q4 Strategy  &  Play \% & Q1 Strategy & Q2 Strategy & Q3 Strategy & Q4 Strategy  &  Play \%\\
\hline
\$0 & \$0 & \$0 & \$3,640,675  & 29.295\%           & \$0 & \$0 & \$0 & \$4,550,844  & 16.087\% \\
\$200,000 & \$0 & \$0 & \$3,512,868  & 20.912\%     & \$200,000 & \$0 & \$0 & \$4,423,037  & 13.585\% \\
\$0& \$200,000 & \$0 & \$3,485,635  & 14.838\%      & \$0& \$200,000 & \$0 & \$4,395,804  & 11.431\% \\
\$0 & \$0& \$200,000 & \$3,440,675  & 10.437\%      & \$0 & \$0& \$200,000 & \$4,350,844  & 9.493\% \\
\$400,000 & \$0 & \$0 & \$3,385,060  & 7.216\%      & \$400,000 & \$0 & \$0 & \$4,295,229  & 8.080\% \\
\$200,000& \$200,000 & \$0 & \$3,357,828  & 5.230\% & \$200,000& \$200,000 & \$0 & \$4,267,997  & 6.785\% \\
\$0& \$400,000 & \$0 & \$3,330,595  & 3.575\%       & \$0& \$400,000 & \$0 & \$4,240,764  & 5.596\% \\
\$200,000 & \$0& \$200,000 & \$3,312,868  & 2.507\% & \$200,000 & \$0& \$200,000 & \$4,223,037  & 4.715\% \\
\$0& \$200,000& \$200,000 & \$3,285,635  & 1.770\%  & \$0& \$200,000& \$200,000 & \$4,195,804  & 3.918\% \\
\$600,000 & \$0 & \$0 & \$3,257,252  & 1.303\%       & \$600,000 & \$0 & \$0 & \$4,167,421  & 3.203\% \\
&&&&                                                & \$400,000& \$200,000 & \$0 & \$4,140,189  & 2.859\% \\
&&&&                                                & \$0 & \$0& \$400,000 & \$4,150,844  & 2.384\% \\
&&&&                                                & \$200,000& \$400,000 & \$0 & \$4,112,957  & 1.919\% \\
&&&&                                                & \$0 & \$600,000 & \$0 & \$4,085,724  & 1.675\% \\
&&&&                                                & \$400,000 & \$0& \$200,000 & \$4,095,229  & 1.449\% \\\
&&&&                                                & \$200,000& \$200,000& \$200,000 & \$4,067,997  & 1.149\% \\
\hline
Other &&&& 2.917\% & Other &&&& 5.672\% \\
\hline
\small \textbf{Game Value:} &&& \small \textbf{-13.53} && \small \textbf{Winner:} &&& \small \textbf{P2}&\\
\hline
\end{tabular}
\caption{Mixed strategy output, when P1 is given an initial budget of \$800,000, P2 is given an initial budget of \$1,000,000, and the Weighted by State, Floored Utility Vector is used.}
\label{8v10table.2}
\end{figure}

\begin{figure}
\centering
\captionsetup{justification=raggedright}
\subfloat[P1: \$800,000 Initial Budget]{
\begin{tikzpicture}
\begin{axis}[
width=7.0cm,
ylabel={\tiny{Amount to Spend (\$1,000s)}},
xlabel={\tiny{Quarter}},
xmin=1, xmax=4,
ymin=-50, ymax=5000,
xtick={1,2,3,4},
ytick={0,1000,2000,3000,4000,5000},
legend pos=north west,
ymajorgrids=true,
grid style=dashed,
]
\addplot[draw=gr10]
coordinates {(1,0)(2,0)(3,200)(4,3440)};
\addplot[draw=gr15]
coordinates {(1,0)(2,200)(3,0)(4,3485)};
\addplot[draw=gr2]
coordinates {(1,0)(2,200)(3,200)(4,3285)};
\addplot[draw=gr4]
coordinates {(1,0)(2,400)(3,0)(4,3330)};
\addplot[draw=gr21]
coordinates {(1,200)(2,0)(3,0)(4,3512)};
\addplot[draw=gr3]
coordinates {(1,200)(2,0)(3,200)(4,3312)};
\addplot[draw=gr5]
coordinates {(1,200)(2,200)(3,0)(4,3357)};
\addplot[draw=gr7]
coordinates {(1,400)(2,0)(3,0)(4,3385)};
\addplot[draw=gr1]
coordinates {(1,600)(2,0)(3,0)(4,3257)};
\addplot[draw=red]
coordinates {(1,0)(2,0)(3,0)(4,3640)};
\end{axis}
\end{tikzpicture}
}
\qquad
\subfloat[P2: \$1,000,000 Initial Budget]{
\begin{tikzpicture}
\begin{axis}[
width=7.0cm,
ylabel={\tiny{Amount to Spend (\$1,000s)}},
xlabel={\tiny{Quarter}},
xmin=1, xmax=4,
ymin=-50, ymax=5000,
xtick={1,2,3,4},
ytick={0,1000,2000,3000,4000,5000},
legend pos=north west,
ymajorgrids=true,
grid style=dashed,
]
\addplot[draw=gr9]
coordinates {(1,0)(2,0)(3,200)(4,4350)};
\addplot[draw=gr2]
coordinates {(1,0)(2,0)(3,400)(4,4150)};
\addplot[draw=gr11]
coordinates {(1,0)(2,200)(3,0)(4,4395)};
\addplot[draw=gr4]
coordinates {(1,0)(2,200)(3,200)(4,4195)};
\addplot[draw=gr6]
coordinates {(1,0)(2,400)(3,0)(4,4240)};
\addplot[draw=gr2]
coordinates {(1,0)(2,600)(3,0)(4,4085)};
\addplot[draw=gr14]
coordinates {(1,200)(2,0)(3,0)(4,4423)};
\addplot[draw=gr5]
coordinates {(1,200)(2,0)(3,200)(4,4223)};
\addplot[draw=gr7]
coordinates {(1,200)(2,200)(3,0)(4,4267)};
\addplot[draw=gr1]
coordinates {(1,200)(2,200)(3,200)(4,4067)};
\addplot[draw=gr2]
coordinates {(1,200)(2,400)(3,0)(4,4112)};
\addplot[draw=gr8]
coordinates {(1,400)(2,0)(3,0)(4,4295)};
\addplot[draw=gr1]
coordinates {(1,400)(2,0)(3,200)(4,4095)};
\addplot[draw=gr3]
coordinates {(1,400)(2,200)(3,0)(4,4140)};
\addplot[draw=gr3]
coordinates {(1,600)(2,0)(3,0)(4,4167)};
\addplot[draw=blue]
coordinates {(1,0)(2,0)(3,0)(4,4550)};
\end{axis}
\end{tikzpicture}
}
\caption{Overlaid strategies for P1 and P2. Colored line indicates top strategy. Darker gray lines indicate heavily used strategies, while lighter gray lines indicated lightly used strategies.}
\label{8v10.2}
\end{figure}

\begin{figure}
\tiny
\begin{tabular}{ |p{1.0cm}p{1.0cm}p{1.0cm}p{2.0cm}|p{1.0cm}||p{1.0cm}p{1.0cm}p{1.0cm}p{2.0cm}|p{1.0cm}|}
\hline
\multicolumn{10}{|c|}{P1 \$800,000 vs. P2 \$1,100,000} \\
\hline
Player 1 & (P1) & & & & Player 2 & (P2) & & & \\
\hline
Q1 Strategy & Q2 Strategy & Q3 Strategy & Q4 Strategy  &  Play \% & Q1 Strategy & Q2 Strategy & Q3 Strategy & Q4 Strategy  &  Play \%\\
\hline
\$0 & \$0 & \$0 & \$3,640,675  & 29.610\%           & \$0 & \$0 & \$0 & \$5,005,929  & 13.086\% \\
\$200,000 & \$0 & \$0 & \$3,512,868  & 20.664\%     & \$200,000 & \$0 & \$0 & \$4,878,121  & 11.437\% \\
\$0& \$200,000 & \$0 & \$3,485,635  & 14.628\%      & \$0& \$200,000 & \$0 & \$4,850,889  & 9.759\% \\
\$0 & \$0& \$200,000 & \$3,440,675  & 10.406\%      & \$0 & \$0& \$200,000 & \$4,805,929  & 8.420\% \\
\$400,000 & \$0 & \$0 & \$3,385,060  & 7.226\%      & \$400,000 & \$0 & \$0 & \$4,750,314  & 7.302\% \\
\$200,000& \$200,000 & \$0 & \$3,357,828  & 5.157\% & \$200,000& \$200,000 & \$0 & \$4,723,081  & 6.499\% \\
\$0& \$400,000 & \$0 & \$3,330,595  & 3.691\%       & \$0& \$400,000 & \$0 & \$4,695,849  & 5.637\% \\
\$200,000 & \$0& \$200,000 & \$3,312,868  & 2.587\% & \$200,000 & \$0& \$200,000 & \$4,678,121  & 4.878\% \\
\$0& \$200,000& \$200,000 & \$3,285,635  & 1.795\%  & \$0& \$200,000& \$200,000 & \$4,650,889  & 4.343\% \\
\$600,000 & \$0 & \$0 & \$3,257,252  & 1.311\%       & \$600,000 & \$0 & \$0 & \$4,622,506  & 3.811\% \\
&&&&                                                & \$400,000& \$200,000 & \$0 & \$4,595,274  & 3.230\% \\
&&&&                                                & \$0 & \$0& \$400,000 & \$4,605,929  & 2.884\% \\
&&&&                                                & \$200,000& \$400,000 & \$0 & \$4,568,041  & 2.518\% \\
&&&&                                                & \$0 & \$600,000 & \$0 & \$4,540,809  & 2.144\% \\
&&&&                                                & \$400,000 & \$0& \$200,000 & \$4,550,314  & 1.869\% \\
&&&&                                                & \$200,000& \$200,000& \$200,000 & \$4,523,081  & 1.584\% \\
&&&&                                                & \$800,000 & \$0 & \$0 & \$4,494,698  & 1.396\% \\
&&&&                                                & \$0& \$400,000& \$200,000 & \$4,495,849  & 1.181\% \\
&&&&                                                & \$600,000& \$200,000 & \$0 & \$4,467,466  & 1.056\% \\
\hline
Other &&&& 2.925\% & Other &&&& 6.966\% \\
\hline
\small \textbf{Game Value:} &&& \small \textbf{-20.40} && \small \textbf{Winner:} &&& \small \textbf{P2}&\\
\hline
\end{tabular}
\caption{Mixed strategy output, when P1 is given an initial budget of \$800,000, P2 is given an initial budget of \$1,100,000, and the Weighted by State, Floored Utility Vector is used.}
\label{8v11table.2}
\end{figure}

\begin{figure}
\centering
\captionsetup{justification=raggedright}
\subfloat[P1: \$800,000 Initial Budget]{
\begin{tikzpicture}
\begin{axis}[
width=7.0cm,
ylabel={\tiny{Amount to Spend (\$1,000s)}},
xlabel={\tiny{Quarter}},
xmin=1, xmax=4,
ymin=-50, ymax=5000,
xtick={1,2,3,4},
ytick={0,1000,2000,3000,4000,5000},
legend pos=north west,
ymajorgrids=true,
grid style=dashed,
]
\addplot[draw=gr10]
coordinates {(1,0)(2,0)(3,200)(4,3440)};
\addplot[draw=gr15]
coordinates {(1,0)(2,200)(3,0)(4,3485)};
\addplot[draw=gr2]
coordinates {(1,0)(2,200)(3,200)(4,3285)};
\addplot[draw=gr4]
coordinates {(1,0)(2,400)(3,0)(4,3330)};
\addplot[draw=gr21]
coordinates {(1,200)(2,0)(3,0)(4,3512)};
\addplot[draw=gr3]
coordinates {(1,200)(2,0)(3,200)(4,3312)};
\addplot[draw=gr5]
coordinates {(1,200)(2,200)(3,0)(4,3357)};
\addplot[draw=gr7]
coordinates {(1,400)(2,0)(3,0)(4,3385)};
\addplot[draw=gr1]
coordinates {(1,600)(2,0)(3,0)(4,3257)};
\addplot[draw=red]
coordinates {(1,0)(2,0)(3,0)(4,3640)};
\end{axis}
\end{tikzpicture}
}
\qquad
\subfloat[P2: \$1,100,000 Initial Budget]{
\begin{tikzpicture}
\begin{axis}[
width=7.0cm,
ylabel={\tiny{Amount to Spend (\$1,000s)}},
xlabel={\tiny{Quarter}},
xmin=1, xmax=4,
ymin=-50, ymax=5000,
xtick={1,2,3,4},
ytick={0,1000,2000,3000,4000,5000},
legend pos=north west,
ymajorgrids=true,
grid style=dashed,
]
\addplot[draw=gr8] coordinates
{(1,0)(2,0)(3,200)(4,4805)};
\addplot[draw=gr3] coordinates
{(1,0)(2,0)(3,400)(4,4605)};
\addplot[draw=gr10] coordinates
{(1,0)(2,200)(3,0)(4,4850)};
\addplot[draw=gr4] coordinates
{(1,0)(2,200)(3,200)(4,4650)};
\addplot[draw=gr6] coordinates
{(1,0)(2,400)(3,0)(4,4695)};
\addplot[draw=gr1] coordinates
{(1,0)(2,400)(3,200)(4,4495)};
\addplot[draw=gr2] coordinates
{(1,0)(2,600)(3,0)(4,4540)};
\addplot[draw=gr11] coordinates
{(1,200)(2,0)(3,0)(4,4878)};
\addplot[draw=gr5] coordinates
{(1,200)(2,0)(3,200)(4,4678)};
\addplot[draw=gr6] coordinates
{(1,200)(2,200)(3,0)(4,4723)};
\addplot[draw=gr2] coordinates
{(1,200)(2,200)(3,200)(4,4523)};
\addplot[draw=gr3] coordinates
{(1,200)(2,400)(3,0)(4,4568)};
\addplot[draw=gr7] coordinates
{(1,400)(2,0)(3,0)(4,4750)};
\addplot[draw=gr2] coordinates
{(1,400)(2,0)(3,200)(4,4550)};
\addplot[draw=gr3] coordinates
{(1,400)(2,200)(3,0)(4,4595)};
\addplot[draw=gr4] coordinates
{(1,600)(2,0)(3,0)(4,4622)};
\addplot[draw=gr1] coordinates
{(1,600)(2,200)(3,0)(4,4467)};
\addplot[draw=gr1] coordinates
{(1,800)(2,0)(3,0)(4,4494)};
\addplot[draw=blue] coordinates
{(1,0)(2,0)(3,0)(4,5005)};
\end{axis}
\end{tikzpicture}
}
\caption{Overlaid strategies for P1 and P2. Colored line indicates top strategy. Darker gray lines indicate heavily used strategies, while lighter gray lines indicated lightly used strategies.}
\label{8v11.2}
\end{figure}

\begin{figure}
\tiny
\begin{tabular}{ |p{1.0cm}p{1.0cm}p{1.0cm}p{2.0cm}|p{1.0cm}||p{1.0cm}p{1.0cm}p{1.0cm}p{2.0cm}|p{1.0cm}|}
\hline
\multicolumn{10}{|c|}{P1 \$800,000 vs. P2 \$1,200,000} \\
\hline
Player 1 & (P1) & & & & Player 2 & (P2) & & & \\
\hline
Q1 Strategy & Q2 Strategy & Q3 Strategy & Q4 Strategy  &  Play \% & Q1 Strategy & Q2 Strategy & Q3 Strategy & Q4 Strategy  &  Play \%\\
\hline
\$0 & \$0 & \$0 & \$3,640,675  & 29.290\%           & \$0 & \$0 & \$0 & \$5,461,013  & 9.873\% \\
\$200,000 & \$0 & \$0 & \$3,512,868  & 20.772\%     & \$200,000 & \$0 & \$0 & \$5,333,206  & 8.695\% \\
\$0& \$200,000 & \$0 & \$3,485,635  & 14.696\%      & \$0& \$200,000 & \$0 & \$5,305,973  & 7.907\% \\
\$0 & \$0& \$200,000 & \$3,440,675  & 10.443\%      & \$0 & \$0& \$200,000 & \$5,261,013  & 6.977\% \\
\$400,000 & \$0 & \$0 & \$3,385,060  & 7.354\%      & \$400,000 & \$0 & \$0 & \$5,205,398  & 6.534\% \\
\$200,000& \$200,000 & \$0 & \$3,357,828  & 5.212\% & \$200,000& \$200,000 & \$0 & \$5,178,166  & 5.893\% \\
\$0& \$400,000 & \$0 & \$3,330,595  & 3.686\%       & \$0& \$400,000 & \$0 & \$5,150,933  & 5.355\% \\
\$200,000 & \$0& \$200,000 & \$3,312,868  & 2.538\% & \$200,000 & \$0& \$200,000 & \$5,133,206  & 4.833\% \\
\$0& \$200,000& \$200,000 & \$3,285,635  & 1.816\%  & \$0& \$200,000& \$200,000 & \$5,105,973  & 4.430\% \\
\$600,000 & \$0 & \$0 & \$3,257,252  & 1.336\%      & \$600,000 & \$0 & \$0 & \$5,077,590  & 3.905\% \\
&&&&                                                & \$400,000& \$200,000 & \$0 & \$5,050,358  & 3.461\% \\
&&&&                                                & \$0 & \$0& \$400,000 & \$5,061,013  & 3.094\% \\
&&&&                                                & \$200,000& \$400,000 & \$0 & \$5,023,126  & 2.801\% \\
&&&&                                                & \$0 & \$600,000 & \$0 & \$4,995,893  & 2.636\% \\
&&&&                                                & \$400,000 & \$0& \$200,000 & \$5,005,398  & 2.369\% \\
&&&&                                                & \$200,000& \$200,000& \$200,000 & \$4,978,166  & 2.099\% \\
&&&&                                                & \$800,000 & \$0 & \$0 & \$4,949,783  & 1.901\% \\
&&&&                                                & \$0& \$400,000& \$200,000 & \$4,950,933  & 1.706\% \\
&&&&                                                & \$600,000& \$200,000 & \$0 & \$4,922,550  & 1.507\% \\
&&&&                                                & \$200,000 & \$0& \$400,000 & \$4,933,206  & 1.409\% \\
&&&&                                                & \$400,000& \$400,000 & \$0 & \$4,895,318  & 1.186\% \\
&&&&                                                & \$0& \$200,000& \$400,000 & \$4,905,973  & 1.116\% \\
&&&&                                                & \$200,000 & \$600,000 & \$0 & \$4,868,086  & 1.049\% \\
\hline
Other &&&& 2.857\% & Other &&&& 9.264\% \\
\hline
\small \textbf{Game Value:} &&& \small \textbf{-26.38} && \small \textbf{Winner:} &&& \small \textbf{P2}&\\
\hline
\end{tabular}
\caption{Mixed strategy output, when P1 is given an initial budget of \$800,000, P2 is given an initial budget of \$1,200,000, and the Weighted by State, Floored Utility Vector is used.}
\label{8v12table.2}
\end{figure}

\begin{figure}
\centering
\captionsetup{justification=raggedright}
\subfloat[P1: \$800,000 Initial Budget]{
\begin{tikzpicture}
\begin{axis}[
width=7.0cm,
ylabel={\tiny{Amount to Spend (\$1,000s)}},
xlabel={\tiny{Quarter}},
xmin=1, xmax=4,
ymin=-50, ymax=6000,
xtick={1,2,3,4},
ytick={0,1000,2000,3000,4000,5000,6000},
legend pos=north west,
ymajorgrids=true,
grid style=dashed,
]
\addplot[draw=gr10]
coordinates {(1,0)(2,0)(3,200)(4,3440)};
\addplot[draw=gr15]
coordinates {(1,0)(2,200)(3,0)(4,3485)};
\addplot[draw=gr2]
coordinates {(1,0)(2,200)(3,200)(4,3285)};
\addplot[draw=gr4]
coordinates {(1,0)(2,400)(3,0)(4,3330)};
\addplot[draw=gr21]
coordinates {(1,200)(2,0)(3,0)(4,3512)};
\addplot[draw=gr3]
coordinates {(1,200)(2,0)(3,200)(4,3312)};
\addplot[draw=gr5]
coordinates {(1,200)(2,200)(3,0)(4,3357)};
\addplot[draw=gr7]
coordinates {(1,400)(2,0)(3,0)(4,3385)};
\addplot[draw=gr1]
coordinates {(1,600)(2,0)(3,0)(4,3257)};
\addplot[draw=red]
coordinates {(1,0)(2,0)(3,0)(4,3640)};
\end{axis}
\end{tikzpicture}
}
\qquad
\subfloat[P2: \$1,200,000 Initial Budget]{
\begin{tikzpicture}
\begin{axis}[
width=7.0cm,
ylabel={\tiny{Amount to Spend (\$1,000s)}},
xlabel={\tiny{Quarter}},
xmin=1, xmax=4,
ymin=-50, ymax=6000,
xtick={1,2,3,4},
ytick={0,1000,2000,3000,4000,5000,6000},
legend pos=north west,
ymajorgrids=true,
grid style=dashed,
]
\addplot[draw=gr7]
coordinates {(1,0)(2,0)(3,200)(4,5261)};
\addplot[draw=gr3]
coordinates {(1,0)(2,0)(3,400)(4,5061)};
\addplot[draw=gr8]
coordinates {(1,0)(2,200)(3,0)(4,5305)};
\addplot[draw=gr4]
coordinates {(1,0)(2,200)(3,200)(4,5105)};
\addplot[draw=gr1]
coordinates {(1,0)(2,200)(3,400)(4,4905)};
\addplot[draw=gr5]
coordinates {(1,0)(2,400)(3,0)(4,5150)};
\addplot[draw=gr2]
coordinates {(1,0)(2,400)(3,200)(4,4950)};
\addplot[draw=gr3]
coordinates {(1,0)(2,600)(3,0)(4,4995)};
\addplot[draw=gr9]
coordinates {(1,200)(2,0)(3,0)(4,5333)};
\addplot[draw=gr5]
coordinates {(1,200)(2,0)(3,200)(4,5133)};
\addplot[draw=gr1]
coordinates {(1,200)(2,0)(3,400)(4,4933)};
\addplot[draw=gr6]
coordinates {(1,200)(2,200)(3,0)(4,5178)};
\addplot[draw=gr2]
coordinates {(1,200)(2,200)(3,200)(4,4978)};
\addplot[draw=gr3]
coordinates {(1,200)(2,400)(3,0)(4,5023)};
\addplot[draw=gr1]
coordinates {(1,200)(2,600)(3,0)(4,4868)};
\addplot[draw=gr7]
coordinates {(1,400)(2,0)(3,0)(4,5205)};
\addplot[draw=gr2]
coordinates {(1,400)(2,0)(3,200)(4,5005)};
\addplot[draw=gr3]
coordinates {(1,400)(2,200)(3,0)(4,5050)};
\addplot[draw=gr1]
coordinates {(1,400)(2,400)(3,0)(4,4895)};
\addplot[draw=gr4]
coordinates {(1,600)(2,0)(3,0)(4,5077)};
\addplot[draw=gr2]
coordinates {(1,600)(2,200)(3,0)(4,4922)};
\addplot[draw=gr2]
coordinates {(1,800)(2,0)(3,0)(4,4949)};
\addplot[draw=blue]
coordinates {(1,0)(2,0)(3,0)(4,5461)};
\end{axis}
\end{tikzpicture}
}
\caption{Overlaid strategies for P1 and P2. Colored line indicates top strategy. Darker gray lines indicate heavily used strategies, while lighter gray lines indicated lightly used strategies.}
\label{8v12.2}
\end{figure}

\begin{figure}
\tiny
\begin{tabular}{ |p{1.0cm}p{1.0cm}p{1.0cm}p{2.0cm}|p{1.0cm}||p{1.0cm}p{1.0cm}p{1.0cm}p{2.0cm}|p{1.0cm}|}
\hline
\multicolumn{10}{|c|}{P1 \$800,000 vs. P2 \$1,300,000} \\
\hline
Player 1 & (P1) & & & & Player 2 & (P2) & & & \\
\hline
Q1 Strategy & Q2 Strategy & Q3 Strategy & Q4 Strategy  &  Play \% & Q1 Strategy & Q2 Strategy & Q3 Strategy & Q4 Strategy  &  Play \%\\
\hline
\$0 & \$0 & \$0 & \$3,640,675  & 29.488\%           & \$0 & \$0 & \$0 & \$5,916,098  & 7.910\% \\
\$200,000 & \$0 & \$0 & \$3,512,868  & 20.956\%     & \$200,000 & \$0 & \$0 & \$5,788,290  & 7.383\% \\
\$0& \$200,000 & \$0 & \$3,485,635  & 14.687\%      & \$0& \$200,000 & \$0 & \$5,761,058  & 6.898\% \\
\$0 & \$0& \$200,000 & \$3,440,675  & 10.497\%      & \$0 & \$0& \$200,000 & \$5,716,098  & 6.328\% \\
\$400,000 & \$0 & \$0 & \$3,385,060  & 7.262\%      & \$400,000 & \$0 & \$0 & \$5,660,483  & 5.774\% \\
\$200,000& \$200,000 & \$0 & \$3,357,828  & 5.170\% & \$200,000& \$200,000 & \$0 & \$5,633,250  & 5.447\% \\
\$0& \$400,000 & \$0 & \$3,330,595  & 3.668\%       & \$0& \$400,000 & \$0 & \$5,606,018  & 4.969\% \\
\$200,000 & \$0& \$200,000 & \$3,312,868  & 2.513\% & \$200,000 & \$0& \$200,000 & \$5,588,290  & 4.484\% \\
\$0& \$200,000& \$200,000 & \$3,285,635  & 1.691\%  & \$0& \$200,000& \$200,000 & \$5,561,058  & 4.104\% \\
\$600,000 & \$0 & \$0 & \$3,257,252  & 1.235\%      & \$600,000 & \$0 & \$0 & \$5,532,675  & 3.889\% \\
&&&&                                                & \$400,000& \$200,000 & \$0 & \$5,505,443  & 3.547\% \\
&&&&                                                & \$0 & \$0& \$400,000 & \$5,516,098  & 3.272\% \\
&&&&                                                & \$200,000& \$400,000 & \$0 & \$5,478,210  & 2.824\% \\
&&&&                                                & \$0 & \$600,000 & \$0 & \$5,450,978  & 2.803\% \\
&&&&                                                & \$400,000 & \$0& \$200,000 & \$5,460,483  & 2.442\% \\
&&&&                                                & \$200,000& \$200,000& \$200,000 & \$5,433,250  & 2.339\% \\
&&&&                                                & \$800,000 & \$0 & \$0 & \$5,404,867  & 2.115\% \\
&&&&                                                & \$0& \$400,000& \$200,000 & \$5,406,018  & 1.967\% \\
&&&&                                                & \$600,000& \$200,000 & \$0 & \$5,377,635  & 1.787\% \\
&&&&                                                & \$200,000 & \$0& \$400,000 & \$5,388,290  & 1.619\% \\
&&&&                                                & \$400,000& \$400,000 & \$0 & \$5,350,403  & 1.479\% \\
&&&&                                                & \$0& \$200,000& \$400,000 & \$5,361,058  & 1.366\% \\
&&&&                                                & \$200,000 & \$600,000 & \$0 & \$5,323,170  & 1.333\% \\
&&&&                                                & \$600,000 & \$0& \$200,000 & \$5,332,675  & 1.145\% \\
&&&&                                                & \$0& \$800,000 & \$0 & \$5,295,938  & 1.083\% \\
\hline
Other &&&& 2.833\% & Other &&&& 11.693\% \\
\hline
\small \textbf{Game Value:} &&& \small \textbf{-34.08} && \small \textbf{Winner:} &&& \small \textbf{P2}&\\
\hline
\end{tabular}
\caption{Mixed strategy output, when P1 is given an initial budget of \$800,000, P2 is given an initial budget of \$1,300,000, and the Weighted by State, Floored Utility Vector is used.}
\label{8v13table.2}
\end{figure}

\begin{figure}
\centering
\captionsetup{justification=raggedright}
\subfloat[P1: \$800,000 Initial Budget]{
\begin{tikzpicture}
\begin{axis}[
width=7.0cm,
ylabel={\tiny{Amount to Spend (\$1,000s)}},
xlabel={\tiny{Quarter}},
xmin=1, xmax=4,
ymin=-50, ymax=6000,
xtick={1,2,3,4},
ytick={0,1000,2000,3000,4000,5000,6000},
legend pos=north west,
ymajorgrids=true,
grid style=dashed,
]
\addplot[draw=gr10]
coordinates {(1,0)(2,0)(3,200)(4,3440)};
\addplot[draw=gr15]
coordinates {(1,0)(2,200)(3,0)(4,3485)};
\addplot[draw=gr2]
coordinates {(1,0)(2,200)(3,200)(4,3285)};
\addplot[draw=gr4]
coordinates {(1,0)(2,400)(3,0)(4,3330)};
\addplot[draw=gr21]
coordinates {(1,200)(2,0)(3,0)(4,3512)};
\addplot[draw=gr3]
coordinates {(1,200)(2,0)(3,200)(4,3312)};
\addplot[draw=gr5]
coordinates {(1,200)(2,200)(3,0)(4,3357)};
\addplot[draw=gr7]
coordinates {(1,400)(2,0)(3,0)(4,3385)};
\addplot[draw=gr1]
coordinates {(1,600)(2,0)(3,0)(4,3257)};
\addplot[draw=red]
coordinates {(1,0)(2,0)(3,0)(4,3640)};
\end{axis}
\end{tikzpicture}
}
\qquad
\subfloat[P2: \$1,300,000 Initial Budget]{
\begin{tikzpicture}
\begin{axis}[
width=7.0cm,
ylabel={\tiny{Amount to Spend (\$1,000s)}},
xlabel={\tiny{Quarter}},
xmin=1, xmax=4,
ymin=-50, ymax=6000,
xtick={1,2,3,4},
ytick={0,1000,2000,3000,4000,5000,6000},
legend pos=north west,
ymajorgrids=true,
grid style=dashed,
]
\addplot[draw=gr6]
coordinates {(1,0)(2,0)(3,200)(4,5716)};
\addplot[draw=gr3]
coordinates {(1,0)(2,0)(3,400)(4,5516)};
\addplot[draw=gr7]
coordinates {(1,0)(2,200)(3,0)(4,5761)};
\addplot[draw=gr4]
coordinates {(1,0)(2,200)(3,200)(4,5561)};
\addplot[draw=gr1]
coordinates {(1,0)(2,200)(3,400)(4,5361)};
\addplot[draw=gr5]
coordinates {(1,0)(2,400)(3,0)(4,5606)};
\addplot[draw=gr2]
coordinates {(1,0)(2,400)(3,200)(4,5406)};
\addplot[draw=gr3]
coordinates {(1,0)(2,600)(3,0)(4,5450)};
\addplot[draw=gr1]
coordinates {(1,0)(2,800)(3,0)(4,5295)};
\addplot[draw=gr7]
coordinates {(1,200)(2,0)(3,0)(4,5788)};
\addplot[draw=gr4]
coordinates {(1,200)(2,0)(3,200)(4,5588)};
\addplot[draw=gr2]
coordinates {(1,200)(2,0)(3,400)(4,5388)};
\addplot[draw=gr5]
coordinates {(1,200)(2,200)(3,0)(4,5633)};
\addplot[draw=gr2]
coordinates {(1,200)(2,200)(3,200)(4,5433)};
\addplot[draw=gr3]
coordinates {(1,200)(2,400)(3,0)(4,5478)};
\addplot[draw=gr1]
coordinates {(1,200)(2,600)(3,0)(4,5323)};
\addplot[draw=gr6]
coordinates {(1,400)(2,0)(3,0)(4,5660)};
\addplot[draw=gr2]
coordinates {(1,400)(2,0)(3,200)(4,5460)};
\addplot[draw=gr4]
coordinates {(1,400)(2,200)(3,0)(4,5505)};
\addplot[draw=gr1]
coordinates {(1,400)(2,400)(3,0)(4,5350)};
\addplot[draw=gr4]
coordinates {(1,600)(2,0)(3,0)(4,5532)};
\addplot[draw=gr1]
coordinates {(1,600)(2,0)(3,200)(4,5332)};
\addplot[draw=gr2]
coordinates {(1,600)(2,200)(3,0)(4,5377)};
\addplot[draw=gr2]
coordinates {(1,800)(2,0)(3,0)(4,5404)};
\addplot[draw=blue]
coordinates {(1,0)(2,0)(3,0)(4,5916)};
\end{axis}
\end{tikzpicture}
}
\caption{Overlaid strategies for P1 and P2. Colored line indicates top strategy. Darker gray lines indicate heavily used strategies, while lighter gray lines indicated lightly used strategies.}
\label{8v13.2}
\end{figure}

\begin{figure}
\tiny
\begin{tabular}{ |p{1.0cm}p{1.0cm}p{1.0cm}p{2.0cm}|p{1.0cm}||p{1.0cm}p{1.0cm}p{1.0cm}p{2.0cm}|p{1.0cm}|}
\hline
\multicolumn{10}{|c|}{P1 \$800,000 vs. P2 \$1,400,000} \\
\hline
Player 1 & (P1) & & & & Player 2 & (P2) & & & \\
\hline
Q1 Strategy & Q2 Strategy & Q3 Strategy & Q4 Strategy  &  Play \% & Q1 Strategy & Q2 Strategy & Q3 Strategy & Q4 Strategy  &  Play \%\\
\hline
\$0 & \$0 & \$0 & \$3,640,675  & 29.600\%           & \$0 & \$0 & \$0 & \$6,371,182  & 6.444\% \\
\$200,000 & \$0 & \$0 & \$3,512,868  & 20.779\%     & \$200,000 & \$0 & \$0 & \$6,243,375  & 5.905\% \\
\$0& \$200,000 & \$0 & \$3,485,635  & 14.723\%      & \$0& \$200,000 & \$0 & \$6,216,142  & 5.641\% \\
\$0 & \$0& \$200,000 & \$3,440,675  & 10.310\%      & \$0 & \$0& \$200,000 & \$6,171,182  & 5.240\% \\
\$400,000 & \$0 & \$0 & \$3,385,060  & 7.338\%      & \$400,000 & \$0 & \$0 & \$6,115,567  & 4.930\% \\
\$200,000& \$200,000 & \$0 & \$3,357,828  & 5.077\% & \$200,000& \$200,000 & \$0 & \$6,088,335  & 4.464\% \\
\$0& \$400,000 & \$0 & \$3,330,595  & 3.623\%       & \$0& \$400,000 & \$0 & \$6,061,102  & 4.305\% \\
\$200,000 & \$0& \$200,000 & \$3,312,868  & 2.552\% & \$200,000 & \$0& \$200,000 & \$6,043,375  & 4.048\% \\
\$0& \$200,000& \$200,000 & \$3,285,635  & 1.755\%  & \$0& \$200,000& \$200,000 & \$6,016,142  & 3.740\% \\
\$600,000 & \$0 & \$0 & \$3,257,252  & 1.249\%      & \$600,000 & \$0 & \$0 & \$5,987,759  & 3.449\% \\
&&&&                                                & \$400,000& \$200,000 & \$0 & \$5,960,527  & 3.286\% \\
&&&&                                                & \$0 & \$0& \$400,000 & \$5,971,182  & 3.089\% \\
&&&&                                                & \$200,000& \$400,000 & \$0 & \$5,933,295  & 2.809\% \\
&&&&                                                & \$0 & \$600,000 & \$0 & \$5,906,062  & 2.764\% \\
&&&&                                                & \$400,000 & \$0& \$200,000 & \$5,915,567  & 2.529\% \\
&&&&                                                & \$200,000& \$200,000& \$200,000 & \$5,888,335  & 2.438\% \\
&&&&                                                & \$800,000 & \$0 & \$0 & \$5,859,952  & 2.185\% \\
&&&&                                                & \$0& \$400,000& \$200,000 & \$5,861,102  & 2.073\% \\
&&&&                                                & \$600,000& \$200,000 & \$0 & \$5,832,719  & 2.020\% \\
&&&&                                                & \$200,000 & \$0& \$400,000 & \$5,843,375  & 1.768\% \\
&&&&                                                & \$400,000& \$400,000 & \$0 & \$5,805,487  & 1.756\% \\
&&&&                                                & \$0& \$200,000& \$400,000 & \$5,816,142  & 1.643\% \\
&&&&                                                & \$200,000 & \$600,000 & \$0 & \$5,778,255  & 1.482\% \\
&&&&                                                & \$600,000 & \$0& \$200,000 & \$5,787,759  & 1.430\% \\
&&&&                                                & \$0& \$800,000 & \$0 & \$5,751,022  & 1.3580\% \\
&&&&                                                & \$400,000& \$200,000& \$200,000 & \$5,760,527  & 1.229\% \\
&&&&                                                & \$1,000,000 & \$0 & \$0 & \$5,732,144  & 1.168\% \\
&&&&                                                & \$0 & \$0 & \$600,000 & \$5,771,182  & 1.058\% \\
&&&&                                                & \$200,000& \$400,000& \$200,000 & \$5,733,295  & 1.053\% \\
\hline
Other &&&& 2.994\% & Other &&&& 14.696\% \\
\hline
\small \textbf{Game Value:} &&& \small \textbf{-40.90} && \small \textbf{Winner:} &&& \small \textbf{P2}&\\
\hline
\end{tabular}
\caption{Mixed strategy output, when P1 is given an initial budget of \$800,000, P2 is given an initial budget of \$1,400,000, and the Weighted by State, Floored Utility Vector is used.}
\label{8v14table.2}
\end{figure}

\begin{figure}
\centering
\captionsetup{justification=raggedright}
\subfloat[P1: \$800,000 Initial Budget]{
\begin{tikzpicture}
\begin{axis}[
width=7.0cm,
ylabel={\tiny{Amount to Spend (\$1,000s)}},
xlabel={\tiny{Quarter}},
xmin=1, xmax=4,
ymin=-50, ymax=7000,
xtick={1,2,3,4},
ytick={0,1000,2000,3000,4000,5000,6000,7000},
legend pos=north west,
ymajorgrids=true,
grid style=dashed,
]
\addplot[draw=gr10]
coordinates {(1,0)(2,0)(3,200)(4,3440)};
\addplot[draw=gr15]
coordinates {(1,0)(2,200)(3,0)(4,3485)};
\addplot[draw=gr2]
coordinates {(1,0)(2,200)(3,200)(4,3285)};
\addplot[draw=gr4]
coordinates {(1,0)(2,400)(3,0)(4,3330)};
\addplot[draw=gr21]
coordinates {(1,200)(2,0)(3,0)(4,3512)};
\addplot[draw=gr3]
coordinates {(1,200)(2,0)(3,200)(4,3312)};
\addplot[draw=gr5]
coordinates {(1,200)(2,200)(3,0)(4,3357)};
\addplot[draw=gr7]
coordinates {(1,400)(2,0)(3,0)(4,3385)};
\addplot[draw=gr1]
coordinates {(1,600)(2,0)(3,0)(4,3257)};
\addplot[draw=red]
coordinates {(1,0)(2,0)(3,0)(4,3640)};
\end{axis}
\end{tikzpicture}
}
\qquad
\subfloat[P2: \$1,400,000 Initial Budget]{
\begin{tikzpicture}
\begin{axis}[
width=7.0cm,
ylabel={\tiny{Amount to Spend (\$1,000s)}},
xlabel={\tiny{Quarter}},
xmin=1, xmax=4,
ymin=-50, ymax=7000,
xtick={1,2,3,4},
ytick={0,1000,2000,3000,4000,5000,6000,7000},
legend pos=north west,
ymajorgrids=true,
grid style=dashed,
]
\addplot[draw=gr5]
coordinates {(1,0)(2,0)(3,200)(4,6171)};
\addplot[draw=gr3]
coordinates {(1,0)(2,0)(3,400)(4,5971)};
\addplot[draw=gr1]
coordinates {(1,0)(2,0)(3,600)(4,5771)};
\addplot[draw=gr6]
coordinates {(1,0)(2,200)(3,0)(4,6216)};
\addplot[draw=gr4]
coordinates {(1,0)(2,200)(3,200)(4,6016)};
\addplot[draw=gr2]
coordinates {(1,0)(2,200)(3,400)(4,5816)};
\addplot[draw=gr4]
coordinates {(1,0)(2,400)(3,0)(4,6061)};
\addplot[draw=gr2]
coordinates {(1,0)(2,400)(3,200)(4,5861)};
\addplot[draw=gr3]
coordinates {(1,0)(2,600)(3,0)(4,5906)};
\addplot[draw=gr1]
coordinates {(1,0)(2,800)(3,0)(4,5751)};
\addplot[draw=gr6]
coordinates {(1,200)(2,0)(3,0)(4,6243)};
\addplot[draw=gr4]
coordinates {(1,200)(2,0)(3,200)(4,6043)};
\addplot[draw=gr2]
coordinates {(1,200)(2,0)(3,400)(4,5843)};
\addplot[draw=gr4]
coordinates {(1,200)(2,200)(3,0)(4,6088)};
\addplot[draw=gr2]
coordinates {(1,200)(2,200)(3,200)(4,5888)};
\addplot[draw=gr3]
coordinates {(1,200)(2,400)(3,0)(4,5933)};
\addplot[draw=gr1]
coordinates {(1,200)(2,400)(3,200)(4,5733)};
\addplot[draw=gr1]
coordinates {(1,200)(2,600)(3,0)(4,5778)};
\addplot[draw=gr5]
coordinates {(1,400)(2,0)(3,0)(4,6115)};
\addplot[draw=gr3]
coordinates {(1,400)(2,0)(3,200)(4,5915)};
\addplot[draw=gr3]
coordinates {(1,400)(2,200)(3,0)(4,5960)};
\addplot[draw=gr1]
coordinates {(1,400)(2,200)(3,200)(4,5760)};
\addplot[draw=gr2]
coordinates {(1,400)(2,400)(3,0)(4,5805)};
\addplot[draw=gr3]
coordinates {(1,600)(2,0)(3,0)(4,5987)};
\addplot[draw=gr1]
coordinates {(1,600)(2,0)(3,200)(4,5787)};
\addplot[draw=gr2]
coordinates {(1,600)(2,200)(3,0)(4,5832)};
\addplot[draw=gr2]
coordinates {(1,800)(2,0)(3,0)(4,5859)};
\addplot[draw=gr1]
coordinates {(1,1000)(2,0)(3,0)(4,5732)};
\addplot[draw=blue]
coordinates {(1,0)(2,0)(3,0)(4,6371)};
\end{axis}
\end{tikzpicture}
}
\caption{Overlaid strategies for P1 and P2. Colored line indicates top strategy. Darker gray lines indicate heavily used strategies, while lighter gray lines indicated lightly used strategies.}
\label{8v14.2}
\end{figure}

\begin{figure}
\tiny
\begin{tabular}{ |p{1.0cm}p{1.0cm}p{1.0cm}p{2.0cm}|p{1.0cm}||p{1.0cm}p{1.0cm}p{1.0cm}p{2.0cm}|p{1.0cm}|}
\hline
\multicolumn{10}{|c|}{P1 \$800,000 vs. P2 \$1,600,000} \\
\hline
Player 1 & (P1) & & & & Player 2 & (P2) & & & \\
\hline
Q1 Strategy & Q2 Strategy & Q3 Strategy & Q4 Strategy  &  Play \% & Q1 Strategy & Q2 Strategy & Q3 Strategy & Q4 Strategy  &  Play \%\\
\hline
\$0 & \$0 & \$0 & \$3,640,675  & 29.361\%           & \$0 & \$0 & \$0 & \$7,281,351  & 4.402\% \\
\$200,000 & \$0 & \$0 & \$3,512,868  & 20.672\%     & \$200,000 & \$0 & \$0 & \$7,153,544  & 4.141\% \\
\$0& \$200,000 & \$0 & \$3,485,635  & 14.740\%      & \$0& \$200,000 & \$0 & \$7,126,311  & 3.950\% \\
\$0 & \$0& \$200,000 & \$3,440,675  & 10.491\%      & \$0 & \$0& \$200,000 & \$7,081,351  & 3.931\% \\
\$400,000 & \$0 & \$0 & \$3,385,060  & 7.263\%      & \$400,000 & \$0 & \$0 & \$7,025,736  & 3.642\% \\
\$200,000& \$200,000 & \$0 & \$3,357,828  & 5.169\% & \$200,000& \$200,000 & \$0 & \$6,998,504  & 3.461\% \\
\$0& \$400,000 & \$0 & \$3,330,595  & 3.751\%       & \$0& \$400,000 & \$0 & \$6,971,271  & 3.252\% \\
\$200,000 & \$0& \$200,000 & \$3,312,868  & 2.598\% & \$200,000 & \$0& \$200,000 & \$6,953,544 & 3.173\% \\
\$0& \$200,000& \$200,000 & \$3,285,635  & 1.776\%  & \$0& \$200,000& \$200,000 & \$6,926,311  & 3.029\% \\
\$600,000 & \$0 & \$0& \$3,257,252  & 1.236\%       & \$600,000 & \$0 & \$0 & \$6,897,928  & 2.918\% \\
&&&&                                                & \$0 & \$0& \$400,000 & \$6,881,351  & 2.740\% \\
&&&&                                                & \$400,000& \$200,000 & \$0 & \$6,870,696  & 2.727\% \\
&&&&                                                & \$200,000& \$400,000 & \$0 & \$6,843,464  & 2.525\% \\
&&&&                                                & \$0 & \$600,000 & \$0 & \$6,816,231  & 2.465\% \\
&&&&                                                & \$400,000 & \$0& \$200,000 & \$6,825,736  & 2.314\% \\
&&&&                                                & \$200,000& \$200,000& \$200,000 & \$6,798,504  & 2.195\% \\
&&&&                                                & \$800,000 & \$0 & \$0 & \$6,770,121  & 2.111\% \\
&&&&                                                & \$0& \$400,000& \$200,000 & \$6,771,271  & 2.036\% \\
&&&&                                                & \$200,000 & \$0& \$400,000 & \$6,753,544  & 1.964\% \\
&&&&                                                & \$600,000& \$200,000 & \$0 & \$6,742,888  & 1.909\% \\
&&&&                                                & \$400,000& \$400,000 & \$0 & \$6,715,656  & 1.749\% \\
&&&&                                                & \$0& \$200,000& \$400,000 & \$6,726,311  & 1.741\% \\
&&&&                                                & \$600,000 & \$0& \$200,000 & \$6,697,928  & 1.623\% \\
&&&&                                                & \$200,000 & \$600,000 & \$0 & \$6,688,424  & 1.613\% \\
&&&&                                                & \$0& \$800,000 & \$0 & \$6,661,191  & 1.494\% \\
&&&&                                                & \$400,000& \$200,000& \$200,000 & \$6,670,696  & 1.416\% \\
&&&&                                                & \$1,000,000 & \$0 & \$0& \$6,642,313  & 1.406\% \\
&&&&                                                & \$0 & \$0 & \$600,000 & \$6,681,351  & 1.321\% \\
&&&&                                                & \$200,000& \$400,000& \$200,000 & \$6,643,464  & 1.287\% \\
&&&&                                                & \$800,000& \$200,000 & \$0 & \$6,615,081  & 1.272\% \\
&&&&                                                & \$0 & \$600,000& \$200,000 & \$6,616,231  & 1.122\% \\
&&&&                                                & \$400,000 & \$0& \$400,000 & \$6,625,736  & 1.087\% \\
&&&&                                                & \$600,000& \$400,000 & \$0 & \$6,587,848  & 1.064\% \\
&&&&                                                & \$200,000& \$200,000& \$400,000 & \$6,598,504  & 1.051\% \\
\hline
Other &&&& 2.943\% & Other &&&& 21.869\% \\
\hline
\small \textbf{Game Value:} &&& \small \textbf{-54.83} && \small \textbf{Winner:} &&& \small \textbf{P2}&\\
\hline
\end{tabular}
\caption{Mixed strategy output, when P1 is given an initial budget of \$800,000, P2 is given an initial budget of \$1,600,000, and the Weighted by State, Floored Utility Vector is used.}
\label{8v16table.2}
\end{figure}

\begin{figure}
\centering
\captionsetup{justification=raggedright}
\subfloat[P1: \$800,000 Initial Budget]{
\begin{tikzpicture}
\begin{axis}[
width=7.0cm,
ylabel={\tiny{Amount to Spend (\$1,000s)}},
xlabel={\tiny{Quarter}},
xmin=1, xmax=4,
ymin=-50, ymax=8000,
xtick={1,2,3,4},
ytick={0,1000,2000,3000,4000,5000,6000,7000,8000},
legend pos=north west,
ymajorgrids=true,
grid style=dashed,
]
\addplot[draw=gr10]
coordinates {(1,0)(2,0)(3,200)(4,3440)};
\addplot[draw=gr15]
coordinates {(1,0)(2,200)(3,0)(4,3485)};
\addplot[draw=gr2]
coordinates {(1,0)(2,200)(3,200)(4,3285)};
\addplot[draw=gr4]
coordinates {(1,0)(2,400)(3,0)(4,3330)};
\addplot[draw=gr21]
coordinates {(1,200)(2,0)(3,0)(4,3512)};
\addplot[draw=gr3]
coordinates {(1,200)(2,0)(3,200)(4,3312)};
\addplot[draw=gr5]
coordinates {(1,200)(2,200)(3,0)(4,3357)};
\addplot[draw=gr7]
coordinates {(1,400)(2,0)(3,0)(4,3385)};
\addplot[draw=gr1]
coordinates {(1,600)(2,0)(3,0)(4,3257)};
\addplot[draw=red]
coordinates {(1,0)(2,0)(3,0)(4,3640)};
\end{axis}
\end{tikzpicture}
}
\qquad
\subfloat[P2: \$1,600,000 Initial Budget]{
\begin{tikzpicture}
\begin{axis}[
width=7.0cm,
ylabel={\tiny{Amount to Spend (\$1,000s)}},
xlabel={\tiny{Quarter}},
xmin=1, xmax=4,
ymin=-50, ymax=8000,
xtick={1,2,3,4},
ytick={0,1000,2000,3000,4000,5000,6000,7000,8000},
legend pos=north west,
ymajorgrids=true,
grid style=dashed,
]
\addplot[draw=gr4]
coordinates {(1,0)(2,0)(3,200)(4,7081)};
\addplot[draw=gr3]
coordinates {(1,0)(2,0)(3,400)(4,6881)};
\addplot[draw=gr1]
coordinates {(1,0)(2,0)(3,600)(4,6681)};
\addplot[draw=gr4]
coordinates {(1,0)(2,200)(3,0)(4,7126)};
\addplot[draw=gr3]
coordinates {(1,0)(2,200)(3,200)(4,6926)};
\addplot[draw=gr2]
coordinates {(1,0)(2,200)(3,400)(4,6726)};
\addplot[draw=gr3]
coordinates {(1,0)(2,400)(3,0)(4,6971)};
\addplot[draw=gr2]
coordinates {(1,0)(2,400)(3,200)(4,6771)};
\addplot[draw=gr2]
coordinates {(1,0)(2,600)(3,0)(4,6816)};
\addplot[draw=gr1]
coordinates {(1,0)(2,600)(3,200)(4,6616)};
\addplot[draw=gr1]
coordinates {(1,0)(2,800)(3,0)(4,6661)};
\addplot[draw=gr4]
coordinates {(1,200)(2,0)(3,0)(4,7153)};
\addplot[draw=gr3]
coordinates {(1,200)(2,0)(3,200)(4,6953)};
\addplot[draw=gr2]
coordinates {(1,200)(2,0)(3,400)(4,6753)};
\addplot[draw=gr3]
coordinates {(1,200)(2,200)(3,0)(4,6998)};
\addplot[draw=gr2]
coordinates {(1,200)(2,200)(3,200)(4,6798)};
\addplot[draw=gr1]
coordinates {(1,200)(2,200)(3,400)(4,6598)};
\addplot[draw=gr3]
coordinates {(1,200)(2,400)(3,0)(4,6843)};
\addplot[draw=gr1]
coordinates {(1,200)(2,400)(3,200)(4,6643)};
\addplot[draw=gr2]
coordinates {(1,200)(2,600)(3,0)(4,6688)};
\addplot[draw=gr4]
coordinates {(1,400)(2,0)(3,0)(4,7025)};
\addplot[draw=gr2]
coordinates {(1,400)(2,0)(3,200)(4,6825)};
\addplot[draw=gr1]
coordinates {(1,400)(2,0)(3,400)(4,6625)};
\addplot[draw=gr3]
coordinates {(1,400)(2,200)(3,0)(4,6870)};
\addplot[draw=gr1]
coordinates {(1,400)(2,200)(3,200)(4,6670)};
\addplot[draw=gr2]
coordinates {(1,400)(2,400)(3,0)(4,6715)};
\addplot[draw=gr3]
coordinates {(1,600)(2,0)(3,0)(4,6897)};
\addplot[draw=gr2]
coordinates {(1,600)(2,0)(3,200)(4,6697)};
\addplot[draw=gr2]
coordinates {(1,600)(2,200)(3,0)(4,6742)};
\addplot[draw=gr1]
coordinates {(1,600)(2,400)(3,0)(4,6587)};
\addplot[draw=gr2]
coordinates {(1,800)(2,0)(3,0)(4,6770)};
\addplot[draw=gr1]
coordinates {(1,800)(2,200)(3,0)(4,6615)};
\addplot[draw=gr1]
coordinates {(1,1000)(2,0)(3,0)(4,6642)};
\addplot[draw=blue]
coordinates {(1,0)(2,0)(3,0)(4,7281)};
\end{axis}
\end{tikzpicture}
}
\caption{Overlaid strategies for P1 and P2. Colored line indicates top strategy. Darker gray lines indicate heavily used strategies, while lighter gray lines indicated lightly used strategies.}
\label{8v16.2}
\end{figure}

\newpage
\subsection{Weighted by Sample, Floored Utility Vector}
This vector was calculated extremely similarly to the Weighted by State, Floored Utility Vector. Like that vector, each individual race's regression slope was calculated the same as the original vector (Unweighted, Floored Utility Vector). The only difference in this calculation was the way in which the weights were found. In this vector, the weights were calculated by 
\begin{equation}
w_{state} = \frac{smp_{state}}{smp_{sum}},
\end{equation}
where $w_{state}$ denotes the weight assigned to each state's senate race, $pop_{smp_{state}}$ denotes the total sample size used in straw polls for a given quarter of any state with a 2018 Senate race, and $smp_{sum}$ denotes the sum of all $smp_{state}$ values that were calculated. The final value for each quarter, then, was given by 
\begin{equation}
v_{q_k} = \sum smp_{state} \mid state \in K,
\end{equation}
where $q_k$ denotes the value of the vector in quarter $k$, and $K$ denotes the set of states with enough data to calculate an individual regression slope in quarter K.\\
\\Recall again that the original utility vector was
\begin{equation}
[ 0.611974882, 0.214168992, 0.234584564, 2.111203038 ]\cdot 2\times 10^{-5},
\end{equation}
while the Weighted by Sample, Floored Utility Vector is
\begin{equation}
[ 0.167306442, 0.089473901, 0.11420262, 2.841187593 ]\cdot 2\times 10^{-5}.
\end{equation}
This vector places the most value on Q4 out of any vector considered, when comparing proportions of vector values. It's unsurprising that these strategies, too, are ordered almost perfectly in order of Q4 spending values. When viewed in context of the past two vectors, this vector yielded relatively uninteresting strategies. Like the Unweighted, Floored Utility Vector and Weighted by State, Floored Utility Vectors, any strategy that did not follow this order was used less than 5\% of the time. 

Like the previous two vectors, this vector also yields only wins for P2, since players are never punished for spending money. This gives a clear advantage to higher-budget candidates.

\begin{figure}
\tiny
\begin{tabular}{ |p{1.0cm}p{1.0cm}p{1.0cm}p{2.0cm}|p{1.0cm}||p{1.0cm}p{1.0cm}p{1.0cm}p{2.0cm}|p{1.0cm}|}
\hline
\multicolumn{10}{|c|}{P1 \$800,000 vs. P2 \$800,000} \\
\hline
Player 1 & (P1) & & & & Player 2 & (P2) & & & \\
\hline
Q1 Strategy & Q2 Strategy & Q3 Strategy & Q4 Strategy  &  Play \% & Q1 Strategy & Q2 Strategy & Q3 Strategy & Q4 Strategy  &  Play \%\\
\hline
\$0 & \$0 & \$0 & \$3,640,675 & 29.374\% & \$0 & \$0 & \$0 & \$3,640,675 & 29.455\% \\
\$200,000 & \$0 & \$0 & \$3,512,868 & 20.795\% & \$200,000 & \$0 & \$0 & \$3,512,868 & 20.855\% \\
\$0 & \$200,000 & \$0 & \$3,485,635 & 14.709\% & \$0 & \$200,000 & \$0 & \$3,485,635 & 14.614\% \\
\$0 & \$0 & \$200,000 & \$3,440,675 & 10.318\% & \$0 & \$0 & \$200,000 & \$3,440,675 & 10.320\% \\
\$400,000 & \$0 & \$0 & \$3,385,060 & 7.424\% & \$400,000 & \$0 & \$0 & \$3,385,060 & 7.539\% \\
\$200,000 & \$200,000 & \$0 & \$3,357,828 & 5.154\% & \$200,000 & \$200,000 & \$0 & \$3,357,828 & 5.056\% \\
\$0 & \$400,000 & \$0 & \$3,330,595 & 3.681\% & \$0 & \$400,000 & \$0 & \$3,330,595 & 3.682\% \\
\$200,000 & \$0 & \$200,000 & \$3,312,868 & 2.572\% & \$200,000 & \$0 & \$200,000 & \$3,312,868 & 2.551\% \\
\$0 & \$200,000 & \$200,000 & \$3,285,635 & 1.805\% & \$0 & \$200,000 & \$200,000 & \$3,285,635 & 1.822\% \\
\$600,000 & \$0 & \$0 & \$3,257,252 & 1.259\% & \$600,000 & \$0 & \$0 & \$3,257,252 & 1.232\% \\
\hline
Other &&&& 2.909\% & Other &&&& 2.874\% \\
\hline
\small \textbf{Game Value:} &&& \small \textbf{0.00} && \small \textbf{Winner:} &&& \small \textbf{P2}&\\
\hline
\end{tabular}
\caption{Mixed strategy output, when P1 is given an initial budget of \$800,000, P2 is given an initial budget of \$800,000, and the Weighted by Sample, Floored Utility Vector is used.}
\label{8v8table.3}
\end{figure}

\begin{figure}
\centering
\captionsetup{justification=raggedright}
\subfloat[P1: \$800,000 Initial Budget]{
\begin{tikzpicture}
\begin{axis}[
width=7.0cm,
ylabel={\tiny{Amount to Spend (\$1,000s)}},
xlabel={\tiny{Quarter}},
xmin=1, xmax=4,
ymin=-50, ymax=4000,
xtick={1,2,3,4},
ytick={0,1000,2000,3000,4000},
legend pos=north west,
ymajorgrids=true,
grid style=dashed,
]
\addplot[draw=gr10]
coordinates {(1,0)(2,0)(3,200)(4,3440.675)};
\addplot[draw=gr15]
coordinates {(1,0)(2,200)(3,0)(4,3485.635)};
\addplot[draw=gr2]
coordinates {(1,0)(2,200)(3,200)(4,3285.635)};
\addplot[draw=gr4]
coordinates {(1,0)(2,400)(3,0)(4,3330.595)};
\addplot[draw=gr21]
coordinates {(1,200)(2,0)(3,0)(4,3512.868)};
\addplot[draw=gr3]
coordinates {(1,200)(2,0)(3,200)(4,3312.868)};
\addplot[draw=gr5]
coordinates {(1,200)(2,200)(3,0)(4,3357.828)};
\addplot[draw=gr7]
coordinates {(1,400)(2,0)(3,0)(4,3385.06)};
\addplot[draw=gr1]
coordinates {(1,600)(2,0)(3,0)(4,3257.252)};
\addplot[draw=red]
coordinates {(1,0)(2,0)(3,0)(4,3640.675)};
\end{axis}
\end{tikzpicture}
}
\qquad
\subfloat[P2: \$800,000 Initial Budget]{
\begin{tikzpicture}
\begin{axis}[
width=7.0cm,
ylabel={\tiny{Amount to Spend (\$1,000s)}},
xlabel={\tiny{Quarter}},
xmin=1, xmax=4,
ymin=-50, ymax=4000,
xtick={1,2,3,4},
ytick={0,1000,2000,3000,4000},
legend pos=north west,
ymajorgrids=true,
grid style=dashed,
]
\addplot[draw=gr10]
coordinates {(1,0)(2,0)(3,200)(4,3440.675)};
\addplot[draw=gr15]
coordinates {(1,0)(2,200)(3,0)(4,3485.635)};
\addplot[draw=gr2]
coordinates {(1,0)(2,200)(3,200)(4,3285.635)};
\addplot[draw=gr4]
coordinates {(1,0)(2,400)(3,0)(4,3330.595)};
\addplot[draw=gr21]
coordinates {(1,200)(2,0)(3,0)(4,3512.868)};
\addplot[draw=gr3]
coordinates {(1,200)(2,0)(3,200)(4,3312.868)};
\addplot[draw=gr5]
coordinates {(1,200)(2,200)(3,0)(4,3357.828)};
\addplot[draw=gr8]
coordinates {(1,400)(2,0)(3,0)(4,3385.06)};
\addplot[draw=gr1]
coordinates {(1,600)(2,0)(3,0)(4,3257.252)};
\addplot[draw=blue]
coordinates {(1,0)(2,0)(3,0)(4,3640.675)};
\end{axis}
\end{tikzpicture}
}
\caption{Overlaid strategies for P1 and P2. Colored line indicates top strategy. Darker gray lines indicate heavily used strategies, while lighter gray lines indicated lightly used strategies.}
\label{8v8.3}
\end{figure}

\begin{figure}
\tiny
\begin{tabular}{ |p{1.0cm}p{1.0cm}p{1.0cm}p{2.0cm}|p{1.0cm}||p{1.0cm}p{1.0cm}p{1.0cm}p{2.0cm}|p{1.0cm}|}
\hline
\multicolumn{10}{|c|}{P1 \$800,000 vs. P2 \$900,000} \\
\hline
Player 1 & (P1) & & & & Player 2 & (P2) & & & \\
\hline
Q1 Strategy & Q2 Strategy & Q3 Strategy & Q4 Strategy  &  Play \% & Q1 Strategy & Q2 Strategy & Q3 Strategy & Q4 Strategy  &  Play \%\\
\hline
\$0 & \$0 & \$0 & \$3,640,675 & 29.360\% & \$0 & \$0 & \$0 & \$4,095,760 & 22.927\% \\
\$200,000 & \$0 & \$0 & \$3,512,868 & 20.772\% & \$200,000 & \$0 & \$0 & \$3,967,952 & 17.767\% \\
\$0 & \$200,000 & \$0 & \$3,485,635 & 14.72\% & \$0 & \$200,000 & \$0 & \$3,940,720 & 13.639\% \\
\$0 & \$0 & \$200,000 & \$3,440,675 & 10.628\% & \$0 & \$0 & \$200,000 & \$3,895,760 & 10.629\% \\
\$400,000 & \$0 & \$0 & \$3,385,060 & 7.255\% & \$400,000 & \$0 & \$0 & \$3,840,145 & 8.216\% \\
\$200,000 & \$200,000 & \$0 & \$3,357,828 & 5.098\% & \$200,000 & \$200,000 & \$0 & \$3,812,912 & 6.184\% \\
\$0 & \$400,000 & \$0 & \$3,330,595 & 3.714\% & \$0 & \$400,000 & \$0 & \$3,785,680 & 4.927\% \\
\$200,000 & \$0 & \$200,000 & \$3,312,868 & 2.607\% & \$200,000 & \$0 & \$200,000 & \$3,767,952 & 3.600\% \\
\$0 & \$200,000 & \$200,000 & \$3,285,635 & 1.758\% & \$0 & \$200,000 & \$200,000 & \$3,740,720 & 2.790\% \\
\$600,000 & \$0 & \$0 & \$3,257,252 & 1.195\% & \$600,000 & \$0 & \$0 & \$3,712,337 & 2.223\% \\
&&&&&\$400,000 & \$200,000 & \$0 & \$3,685,105 & 1.680\% \\
&&&&&\$0 & \$0 & \$400,000 & \$3,695,760 & 1.273\% \\
\hline
Other &&&& 2.893\% & Other &&&& 4.145\% \\
\hline
\small \textbf{Game Value:} &&& \small \textbf{-6.07} && \small \textbf{Winner:} &&& \small \textbf{P2}&\\
\hline
\end{tabular}
\caption{Mixed strategy output, when P1 is given an initial budget of \$800,000, P2 is given an initial budget of \$900,000, and the Weighted by Sample, Floored Utility Vector is used.}
\label{8v9table.3}
\end{figure}

\begin{figure}
\centering
\captionsetup{justification=raggedright}
\subfloat[P1: \$800,000 Initial Budget]{
\begin{tikzpicture}
\begin{axis}[
width=7.0cm,
ylabel={\tiny{Amount to Spend (\$1,000s)}},
xlabel={\tiny{Quarter}},
xmin=1, xmax=4,
ymin=-50, ymax=5000,
xtick={1,2,3,4},
ytick={0,1000,2000,3000,4000,5000},
legend pos=north west,
ymajorgrids=true,
grid style=dashed,
]
\addplot[draw=gr11]
coordinates {(1,0)(2,0)(3,200)(4,3440.675)};
\addplot[draw=gr15]
coordinates {(1,0)(2,200)(3,0)(4,3485.635)};
\addplot[draw=gr2]
coordinates {(1,0)(2,200)(3,200)(4,3285.635)};
\addplot[draw=gr4]
coordinates {(1,0)(2,400)(3,0)(4,3330.595)};
\addplot[draw=gr21]
coordinates {(1,200)(2,0)(3,0)(4,3512.868)};
\addplot[draw=gr3]
coordinates {(1,200)(2,0)(3,200)(4,3312.868)};
\addplot[draw=gr5]
coordinates {(1,200)(2,200)(3,0)(4,3357.828)};
\addplot[draw=gr7]
coordinates {(1,400)(2,0)(3,0)(4,3385.06)};
\addplot[draw=gr1]
coordinates {(1,600)(2,0)(3,0)(4,3257.252)};
\addplot[draw=red]
coordinates {(1,0)(2,0)(3,0)(4,3640.675)};
\end{axis}
\end{tikzpicture}
}
\qquad
\subfloat[P2: \$900,000 Initial Budget]{
\begin{tikzpicture}
\begin{axis}[
width=7.0cm,
ylabel={\tiny{Amount to Spend (\$1,000s)}},
xlabel={\tiny{Quarter}},
xmin=1, xmax=4,
ymin=-50, ymax=5000,
xtick={1,2,3,4},
ytick={0,1000,2000,3000,4000,5000},
legend pos=north west,
ymajorgrids=true,
grid style=dashed,
]
\addplot[draw=gr11]
coordinates {(1,0)(2,0)(3,200)(4,3895.76)};
\addplot[draw=gr1]
coordinates {(1,0)(2,0)(3,400)(4,3695.76)};
\addplot[draw=gr14]
coordinates {(1,0)(2,200)(3,0)(4,3940.72)};
\addplot[draw=gr3]
coordinates {(1,0)(2,200)(3,200)(4,3740.72)};
\addplot[draw=gr5]
coordinates {(1,0)(2,400)(3,0)(4,3785.68)};
\addplot[draw=gr18]
coordinates {(1,200)(2,0)(3,0)(4,3967.952)};
\addplot[draw=gr4]
coordinates {(1,200)(2,0)(3,200)(4,3767.952)};
\addplot[draw=gr6]
coordinates {(1,200)(2,200)(3,0)(4,3812.912)};
\addplot[draw=gr8]
coordinates {(1,400)(2,0)(3,0)(4,3840.145)};
\addplot[draw=gr2]
coordinates {(1,400)(2,200)(3,0)(4,3685.105)};
\addplot[draw=gr2]
coordinates {(1,600)(2,0)(3,0)(4,3712.337)};
\addplot[draw=blue]
coordinates {(1,0)(2,0)(3,0)(4,4095.76)};
\end{axis}
\end{tikzpicture}
}
\caption{Overlaid strategies for P1 and P2. Colored line indicates top strategy. Darker gray lines indicate heavily used strategies, while lighter gray lines indicated lightly used strategies.}
\label{8v9.3}
\end{figure}

\begin{figure}
\tiny
\begin{tabular}{ |p{1.0cm}p{1.0cm}p{1.0cm}p{2.0cm}|p{1.0cm}||p{1.0cm}p{1.0cm}p{1.0cm}p{2.0cm}|p{1.0cm}|}
\hline
\multicolumn{10}{|c|}{P1 \$800,000 vs. P2 \$1,000,000} \\
\hline
Player 1 & (P1) & & & & Player 2 & (P2) & & & \\
\hline
Q1 Strategy & Q2 Strategy & Q3 Strategy & Q4 Strategy  &  Play \% & Q1 Strategy & Q2 Strategy & Q3 Strategy & Q4 Strategy  &  Play \%\\
\hline
\$0 & \$0 & \$0 & \$3,640,675 & 29.312\% & \$0 & \$0 & \$0 & \$4,550,844 & 16.166\% \\
\$200,000 & \$0 & \$0 & \$3,512,868 & 20.898\% & \$200,000 & \$0 & \$0 & \$4,423,037 & 13.644\% \\
\$0 & \$200,000 & \$0 & \$3,485,635 & 14.752\% & \$0 & \$200,000 & \$0 & \$4,395,804 & 11.331\% \\
\$0 & \$0 & \$200,000 & \$3,440,675 & 10.347\% & \$0 & \$0 & \$200,000 & \$4,350,844 & 9.676\% \\
\$400,000 & \$0 & \$0 & \$3,385,060 & 7.333\% & \$400,000 & \$0 & \$0 & \$4,295,229 & 8.085\% \\
\$200,000 & \$200,000 & \$0 & \$3,357,828 & 5.208\% & \$200,000 & \$200,000 & \$0 & \$4,267,997 & 6.846\% \\
\$0 & \$400,000 & \$0 & \$3,330,595 & 3.675\% & \$0 & \$400,000 & \$0 & \$4,240,764 & 5.616\% \\
\$200,000 & \$0 & \$200,000 & \$3,312,868 & 2.544\% & \$200,000 & \$0 & \$200,000 & \$4,223,037 & 4.720\% \\
\$0 & \$200,000 & \$200,000 & \$3,285,635 & 1.759\% & \$0 & \$200,000 & \$200,000 & \$4,195,804 & 3.835\% \\
\$600,000 & \$0 & \$0 & \$3,257,252 & 1.260\% & \$600,000 & \$0 & \$0 & \$4,167,421 & 3.357\% \\
&&&&&\$400,000 & \$200,000 & \$0 & \$4,140,189 & 2.751\% \\
&&&&&\$0 & \$0 & \$400,000 & \$4,150,844 & 2.255\% \\
&&&&&\$200,000 & \$400,000 & \$0 & \$4,112,957 & 1.929\% \\
&&&&&\$400,000 & \$0 & \$200,000 & \$4,095,229 & 1.615\% \\
&&&&&\$0 & \$600,000 & \$0 & \$4,085,724 & 1.365\% \\
&&&&&\$200,000 & \$200,000 & \$200,000 & \$4,067,997 & 1.165\% \\
\hline
Other &&&& 2.912\% & Other &&&& 5.644\% \\
\hline
\small \textbf{Game Value:} &&& \small \textbf{-11.92} && \small \textbf{Winner:} &&& \small \textbf{P2}&\\
\hline
\end{tabular}
\caption{Mixed strategy output, when P1 is given an initial budget of \$800,000, P2 is given an initial budget of \$1,000,000, and the Weighted by Sample, Floored Utility Vector is used.}
\label{8v10table.3}
\end{figure}

\begin{figure}
\centering
\captionsetup{justification=raggedright}
\subfloat[P1: \$800,000 Initial Budget]{
\begin{tikzpicture}
\begin{axis}[
width=7.0cm,
ylabel={\tiny{Amount to Spend (\$1,000s)}},
xlabel={\tiny{Quarter}},
xmin=1, xmax=4,
ymin=-50, ymax=5000,
xtick={1,2,3,4},
ytick={0,1000,2000,3000,4000,5000},
legend pos=north west,
ymajorgrids=true,
grid style=dashed,
]
\addplot[draw=gr10]
coordinates {(1,0)(2,0)(3,200)(4,3440.675)};
\addplot[draw=gr15]
coordinates {(1,0)(2,200)(3,0)(4,3485.635)};
\addplot[draw=gr2]
coordinates {(1,0)(2,200)(3,200)(4,3285.635)};
\addplot[draw=gr4]
coordinates {(1,0)(2,400)(3,0)(4,3330.595)};
\addplot[draw=gr21]
coordinates {(1,200)(2,0)(3,0)(4,3512.868)};
\addplot[draw=gr3]
coordinates {(1,200)(2,0)(3,200)(4,3312.868)};
\addplot[draw=gr5]
coordinates {(1,200)(2,200)(3,0)(4,3357.828)};
\addplot[draw=gr7]
coordinates {(1,400)(2,0)(3,0)(4,3385.06)};
\addplot[draw=gr1]
coordinates {(1,600)(2,0)(3,0)(4,3257.252)};
\addplot[draw=red]
coordinates {(1,0)(2,0)(3,0)(4,3640.675)};
\end{axis}
\end{tikzpicture}
}
\qquad
\subfloat[P2: \$1,000,000 Initial Budget]{
\begin{tikzpicture}
\begin{axis}[
width=7.0cm,
ylabel={\tiny{Amount to Spend (\$1,000s)}},
xlabel={\tiny{Quarter}},
xmin=1, xmax=4,
ymin=-50, ymax=5000,
xtick={1,2,3,4},
ytick={0,1000,2000,3000,4000,5000},
legend pos=north west,
ymajorgrids=true,
grid style=dashed,
]
\addplot[draw=gr10]
coordinates {(1,0)(2,0)(3,200)(4,4350.844)};
\addplot[draw=gr2]
coordinates {(1,0)(2,0)(3,400)(4,4150.844)};
\addplot[draw=gr11]
coordinates {(1,0)(2,200)(3,0)(4,4395.804)};
\addplot[draw=gr4]
coordinates {(1,0)(2,200)(3,200)(4,4195.804)};
\addplot[draw=gr6]
coordinates {(1,0)(2,400)(3,0)(4,4240.764)};
\addplot[draw=gr1]
coordinates {(1,0)(2,600)(3,0)(4,4085.724)};
\addplot[draw=gr14]
coordinates {(1,200)(2,0)(3,0)(4,4423.037)};
\addplot[draw=gr5]
coordinates {(1,200)(2,0)(3,200)(4,4223.037)};
\addplot[draw=gr7]
coordinates {(1,200)(2,200)(3,0)(4,4267.997)};
\addplot[draw=gr1]
coordinates {(1,200)(2,200)(3,200)(4,4067.997)};
\addplot[draw=gr2]
coordinates {(1,200)(2,400)(3,0)(4,4112.957)};
\addplot[draw=gr8]
coordinates {(1,400)(2,0)(3,0)(4,4295.229)};
\addplot[draw=gr2]
coordinates {(1,400)(2,0)(3,200)(4,4095.229)};
\addplot[draw=gr3]
coordinates {(1,400)(2,200)(3,0)(4,4140.189)};
\addplot[draw=gr3]
coordinates {(1,600)(2,0)(3,0)(4,4167.421)};
\addplot[draw=blue]
coordinates {(1,0)(2,0)(3,0)(4,4550.844)};
\end{axis}
\end{tikzpicture}
}
\caption{Overlaid strategies for P1 and P2. Colored line indicates top strategy. Darker gray lines indicate heavily used strategies, while lighter gray lines indicated lightly used strategies.}
\label{8v10.3}
\end{figure}

\begin{figure}
\tiny
\begin{tabular}{ |p{1.0cm}p{1.0cm}p{1.0cm}p{2.0cm}|p{1.0cm}||p{1.0cm}p{1.0cm}p{1.0cm}p{2.0cm}|p{1.0cm}|}
\hline
\multicolumn{10}{|c|}{P1 \$800,000 vs. P2 \$1,100,000} \\
\hline
Player 1 & (P1) & & & & Player 2 & (P2) & & & \\
\hline
Q1 Strategy & Q2 Strategy & Q3 Strategy & Q4 Strategy  &  Play \% & Q1 Strategy & Q2 Strategy & Q3 Strategy & Q4 Strategy  &  Play \%\\
\hline
\$0 & \$0 & \$0 & \$3,640,675 & 29.475\% & \$0 & \$0 & \$0 & \$5,005,929 & 12.809\% \\
\$200,000 & \$0 & \$0 & \$3,512,868 & 20.767\% & \$200,000 & \$0 & \$0 & \$4,878,121 & 11.402\% \\
\$0 & \$200,000 & \$0 & \$3,485,635 & 14.858\% & \$0 & \$200,000 & \$0 & \$4,850,889 & 9.86\% \\
\$0 & \$0 & \$200,000 & \$3,440,675 & 10.415\% & \$0 & \$0 & \$200,000 & \$4,805,929 & 8.652\% \\
\$400,000 & \$0 & \$0 & \$3,385,060 & 7.340\% & \$400,000 & \$0 & \$0 & \$4,750,314 & 7.600\% \\
\$200,000 & \$200,000 & \$0 & \$3,357,828 & 5.130\% & \$200,000 & \$200,000 & \$0 & \$4,723,081 & 6.502\% \\
\$0 & \$400,000 & \$0 & \$3,330,595 & 3.627\% & \$0 & \$400,000 & \$0 & \$4,695,849 & 5.708\% \\
\$200,000 & \$0 & \$200,000 & \$3,312,868 & 2.434\% & \$200,000 & \$0 & \$200,000 & \$4,678,121 & 5.006\% \\
\$0 & \$200,000 & \$200,000 & \$3,285,635 & 1.784\% & \$0 & \$200,000 & \$200,000 & \$4,650,889 & 4.272\% \\
\$600,000 & \$0 & \$0 & \$3,257,252 & 1.272\% & \$600,000 & \$0 & \$0 & \$4,622,506 & 3.736\% \\
&&&&&\$400,000 & \$200,000 & \$0 & \$4,595,274 & 3.176\% \\
&&&&&\$0 & \$0 & \$400,000 & \$4,605,929 & 2.845\% \\
&&&&&\$200,000 & \$400,000 & \$0 & \$4,568,041 & 2.387\% \\
&&&&&\$400,000 & \$0 & \$200,000 & \$4,550,314 & 2.128\% \\
&&&&&\$0 & \$600,000 & \$0 & \$4,540,809 & 1.835\% \\
&&&&&\$200,000 & \$200,000 & \$200,000 & \$4,523,081 & 1.620\% \\
&&&&&\$800,000 & \$0 & \$0 & \$4,494,698 & 1.407\% \\
&&&&&\$0 & \$400,000 & \$200,000 & \$4,495,849 & 1.223\% \\
&&&&&\$600,000 & \$200,000 & \$0 & \$4,467,466 & 1.048\% \\
\hline
Other &&&& 2.898\% & Other &&&& 6.784\% \\
\hline
\small \textbf{Game Value:} &&& \small \textbf{-17.96} && \small \textbf{Winner:} &&& \small \textbf{P2}&\\
\hline
\end{tabular}
\caption{Mixed strategy output, when P1 is given an initial budget of \$800,000, P2 is given an initial budget of \$1,100,000, and the Weighted by Sample, Floored Utility Vector is used.}
\label{8v11table.3}
\end{figure}

\begin{figure}
\centering
\captionsetup{justification=raggedright}
\subfloat[P1: \$800,000 Initial Budget]{
\begin{tikzpicture}
\begin{axis}[
width=7.0cm,
ylabel={\tiny{Amount to Spend (\$1,000s)}},
xlabel={\tiny{Quarter}},
xmin=1, xmax=4,
ymin=-50, ymax=6000,
xtick={1,2,3,4},
ytick={0,1000,2000,3000,4000,5000,6000},
legend pos=north west,
ymajorgrids=true,
grid style=dashed,
]
\addplot[draw=gr10]
coordinates {(1,0)(2,0)(3,200)(4,3440.675)};
\addplot[draw=gr15]
coordinates {(1,0)(2,200)(3,0)(4,3485.635)};
\addplot[draw=gr2]
coordinates {(1,0)(2,200)(3,200)(4,3285.635)};
\addplot[draw=gr4]
coordinates {(1,0)(2,400)(3,0)(4,3330.595)};
\addplot[draw=gr21]
coordinates {(1,200)(2,0)(3,0)(4,3512.868)};
\addplot[draw=gr2]
coordinates {(1,200)(2,0)(3,200)(4,3312.868)};
\addplot[draw=gr5]
coordinates {(1,200)(2,200)(3,0)(4,3357.828)};
\addplot[draw=gr7]
coordinates {(1,400)(2,0)(3,0)(4,3385.06)};
\addplot[draw=gr1]
coordinates {(1,600)(2,0)(3,0)(4,3257.252)};
\addplot[draw=red]
coordinates {(1,0)(2,0)(3,0)(4,3640.675)};
\end{axis}
\end{tikzpicture}
}
\qquad
\subfloat[P2: \$1,100,000 Initial Budget]{
\begin{tikzpicture}
\begin{axis}[
width=7.0cm,
ylabel={\tiny{Amount to Spend (\$1,000s)}},
xlabel={\tiny{Quarter}},
xmin=1, xmax=4,
ymin=-50, ymax=6000,
xtick={1,2,3,4},
ytick={0,1000,2000,3000,4000,5000,6000},
legend pos=north west,
ymajorgrids=true,
grid style=dashed,
]
\addplot[draw=gr9]
coordinates {(1,0)(2,0)(3,200)(4,4805.929)};
\addplot[draw=gr3]
coordinates {(1,0)(2,0)(3,400)(4,4605.929)};
\addplot[draw=gr10]
coordinates {(1,0)(2,200)(3,0)(4,4850.889)};
\addplot[draw=gr4]
coordinates {(1,0)(2,200)(3,200)(4,4650.889)};
\addplot[draw=gr6]
coordinates {(1,0)(2,400)(3,0)(4,4695.849)};
\addplot[draw=gr1]
coordinates {(1,0)(2,400)(3,200)(4,4495.849)};
\addplot[draw=gr2]
coordinates {(1,0)(2,600)(3,0)(4,4540.809)};
\addplot[draw=gr11]
coordinates {(1,200)(2,0)(3,0)(4,4878.121)};
\addplot[draw=gr5]
coordinates {(1,200)(2,0)(3,200)(4,4678.121)};
\addplot[draw=gr7]
coordinates {(1,200)(2,200)(3,0)(4,4723.081)};
\addplot[draw=gr2]
coordinates {(1,200)(2,200)(3,200)(4,4523.081)};
\addplot[draw=gr2]
coordinates {(1,200)(2,400)(3,0)(4,4568.041)};
\addplot[draw=gr8]
coordinates {(1,400)(2,0)(3,0)(4,4750.314)};
\addplot[draw=gr2]
coordinates {(1,400)(2,0)(3,200)(4,4550.314)};
\addplot[draw=gr3]
coordinates {(1,400)(2,200)(3,0)(4,4595.274)};
\addplot[draw=gr4]
coordinates {(1,600)(2,0)(3,0)(4,4622.506)};
\addplot[draw=gr1]
coordinates {(1,600)(2,200)(3,0)(4,4467.466)};
\addplot[draw=gr1]
coordinates {(1,800)(2,0)(3,0)(4,4494.698)};
\addplot[draw=blue]
coordinates {(1,0)(2,0)(3,0)(4,5005.929)};
\end{axis}
\end{tikzpicture}
}
\caption{Overlaid strategies for P1 and P2. Colored line indicates top strategy. Darker gray lines indicate heavily used strategies, while lighter gray lines indicated lightly used strategies.}
\label{8v11.3}
\end{figure}

\begin{figure}
\tiny
\begin{tabular}{ |p{1.0cm}p{1.0cm}p{1.0cm}p{2.0cm}|p{1.0cm}||p{1.0cm}p{1.0cm}p{1.0cm}p{2.0cm}|p{1.0cm}|}
\hline
\multicolumn{10}{|c|}{P1 \$800,000 vs. P2 \$1,200,000} \\
\hline
Player 1 & (P1) & & & & Player 2 & (P2) & & & \\
\hline
Q1 Strategy & Q2 Strategy & Q3 Strategy & Q4 Strategy  &  Play \% & Q1 Strategy & Q2 Strategy & Q3 Strategy & Q4 Strategy  &  Play \%\\
\hline
\$0 & \$0 & \$0 & \$3,640,675 & 29.231\% & \$0 & \$0 & \$0 & \$5,461,013 & 9.667\% \\
\$200,000 & \$0 & \$0 & \$3,512,868 & 20.818\% & \$200,000 & \$0 & \$0 & \$5,333,206 & 8.807\% \\
\$0 & \$200,000 & \$0 & \$3,485,635 & 14.767\% & \$0 & \$200,000 & \$0 & \$5,305,973 & 8.139\% \\
\$0 & \$0 & \$200,000 & \$3,440,675 & 10.491\% & \$0 & \$0 & \$200,000 & \$5,261,013 & 7.061\% \\
\$400,000 & \$0 & \$0 & \$3,385,060 & 7.239\% & \$400,000 & \$0 & \$0 & \$5,205,398 & 6.44\% \\
\$200,000 & \$200,000 & \$0 & \$3,357,828 & 5.136\% & \$200,000 & \$200,000 & \$0 & \$5,178,166 & 5.835\% \\
\$0 & \$400,000 & \$0 & \$3,330,595 & 3.705\% & \$0 & \$400,000 & \$0 & \$5,150,933 & 5.336\% \\
\$200,000 & \$0 & \$200,000 & \$3,312,868 & 2.606\% & \$200,000 & \$0 & \$200,000 & \$5,133,206 & 4.797\% \\
\$0 & \$200,000 & \$200,000 & \$3,285,635 & 1.778\% & \$0 & \$200,000 & \$200,000 & \$5,105,973 & 4.229\% \\
\$600,000 & \$0 & \$0 & \$3,257,252 & 1.214\% & \$600,000 & \$0 & \$0 & \$5,077,590 & 3.861\% \\
&&&&&\$400,000 & \$200,000 & \$0 & \$5,050,358 & 3.447\% \\
&&&&&\$0 & \$0 & \$400,000 & \$5,061,013 & 3.163\% \\
&&&&&\$200,000 & \$400,000 & \$0 & \$5,023,126 & 2.887\% \\
&&&&&\$400,000 & \$0 & \$200,000 & \$5,005,398 & 2.633\% \\
&&&&&\$0 & \$600,000 & \$0 & \$4,995,893 & 2.381\% \\
&&&&&\$200,000 & \$200,000 & \$200,000 & \$4,978,166 & 2.104\% \\
&&&&&\$800,000 & \$0 & \$0 & \$4,949,783 & 1.978\% \\
&&&&&\$0 & \$400,000 & \$200,000 & \$4,950,933 & 1.701\% \\
&&&&&\$600,000 & \$200,000 & \$0 & \$4,922,550 & 1.535\% \\
&&&&&\$200,000 & \$0 & \$400,000 & \$4,933,206 & 1.402\% \\
&&&&&\$400,000 & \$400,000 & \$0 & \$4,895,318 & 1.302\% \\
&&&&&\$0 & \$200,000 & \$400,000 & \$4,905,973 & 1.138\% \\
\hline
Other &&&& 3.015\% & Other &&&& 10.157\% \\
\hline
\small \textbf{Game Value:} &&& \small \textbf{-23.93} && \small \textbf{Winner:} &&& \small \textbf{P2}&\\
\hline
\end{tabular}
\caption{Mixed strategy output, when P1 is given an initial budget of \$800,000, P2 is given an initial budget of \$1,200,000, and the Weighted by Sample, Floored Utility Vector is used.}
\label{8v12table.3}
\end{figure}

\begin{figure}
\centering
\captionsetup{justification=raggedright}
\subfloat[P1: \$800,000 Initial Budget]{
\begin{tikzpicture}
\begin{axis}[
width=7.0cm,
ylabel={\tiny{Amount to Spend (\$1,000s)}},
xlabel={\tiny{Quarter}},
xmin=1, xmax=4,
ymin=-50, ymax=6000,
xtick={1,2,3,4},
ytick={0,1000,2000,3000,4000,5000,6000},
legend pos=north west,
ymajorgrids=true,
grid style=dashed,
]
\addplot[draw=gr10]
coordinates {(1,0)(2,0)(3,200)(4,3440.675)};
\addplot[draw=gr15]
coordinates {(1,0)(2,200)(3,0)(4,3485.635)};
\addplot[draw=gr2]
coordinates {(1,0)(2,200)(3,200)(4,3285.635)};
\addplot[draw=gr4]
coordinates {(1,0)(2,400)(3,0)(4,3330.595)};
\addplot[draw=gr21]
coordinates {(1,200)(2,0)(3,0)(4,3512.868)};
\addplot[draw=gr3]
coordinates {(1,200)(2,0)(3,200)(4,3312.868)};
\addplot[draw=gr5]
coordinates {(1,200)(2,200)(3,0)(4,3357.828)};
\addplot[draw=gr7]
coordinates {(1,400)(2,0)(3,0)(4,3385.06)};
\addplot[draw=gr1]
coordinates {(1,600)(2,0)(3,0)(4,3257.252)};
\addplot[draw=red]
coordinates {(1,0)(2,0)(3,0)(4,3640.675)};
\end{axis}
\end{tikzpicture}
}
\qquad
\subfloat[P2: \$1,200,000 Initial Budget]{
\begin{tikzpicture}
\begin{axis}[
width=7.0cm,
ylabel={\tiny{Amount to Spend (\$1,000s)}},
xlabel={\tiny{Quarter}},
xmin=1, xmax=4,
ymin=-50, ymax=6000,
xtick={1,2,3,4},
ytick={0,1000,2000,3000,4000,5000,6000},
legend pos=north west,
ymajorgrids=true,
grid style=dashed,
]
\addplot[draw=gr7]
coordinates {(1,0)(2,0)(3,200)(4,5261.013)};
\addplot[draw=gr3]
coordinates {(1,0)(2,0)(3,400)(4,5061.013)};
\addplot[draw=gr8]
coordinates {(1,0)(2,200)(3,0)(4,5305.973)};
\addplot[draw=gr4]
coordinates {(1,0)(2,200)(3,200)(4,5105.973)};
\addplot[draw=gr1]
coordinates {(1,0)(2,200)(3,400)(4,4905.973)};
\addplot[draw=gr5]
coordinates {(1,0)(2,400)(3,0)(4,5150.933)};
\addplot[draw=gr2]
coordinates {(1,0)(2,400)(3,200)(4,4950.933)};
\addplot[draw=gr2]
coordinates {(1,0)(2,600)(3,0)(4,4995.893)};
\addplot[draw=gr9]
coordinates {(1,200)(2,0)(3,0)(4,5333.206)};
\addplot[draw=gr5]
coordinates {(1,200)(2,0)(3,200)(4,5133.206)};
\addplot[draw=gr1]
coordinates {(1,200)(2,0)(3,400)(4,4933.206)};
\addplot[draw=gr6]
coordinates {(1,200)(2,200)(3,0)(4,5178.166)};
\addplot[draw=gr2]
coordinates {(1,200)(2,200)(3,200)(4,4978.166)};
\addplot[draw=gr3]
coordinates {(1,200)(2,400)(3,0)(4,5023.126)};
\addplot[draw=gr6]
coordinates {(1,400)(2,0)(3,0)(4,5205.398)};
\addplot[draw=gr3]
coordinates {(1,400)(2,0)(3,200)(4,5005.398)};
\addplot[draw=gr3]
coordinates {(1,400)(2,200)(3,0)(4,5050.358)};
\addplot[draw=gr1]
coordinates {(1,400)(2,400)(3,0)(4,4895.318)};
\addplot[draw=gr4]
coordinates {(1,600)(2,0)(3,0)(4,5077.59)};
\addplot[draw=gr2]
coordinates {(1,600)(2,200)(3,0)(4,4922.55)};
\addplot[draw=gr2]
coordinates {(1,800)(2,0)(3,0)(4,4949.783)};
\addplot[draw=blue]
coordinates {(1,0)(2,0)(3,0)(4,5461.013)};
\end{axis}
\end{tikzpicture}
}
\caption{Overlaid strategies for P1 and P2. Colored line indicates top strategy. Darker gray lines indicate heavily used strategies, while lighter gray lines indicated lightly used strategies.}
\label{8v12.3}
\end{figure}

\begin{figure}
\tiny
\begin{tabular}{ |p{1.0cm}p{1.0cm}p{1.0cm}p{2.0cm}|p{1.0cm}||p{1.0cm}p{1.0cm}p{1.0cm}p{2.0cm}|p{1.0cm}|}
\hline
\multicolumn{10}{|c|}{P1 \$800,000 vs. P2 \$1,300,000} \\
\hline
Player 1 & (P1) & & & & Player 2 & (P2) & & & \\
\hline
Q1 Strategy & Q2 Strategy & Q3 Strategy & Q4 Strategy  &  Play \% & Q1 Strategy & Q2 Strategy & Q3 Strategy & Q4 Strategy  &  Play \%\\
\hline
\$0 & \$0 & \$0 & \$3,640,675 & 29.470\% & \$0 & \$0 & \$0 & \$5,916,098 & 8.306\% \\
\$200,000 & \$0 & \$0 & \$3,512,868 & 20.977\% & \$200,000 & \$0 & \$0 & \$5,788,290 & 7.385\% \\
\$0 & \$200,000 & \$0 & \$3,485,635 & 14.591\% & \$0 & \$200,000 & \$0 & \$5,761,058 & 6.886\% \\
\$0 & \$0 & \$200,000 & \$3,440,675 & 10.490\% & \$0 & \$0 & \$200,000 & \$5,716,098 & 6.375\% \\
\$400,000 & \$0 & \$0 & \$3,385,060 & 7.248\% & \$400,000 & \$0 & \$0 & \$5,660,483 & 5.740\% \\
\$200,000 & \$200,000 & \$0 & \$3,357,828 & 5.139\% & \$200,000 & \$200,000 & \$0 & \$5,633,250 & 5.217\% \\
\$0 & \$400,000 & \$0 & \$3,330,595 & 3.660\% & \$0 & \$400,000 & \$0 & \$5,606,018 & 4.863\% \\
\$200,000 & \$0 & \$200,000 & \$3,312,868 & 2.514\% & \$200,000 & \$0 & \$200,000 & \$5,588,290 & 4.501\% \\
\$0 & \$200,000 & \$200,000 & \$3,285,635 & 1.766\% & \$0 & \$200,000 & \$200,000 & \$5,561,058 & 4.181\% \\
\$600,000 & \$0 & \$0 & \$3,257,252 & 1.230\% & \$600,000 & \$0 & \$0 & \$5,532,675 & 3.752\% \\
&&&&&\$400,000 & \$200,000 & \$0 & \$5,505,443 & 3.539\% \\
&&&&&\$0 & \$0 & \$400,000 & \$5,516,098 & 3.230\% \\
&&&&&\$200,000 & \$400,000 & \$0 & \$5,478,210 & 2.927\% \\
&&&&&\$400,000 & \$0 & \$200,000 & \$5,460,483 & 2.702\% \\
&&&&&\$0 & \$600,000 & \$0 & \$5,450,978 & 2.481\% \\
&&&&&\$200,000 & \$200,000 & \$200,000 & \$5,433,250 & 2.290\% \\
&&&&&\$800,000 & \$0 & \$0 & \$5,404,867 & 2.065\% \\
&&&&&\$0 & \$400,000 & \$200,000 & \$5,406,018 & 1.933\% \\
&&&&&\$600,000 & \$200,000 & \$0 & \$5,377,635 & 1.824\% \\
&&&&&\$200,000 & \$0 & \$400,000 & \$5,388,290 & 1.612\% \\
&&&&&\$400,000 & \$400,000 & \$0 & \$5,350,403 & 1.450\% \\
&&&&&\$0 & \$200,000 & \$400,000 & \$5,361,058 & 1.410\% \\
&&&&&\$600,000 & \$0 & \$200,000 & \$5,332,675 & 1.311\% \\
&&&&&\$200,000 & \$600,000 & \$0 & \$5,323,170 & 1.128\% \\
&&&&&\$400,000 & \$200,000 & \$200,000 & \$5,305,443 & 1.071\% \\
\hline
Other &&&& 2.915\% & Other &&&& 11.821\% \\
\hline
\small \textbf{Game Value:} &&& \small \textbf{-30.04} && \small \textbf{Winner:} &&& \small \textbf{P2}&\\
\hline
\end{tabular}
\caption{Mixed strategy output, when P1 is given an initial budget of \$800,000, P2 is given an initial budget of \$1,300,000, and the Weighted by Sample, Floored Utility Vector is used.}
\label{8v13table.3}
\end{figure}

\begin{figure}
\centering
\captionsetup{justification=raggedright}
\subfloat[P1: \$800,000 Initial Budget]{
\begin{tikzpicture}
\begin{axis}[
width=7.0cm,
ylabel={\tiny{Amount to Spend (\$1,000s)}},
xlabel={\tiny{Quarter}},
xmin=1, xmax=4,
ymin=-50, ymax=6000,
xtick={1,2,3,4},
ytick={0,1000,2000,3000,4000,5000,6000},
legend pos=north west,
ymajorgrids=true,
grid style=dashed,
]
\addplot[draw=gr10]
coordinates {(1,0)(2,0)(3,200)(4,3440.675)};
\addplot[draw=gr15]
coordinates {(1,0)(2,200)(3,0)(4,3485.635)};
\addplot[draw=gr2]
coordinates {(1,0)(2,200)(3,200)(4,3285.635)};
\addplot[draw=gr4]
coordinates {(1,0)(2,400)(3,0)(4,3330.595)};
\addplot[draw=gr21]
coordinates {(1,200)(2,0)(3,0)(4,3512.868)};
\addplot[draw=gr3]
coordinates {(1,200)(2,0)(3,200)(4,3312.868)};
\addplot[draw=gr5]
coordinates {(1,200)(2,200)(3,0)(4,3357.828)};
\addplot[draw=gr7]
coordinates {(1,400)(2,0)(3,0)(4,3385.06)};
\addplot[draw=gr1]
coordinates {(1,600)(2,0)(3,0)(4,3257.252)};
\addplot[draw=red]
coordinates {(1,0)(2,0)(3,0)(4,3640.675)};
\end{axis}
\end{tikzpicture}
}
\qquad
\subfloat[P2: \$1,300,000 Initial Budget]{
\begin{tikzpicture}
\begin{axis}[
width=7.0cm,
ylabel={\tiny{Amount to Spend (\$1,000s)}},
xlabel={\tiny{Quarter}},
xmin=1, xmax=4,
ymin=-50, ymax=6000,
xtick={1,2,3,4},
ytick={0,1000,2000,3000,4000,5000,6000},
legend pos=north west,
ymajorgrids=true,
grid style=dashed,
]
\addplot[draw=gr6]
coordinates {(1,0)(2,0)(3,200)(4,5716.098)};
\addplot[draw=gr3]
coordinates {(1,0)(2,0)(3,400)(4,5516.098)};
\addplot[draw=gr7]
coordinates {(1,0)(2,200)(3,0)(4,5761.058)};
\addplot[draw=gr4]
coordinates {(1,0)(2,200)(3,200)(4,5561.058)};
\addplot[draw=gr1]
coordinates {(1,0)(2,200)(3,400)(4,5361.058)};
\addplot[draw=gr5]
coordinates {(1,0)(2,400)(3,0)(4,5606.018)};
\addplot[draw=gr2]
coordinates {(1,0)(2,400)(3,200)(4,5406.018)};
\addplot[draw=gr2]
coordinates {(1,0)(2,600)(3,0)(4,5450.978)};
\addplot[draw=gr7]
coordinates {(1,200)(2,0)(3,0)(4,5788.29)};
\addplot[draw=gr5]
coordinates {(1,200)(2,0)(3,200)(4,5588.29)};
\addplot[draw=gr2]
coordinates {(1,200)(2,0)(3,400)(4,5388.29)};
\addplot[draw=gr5]
coordinates {(1,200)(2,200)(3,0)(4,5633.25)};
\addplot[draw=gr2]
coordinates {(1,200)(2,200)(3,200)(4,5433.25)};
\addplot[draw=gr3]
coordinates {(1,200)(2,400)(3,0)(4,5478.21)};
\addplot[draw=gr1]
coordinates {(1,200)(2,600)(3,0)(4,5323.17)};
\addplot[draw=gr6]
coordinates {(1,400)(2,0)(3,0)(4,5660.483)};
\addplot[draw=gr3]
coordinates {(1,400)(2,0)(3,200)(4,5460.483)};
\addplot[draw=gr4]
coordinates {(1,400)(2,200)(3,0)(4,5505.443)};
\addplot[draw=gr1]
coordinates {(1,400)(2,200)(3,200)(4,5305.443)};
\addplot[draw=gr1]
coordinates {(1,400)(2,400)(3,0)(4,5350.403)};
\addplot[draw=gr4]
coordinates {(1,600)(2,0)(3,0)(4,5532.675)};
\addplot[draw=gr1]
coordinates {(1,600)(2,0)(3,200)(4,5332.675)};
\addplot[draw=gr2]
coordinates {(1,600)(2,200)(3,0)(4,5377.635)};
\addplot[draw=gr2]
coordinates {(1,800)(2,0)(3,0)(4,5404.867)};
\addplot[draw=blue]
coordinates {(1,0)(2,0)(3,0)(4,5916.098)};
\end{axis}
\end{tikzpicture}
}
\caption{Overlaid strategies for P1 and P2. Colored line indicates top strategy. Darker gray lines indicate heavily used strategies, while lighter gray lines indicated lightly used strategies.}
\label{8v13.3}
\end{figure}

\begin{figure}
\tiny
\begin{tabular}{ |p{1.0cm}p{1.0cm}p{1.0cm}p{2.0cm}|p{1.0cm}||p{1.0cm}p{1.0cm}p{1.0cm}p{2.0cm}|p{1.0cm}|}
\hline
\multicolumn{10}{|c|}{P1 \$800,000 vs. P2 \$1,400,000} \\
\hline
Player 1 & (P1) & & & & Player 2 & (P2) & & & \\
\hline
Q1 Strategy & Q2 Strategy & Q3 Strategy & Q4 Strategy  &  Play \% & Q1 Strategy & Q2 Strategy & Q3 Strategy & Q4 Strategy  &  Play \%\\
\hline
\$0 & \$0 & \$0 & \$3,640,675 & 29.352\% & \$0 & \$0 & \$0 & \$6,371,182 & 6.343\% \\
\$200,000 & \$0 & \$0 & \$3,512,868 & 21.08\% & \$200,000 & \$0 & \$0 & \$6,243,375 & 5.891\% \\
\$0 & \$200,000 & \$0 & \$3,485,635 & 14.754\% & \$0 & \$200,000 & \$0 & \$6,216,142 & 5.58\% \\
\$0 & \$0 & \$200,000 & \$3,440,675 & 10.140\% & \$0 & \$0 & \$200,000 & \$6,171,182 & 5.178\% \\
\$400,000 & \$0 & \$0 & \$3,385,060 & 7.270\% & \$400,000 & \$0 & \$0 & \$6,115,567 & 4.969\% \\
\$200,000 & \$200,000 & \$0 & \$3,357,828 & 5.191\% & \$200,000 & \$200,000 & \$0 & \$6,088,335 & 4.701\% \\
\$0 & \$400,000 & \$0 & \$3,330,595 & 3.588\% & \$0 & \$400,000 & \$0 & \$6,061,102 & 4.272\% \\
\$200,000 & \$0 & \$200,000 & \$3,312,868 & 2.649\% & \$200,000 & \$0 & \$200,000 & \$6,043,375 & 4.069\% \\
\$0 & \$200,000 & \$200,000 & \$3,285,635 & 1.816\% & \$0 & \$200,000 & \$200,000 & \$6,016,142 & 3.667\% \\
\$600,000 & \$0 & \$0 & \$3,257,252 & 1.244\% & \$600,000 & \$0 & \$0 & \$5,987,759 & 3.524\% \\
&&&&&\$400,000 & \$200,000 & \$0 & \$5,960,527 & 3.329\% \\
&&&&&\$0 & \$0 & \$400,000 & \$5,971,182 & 3.113\% \\
&&&&&\$200,000 & \$400,000 & \$0 & \$5,933,295 & 2.946\% \\
&&&&&\$400,000 & \$0 & \$200,000 & \$5,915,567 & 2.718\% \\
&&&&&\$0 & \$600,000 & \$0 & \$5,906,062 & 2.535\% \\
&&&&&\$200,000 & \$200,000 & \$200,000 & \$5,888,335 & 2.446\% \\
&&&&&\$800,000 & \$0 & \$0 & \$5,859,952 & 2.161\% \\
&&&&&\$0 & \$400,000 & \$200,000 & \$5,861,102 & 2.151\% \\
&&&&&\$600,000 & \$200,000 & \$0 & \$5,832,719 & 1.891\% \\
&&&&&\$200,000 & \$0 & \$400,000 & \$5,843,375 & 1.758\% \\
&&&&&\$400,000 & \$400,000 & \$0 & \$5,805,487 & 1.711\% \\
&&&&&\$0 & \$200,000 & \$400,000 & \$5,816,142 & 1.567\% \\
&&&&&\$600,000 & \$0 & \$200,000 & \$5,787,759 & 1.501\% \\
&&&&&\$200,000 & \$600,000 & \$0 & \$5,778,255 & 1.416\% \\
&&&&&\$400,000 & \$200,000 & \$200,000 & \$5,760,527 & 1.308\% \\
&&&&&\$1,000,000 & \$0 & \$0 & \$5,732,144 & 1.184\% \\
&&&&&\$0 & \$0 & \$600,000 & \$5,771,182 & 1.178\% \\
&&&&&\$0 & \$800,000 & \$0 & \$5,751,022 & 1.062\% \\
&&&&&\$200,000 & \$400,000 & \$200,000 & \$5,733,295 & 1.017\% \\
\hline
Other &&&& 2.916\% & Other &&&& 14.814\% \\
\hline
\small \textbf{Game Value:} &&& \small \textbf{-36.03} && \small \textbf{Winner:} &&& \small \textbf{P2}&\\
\hline
\end{tabular}
\caption{Mixed strategy output, when P1 is given an initial budget of \$800,000, P2 is given an initial budget of \$1,400,000, and the Weighted by Sample, Floored Utility Vector is used.}
\label{8v14table.3}
\end{figure}

\begin{figure}
\centering
\captionsetup{justification=raggedright}
\subfloat[P1: \$800,000 Initial Budget]{
\begin{tikzpicture}
\begin{axis}[
width=7.0cm,
ylabel={\tiny{Amount to Spend (\$1,000s)}},
xlabel={\tiny{Quarter}},
xmin=1, xmax=4,
ymin=-50, ymax=7000,
xtick={1,2,3,4},
ytick={0,1000,2000,3000,4000,5000,6000,7000},
legend pos=north west,
ymajorgrids=true,
grid style=dashed,
]
\addplot[draw=gr10]
coordinates {(1,0)(2,0)(3,200)(4,3440.675)};
\addplot[draw=gr15]
coordinates {(1,0)(2,200)(3,0)(4,3485.635)};
\addplot[draw=gr2]
coordinates {(1,0)(2,200)(3,200)(4,3285.635)};
\addplot[draw=gr4]
coordinates {(1,0)(2,400)(3,0)(4,3330.595)};
\addplot[draw=gr21]
coordinates {(1,200)(2,0)(3,0)(4,3512.868)};
\addplot[draw=gr3]
coordinates {(1,200)(2,0)(3,200)(4,3312.868)};
\addplot[draw=gr5]
coordinates {(1,200)(2,200)(3,0)(4,3357.828)};
\addplot[draw=gr7]
coordinates {(1,400)(2,0)(3,0)(4,3385.06)};
\addplot[draw=gr1]
coordinates {(1,600)(2,0)(3,0)(4,3257.252)};
\addplot[draw=red]
coordinates {(1,0)(2,0)(3,0)(4,3640.675)};
\end{axis}
\end{tikzpicture}
}
\qquad
\subfloat[P2: \$1,400,000 Initial Budget]{
\begin{tikzpicture}
\begin{axis}[
width=7.0cm,
ylabel={\tiny{Amount to Spend (\$1,000s)}},
xlabel={\tiny{Quarter}},
xmin=1, xmax=4,
ymin=-50, ymax=7000,
xtick={1,2,3,4},
ytick={0,1000,2000,3000,4000,5000,6000,7000},
legend pos=north west,
ymajorgrids=true,
grid style=dashed,
]
\addplot[draw=gr5]
coordinates {(1,0)(2,0)(3,200)(4,6171.182)};
\addplot[draw=gr3]
coordinates {(1,0)(2,0)(3,400)(4,5971.182)};
\addplot[draw=gr1]
coordinates {(1,0)(2,0)(3,600)(4,5771.182)};
\addplot[draw=gr6]
coordinates {(1,0)(2,200)(3,0)(4,6216.142)};
\addplot[draw=gr4]
coordinates {(1,0)(2,200)(3,200)(4,6016.142)};
\addplot[draw=gr2]
coordinates {(1,0)(2,200)(3,400)(4,5816.142)};
\addplot[draw=gr4]
coordinates {(1,0)(2,400)(3,0)(4,6061.102)};
\addplot[draw=gr2]
coordinates {(1,0)(2,400)(3,200)(4,5861.102)};
\addplot[draw=gr3]
coordinates {(1,0)(2,600)(3,0)(4,5906.062)};
\addplot[draw=gr1]
coordinates {(1,0)(2,800)(3,0)(4,5751.022)};
\addplot[draw=gr6]
coordinates {(1,200)(2,0)(3,0)(4,6243.375)};
\addplot[draw=gr4]
coordinates {(1,200)(2,0)(3,200)(4,6043.375)};
\addplot[draw=gr2]
coordinates {(1,200)(2,0)(3,400)(4,5843.375)};
\addplot[draw=gr5]
coordinates {(1,200)(2,200)(3,0)(4,6088.335)};
\addplot[draw=gr2]
coordinates {(1,200)(2,200)(3,200)(4,5888.335)};
\addplot[draw=gr3]
coordinates {(1,200)(2,400)(3,0)(4,5933.295)};
\addplot[draw=gr1]
coordinates {(1,200)(2,400)(3,200)(4,5733.295)};
\addplot[draw=gr1]
coordinates {(1,200)(2,600)(3,0)(4,5778.255)};
\addplot[draw=gr5]
coordinates {(1,400)(2,0)(3,0)(4,6115.567)};
\addplot[draw=gr3]
coordinates {(1,400)(2,0)(3,200)(4,5915.567)};
\addplot[draw=gr3]
coordinates {(1,400)(2,200)(3,0)(4,5960.527)};
\addplot[draw=gr1]
coordinates {(1,400)(2,200)(3,200)(4,5760.527)};
\addplot[draw=gr2]
coordinates {(1,400)(2,400)(3,0)(4,5805.487)};
\addplot[draw=gr4]
coordinates {(1,600)(2,0)(3,0)(4,5987.759)};
\addplot[draw=gr2]
coordinates {(1,600)(2,0)(3,200)(4,5787.759)};
\addplot[draw=gr2]
coordinates {(1,600)(2,200)(3,0)(4,5832.719)};
\addplot[draw=gr2]
coordinates {(1,800)(2,0)(3,0)(4,5859.952)};
\addplot[draw=gr1]
coordinates {(1,1000)(2,0)(3,0)(4,5732.144)};
\addplot[draw=blue]
coordinates {(1,0)(2,0)(3,0)(4,6371.182)};
\end{axis}
\end{tikzpicture}
}
\caption{Overlaid strategies for P1 and P2. Colored line indicates top strategy. Darker gray lines indicate heavily used strategies, while lighter gray lines indicated lightly used strategies.}
\label{8v14.3}
\end{figure}

\begin{figure}
\tiny
\begin{tabular}{ |p{1.0cm}p{1.0cm}p{1.0cm}p{2.0cm}|p{1.0cm}||p{1.0cm}p{1.0cm}p{1.0cm}p{2.0cm}|p{1.0cm}|}
\hline
\multicolumn{10}{|c|}{P1 \$800,000 vs. P2 \$1,600,000} \\
\hline
Player 1 & (P1) & & & & Player 2 & (P2) & & & \\
\hline
Q1 Strategy & Q2 Strategy & Q3 Strategy & Q4 Strategy  &  Play \% & Q1 Strategy & Q2 Strategy & Q3 Strategy & Q4 Strategy  &  Play \%\\
\hline
\$0 & \$0 & \$0 & \$3,640,675 & 29.405\% & \$0 & \$0 & \$0 & \$7,281,351 & 4.386\% \\
\$200,000 & \$0 & \$0 & \$3,512,868 & 20.81\% & \$200,000 & \$0 & \$0 & \$7,153,544 & 4.176\% \\
\$0 & \$200,000 & \$0 & \$3,485,635 & 14.639\% & \$0 & \$200,000 & \$0 & \$7,126,311 & 4.085\% \\
\$0 & \$0 & \$200,000 & \$3,440,675 & 10.369\% & \$0 & \$0 & \$200,000 & \$7,081,351 & 3.793\% \\
\$400,000 & \$0 & \$0 & \$3,385,060 & 7.400\% & \$400,000 & \$0 & \$0 & \$7,025,736 & 3.562\% \\
\$200,000 & \$200,000 & \$0 & \$3,357,828 & 5.174\% & \$200,000 & \$200,000 & \$0 & \$6,998,504 & 3.493\% \\
\$0 & \$400,000 & \$0 & \$3,330,595 & 3.638\% & \$0 & \$400,000 & \$0 & \$6,971,271 & 3.279\% \\
\$200,000 & \$0 & \$200,000 & \$3,312,868 & 2.577\% & \$200,000 & \$0 & \$200,000 & \$6,953,544 & 3.274\% \\
\$0 & \$200,000 & \$200,000 & \$3,285,635 & 1.804\% & \$0 & \$200,000 & \$200,000 & \$6,926,311 & 3.026\% \\
\$600,000 & \$0 & \$0 & \$3,257,252 & 1.232\% & \$600,000 & \$0 & \$0 & \$6,897,928 & 2.999\% \\
&&&&&\$400,000 & \$200,000 & \$0 & \$6,870,696 & 2.845\% \\
&&&&&\$0 & \$0 & \$400,000 & \$6,881,351 & 2.649\% \\
&&&&&\$200,000 & \$400,000 & \$0 & \$6,843,464 & 2.569\% \\
&&&&&\$400,000 & \$0 & \$200,000 & \$6,825,736 & 2.503\% \\
&&&&&\$0 & \$600,000 & \$0 & \$6,816,231 & 2.406\% \\
&&&&&\$200,000 & \$200,000 & \$200,000 & \$6,798,504 & 2.153\% \\
&&&&&\$800,000 & \$0 & \$0 & \$6,770,121 & 2.083\% \\
&&&&&\$0 & \$400,000 & \$200,000 & \$6,771,271 & 2.036\% \\
&&&&&\$600,000 & \$200,000 & \$0 & \$6,742,888 & 2.000\% \\
&&&&&\$200,000 & \$0 & \$400,000 & \$6,753,544 & 1.873\% \\
&&&&&\$400,000 & \$400,000 & \$0 & \$6,715,656 & 1.774\% \\
&&&&&\$0 & \$200,000 & \$400,000 & \$6,726,311 & 1.736\% \\
&&&&&\$600,000 & \$0 & \$200,000 & \$6,697,928 & 1.645\% \\
&&&&&\$200,000 & \$600,000 & \$0 & \$6,688,424 & 1.557\% \\
&&&&&\$0 & \$0 & \$600,000 & \$6,681,351 & 1.487\% \\
&&&&&\$400,000 & \$200,000 & \$200,000 & \$6,670,696 & 1.483\% \\
&&&&&\$1,000,000 & \$0 & \$0 & \$6,642,313 & 1.379\% \\
&&&&&\$0 & \$800,000 & \$0 & \$6,661,191 & 1.297\% \\
&&&&&\$200,000 & \$400,000 & \$200,000 & \$6,643,464 & 1.259\% \\
&&&&&\$800,000 & \$200,000 & \$0 & \$6,615,081 & 1.227\% \\
&&&&&\$400,000 & \$0 & \$400,000 & \$6,625,736 & 1.198\% \\
&&&&&\$600,000 & \$400,000 & \$0 & \$6,587,848 & 1.062\% \\
&&&&&\$0 & \$600,000 & \$200,000 & \$6,616,231 & 1.06\% \\
\hline
Other &&&& 2.952\% & Other &&&& 22.646\% \\
\hline
\small \textbf{Game Value:} &&& \small \textbf{-48.36} && \small \textbf{Winner:} &&& \small \textbf{P2}&\\
\hline
\end{tabular}
\caption{Mixed strategy output, when P1 is given an initial budget of \$800,000, P2 is given an initial budget of \$1,600,000, and the Weighted by Sample, Floored Utility Vector is used.}
\label{8v16table.3}
\end{figure}

\begin{figure}
\centering
\captionsetup{justification=raggedright}
\subfloat[P1: \$800,000 Initial Budget]{
\begin{tikzpicture}
\begin{axis}[
width=7.0cm,
ylabel={\tiny{Amount to Spend (\$1,000s)}},
xlabel={\tiny{Quarter}},
xmin=1, xmax=4,
ymin=-50, ymax=8000,
xtick={1,2,3,4},
ytick={0,1000,2000,3000,4000,5000,6000,7000,8000},
legend pos=north west,
ymajorgrids=true,
grid style=dashed,
]
\addplot[draw=gr10]
coordinates {(1,0)(2,0)(3,200)(4,3440.675)};
\addplot[draw=gr15]
coordinates {(1,0)(2,200)(3,0)(4,3485.635)};
\addplot[draw=gr2]
coordinates {(1,0)(2,200)(3,200)(4,3285.635)};
\addplot[draw=gr4]
coordinates {(1,0)(2,400)(3,0)(4,3330.595)};
\addplot[draw=gr21]
coordinates {(1,200)(2,0)(3,0)(4,3512.868)};
\addplot[draw=gr3]
coordinates {(1,200)(2,0)(3,200)(4,3312.868)};
\addplot[draw=gr5]
coordinates {(1,200)(2,200)(3,0)(4,3357.828)};
\addplot[draw=gr7]
coordinates {(1,400)(2,0)(3,0)(4,3385.06)};
\addplot[draw=gr1]
coordinates {(1,600)(2,0)(3,0)(4,3257.252)};
\addplot[draw=red]
coordinates {(1,0)(2,0)(3,0)(4,3640.675)};
\end{axis}
\end{tikzpicture}
}
\qquad
\subfloat[P2: \$1,600,000 Initial Budget]{
\begin{tikzpicture}
\begin{axis}[
width=7.0cm,
ylabel={\tiny{Amount to Spend (\$1,000s)}},
xlabel={\tiny{Quarter}},
xmin=1, xmax=4,
ymin=-50, ymax=8000,
xtick={1,2,3,4},
ytick={0,1000,2000,3000,4000,5000,6000,7000,8000},
legend pos=north west,
ymajorgrids=true,
grid style=dashed,
]
\addplot[draw=gr4]
coordinates {(1,0)(2,0)(3,200)(4,7081.351)};
\addplot[draw=gr3]
coordinates {(1,0)(2,0)(3,400)(4,6881.351)};
\addplot[draw=gr1]
coordinates {(1,0)(2,0)(3,600)(4,6681.351)};
\addplot[draw=gr4]
coordinates {(1,0)(2,200)(3,0)(4,7126.311)};
\addplot[draw=gr3]
coordinates {(1,0)(2,200)(3,200)(4,6926.311)};
\addplot[draw=gr2]
coordinates {(1,0)(2,200)(3,400)(4,6726.311)};
\addplot[draw=gr3]
coordinates {(1,0)(2,400)(3,0)(4,6971.271)};
\addplot[draw=gr2]
coordinates {(1,0)(2,400)(3,200)(4,6771.271)};
\addplot[draw=gr2]
coordinates {(1,0)(2,600)(3,0)(4,6816.231)};
\addplot[draw=gr1]
coordinates {(1,0)(2,600)(3,200)(4,6616.231)};
\addplot[draw=gr1]
coordinates {(1,0)(2,800)(3,0)(4,6661.191)};
\addplot[draw=gr4]
coordinates {(1,200)(2,0)(3,0)(4,7153.544)};
\addplot[draw=gr3]
coordinates {(1,200)(2,0)(3,200)(4,6953.544)};
\addplot[draw=gr2]
coordinates {(1,200)(2,0)(3,400)(4,6753.544)};
\addplot[draw=gr3]
coordinates {(1,200)(2,200)(3,0)(4,6998.504)};
\addplot[draw=gr2]
coordinates {(1,200)(2,200)(3,200)(4,6798.504)};
\addplot[draw=gr3]
coordinates {(1,200)(2,400)(3,0)(4,6843.464)};
\addplot[draw=gr1]
coordinates {(1,200)(2,400)(3,200)(4,6643.464)};
\addplot[draw=gr2]
coordinates {(1,200)(2,600)(3,0)(4,6688.424)};
\addplot[draw=gr4]
coordinates {(1,400)(2,0)(3,0)(4,7025.736)};
\addplot[draw=gr3]
coordinates {(1,400)(2,0)(3,200)(4,6825.736)};
\addplot[draw=gr1]
coordinates {(1,400)(2,0)(3,400)(4,6625.736)};
\addplot[draw=gr3]
coordinates {(1,400)(2,200)(3,0)(4,6870.696)};
\addplot[draw=gr1]
coordinates {(1,400)(2,200)(3,200)(4,6670.696)};
\addplot[draw=gr2]
coordinates {(1,400)(2,400)(3,0)(4,6715.656)};
\addplot[draw=gr3]
coordinates {(1,600)(2,0)(3,0)(4,6897.928)};
\addplot[draw=gr2]
coordinates {(1,600)(2,0)(3,200)(4,6697.928)};
\addplot[draw=gr2]
coordinates {(1,600)(2,200)(3,0)(4,6742.888)};
\addplot[draw=gr1]
coordinates {(1,600)(2,400)(3,0)(4,6587.848)};
\addplot[draw=gr2]
coordinates {(1,800)(2,0)(3,0)(4,6770.121)};
\addplot[draw=gr1]
coordinates {(1,800)(2,200)(3,0)(4,6615.081)};
\addplot[draw=gr1]
coordinates {(1,1000)(2,0)(3,0)(4,6642.313)};
\addplot[draw=blue]
coordinates {(1,0)(2,0)(3,0)(4,7281.351)};
\end{axis}
\end{tikzpicture}
}
\caption{Overlaid strategies for P1 and P2. Colored line indicates top strategy. Darker gray lines indicate heavily used strategies, while lighter gray lines indicated lightly used strategies.}
\label{8v16.3}
\end{figure}

\newpage
\subsection{Unweighted, Unfloored Utility Vector}
Until this point, only floored utility vectors have been examined. Floored vectors are likely the most realistic, since it would go against much election literature and common practice if spending money did not help with polls \cite{fundresearch1}\cite{jackie}\cite{max}. However, it is possible that spending money actually hurts a campaign. This could be caused by voters getting tired of hearing a candidate's name, or by scrutiny of policy issues that accompanies name recognition. In all likelihood, the negative numbers are almost definitely the direct result of a poorly-constructed utility function. As Section 7: \textbf{Critiques} will outline, a more accurate utility vector is definitely possible, but was not explored in this report.

The Unweighted, Unfloored Utility Vector is the first of these vectors. It takes the form
\begin{equation}
[ 6.30872096, 2.549945091, 0.401012477, -1.653903518 ]\cdot 2\times 10^{-5}.
\end{equation}
Note that this utility vector rewards players for spending at the beginning of the campaign, but these rewards get progressively worse until finally a player is punished for spending in the Q4. What makes this utility vector epecially interesting, though, is the way it interacts with the fundraising return function and the rules of the game. Recall that the fundraising return function takes the form 
\begin{equation*}
f_i(f_{i-1}, s_i) = 0.1124s_i + 1.1057f_{i-1},
\end{equation*}
where $f_k$ denotes the amount a candidate fundraised in quarter $k$ and $s_k$ denotes the amount a  candidate spent in quarter $k$. This yields higher and higher fundraising returns each quarter, and gives an extra boost for previous spending. Now, recall also that a player \textit{must} spend their entire remaining budget in Q4. This rule makes more sense when working with floored vectors; however, to lift it would drastically increase the run time of the code, since it would require generating strategies for every spending option for Q4, rather than tacking on a single calculated number to each strategy (see Section 4.3: \textbf{Generating Strategies}).

While this rule is unrealistic, it generates some exciting and unexpected optimal strategy sets. Regardless of budget, every game seems to reward an s-curve-like top strategy, where the difference in spending between Q1 and Q2 is notably less than that between Q2 and Q3. This shape is likely generated because players are incentivized to spend their entire remaining budget in Q3, since spending in Q4 is punished. While this does give players a higher fundraising return in Q4, the amount they gain in fundraising from spending this amount in Q3 is less than the amount they would retain by not spending it. 

Another interesting aspect of these graphs is the tendency not to spend as strongly early in the game; specifically, in Q2. However, note that the fundraising return is a function of both the past quarter of spending \textit{and} the past quarter of fundraising. Rather than being written recursively, the fundraising function can be written as 
\begin{equation}
f(q_k) = \begin{cases}
0 & k=1 \\
0.1124s_1 + 1.1057y_0 & k=2 \\
0.1124s_2 + 1.1057(0.1124s_1 + 1.1057y_0) & k=3 \\
0.1124s_3 + 1.1057(0.1124s_2 + 1.1057(0.1124s_1 + 1.1057y_0)) & k=4
\end{cases}.
\end{equation}
Simplified,
\begin{equation}
\label{expandedfunds}
f(q_k) = \begin{cases}
0 & k=1 \\
0.1124s_1 + 1.1057y_0 & k=2 \\
0.1124s_2 + 0.1243s_1 + 1.2223y_0 & k=3 \\
0.1124s_3 + 1.1057s_2 + 0.1374s_1 + 1.3518y_0 & k=4
\end{cases},
\end{equation}
where $f(q_k)$ denotes the amount a player fundraises immediately before quarter $k$, $s_k$ denotes the amount a player spends in quarter $k$, and $y_0$ indicates a player's initial budget. This results in a lower spend amount in Q2. However, in Q1, this incentive not to increase later fundraising amounts is outweighed heavily by the high utility of spending that quarter. 

In every budget pairing, another highly used strategy is spending a relatively even amount each quarter. The benefit of this strategy is keeping the fundraising return low relative to other strategies, since spending a lot in a single quarter creates an abnormally high coefficient for that $s$ value in (\ref{expandedfunds}). In lesser-used strategies, however, spending amounts become more and more variable in Q2 and Q3. The purpose of this is likely to create large differentials in the voter utility function to maximize voters won in higher-spending months, while minimizing the amount fundraises --- and therefore must spend --- for Q4. These graphs are especially interesting because lesser-used strategies for each quarter are not bounded by the top strategy, but are rather both higher and lower than it, depending on the strategy.

This vector only yields wins for P2. This is more significant than previous vectors, however, because this result is no longer trivial. Since the vector punishes players for spending in the last quarter, it's feasible that P1 could win different simulations than in previous games. For this reason, players are highly discouraged from saving their entire budget for the last quarter, which in previous games has been a very popular dominant strategy.

\begin{figure}
\tiny
\begin{tabular}{ |p{1.0cm}p{1.0cm}p{1.0cm}p{2.0cm}|p{1.0cm}||p{1.0cm}p{1.0cm}p{1.0cm}p{2.0cm}|p{1.0cm}|}
\hline
\multicolumn{10}{|c|}{P1 \$800,000 vs. P2 \$800,000} \\
\hline
Player 1 & (P1) & & & & Player 2 & (P2) & & & \\
\hline
Q1 Strategy & Q2 Strategy & Q3 Strategy & Q4 Strategy  &  Play \% & Q1 Strategy & Q2 Strategy & Q3 Strategy & Q4 Strategy  &  Play \%\\
\hline
\$800,000 & \$600,000 & \$1,400,000 & \$1,264,325 & 29.263\% & \$800,000 & \$600,000 & \$1,400,000 & \$1,264,325 & 29.368\% \\
\$800,000 & \$800,000 & \$1,000,000 & \$1,509,285 & 20.772\% & \$800,000 & \$800,000 & \$1,000,000 & \$1,509,285 & 20.676\% \\
\$600,000 & \$1,000,000 & \$1,200,000 & \$1,282,052 & 14.836\% & \$600,000 & \$1,000,000 & \$1,200,000 & \$1,282,052 & 14.793\% \\
\$800,000 & \$400,000 & \$1,600,000 & \$1,219,365 & 10.407\% & \$800,000 & \$400,000 & \$1,600,000 & \$1,219,365 & 10.531\% \\
\$800,000 & \$600,000 & \$1,200,000 & \$1,464,325 & 7.458\% & \$800,000 & \$600,000 & \$1,200,000 & \$1,464,325 & 7.267\% \\
\$800,000 & \$800,000 & \$800,000 & \$1,709,285 & 5.206\% & \$800,000 & \$800,000 & \$800,000 & \$1,709,285 & 5.199\% \\
\$600,000 & \$800,000 & \$1,400,000 & \$1,237,092 & 3.523\% & \$600,000 & \$800,000 & \$1,400,000 & \$1,237,092 & 3.633\% \\
\$800,000 & \$200,000 & \$1,800,000 & \$1,174,405 & 2.496\% & \$800,000 & \$200,000 & \$1,800,000 & \$1,174,405 & 2.608\% \\
\$600,000 & \$1,000,000 & \$1,000,000 & \$1,482,052 & 1.832\% & \$600,000 & \$1,000,000 & \$1,000,000 & \$1,482,052 & 1.736\% \\
\$800,000 & \$400,000 & \$1,400,000 & \$1,419,365 & 1.232\% & \$800,000 & \$400,000 & \$1,400,000 & \$1,419,365 & 1.270\% \\
\hline
Other& & & & 2.975\% & Other& & & & 2.919\% \\
\hline
\small \textbf{Game Value:} &&& \small \textbf{0.00} && \small \textbf{Winner:} &&& \small \textbf{P1}&\\
\hline
\end{tabular}
\caption{Mixed strategy output, when P1 is given an initial budget of \$800,000, P2 is given an initial budget of \$800,000, and the Unweighted, Unfloored Utility Vector is used.}
\label{8v8table.4}
\end{figure}

\begin{figure}
\centering
\captionsetup{justification=raggedright}
\subfloat[P1: \$800,000 Initial Budget]{
\begin{tikzpicture}
\begin{axis}[
width=7.0cm,
ylabel={\tiny{Amount to Spend (\$1,000s)}},
xlabel={\tiny{Quarter}},
xmin=1, xmax=4,
ymin=-50, ymax=3000,
xtick={1,2,3,4},
ytick={0,1000,2000,3000},
legend pos=north west,
ymajorgrids=true,
grid style=dashed,
]
\addplot[draw=gr1]
coordinates {(1,800)(2,400)(3,1400)(4,1419.365)};
\addplot[draw=gr2]
coordinates {(1,600)(2,1000)(3,1000)(4,1482.052)};
\addplot[draw=gr2]
coordinates {(1,800)(2,200)(3,1800)(4,1174.405)};
\addplot[draw=gr4]
coordinates {(1,600)(2,800)(3,1400)(4,1237.092)};
\addplot[draw=gr5]
coordinates {(1,800)(2,800)(3,800)(4,1709.285)};
\addplot[draw=gr7]
coordinates {(1,800)(2,600)(3,1200)(4,1464.325)};
\addplot[draw=gr10]
coordinates {(1,800)(2,400)(3,1600)(4,1219.365)};
\addplot[draw=gr15]
coordinates {(1,600)(2,1000)(3,1200)(4,1282.052)};
\addplot[draw=gr21]
coordinates {(1,800)(2,800)(3,1000)(4,1509.285)};
\addplot[draw=red]
coordinates {(1,800)(2,600)(3,1400)(4,1264.325)};
\end{axis}
\end{tikzpicture}
}
\qquad
\subfloat[P2: \$800,000 Initial Budget]{
\begin{tikzpicture}
\begin{axis}[
width=7.0cm,
ylabel={\tiny{Amount to Spend (\$1,000s)}},
xlabel={\tiny{Quarter}},
xmin=1, xmax=4,
ymin=-50, ymax=3000,
xtick={1,2,3,4},
ytick={0,1000,2000,3000},
legend pos=north west,
ymajorgrids=true,
grid style=dashed,
]
\addplot[draw=gr1]
coordinates {(1,800)(2,400)(3,1400)(4,1419.365)};
\addplot[draw=gr2]
coordinates {(1,600)(2,1000)(3,1000)(4,1482.052)};
\addplot[draw=gr3]
coordinates {(1,800)(2,200)(3,1800)(4,1174.405)};
\addplot[draw=gr4]
coordinates {(1,600)(2,800)(3,1400)(4,1237.092)};
\addplot[draw=gr5]
coordinates {(1,800)(2,800)(3,800)(4,1709.285)};
\addplot[draw=gr7]
coordinates {(1,800)(2,600)(3,1200)(4,1464.325)};
\addplot[draw=gr11]
coordinates {(1,800)(2,400)(3,1600)(4,1219.365)};
\addplot[draw=gr15]
coordinates {(1,600)(2,1000)(3,1200)(4,1282.052)};
\addplot[draw=gr21]
coordinates {(1,800)(2,800)(3,1000)(4,1509.285)};
\addplot[draw=blue]
coordinates {(1,800)(2,600)(3,1400)(4,1264.325)};
\end{axis}
\end{tikzpicture}
}
\caption{Overlaid strategies for P1 and P2. Colored line indicates top strategy. Darker gray lines indicate heavily used strategies, while lighter gray lines indicated lightly used strategies.}
\label{8v8.4}
\end{figure}

\begin{figure}
\tiny
\begin{tabular}{ |p{1.0cm}p{1.0cm}p{1.0cm}p{2.0cm}|p{1.0cm}||p{1.0cm}p{1.0cm}p{1.0cm}p{2.0cm}|p{1.0cm}|}
\hline
\multicolumn{10}{|c|}{P1 \$800,000 vs. P2 \$900,000} \\
\hline
Player 1 & (P1) & & & & Player 2 & (P2) & & & \\
\hline
Q1 Strategy & Q2 Strategy & Q3 Strategy & Q4 Strategy  &  Play \% & Q1 Strategy & Q2 Strategy & Q3 Strategy & Q4 Strategy  &  Play \%\\
\hline
\$800,000 & \$600,000 & \$1,400,000 & \$1,264,325 & 29.587\% & \$800,000 & \$800,000 & \$1,600,000 & \$1,364,369 & 22.952\% \\
\$800,000 & \$800,000 & \$1,000,000 & \$1,509,285 & 20.626\% & \$800,000 & \$1,000,000 & \$1,200,000 & \$1,609,329 & 17.574\% \\
\$600,000 & \$1,000,000 & \$1,200,000 & \$1,282,052 & 14.726\% & \$600,000 & \$1,200,000 & \$1,400,000 & \$1,382,097 & 13.698\% \\
\$800,000 & \$400,000 & \$1,600,000 & \$1,219,365 & 10.363\% & \$800,000 & \$600,000 & \$1,800,000 & \$1,319,409 & 10.681\% \\
\$800,000 & \$600,000 & \$1,200,000 & \$1,464,325 & 7.345\% & \$800,000 & \$800,000 & \$1,400,000 & \$1,564,369 & 8.102\% \\
\$800,000 & \$800,000 & \$800,000 & \$1,709,285 & 5.182\% & \$800,000 & \$1,000,000 & \$1,000,000 & \$1,809,329 & 6.202\% \\
\$600,000 & \$800,000 & \$1,400,000 & \$1,237,092 & 3.598\% & \$600,000 & \$1,000,000 & \$1,600,000 & \$1,337,137 & 4.783\% \\
\$800,000 & \$200,000 & \$1,800,000 & \$1,174,405 & 2.551\% & \$800,000 & \$400,000 & \$2,000,000 & \$1,274,449 & 3.631\% \\
\$600,000 & \$1,000,000 & \$1,000,000 & \$1,482,052 & 1.851\% & \$600,000 & \$1,200,000 & \$1,200,000 & \$1,582,097 & 2.888\% \\
\$800,000 & \$400,000 & \$1,400,000 & \$1,419,365 & 1.248\% & \$800,000 & \$600,000 & \$1,600,000 & \$1,519,409 & 2.204\% \\
& & & & & \$800,000 & \$800,000 & \$1,200,000 & \$1,764,369 & 1.694\% \\
& & & & & \$800,000 & \$1,000,000 & \$800,000 & \$2,009,329 & 1.289\% \\
& & & & & \$400,000 & \$1,400,000 & \$1,400,000 & \$1,354,865 & 1.069\% \\
\hline
Other& & & & 2.923\% & Other& & & & 3.233\% \\
\hline
\small \textbf{Game Value:} &&& \small \textbf{-1.75} && \small \textbf{Winner:} &&& \small \textbf{P2}&\\
\hline
\end{tabular}
\caption{Mixed strategy output, when P1 is given an initial budget of \$800,000, P2 is given an initial budget of \$900,000, and the Unweighted, Unfloored Utility Vector is used.}
\label{8v9table.4}
\end{figure}

\begin{figure}
\centering
\captionsetup{justification=raggedright}
\subfloat[P1: \$800,000 Initial Budget]{
\begin{tikzpicture}
\begin{axis}[
width=7.0cm,
ylabel={\tiny{Amount to Spend (\$1,000s)}},
xlabel={\tiny{Quarter}},
xmin=1, xmax=4,
ymin=-50, ymax=3000,
xtick={1,2,3,4},
ytick={0,1000,2000,3000},
legend pos=north west,
ymajorgrids=true,
grid style=dashed,
]
\addplot[draw=gr1]
coordinates {(1,800)(2,400)(3,1400)(4,1419.365)};
\addplot[draw=gr2]
coordinates {(1,600)(2,1000)(3,1000)(4,1482.052)};
\addplot[draw=gr3]
coordinates {(1,800)(2,200)(3,1800)(4,1174.405)};
\addplot[draw=gr4]
coordinates {(1,600)(2,800)(3,1400)(4,1237.092)};
\addplot[draw=gr5]
coordinates {(1,800)(2,800)(3,800)(4,1709.285)};
\addplot[draw=gr7]
coordinates {(1,800)(2,600)(3,1200)(4,1464.325)};
\addplot[draw=gr10]
coordinates {(1,800)(2,400)(3,1600)(4,1219.365)};
\addplot[draw=gr15]
coordinates {(1,600)(2,1000)(3,1200)(4,1282.052)};
\addplot[draw=gr21]
coordinates {(1,800)(2,800)(3,1000)(4,1509.285)};
\addplot[draw=red]
coordinates {(1,800)(2,600)(3,1400)(4,1264.325)};
\end{axis}
\end{tikzpicture}
}
\qquad
\subfloat[P2: \$900,000 Initial Budget]{
\begin{tikzpicture}
\begin{axis}[
width=7.0cm,
ylabel={\tiny{Amount to Spend (\$1,000s)}},
xlabel={\tiny{Quarter}},
xmin=1, xmax=4,
ymin=-50, ymax=3000,
xtick={1,2,3,4},
ytick={0,1000,2000,3000},
legend pos=north west,
ymajorgrids=true,
grid style=dashed,
]
\addplot[draw=gr1]
coordinates {(1,400)(2,1400)(3,1400)(4,1354.865)};
\addplot[draw=gr1]
coordinates {(1,800)(2,1000)(3,800)(4,2009.329)};
\addplot[draw=gr2]
coordinates {(1,800)(2,800)(3,1200)(4,1764.369)};
\addplot[draw=gr2]
coordinates {(1,800)(2,600)(3,1600)(4,1519.409)};
\addplot[draw=gr3]
coordinates {(1,600)(2,1200)(3,1200)(4,1582.097)};
\addplot[draw=gr4]
coordinates {(1,800)(2,400)(3,2000)(4,1274.449)};
\addplot[draw=gr5]
coordinates {(1,600)(2,1000)(3,1600)(4,1337.137)};
\addplot[draw=gr6]
coordinates {(1,800)(2,1000)(3,1000)(4,1809.329)};
\addplot[draw=gr8]
coordinates {(1,800)(2,800)(3,1400)(4,1564.369)};
\addplot[draw=gr11]
coordinates {(1,800)(2,600)(3,1800)(4,1319.409)};
\addplot[draw=gr14]
coordinates {(1,600)(2,1200)(3,1400)(4,1382.097)};
\addplot[draw=gr18]
coordinates {(1,800)(2,1000)(3,1200)(4,1609.329)};
\addplot[draw=blue]
coordinates {(1,800)(2,800)(3,1600)(4,1364.369)};
\end{axis}
\end{tikzpicture}
}
\caption{Overlaid strategies for P1 and P2. Colored line indicates top strategy. Darker gray lines indicate heavily used strategies, while lighter gray lines indicated lightly used strategies.}
\label{8v9.4}
\end{figure}

\begin{figure}
\tiny
\begin{tabular}{ |p{1.0cm}p{1.0cm}p{1.0cm}p{2.0cm}|p{1.0cm}||p{1.0cm}p{1.0cm}p{1.0cm}p{2.0cm}|p{1.0cm}|}
\hline
\multicolumn{10}{|c|}{P1 \$800,000 vs. P2 \$1,000,000} \\
\hline
Player 1 & (P1) & & & & Player 2 & (P2) & & & \\
\hline
Q1 Strategy & Q2 Strategy & Q3 Strategy & Q4 Strategy  &  Play \% & Q1 Strategy & Q2 Strategy & Q3 Strategy & Q4 Strategy  &  Play \%\\
\hline
\$800,000 & \$600,000 & \$1,400,000 & \$1,264,325 & 29.373\% & \$1,000,000 & \$1,000,000 & \$1,600,000 & \$1,536,606 & 16.190\% \\
\$800,000 & \$800,000 & \$1,000,000 & \$1,509,285 & 20.941\% & \$1,000,000 & \$1,200,000 & \$1,200,000 & \$1,781,566 & 13.529\% \\
\$600,000 & \$1,000,000 & \$1,200,000 & \$1,282,052 & 14.749\% & \$1,000,000 & \$800,000 & \$1,800,000 & \$1,491,646 & 11.424\% \\
\$800,000 & \$400,000 & \$1,600,000 & \$1,219,365 & 10.276\% & \$1,000,000 & \$1,000,000 & \$1,400,000 & \$1,736,606 & 9.513\% \\
\$800,000 & \$600,000 & \$1,200,000 & \$1,464,325 & 7.398\% & \$1,000,000 & \$1,200,000 & \$1,000,000 & \$1,981,566 & 7.899\% \\
\$800,000 & \$800,000 & \$800,000 & \$1,709,285 & 5.096\% & \$800,000 & \$1,200,000 & \$1,600,000 & \$1,509,374 & 6.819\% \\
\$600,000 & \$800,000 & \$1,400,000 & \$1,237,092 & 3.663\% & \$1,000,000 & \$600,000 & \$2,000,000 & \$1,446,686 & 5.540\% \\
\$800,000 & \$200,000 & \$1,800,000 & \$1,174,405 & 2.583\% & \$1,000,000 & \$800,000 & \$1,600,000 & \$1,691,646 & 4.719\% \\
\$600,000 & \$1,000,000 & \$1,000,000 & \$1,482,052 & 1.837\% & \$1,000,000 & \$1,000,000 & \$1,200,000 & \$1,936,606 & 3.950\% \\
\$800,000 & \$400,000 & \$1,400,000 & \$1,419,365 & 1.240\% & \$1,000,000 & \$1,200,000 & \$800,000 & \$2,181,566 & 3.265\% \\
& & & & & \$800,000 & \$1,000,000 & \$1,800,000 & \$1,464,414 & 2.764\% \\
& & & & & \$1,000,000 & \$400,000 & \$2,200,000 & \$1,401,726 & 2.408\% \\
& & & & & \$800,000 & \$1,200,000 & \$1,400,000 & \$1,709,374 & 1.985\% \\
& & & & & \$1,000,000 & \$600,000 & \$1,800,000 & \$1,646,686 & 1.709\% \\
& & & & & \$1,000,000 & \$800,000 & \$1,400,000 & \$1,891,646 & 1.390\% \\
& & & & & \$1,000,000 & \$1,000,000 & \$1,000,000 & \$2,136,606 & 1.134\% \\
\hline
\small \textbf{Game Value:} &&& \small \textbf{-8.39} && \small \textbf{Winner:} &&& \small \textbf{P2}&\\
\hline
Other& & & & 2.844\% & Other& & & & 5.762\% \\
\hline
\end{tabular}
\caption{Mixed strategy output, when P1 is given an initial budget of \$800,000, P2 is given an initial budget of \$1,000,000, and the Unweighted, Unfloored Utility Vector is used.}
\label{8v10table.4}
\end{figure}

\begin{figure}
\centering
\captionsetup{justification=raggedright}
\subfloat[P1: \$800,000 Initial Budget]{
\begin{tikzpicture}
\begin{axis}[
width=7.0cm,
ylabel={\tiny{Amount to Spend (\$1,000s)}},
xlabel={\tiny{Quarter}},
xmin=1, xmax=4,
ymin=-50, ymax=3000,
xtick={1,2,3,4},
ytick={0,1000,2000,3000},
legend pos=north west,
ymajorgrids=true,
grid style=dashed,
]
\addplot[draw=gr1]
coordinates {(1,800)(2,400)(3,1400)(4,1419.365)};
\addplot[draw=gr2]
coordinates {(1,600)(2,1000)(3,1000)(4,1482.052)};
\addplot[draw=gr3]
coordinates {(1,800)(2,200)(3,1800)(4,1174.405)};
\addplot[draw=gr4]
coordinates {(1,600)(2,800)(3,1400)(4,1237.092)};
\addplot[draw=gr5]
coordinates {(1,800)(2,800)(3,800)(4,1709.285)};
\addplot[draw=gr7]
coordinates {(1,800)(2,600)(3,1200)(4,1464.325)};
\addplot[draw=gr10]
coordinates {(1,800)(2,400)(3,1600)(4,1219.365)};
\addplot[draw=gr15]
coordinates {(1,600)(2,1000)(3,1200)(4,1282.052)};
\addplot[draw=gr21]
coordinates {(1,800)(2,800)(3,1000)(4,1509.285)};
\addplot[draw=red]
coordinates {(1,800)(2,600)(3,1400)(4,1264.325)};
\end{axis}
\end{tikzpicture}
}
\qquad
\subfloat[P2: \$1,000,000 Initial Budget]{
\begin{tikzpicture}
\begin{axis}[
width=7.0cm,
ylabel={\tiny{Amount to Spend (\$1,000s)}},
xlabel={\tiny{Quarter}},
xmin=1, xmax=4,
ymin=-50, ymax=3000,
xtick={1,2,3,4},
ytick={0,1000,2000,3000},
legend pos=north west,
ymajorgrids=true,
grid style=dashed,
]
\addplot[draw=gr1]
coordinates {(1,1000)(2,1000)(3,1000)(4,2136.606)};
\addplot[draw=gr1]
coordinates {(1,1000)(2,800)(3,1400)(4,1891.646)};
\addplot[draw=gr2]
coordinates {(1,1000)(2,600)(3,1800)(4,1646.686)};
\addplot[draw=gr2]
coordinates {(1,800)(2,1200)(3,1400)(4,1709.374)};
\addplot[draw=gr2]
coordinates {(1,1000)(2,400)(3,2200)(4,1401.726)};
\addplot[draw=gr3]
coordinates {(1,800)(2,1000)(3,1800)(4,1464.414)};
\addplot[draw=gr3]
coordinates {(1,1000)(2,1200)(3,800)(4,2181.566)};
\addplot[draw=gr4]
coordinates {(1,1000)(2,1000)(3,1200)(4,1936.606)};
\addplot[draw=gr5]
coordinates {(1,1000)(2,800)(3,1600)(4,1691.646)};
\addplot[draw=gr6]
coordinates {(1,1000)(2,600)(3,2000)(4,1446.686)};
\addplot[draw=gr7]
coordinates {(1,800)(2,1200)(3,1600)(4,1509.374)};
\addplot[draw=gr8]
coordinates {(1,1000)(2,1200)(3,1000)(4,1981.566)};
\addplot[draw=gr10]
coordinates {(1,1000)(2,1000)(3,1400)(4,1736.606)};
\addplot[draw=gr11]
coordinates {(1,1000)(2,800)(3,1800)(4,1491.646)};
\addplot[draw=gr14]
coordinates {(1,1000)(2,1200)(3,1200)(4,1781.566)};
\addplot[draw=blue]
coordinates {(1,1000)(2,1000)(3,1600)(4,1536.606)};
\end{axis}
\end{tikzpicture}
}
\caption{Overlaid strategies for P1 and P2. Colored line indicates top strategy. Darker gray lines indicate heavily used strategies, while lighter gray lines indicated lightly used strategies.}
\label{8v10.4}
\end{figure}

\begin{figure}
\tiny
\begin{tabular}{ |p{1.0cm}p{1.0cm}p{1.0cm}p{2.0cm}|p{1.0cm}||p{1.0cm}p{1.0cm}p{1.0cm}p{2.0cm}|p{1.0cm}|}
\hline
\multicolumn{10}{|c|}{P1 \$800,000 vs. P2 \$1,100,000} \\
\hline
Player 1 & (P1) & & & & Player 2 & (P2) & & & \\
\hline
Q1 Strategy & Q2 Strategy & Q3 Strategy & Q4 Strategy  &  Play \% & Q1 Strategy & Q2 Strategy & Q3 Strategy & Q4 Strategy  &  Play \%\\
\hline
\$800,000 & \$600,000 & \$1,400,000 & \$1,264,325 & 29.356\% & \$1,000,000 & \$1,200,000 & \$1,800,000 & \$1,636,650 & 12.95\% \\
\$800,000 & \$800,000 & \$1,000,000 & \$1,509,285 & 20.991\% & \$1,000,000 & \$1,400,000 & \$1,400,000 & \$1,881,610 & 11.353\% \\
\$600,000 & \$1,000,000 & \$1,200,000 & \$1,282,052 & 14.699\% & \$800,000 & \$1,600,000 & \$1,600,000 & \$1,654,378 & 9.862\% \\
\$800,000 & \$400,000 & \$1,600,000 & \$1,219,365 & 10.481\% & \$1,000,000 & \$1,000,000 & \$2,000,000 & \$1,591,690 & 8.709\% \\
\$800,000 & \$600,000 & \$1,200,000 & \$1,464,325 & 7.205\% & \$1,000,000 & \$1,200,000 & \$1,600,000 & \$1,836,650 & 7.605\% \\
\$800,000 & \$800,000 & \$800,000 & \$1,709,285 & 5.168\% & \$1,000,000 & \$1,400,000 & \$1,200,000 & \$2,081,610 & 6.572\% \\
\$600,000 & \$800,000 & \$1,400,000 & \$1,237,092 & 3.744\% & \$800,000 & \$1,400,000 & \$1,800,000 & \$1,609,418 & 5.659\% \\
\$800,000 & \$200,000 & \$1,800,000 & \$1,174,405 & 2.478\% & \$800,000 & \$1,600,000 & \$1,400,000 & \$1,854,378 & 4.980\% \\
\$600,000 & \$1,000,000 & \$1,000,000 & \$1,482,052 & 1.805\% & \$1,000,000 & \$1,000,000 & \$1,800,000 & \$1,791,690 & 4.304\% \\
\$800,000 & \$400,000 & \$1,400,000 & \$1,419,365 & 1.233\% & \$1,000,000 & \$1,200,000 & \$1,400,000 & \$2,036,650 & 3.752\% \\
& & & & & \$1,000,000 & \$1,400,000 & \$1,000,000 & \$2,281,610 & 3.15\% \\
& & & & & \$800,000 & \$1,400,000 & \$1,600,000 & \$1,809,418 & 2.909\% \\
& & & & & \$1,000,000 & \$800,000 & \$2,000,000 & \$1,746,730 & 2.325\% \\
& & & & & \$800,000 & \$1,600,000 & \$1,200,000 & \$2,054,378 & 2.123\% \\
& & & & & \$1,000,000 & \$1,000,000 & \$1,600,000 & \$1,991,690 & 1.853\% \\
& & & & & \$1,000,000 & \$1,200,000 & \$1,200,000 & \$2,236,650 & 1.582\% \\
& & & & & \$1,000,000 & \$1,400,000 & \$800,000 & \$2,481,610 & 1.376\% \\
& & & & & \$800,000 & \$1,200,000 & \$1,800,000 & \$1,764,458 & 1.223\% \\
& & & & & \$1,000,000 & \$600,000 & \$2,200,000 & \$1,701,770 & 1.049\% \\
\hline
Other& & & & 2.84\% & Other& & & & 6.664\% \\
\hline
\small \textbf{Game Value:} &&& \small \textbf{-10.25} && \small \textbf{Winner:} &&& \small \textbf{P2}&\\
\hline
\end{tabular}
\caption{Mixed strategy output, when P1 is given an initial budget of \$800,000, P2 is given an initial budget of \$1,100,000, and the Unweighted, Unfloored Utility Vector is used.}
\label{8v11table.4}
\end{figure}

\begin{figure}
\centering
\captionsetup{justification=raggedright}
\subfloat[P1: \$800,000 Initial Budget]{
\begin{tikzpicture}
\begin{axis}[
width=7.0cm,
ylabel={\tiny{Amount to Spend (\$1,000s)}},
xlabel={\tiny{Quarter}},
xmin=1, xmax=4,
ymin=-50, ymax=3000,
xtick={1,2,3,4},
ytick={0,1000,2000,3000},
legend pos=north west,
ymajorgrids=true,
grid style=dashed,
]
\addplot[draw=gr1]
coordinates {(1,800)(2,400)(3,1400)(4,1419.365)};
\addplot[draw=gr2]
coordinates {(1,600)(2,1000)(3,1000)(4,1482.052)};
\addplot[draw=gr2]
coordinates {(1,800)(2,200)(3,1800)(4,1174.405)};
\addplot[draw=gr4]
coordinates {(1,600)(2,800)(3,1400)(4,1237.092)};
\addplot[draw=gr5]
coordinates {(1,800)(2,800)(3,800)(4,1709.285)};
\addplot[draw=gr7]
coordinates {(1,800)(2,600)(3,1200)(4,1464.325)};
\addplot[draw=gr10]
coordinates {(1,800)(2,400)(3,1600)(4,1219.365)};
\addplot[draw=gr15]
coordinates {(1,600)(2,1000)(3,1200)(4,1282.052)};
\addplot[draw=gr21]
coordinates {(1,800)(2,800)(3,1000)(4,1509.285)};
\addplot[draw=red]
coordinates {(1,800)(2,600)(3,1400)(4,1264.325)};
\end{axis}
\end{tikzpicture}
}
\qquad
\subfloat[P2: \$1,100,000 Initial Budget]{
\begin{tikzpicture}
\begin{axis}[
width=7.0cm,
ylabel={\tiny{Amount to Spend (\$1,000s)}},
xlabel={\tiny{Quarter}},
xmin=1, xmax=4,
ymin=-50, ymax=3000,
xtick={1,2,3,4},
ytick={0,1000,2000,3000},
legend pos=north west,
ymajorgrids=true,
grid style=dashed,
]
\addplot[draw=gr1]
coordinates {(1,1000)(2,600)(3,2200)(4,1701.77)};
\addplot[draw=gr1]
coordinates {(1,800)(2,1200)(3,1800)(4,1764.458)};
\addplot[draw=gr1]
coordinates {(1,1000)(2,1400)(3,800)(4,2481.61)};
\addplot[draw=gr2]
coordinates {(1,1000)(2,1200)(3,1200)(4,2236.65)};
\addplot[draw=gr2]
coordinates {(1,1000)(2,1000)(3,1600)(4,1991.69)};
\addplot[draw=gr2]
coordinates {(1,800)(2,1600)(3,1200)(4,2054.378)};
\addplot[draw=gr2]
coordinates {(1,1000)(2,800)(3,2000)(4,1746.73)};
\addplot[draw=gr3]
coordinates {(1,800)(2,1400)(3,1600)(4,1809.418)};
\addplot[draw=gr3]
coordinates {(1,1000)(2,1400)(3,1000)(4,2281.61)};
\addplot[draw=gr4]
coordinates {(1,1000)(2,1200)(3,1400)(4,2036.65)};
\addplot[draw=gr4]
coordinates {(1,1000)(2,1000)(3,1800)(4,1791.69)};
\addplot[draw=gr5]
coordinates {(1,800)(2,1600)(3,1400)(4,1854.378)};
\addplot[draw=gr6]
coordinates {(1,800)(2,1400)(3,1800)(4,1609.418)};
\addplot[draw=gr7]
coordinates {(1,1000)(2,1400)(3,1200)(4,2081.61)};
\addplot[draw=gr8]
coordinates {(1,1000)(2,1200)(3,1600)(4,1836.65)};
\addplot[draw=gr9]
coordinates {(1,1000)(2,1000)(3,2000)(4,1591.69)};
\addplot[draw=gr10]
coordinates {(1,800)(2,1600)(3,1600)(4,1654.378)};
\addplot[draw=gr11]
coordinates {(1,1000)(2,1400)(3,1400)(4,1881.61)};
\addplot[draw=blue]
coordinates {(1,1000)(2,1200)(3,1800)(4,1636.65)};
\end{axis}
\end{tikzpicture}
}
\caption{Overlaid strategies for P1 and P2. Colored line indicates top strategy. Darker gray lines indicate heavily used strategies, while lighter gray lines indicated lightly used strategies.}
\label{8v11.4}
\end{figure}

\begin{figure}
\tiny
\begin{tabular}{ |p{1.0cm}p{1.0cm}p{1.0cm}p{2.0cm}|p{1.0cm}||p{1.0cm}p{1.0cm}p{1.0cm}p{2.0cm}|p{1.0cm}|}
\hline
\multicolumn{10}{|c|}{P1 \$800,000 vs. P2 \$1,200,000} \\
\hline
Player 1 & (P1) & & & & Player 2 & (P2) & & & \\
\hline
Q1 Strategy & Q2 Strategy & Q3 Strategy & Q4 Strategy  &  Play \% & Q1 Strategy & Q2 Strategy & Q3 Strategy & Q4 Strategy  &  Play \%\\
\hline
\$800,000 & \$600,000 & \$1,400,000 & \$1,264,325 & 29.447\% & \$1,200,000 & \$1,200,000 & \$2,000,000 & \$1,763,927 & 9.798\% \\
\$800,000 & \$800,000 & \$1,000,000 & \$1,509,285 & 20.903\% & \$1,200,000 & \$1,400,000 & \$1,600,000 & \$2,008,887 & 8.736\% \\
\$600,000 & \$1,000,000 & \$1,200,000 & \$1,282,052 & 14.619\% & \$1,000,000 & \$1,600,000 & \$1,800,000 & \$1,781,655 & 7.889\% \\
\$800,000 & \$400,000 & \$1,600,000 & \$1,219,365 & 10.312\% & \$1,200,000 & \$1,200,000 & \$1,800,000 & \$1,963,927 & 7.106\% \\
\$800,000 & \$600,000 & \$1,200,000 & \$1,464,325 & 7.583\% & \$1,200,000 & \$1,400,000 & \$1,400,000 & \$2,208,887 & 6.442\% \\
\$800,000 & \$800,000 & \$800,000 & \$1,709,285 & 5.209\% & \$1,000,000 & \$1,600,000 & \$1,600,000 & \$1,981,655 & 5.862\% \\
\$600,000 & \$800,000 & \$1,400,000 & \$1,237,092 & 3.605\% & \$1,200,000 & \$1,000,000 & \$2,000,000 & \$1,918,967 & 5.218\% \\
\$800,000 & \$200,000 & \$1,800,000 & \$1,174,405 & 2.571\% & \$1,200,000 & \$1,200,000 & \$1,600,000 & \$2,163,927 & 4.781\% \\
\$600,000 & \$1,000,000 & \$1,000,000 & \$1,482,052 & 1.727\% & \$1,200,000 & \$1,400,000 & \$1,200,000 & \$2,408,887 & 4.324\% \\
\$800,000 & \$400,000 & \$1,400,000 & \$1,419,365 & 1.242\% & \$1,000,000 & \$1,400,000 & \$1,800,000 & \$1,936,695 & 3.917\% \\
& & & & & \$1,200,000 & \$800,000 & \$2,200,000 & \$1,874,007 & 3.485\% \\
& & & & & \$1,000,000 & \$1,600,000 & \$1,400,000 & \$2,181,655 & 3.182\% \\
& & & & & \$1,200,000 & \$1,000,000 & \$1,800,000 & \$2,118,967 & 2.836\% \\
& & & & & \$1,200,000 & \$1,200,000 & \$1,400,000 & \$2,363,927 & 2.585\% \\
& & & & & \$1,200,000 & \$1,400,000 & \$1,000,000 & \$2,608,887 & 2.278\% \\
& & & & & \$800,000 & \$1,800,000 & \$1,600,000 & \$1,954,423 & 2.201\% \\
& & & & & \$1,000,000 & \$1,200,000 & \$2,000,000 & \$1,891,735 & 1.918\% \\
& & & & & \$1,200,000 & \$600,000 & \$2,400,000 & \$1,829,047 & 1.727\% \\
& & & & & \$1,000,000 & \$1,400,000 & \$1,600,000 & \$2,136,695 & 1.576\% \\
& & & & & \$1,200,000 & \$800,000 & \$2,000,000 & \$2,074,007 & 1.424\% \\
& & & & & \$1,000,000 & \$1,600,000 & \$1,200,000 & \$2,381,655 & 1.313\% \\
& & & & & \$1,200,000 & \$1,000,000 & \$1,600,000 & \$2,318,967 & 1.128\% \\
& & & & & \$1,200,000 & \$1,200,000 & \$1,200,000 & \$2,563,927 & 1.020\% \\
\hline
Other& & & & 2.782\% & Other& & & & 9.254\% \\
\hline
\small \textbf{Game Value:} &&& \small \textbf{-14.86} && \small \textbf{Winner:} &&& \small \textbf{P2}&\\
\hline
\end{tabular}
\caption{Mixed strategy output, when P1 is given an initial budget of \$800,000, P2 is given an initial budget of \$1,200,000, and the Unweighted, Unfloored Utility Vector is used.}
\label{8v12table.4}
\end{figure}

\begin{figure}
\centering
\captionsetup{justification=raggedright}
\subfloat[P1: \$800,000 Initial Budget]{
\begin{tikzpicture}
\begin{axis}[
width=7.0cm,
ylabel={\tiny{Amount to Spend (\$1,000s)}},
xlabel={\tiny{Quarter}},
xmin=1, xmax=4,
ymin=-50, ymax=3000,
xtick={1,2,3,4},
ytick={0,1000,2000,3000},
legend pos=north west,
ymajorgrids=true,
grid style=dashed,
]
\addplot[draw=gr1]
coordinates {(1,800)(2,400)(3,1400)(4,1419.365)};
\addplot[draw=gr2]
coordinates {(1,600)(2,1000)(3,1000)(4,1482.052)};
\addplot[draw=gr3]
coordinates {(1,800)(2,200)(3,1800)(4,1174.405)};
\addplot[draw=gr4]
coordinates {(1,600)(2,800)(3,1400)(4,1237.092)};
\addplot[draw=gr5]
coordinates {(1,800)(2,800)(3,800)(4,1709.285)};
\addplot[draw=gr8]
coordinates {(1,800)(2,600)(3,1200)(4,1464.325)};
\addplot[draw=gr10]
coordinates {(1,800)(2,400)(3,1600)(4,1219.365)};
\addplot[draw=gr15]
coordinates {(1,600)(2,1000)(3,1200)(4,1282.052)};
\addplot[draw=gr21]
coordinates {(1,800)(2,800)(3,1000)(4,1509.285)};
\addplot[draw=red]
coordinates {(1,800)(2,600)(3,1400)(4,1264.325)};
\end{axis}
\end{tikzpicture}
}
\qquad
\subfloat[P2: \$1,200,000 Initial Budget]{
\begin{tikzpicture}
\begin{axis}[
width=7.0cm,
ylabel={\tiny{Amount to Spend (\$1,000s)}},
xlabel={\tiny{Quarter}},
xmin=1, xmax=4,
ymin=-50, ymax=3000,
xtick={1,2,3,4},
ytick={0,1000,2000,3000},
legend pos=north west,
ymajorgrids=true,
grid style=dashed,
]
\addplot[draw=gr1]
coordinates {(1,1200)(2,1200)(3,1200)(4,2563.927)};
\addplot[draw=gr1]
coordinates {(1,1200)(2,1000)(3,1600)(4,2318.967)};
\addplot[draw=gr1]
coordinates {(1,1000)(2,1600)(3,1200)(4,2381.655)};
\addplot[draw=gr1]
coordinates {(1,1200)(2,800)(3,2000)(4,2074.007)};
\addplot[draw=gr2]
coordinates {(1,1000)(2,1400)(3,1600)(4,2136.695)};
\addplot[draw=gr2]
coordinates {(1,1200)(2,600)(3,2400)(4,1829.047)};
\addplot[draw=gr2]
coordinates {(1,1000)(2,1200)(3,2000)(4,1891.735)};
\addplot[draw=gr2]
coordinates {(1,800)(2,1800)(3,1600)(4,1954.423)};
\addplot[draw=gr2]
coordinates {(1,1200)(2,1400)(3,1000)(4,2608.887)};
\addplot[draw=gr3]
coordinates {(1,1200)(2,1200)(3,1400)(4,2363.927)};
\addplot[draw=gr3]
coordinates {(1,1200)(2,1000)(3,1800)(4,2118.967)};
\addplot[draw=gr3]
coordinates {(1,1000)(2,1600)(3,1400)(4,2181.655)};
\addplot[draw=gr3]
coordinates {(1,1200)(2,800)(3,2200)(4,1874.007)};
\addplot[draw=gr4]
coordinates {(1,1000)(2,1400)(3,1800)(4,1936.695)};
\addplot[draw=gr4]
coordinates {(1,1200)(2,1400)(3,1200)(4,2408.887)};
\addplot[draw=gr5]
coordinates {(1,1200)(2,1200)(3,1600)(4,2163.927)};
\addplot[draw=gr5]
coordinates {(1,1200)(2,1000)(3,2000)(4,1918.967)};
\addplot[draw=gr6]
coordinates {(1,1000)(2,1600)(3,1600)(4,1981.655)};
\addplot[draw=gr6]
coordinates {(1,1200)(2,1400)(3,1400)(4,2208.887)};
\addplot[draw=gr7]
coordinates {(1,1200)(2,1200)(3,1800)(4,1963.927)};
\addplot[draw=gr8]
coordinates {(1,1000)(2,1600)(3,1800)(4,1781.655)};
\addplot[draw=gr9]
coordinates {(1,1200)(2,1400)(3,1600)(4,2008.887)};
\addplot[draw=blue]
coordinates {(1,1200)(2,1200)(3,2000)(4,1763.927)};
\end{axis}
\end{tikzpicture}
}
\caption{Overlaid strategies for P1 and P2. Colored line indicates top strategy. Darker gray lines indicate heavily used strategies, while lighter gray lines indicated lightly used strategies.}
\label{8v12.4}
\end{figure}

\begin{figure}
\tiny
\begin{tabular}{ |p{1.0cm}p{1.0cm}p{1.0cm}p{2.0cm}|p{1.0cm}||p{1.0cm}p{1.0cm}p{1.0cm}p{2.0cm}|p{1.0cm}|}
\hline
\multicolumn{10}{|c|}{P1 \$800,000 vs. P2 \$1,300,000} \\
\hline
Player 1 & (P1) & & & & Player 2 & (P2) & & & \\
\hline
Q1 Strategy & Q2 Strategy & Q3 Strategy & Q4 Strategy  &  Play \% & Q1 Strategy & Q2 Strategy & Q3 Strategy & Q4 Strategy  &  Play \%\\
\hline
\$800,000 & \$600,000 & \$1,400,000 & \$1,264,325 & 29.358\% & \$1,200,000 & \$1,400,000 & \$2,000,000 & \$2,063,972 & 8.018\% \\
\$800,000 & \$800,000 & \$1,000,000 & \$1,509,285 & 20.738\% & \$1,200,000 & \$1,600,000 & \$1,600,000 & \$2,308,932 & 7.5\% \\
\$600,000 & \$1,000,000 & \$1,200,000 & \$1,282,052 & 14.760\% & \$1,000,000 & \$1,800,000 & \$1,800,000 & \$2,081,699 & 6.740\% \\
\$800,000 & \$400,000 & \$1,600,000 & \$1,219,365 & 10.455\% & \$1,200,000 & \$1,200,000 & \$2,200,000 & \$2,019,012 & 6.415\% \\
\$800,000 & \$600,000 & \$1,200,000 & \$1,464,325 & 7.331\% & \$1,200,000 & \$1,400,000 & \$1,800,000 & \$2,263,972 & 5.886\% \\
\$800,000 & \$800,000 & \$800,000 & \$1,709,285 & 5.162\% & \$1,200,000 & \$1,600,000 & \$1,400,000 & \$2,508,932 & 5.434\% \\
\$600,000 & \$800,000 & \$1,400,000 & \$1,237,092 & 3.677\% & \$1,000,000 & \$1,600,000 & \$2,000,000 & \$2,036,739 & 4.853\% \\
\$800,000 & \$200,000 & \$1,800,000 & \$1,174,405 & 2.512\% & \$1,200,000 & \$1,000,000 & \$2,400,000 & \$1,974,052 & 4.399\% \\
\$600,000 & \$1,000,000 & \$1,000,000 & \$1,482,052 & 1.815\% & \$1,000,000 & \$1,800,000 & \$1,600,000 & \$2,281,699 & 3.990\% \\
\$800,000 & \$400,000 & \$1,400,000 & \$1,419,365 & 1.258\% & \$1,200,000 & \$1,200,000 & \$2,000,000 & \$2,219,012 & 3.875\% \\
& & & & & \$1,200,000 & \$1,400,000 & \$1,600,000 & \$2,463,972 & 3.545\% \\
& & & & & \$1,200,000 & \$1,600,000 & \$1,200,000 & \$2,708,932 & 3.200\% \\
& & & & & \$800,000 & \$2,000,000 & \$1,800,000 & \$2,054,467 & 2.992\% \\
& & & & & \$1,000,000 & \$1,400,000 & \$2,200,000 & \$1,991,779 & 2.764\% \\
& & & & & \$1,200,000 & \$800,000 & \$2,600,000 & \$1,929,092 & 2.488\% \\
& & & & & \$1,000,000 & \$1,600,000 & \$1,800,000 & \$2,236,739 & 2.312\% \\
& & & & & \$1,200,000 & \$1,000,000 & \$2,200,000 & \$2,174,052 & 2.076\% \\
& & & & & \$1,000,000 & \$1,800,000 & \$1,400,000 & \$2,481,699 & 1.980\% \\
& & & & & \$1,200,000 & \$1,200,000 & \$1,800,000 & \$2,419,012 & 1.768\% \\
& & & & & \$1,200,000 & \$1,400,000 & \$1,400,000 & \$2,663,972 & 1.576\% \\
& & & & & \$1,200,000 & \$1,600,000 & \$1,000,000 & \$2,908,932 & 1.471\% \\
& & & & & \$800,000 & \$1,800,000 & \$2,000,000 & \$2,009,507 & 1.425\% \\
& & & & & \$1,000,000 & \$1,200,000 & \$2,400,000 & \$1,946,819 & 1.232\% \\
& & & & & \$1,200,000 & \$600,000 & \$2,800,000 & \$1,884,132 & 1.191\% \\
& & & & & \$800,000 & \$2,000,000 & \$1,600,000 & \$2,254,467 & 1.076\% \\
& & & & & \$1,000,000 & \$1,400,000 & \$2,000,000 & \$2,191,779 & 1.041\% \\
\hline
Other& & & & 2.934\% & Other& & & & 10.753\% \\
\hline
\small \textbf{Game Value:} &&& \small \textbf{-15.26} && \small \textbf{Winner:} &&& \small \textbf{P2}&\\
\hline
\end{tabular}
\caption{Mixed strategy output, when P1 is given an initial budget of \$800,000, P2 is given an initial budget of \$1,300,000, and the Unweighted, Unfloored Utility Vector is used.}
\label{8v13table.4}
\end{figure}

\begin{figure}
\centering
\captionsetup{justification=raggedright}
\subfloat[P1: \$800,000 Initial Budget]{
\begin{tikzpicture}
\begin{axis}[
width=7.0cm,
ylabel={\tiny{Amount to Spend (\$1,000s)}},
xlabel={\tiny{Quarter}},
xmin=1, xmax=4,
ymin=-50, ymax=3000,
xtick={1,2,3,4},
ytick={0,1000,2000,3000},
legend pos=north west,
ymajorgrids=true,
grid style=dashed,
]
\addplot[draw=gr1]
coordinates {(1,800)(2,400)(3,1400)(4,1419.365)};
\addplot[draw=gr2]
coordinates {(1,600)(2,1000)(3,1000)(4,1482.052)};
\addplot[draw=gr3]
coordinates {(1,800)(2,200)(3,1800)(4,1174.405)};
\addplot[draw=gr4]
coordinates {(1,600)(2,800)(3,1400)(4,1237.092)};
\addplot[draw=gr5]
coordinates {(1,800)(2,800)(3,800)(4,1709.285)};
\addplot[draw=gr7]
coordinates {(1,800)(2,600)(3,1200)(4,1464.325)};
\addplot[draw=gr10]
coordinates {(1,800)(2,400)(3,1600)(4,1219.365)};
\addplot[draw=gr15]
coordinates {(1,600)(2,1000)(3,1200)(4,1282.052)};
\addplot[draw=gr21]
coordinates {(1,800)(2,800)(3,1000)(4,1509.285)};
\addplot[draw=red]
coordinates {(1,800)(2,600)(3,1400)(4,1264.325)};
\end{axis}
\end{tikzpicture}
}
\qquad
\subfloat[P2: \$1,300,000 Initial Budget]{
\begin{tikzpicture}
\begin{axis}[
width=7.0cm,
ylabel={\tiny{Amount to Spend (\$1,000s)}},
xlabel={\tiny{Quarter}},
xmin=1, xmax=4,
ymin=-50, ymax=3000,
xtick={1,2,3,4},
ytick={0,1000,2000,3000},
legend pos=north west,
ymajorgrids=true,
grid style=dashed,
]
\addplot[draw=gr1]
coordinates {(1,1000)(2,1400)(3,2000)(4,2191.779)};
\addplot[draw=gr1]
coordinates {(1,800)(2,2000)(3,1600)(4,2254.467)};
\addplot[draw=gr1]
coordinates {(1,1200)(2,600)(3,2800)(4,1884.132)};
\addplot[draw=gr1]
coordinates {(1,1000)(2,1200)(3,2400)(4,1946.819)};
\addplot[draw=gr1]
coordinates {(1,800)(2,1800)(3,2000)(4,2009.507)};
\addplot[draw=gr1]
coordinates {(1,1200)(2,1600)(3,1000)(4,2908.932)};
\addplot[draw=gr2]
coordinates {(1,1200)(2,1400)(3,1400)(4,2663.972)};
\addplot[draw=gr2]
coordinates {(1,1200)(2,1200)(3,1800)(4,2419.012)};
\addplot[draw=gr2]
coordinates {(1,1000)(2,1800)(3,1400)(4,2481.699)};
\addplot[draw=gr2]
coordinates {(1,1200)(2,1000)(3,2200)(4,2174.052)};
\addplot[draw=gr2]
coordinates {(1,1000)(2,1600)(3,1800)(4,2236.739)};
\addplot[draw=gr2]
coordinates {(1,1200)(2,800)(3,2600)(4,1929.092)};
\addplot[draw=gr3]
coordinates {(1,1000)(2,1400)(3,2200)(4,1991.779)};
\addplot[draw=gr3]
coordinates {(1,800)(2,2000)(3,1800)(4,2054.467)};
\addplot[draw=gr3]
coordinates {(1,1200)(2,1600)(3,1200)(4,2708.932)};
\addplot[draw=gr4]
coordinates {(1,1200)(2,1400)(3,1600)(4,2463.972)};
\addplot[draw=gr4]
coordinates {(1,1200)(2,1200)(3,2000)(4,2219.012)};
\addplot[draw=gr4]
coordinates {(1,1000)(2,1800)(3,1600)(4,2281.699)};
\addplot[draw=gr4]
coordinates {(1,1200)(2,1000)(3,2400)(4,1974.052)};
\addplot[draw=gr5]
coordinates {(1,1000)(2,1600)(3,2000)(4,2036.739)};
\addplot[draw=gr5]
coordinates {(1,1200)(2,1600)(3,1400)(4,2508.932)};
\addplot[draw=gr6]
coordinates {(1,1200)(2,1400)(3,1800)(4,2263.972)};
\addplot[draw=gr6]
coordinates {(1,1200)(2,1200)(3,2200)(4,2019.012)};
\addplot[draw=gr7]
coordinates {(1,1000)(2,1800)(3,1800)(4,2081.699)};
\addplot[draw=gr8]
coordinates {(1,1200)(2,1600)(3,1600)(4,2308.932)};
\addplot[draw=blue]
coordinates {(1,1200)(2,1400)(3,2000)(4,2063.972)};
\end{axis}
\end{tikzpicture}
}
\caption{Overlaid strategies for P1 and P2. Colored line indicates top strategy. Darker gray lines indicate heavily used strategies, while lighter gray lines indicated lightly used strategies.}
\label{8v13.4}
\end{figure}

\begin{figure}
\tiny
\begin{tabular}{ |p{1.0cm}p{1.0cm}p{1.0cm}p{2.0cm}|p{1.0cm}||p{1.0cm}p{1.0cm}p{1.0cm}p{2.0cm}|p{1.0cm}|}
\hline
\multicolumn{10}{|c|}{P1 \$800,000 vs. P2 \$1,400,000} \\
\hline
Player 1 & (P1) & & & & Player 2 & (P2) & & & \\
\hline
Q1 Strategy & Q2 Strategy & Q3 Strategy & Q4 Strategy  &  Play \% & Q1 Strategy & Q2 Strategy & Q3 Strategy & Q4 Strategy  &  Play \%\\
\hline
\$800,000 & \$600,000 & \$1,400,000 & \$1,264,325 & 29.349\% & \$1,400,000 & \$1,400,000 & \$2,200,000 & \$2,191,248 & 6.338\% \\
\$800,000 & \$800,000 & \$1,000,000 & \$1,509,285 & 20.833\% & \$1,400,000 & \$1,600,000 & \$1,800,000 & \$2,436,208 & 5.893\% \\
\$600,000 & \$1,000,000 & \$1,200,000 & \$1,282,052 & 14.612\% & \$1,200,000 & \$1,800,000 & \$2,000,000 & \$2,208,976 & 5.583\% \\
\$800,000 & \$400,000 & \$1,600,000 & \$1,219,365 & 10.405\% & \$1,400,000 & \$1,200,000 & \$2,400,000 & \$2,146,288 & 5.320\% \\
\$800,000 & \$600,000 & \$1,200,000 & \$1,464,325 & 7.466\% & \$1,400,000 & \$1,400,000 & \$2,000,000 & \$2,391,248 & 4.977\% \\
\$800,000 & \$800,000 & \$800,000 & \$1,709,285 & 5.085\% & \$1,400,000 & \$1,600,000 & \$1,600,000 & \$2,636,208 & 4.505\% \\
\$600,000 & \$800,000 & \$1,400,000 & \$1,237,092 & 3.674\% & \$1,200,000 & \$1,600,000 & \$2,200,000 & \$2,164,016 & 4.314\% \\
\$800,000 & \$200,000 & \$1,800,000 & \$1,174,405 & 2.610\% & \$1,400,000 & \$1,000,000 & \$2,600,000 & \$2,101,328 & 4.023\% \\
\$600,000 & \$1,000,000 & \$1,000,000 & \$1,482,052 & 1.865\% & \$1,200,000 & \$1,800,000 & \$1,800,000 & \$2,408,976 & 3.639\% \\
\$800,000 & \$400,000 & \$1,400,000 & \$1,419,365 & 1.253\% & \$1,400,000 & \$1,200,000 & \$2,200,000 & \$2,346,288 & 3.463\% \\
& & & & & \$1,400,000 & \$1,400,000 & \$1,800,000 & \$2,591,248 & 3.252\% \\
& & & & & \$1,400,000 & \$1,600,000 & \$1,400,000 & \$2,836,208 & 3.101\% \\
& & & & & \$1,000,000 & \$2,000,000 & \$2,000,000 & \$2,181,744 & 2.967\% \\
& & & & & \$1,200,000 & \$1,400,000 & \$2,400,000 & \$2,119,056 & 2.838\% \\
& & & & & \$1,400,000 & \$800,000 & \$2,800,000 & \$2,056,368 & 2.539\% \\
& & & & & \$1,200,000 & \$1,600,000 & \$2,000,000 & \$2,364,016 & 2.335\% \\
& & & & & \$1,400,000 & \$1,000,000 & \$2,400,000 & \$2,301,328 & 2.275\% \\
& & & & & \$1,200,000 & \$1,800,000 & \$1,600,000 & \$2,608,976 & 2.137\% \\
& & & & & \$1,400,000 & \$1,200,000 & \$2,000,000 & \$2,546,288 & 1.981\% \\
& & & & & \$1,400,000 & \$1,400,000 & \$1,600,000 & \$2,791,248 & 1.902\% \\
& & & & & \$1,400,000 & \$1,600,000 & \$1,200,000 & \$3,036,208 & 1.733\% \\
& & & & & \$1,000,000 & \$1,800,000 & \$2,200,000 & \$2,136,784 & 1.591\% \\
& & & & & \$1,200,000 & \$1,200,000 & \$2,600,000 & \$2,074,096 & 1.489\% \\
& & & & & \$1,400,000 & \$600,000 & \$3,000,000 & \$2,011,408 & 1.314\% \\
& & & & & \$1,000,000 & \$2,000,000 & \$1,800,000 & \$2,381,744 & 1.282\% \\
& & & & & \$1,200,000 & \$1,400,000 & \$2,200,000 & \$2,319,056 & 1.252\% \\
& & & & & \$1,200,000 & \$1,600,000 & \$1,800,000 & \$2,564,016 & 1.095\% \\
& & & & & \$1,400,000 & \$800,000 & \$2,600,000 & \$2,256,368 & 1.093\% \\
\hline
Other& & & & 2.848\% & Other& & & & 15.769\% \\
\hline
\small \textbf{Game Value:} &&& \small \textbf{-20.55} && \small \textbf{Winner:} &&& \small \textbf{P2}&\\
\hline
\end{tabular}
\caption{Mixed strategy output, when P1 is given an initial budget of \$800,000, P2 is given an initial budget of \$1,400,000, and the Unweighted, Unfloored Utility Vector is used.}
\label{8v14table.4}
\end{figure}

\begin{figure}
\centering
\captionsetup{justification=raggedright}
\subfloat[P1: \$800,000 Initial Budget]{
\begin{tikzpicture}
\begin{axis}[
width=7.0cm,
ylabel={\tiny{Amount to Spend (\$1,000s)}},
xlabel={\tiny{Quarter}},
xmin=1, xmax=4,
ymin=-50, ymax=4000,
xtick={1,2,3,4},
ytick={0,1000,2000,3000,4000},
legend pos=north west,
ymajorgrids=true,
grid style=dashed,
]
\addplot[draw=gr1]
coordinates {(1,800)(2,400)(3,1400)(4,1419.365)};
\addplot[draw=gr2]
coordinates {(1,600)(2,1000)(3,1000)(4,1482.052)};
\addplot[draw=gr3]
coordinates {(1,800)(2,200)(3,1800)(4,1174.405)};
\addplot[draw=gr4]
coordinates {(1,600)(2,800)(3,1400)(4,1237.092)};
\addplot[draw=gr5]
coordinates {(1,800)(2,800)(3,800)(4,1709.285)};
\addplot[draw=gr7]
coordinates {(1,800)(2,600)(3,1200)(4,1464.325)};
\addplot[draw=gr10]
coordinates {(1,800)(2,400)(3,1600)(4,1219.365)};
\addplot[draw=gr15]
coordinates {(1,600)(2,1000)(3,1200)(4,1282.052)};
\addplot[draw=gr21]
coordinates {(1,800)(2,800)(3,1000)(4,1509.285)};
\addplot[draw=red]
coordinates {(1,800)(2,600)(3,1400)(4,1264.325)};
\end{axis}
\end{tikzpicture}
}
\qquad
\subfloat[P2: \$1,400,000 Initial Budget]{
\begin{tikzpicture}
\begin{axis}[
width=7.0cm,
ylabel={\tiny{Amount to Spend (\$1,000s)}},
xlabel={\tiny{Quarter}},
xmin=1, xmax=4,
ymin=-50, ymax=4000,
xtick={1,2,3,4},
ytick={0,1000,2000,3000,4000},
legend pos=north west,
ymajorgrids=true,
grid style=dashed,
]
\addplot[draw=gr1]
coordinates {(1,1400)(2,800)(3,2600)(4,2256.368)};
\addplot[draw=gr1]
coordinates {(1,1200)(2,1600)(3,1800)(4,2564.016)};
\addplot[draw=gr1]
coordinates {(1,1200)(2,1400)(3,2200)(4,2319.056)};
\addplot[draw=gr1]
coordinates {(1,1000)(2,2000)(3,1800)(4,2381.744)};
\addplot[draw=gr1]
coordinates {(1,1400)(2,600)(3,3000)(4,2011.408)};
\addplot[draw=gr1]
coordinates {(1,1200)(2,1200)(3,2600)(4,2074.096)};
\addplot[draw=gr2]
coordinates {(1,1000)(2,1800)(3,2200)(4,2136.784)};
\addplot[draw=gr2]
coordinates {(1,1400)(2,1600)(3,1200)(4,3036.208)};
\addplot[draw=gr2]
coordinates {(1,1400)(2,1400)(3,1600)(4,2791.248)};
\addplot[draw=gr2]
coordinates {(1,1400)(2,1200)(3,2000)(4,2546.288)};
\addplot[draw=gr2]
coordinates {(1,1200)(2,1800)(3,1600)(4,2608.976)};
\addplot[draw=gr2]
coordinates {(1,1400)(2,1000)(3,2400)(4,2301.328)};
\addplot[draw=gr2]
coordinates {(1,1200)(2,1600)(3,2000)(4,2364.016)};
\addplot[draw=gr3]
coordinates {(1,1400)(2,800)(3,2800)(4,2056.368)};
\addplot[draw=gr3]
coordinates {(1,1200)(2,1400)(3,2400)(4,2119.056)};
\addplot[draw=gr3]
coordinates {(1,1000)(2,2000)(3,2000)(4,2181.744)};
\addplot[draw=gr3]
coordinates {(1,1400)(2,1600)(3,1400)(4,2836.208)};
\addplot[draw=gr3]
coordinates {(1,1400)(2,1400)(3,1800)(4,2591.248)};
\addplot[draw=gr3]
coordinates {(1,1400)(2,1200)(3,2200)(4,2346.288)};
\addplot[draw=gr4]
coordinates {(1,1200)(2,1800)(3,1800)(4,2408.976)};
\addplot[draw=gr4]
coordinates {(1,1400)(2,1000)(3,2600)(4,2101.328)};
\addplot[draw=gr4]
coordinates {(1,1200)(2,1600)(3,2200)(4,2164.016)};
\addplot[draw=gr5]
coordinates {(1,1400)(2,1600)(3,1600)(4,2636.208)};
\addplot[draw=gr5]
coordinates {(1,1400)(2,1400)(3,2000)(4,2391.248)};
\addplot[draw=gr5]
coordinates {(1,1400)(2,1200)(3,2400)(4,2146.288)};
\addplot[draw=gr6]
coordinates {(1,1200)(2,1800)(3,2000)(4,2208.976)};
\addplot[draw=gr6]
coordinates {(1,1400)(2,1600)(3,1800)(4,2436.208)};
\addplot[draw=blue]
coordinates {(1,1400)(2,1400)(3,2200)(4,2191.248)};
\end{axis}
\end{tikzpicture}
}
\caption{Overlaid strategies for P1 and P2. Colored line indicates top strategy. Darker gray lines indicate heavily used strategies, while lighter gray lines indicated lightly used strategies.}
\label{8v14.4}
\end{figure}

\begin{figure}
\tiny
\begin{tabular}{ |p{1.0cm}p{1.0cm}p{1.0cm}p{2.0cm}|p{1.0cm}||p{1.0cm}p{1.0cm}p{1.0cm}p{2.0cm}|p{1.0cm}|}
\hline
\multicolumn{10}{|c|}{P1 \$800,000 vs. P2 \$1,600,000} \\
\hline
Player 1 & (P1) & & & & Player 2 & (P2) & & & \\
\hline
Q1 Strategy & Q2 Strategy & Q3 Strategy & Q4 Strategy  &  Play \% & Q1 Strategy & Q2 Strategy & Q3 Strategy & Q4 Strategy  &  Play \%\\
\hline
\$800,000 & \$600,000 & \$1,400,000 & \$1,264,325 & 29.600\% & \$1,600,000 & \$1,600,000 & \$2,600,000 & \$2,418,570 & 4.295\% \\
\$800,000 & \$800,000 & \$1,000,000 & \$1,509,285 & 20.765\% & \$1,600,000 & \$1,800,000 & \$2,200,000 & \$2,663,530 & 4.092\% \\
\$600,000 & \$1,000,000 & \$1,200,000 & \$1,282,052 & 14.606\% & \$1,400,000 & \$2,000,000 & \$2,400,000 & \$2,436,297 & 3.998\% \\
\$800,000 & \$400,000 & \$1,600,000 & \$1,219,365 & 10.306\% & \$1,600,000 & \$1,400,000 & \$2,800,000 & \$2,373,610 & 3.756\% \\
\$800,000 & \$600,000 & \$1,200,000 & \$1,464,325 & 7.335\% & \$1,600,000 & \$1,600,000 & \$2,400,000 & \$2,618,570 & 3.657\% \\
\$800,000 & \$800,000 & \$800,000 & \$1,709,285 & 5.203\% & \$1,600,000 & \$1,800,000 & \$2,000,000 & \$2,863,530 & 3.520\% \\
\$600,000 & \$800,000 & \$1,400,000 & \$1,237,092 & 3.729\% & \$1,400,000 & \$1,800,000 & \$2,600,000 & \$2,391,337 & 3.303\% \\
\$800,000 & \$200,000 & \$1,800,000 & \$1,174,405 & 2.571\% & \$1,600,000 & \$1,200,000 & \$3,000,000 & \$2,328,650 & 3.233\% \\
\$600,000 & \$1,000,000 & \$1,000,000 & \$1,482,052 & 1.737\% & \$1,400,000 & \$2,000,000 & \$2,200,000 & \$2,636,297 & 3.094\% \\
\$800,000 & \$400,000 & \$1,400,000 & \$1,419,365 & 1.277\% & \$1,600,000 & \$1,400,000 & \$2,600,000 & \$2,573,610 & 3.031\% \\
& & & & & \$1,600,000 & \$1,600,000 & \$2,200,000 & \$2,818,570 & 2.847\% \\
& & & & & \$1,600,000 & \$1,800,000 & \$1,800,000 & \$3,063,530 & 2.685\% \\
& & & & & \$1,400,000 & \$1,600,000 & \$2,800,000 & \$2,346,377 & 2.595\% \\
& & & & & \$1,200,000 & \$2,200,000 & \$2,400,000 & \$2,409,065 & 2.477\% \\
& & & & & \$1,600,000 & \$1,000,000 & \$3,200,000 & \$2,283,690 & 2.298\% \\
& & & & & \$1,400,000 & \$1,800,000 & \$2,400,000 & \$2,591,337 & 2.158\% \\
& & & & & \$1,600,000 & \$1,200,000 & \$2,800,000 & \$2,528,650 & 2.108\% \\
& & & & & \$1,600,000 & \$1,400,000 & \$2,400,000 & \$2,773,610 & 2.000\% \\
& & & & & \$1,400,000 & \$2,000,000 & \$2,000,000 & \$2,836,297 & 1.999\% \\
& & & & & \$1,600,000 & \$1,600,000 & \$2,000,000 & \$3,018,570 & 1.936\% \\
& & & & & \$1,600,000 & \$1,800,000 & \$1,600,000 & \$3,263,530 & 1.881\% \\
& & & & & \$1,200,000 & \$2,000,000 & \$2,600,000 & \$2,364,105 & 1.646\% \\
& & & & & \$1,400,000 & \$1,400,000 & \$3,000,000 & \$2,301,417 & 1.631\% \\
& & & & & \$1,400,000 & \$1,600,000 & \$2,600,000 & \$2,546,377 & 1.574\% \\
& & & & & \$1,200,000 & \$2,200,000 & \$2,200,000 & \$2,609,065 & 1.558\% \\
& & & & & \$1,400,000 & \$1,800,000 & \$2,200,000 & \$2,791,337 & 1.396\% \\
& & & & & \$1,600,000 & \$1,000,000 & \$3,000,000 & \$2,483,690 & 1.330\% \\
& & & & & \$1,400,000 & \$2,000,000 & \$1,800,000 & \$3,036,297 & 1.327\% \\
& & & & & \$1,600,000 & \$1,200,000 & \$2,600,000 & \$2,728,650 & 1.279\% \\
& & & & & \$1,600,000 & \$1,400,000 & \$2,200,000 & \$2,973,610 & 1.249\% \\
& & & & & \$1,600,000 & \$1,600,000 & \$1,800,000 & \$3,218,570 & 1.132\% \\
& & & & & \$1,600,000 & \$1,800,000 & \$1,400,000 & \$3,463,530 & 1.107\% \\
& & & & & \$1,200,000 & \$1,800,000 & \$2,800,000 & \$2,319,145 & 1.068\% \\
& & & & & \$1,000,000 & \$2,400,000 & \$2,400,000 & \$2,381,833 & 1.058\% \\
\hline
Other& & & & 2.871\% & Other& & & & 21.682\%  \\
\hline
\small \textbf{Game Value:} &&& \small \textbf{-27.57} && \small \textbf{Winner:} &&& \small \textbf{P2}&\\
\hline
\end{tabular}
\caption{Mixed strategy output, when P1 is given an initial budget of \$800,000, P2 is given an initial budget of \$1,600,000, and the Unweighted, Unfloored Utility Vector is used.}
\label{8v16table.4}
\end{figure}

\begin{figure}
\centering
\captionsetup{justification=raggedright}
\subfloat[P1: \$800,000 Initial Budget]{
\begin{tikzpicture}
\begin{axis}[
width=7.0cm,
ylabel={\tiny{Amount to Spend (\$1,000s)}},
xlabel={\tiny{Quarter}},
xmin=1, xmax=4,
ymin=-50, ymax=4000,
xtick={1,2,3,4},
ytick={0,1000,2000,3000,4000},
legend pos=north west,
ymajorgrids=true,
grid style=dashed,
]
\addplot[draw=gr1]
coordinates {(1,800)(2,400)(3,1400)(4,1419.365)};
\addplot[draw=gr2]
coordinates {(1,600)(2,1000)(3,1000)(4,1482.052)};
\addplot[draw=gr3]
coordinates {(1,800)(2,200)(3,1800)(4,1174.405)};
\addplot[draw=gr4]
coordinates {(1,600)(2,800)(3,1400)(4,1237.092)};
\addplot[draw=gr5]
coordinates {(1,800)(2,800)(3,800)(4,1709.285)};
\addplot[draw=gr7]
coordinates {(1,800)(2,600)(3,1200)(4,1464.325)};
\addplot[draw=gr10]
coordinates {(1,800)(2,400)(3,1600)(4,1219.365)};
\addplot[draw=gr15]
coordinates {(1,600)(2,1000)(3,1200)(4,1282.052)};
\addplot[draw=gr21]
coordinates {(1,800)(2,800)(3,1000)(4,1509.285)};
\addplot[draw=red]
coordinates {(1,800)(2,600)(3,1400)(4,1264.325)};
\end{axis}
\end{tikzpicture}
}
\qquad
\subfloat[P2: \$1,600,000 Initial Budget]{
\begin{tikzpicture}
\begin{axis}[
width=7.0cm,
ylabel={\tiny{Amount to Spend (\$1,000s)}},
xlabel={\tiny{Quarter}},
xmin=1, xmax=4,
ymin=-50, ymax=4000,
xtick={1,2,3,4},
ytick={0,1000,2000,3000,4000},
legend pos=north west,
ymajorgrids=true,
grid style=dashed,
]
\addplot[draw=gr1]
coordinates {(1,1000)(2,2400)(3,2400)(4,2381.833)};
\addplot[draw=gr1]
coordinates {(1,1200)(2,1800)(3,2800)(4,2319.145)};
\addplot[draw=gr1]
coordinates {(1,1600)(2,1800)(3,1400)(4,3463.53)};
\addplot[draw=gr1]
coordinates {(1,1600)(2,1600)(3,1800)(4,3218.57)};
\addplot[draw=gr1]
coordinates {(1,1600)(2,1400)(3,2200)(4,2973.61)};
\addplot[draw=gr1]
coordinates {(1,1600)(2,1200)(3,2600)(4,2728.65)};
\addplot[draw=gr1]
coordinates {(1,1400)(2,2000)(3,1800)(4,3036.297)};
\addplot[draw=gr1]
coordinates {(1,1600)(2,1000)(3,3000)(4,2483.69)};
\addplot[draw=gr1]
coordinates {(1,1400)(2,1800)(3,2200)(4,2791.337)};
\addplot[draw=gr2]
coordinates {(1,1200)(2,2200)(3,2200)(4,2609.065)};
\addplot[draw=gr2]
coordinates {(1,1400)(2,1600)(3,2600)(4,2546.377)};
\addplot[draw=gr2]
coordinates {(1,1400)(2,1400)(3,3000)(4,2301.417)};
\addplot[draw=gr2]
coordinates {(1,1200)(2,2000)(3,2600)(4,2364.105)};
\addplot[draw=gr2]
coordinates {(1,1600)(2,1800)(3,1600)(4,3263.53)};
\addplot[draw=gr2]
coordinates {(1,1600)(2,1600)(3,2000)(4,3018.57)};
\addplot[draw=gr2]
coordinates {(1,1400)(2,2000)(3,2000)(4,2836.297)};
\addplot[draw=gr2]
coordinates {(1,1600)(2,1400)(3,2400)(4,2773.61)};
\addplot[draw=gr2]
coordinates {(1,1600)(2,1200)(3,2800)(4,2528.65)};
\addplot[draw=gr2]
coordinates {(1,1400)(2,1800)(3,2400)(4,2591.337)};
\addplot[draw=gr2]
coordinates {(1,1600)(2,1000)(3,3200)(4,2283.69)};
\addplot[draw=gr2]
coordinates {(1,1200)(2,2200)(3,2400)(4,2409.065)};
\addplot[draw=gr3]
coordinates {(1,1400)(2,1600)(3,2800)(4,2346.377)};
\addplot[draw=gr3]
coordinates {(1,1600)(2,1800)(3,1800)(4,3063.53)};
\addplot[draw=gr3]
coordinates {(1,1600)(2,1600)(3,2200)(4,2818.57)};
\addplot[draw=gr3]
coordinates {(1,1600)(2,1400)(3,2600)(4,2573.61)};
\addplot[draw=gr3]
coordinates {(1,1400)(2,2000)(3,2200)(4,2636.297)};
\addplot[draw=gr3]
coordinates {(1,1600)(2,1200)(3,3000)(4,2328.65)};
\addplot[draw=gr3]
coordinates {(1,1400)(2,1800)(3,2600)(4,2391.337)};
\addplot[draw=gr4]
coordinates {(1,1600)(2,1800)(3,2000)(4,2863.53)};
\addplot[draw=gr4]
coordinates {(1,1600)(2,1600)(3,2400)(4,2618.57)};
\addplot[draw=gr4]
coordinates {(1,1600)(2,1400)(3,2800)(4,2373.61)};
\addplot[draw=gr4]
coordinates {(1,1400)(2,2000)(3,2400)(4,2436.297)};
\addplot[draw=gr4]
coordinates {(1,1600)(2,1800)(3,2200)(4,2663.53)};
\addplot[draw=blue]
coordinates {(1,1600)(2,1600)(3,2600)(4,2418.57)};
\end{axis}
\end{tikzpicture}
}
\caption{Overlaid strategies for P1 and P2. Colored line indicates top strategy. Darker gray lines indicate heavily used strategies, while lighter gray lines indicated lightly used strategies.}
\label{8v16.4}
\end{figure}

\newpage
\subsection{Weighted by State, Unfloored Utility Vector}
The Weighted by State, Unfloored Utility Vector is given by 
\begin{equation}
[ 0.353833059, -0.460278059, 0.204454969, -0.480740837 ]\cdot 2\times 10^{-5}.
\end{equation}
Unlike the Unweighted, Unfloored Utility Vector, this vector's values are all relatively close to each other. Additionally, players get punished in Q2 and Q4, and they are punished more heavily in these quarters than they are rewarded in Q1 and Q3. While this creates a very unrealistic model, it also creates some great graphs. The optimal strategy for any player, regardless of budget, is generally to minimize spending in Q2 and Q4 and maximize spending in Q1 and Q3. In fact, every strategy used >1\% of the time has a maximum spend in Q1. In most strategies, players spend \$0 in Q2, and the Q4 column is ordered roughly descending. These trends are evident from the gradients in the overlay graphs, especially in Q3 and Q4 (\textbf{Figures \ref{8v8.4}, \ref{8v9.4}, \ref{8v10.4}, \ref{8v11.4}, \ref{8v11.4}, \ref{8v12.4}, \ref{8v13.4}, \ref{8v14.4}, \ref{8v16.4}}). The optimal strategies create alternately upper and lower bounds for spending strategies --- an upper bound for Q1 (and a lower bound, in fact), lower bound for Q2, upper bound for Q3, and a near-lower bound for Q4.

The most exciting part about these simulations, however, is the winner - P1 actually begins winning games for the first time! This is also the first vector in which the negative values outweigh the positive ones. This punishes players more heavily for spending than in any previous game. As a result, players actually have an advantage by starting with a low budget, since it means they don't have to spend as much. The values are very close to zero, but they are definitively positive; P1 definitely has an advantage, for once.

\begin{figure}
\tiny
\begin{tabular}{ |p{1.0cm}p{1.0cm}p{1.0cm}p{2.0cm}|p{1.0cm}||p{1.0cm}p{1.0cm}p{1.0cm}p{2.0cm}|p{1.0cm}|}
\hline
\multicolumn{10}{|c|}{P1 \$800,000 vs. P2 \$800,000} \\
\hline
Player 1 & (P1) & & & & Player 2 & (P2) & & & \\
\hline
Q1 Strategy & Q2 Strategy & Q3 Strategy & Q4 Strategy  &  Play \% & Q1 Strategy & Q2 Strategy & Q3 Strategy & Q4 Strategy  &  Play \%\\
\hline
\$800,000 & \$0 & \$2,000,000 & \$1,129,445 & 29.356\% & \$800,000 & \$0 & \$2,000,000 & \$1,129,445 & 29.402\% \\
\$800,000 & \$0 & \$1,800,000 & \$1,329,445 & 20.984\% & \$800,000 & \$0 & \$1,800,000 & \$1,329,445 & 20.816\% \\
\$800,000 & \$0 & \$1,600,000 & \$1,529,445 & 14.859\% & \$800,000 & \$0 & \$1,600,000 & \$1,529,445 & 14.585\% \\
\$800,000 & \$0 & \$1,400,000 & \$1,729,445 & 10.314\% & \$800,000 & \$0 & \$1,400,000 & \$1,729,445 & 10.524\% \\
\$800,000 & \$0 & \$1,200,000 & \$1,929,445 & 7.285\% & \$800,000 & \$0 & \$1,200,000 & \$1,929,445 & 7.348\% \\
\$800,000 & \$0 & \$1,000,000 & \$2,129,445 & 5.014\% & \$800,000 & \$0 & \$1,000,000 & \$2,129,445 & 5.133\% \\
\$800,000 & \$200,000 & \$1,800,000 & \$1,174,405 & 3.715\% & \$800,000 & \$200,000 & \$1,800,000 & \$1,174,405 & 3.606\% \\
\$800,000 & \$0 & \$800,000 & \$2,329,445 & 2.601\% & \$800,000 & \$0 & \$800,000 & \$2,329,445 & 2.546\% \\
\$800,000 & \$200,000 & \$1,600,000 & \$1,374,405 & 1.761\% & \$800,000 & \$200,000 & \$1,600,000 & \$1,374,405 & 1.862\% \\
\$800,000 & \$0 & \$600,000 & \$2,529,445 & 1.229\% & \$800,000 & \$0 & \$600,000 & \$2,529,445 & 1.223\% \\
\hline
Other &  &  &  & 2.882 \% & Other &  &  &  & 2.955\% \\
\hline
\small \textbf{Game Value:} &&& \small \textbf{0.00} && \small \textbf{Winner:} &&& \small \textbf{P1}&\\
\hline
\end{tabular}
\caption{Mixed strategy output, when P1 is given an initial budget of \$800,000, P2 is given an initial budget of \$800,000, and the Weighted by State, Unfloored Utility Vector is used.}
\label{8v8table.5}
\end{figure}

\begin{figure}
\centering
\captionsetup{justification=raggedright}
\subfloat[P1: \$800,000 Initial Budget]{
\begin{tikzpicture}
\begin{axis}[
width=7.0cm,
ylabel={\tiny{Amount to Spend (\$1,000s)}},
xlabel={\tiny{Quarter}},
xmin=1, xmax=4,
ymin=-50, ymax=3000,
xtick={1,2,3,4},
ytick={0,1000,2000,3000},
legend pos=north west,
ymajorgrids=true,
grid style=dashed,
]
\addplot[draw=gr1]
coordinates {(1,800)(2,0)(3,600)(4,2529)};
\addplot[draw=gr2]
coordinates {(1,800)(2,200)(3,1600)(4,1374)};
\addplot[draw=gr3]
coordinates {(1,800)(2,0)(3,800)(4,2329)};
\addplot[draw=gr4]
coordinates {(1,800)(2,200)(3,1800)(4,1174)};
\addplot[draw=gr5]
coordinates {(1,800)(2,0)(3,1000)(4,2129)};
\addplot[draw=gr7]
coordinates {(1,800)(2,0)(3,1200)(4,1929)};
\addplot[draw=gr10]
coordinates {(1,800)(2,0)(3,1400)(4,1729)};
\addplot[draw=gr15]
coordinates {(1,800)(2,0)(3,1600)(4,1529)};
\addplot[draw=gr21]
coordinates {(1,800)(2,0)(3,1800)(4,1329)};
\addplot[draw=red]
coordinates {(1,800)(2,0)(3,2000)(4,1129)};
\end{axis}
\end{tikzpicture}
}
\qquad
\subfloat[P2: \$800,000 Initial Budget]{
\begin{tikzpicture}
\begin{axis}[
width=7.0cm,
ylabel={\tiny{Amount to Spend (\$1,000s)}},
xlabel={\tiny{Quarter}},
xmin=1, xmax=4,
ymin=-50, ymax=3000,
xtick={1,2,3,4},
ytick={0,1000,2000,3000},
legend pos=north west,
ymajorgrids=true,
grid style=dashed,
]
\addplot[draw=gr1]
coordinates {(1,800)(2,0)(3,600)(4,2529)};
\addplot[draw=gr2]
coordinates {(1,800)(2,200)(3,1600)(4,1374)};
\addplot[draw=gr3]
coordinates {(1,800)(2,0)(3,800)(4,2329)};
\addplot[draw=gr4]
coordinates {(1,800)(2,200)(3,1800)(4,1174)};
\addplot[draw=gr5]
coordinates {(1,800)(2,0)(3,1000)(4,2129)};
\addplot[draw=gr7]
coordinates {(1,800)(2,0)(3,1200)(4,1929)};
\addplot[draw=gr11]
coordinates {(1,800)(2,0)(3,1400)(4,1729)};
\addplot[draw=gr15]
coordinates {(1,800)(2,0)(3,1600)(4,1529)};
\addplot[draw=gr21]
coordinates {(1,800)(2,0)(3,1800)(4,1329)};
\addplot[draw=blue]
coordinates {(1,800)(2,0)(3,2000)(4,1129)};
\end{axis}
\end{tikzpicture}
}
\caption{Overlaid strategies for P1 and P2. Colored line indicates top strategy. Darker gray lines indicate heavily used strategies, while lighter gray lines indicated lightly used strategies.}
\label{8v8.5}
\end{figure}

\begin{figure}
\tiny
\begin{tabular}{ |p{1.0cm}p{1.0cm}p{1.0cm}p{2.0cm}|p{1.0cm}||p{1.0cm}p{1.0cm}p{1.0cm}p{2.0cm}|p{1.0cm}|}
\hline
\multicolumn{10}{|c|}{P1 \$800,000 vs. P2 \$900,000} \\
\hline
Player 1 & (P1) & & & & Player 2 & (P2) & & & \\
\hline
Q1 Strategy & Q2 Strategy & Q3 Strategy & Q4 Strategy  &  Play \% & Q1 Strategy & Q2 Strategy & Q3 Strategy & Q4 Strategy  &  Play \%\\
\hline
\$800,000 & \$0 & \$2,000,000 & \$1,129,445 & 29.277\% & \$800,000 & \$0 & \$2,200,000 & \$1,384,529 & 22.572\% \\
\$800,000 & \$0 & \$1,800,000 & \$1,329,445 & 20.868\% & \$800,000 & \$0 & \$2,000,000 & \$1,584,529 & 17.787\% \\
\$800,000 & \$0 & \$1,600,000 & \$1,529,445 & 14.872\% & \$800,000 & \$0 & \$1,800,000 & \$1,784,529 & 13.811\% \\
\$800,000 & \$0 & \$1,400,000 & \$1,729,445 & 10.424\% & \$800,000 & \$0 & \$1,600,000 & \$1,984,529 & 10.605\% \\
\$800,000 & \$0 & \$1,200,000 & \$1,929,445 & 7.195\% & \$800,000 & \$0 & \$1,400,000 & \$2,184,529 & 8.142\% \\
\$800,000 & \$0 & \$1,000,000 & \$2,129,445 & 5.256\% & \$800,000 & \$200,000 & \$2,200,000 & \$1,229,489 & 6.234\% \\
\$800,000 & \$200,000 & \$1,800,000 & \$1,174,405 & 3.596\% & \$800,000 & \$0 & \$1,200,000 & \$2,384,529 & 4.954\% \\
\$800,000 & \$0 & \$800,000 & \$2,329,445 & 2.554\% & \$800,000 & \$200,000 & \$2,000,000 & \$1,429,489 & 3.682\% \\
\$800,000 & \$200,000 & \$1,600,000 & \$1,374,405 & 1.774\% & \$800,000 & \$0 & \$1,000,000 & \$2,584,529 & 2.910\% \\
\$800,000 & \$0 & \$600,000 & \$2,529,445 & 1.258\% & \$800,000 & \$200,000 & \$1,800,000 & \$1,629,489 & 2.151\% \\
&  &  &  &  & \$800,000 & \$0 & \$800,000 & \$2,784,529 & 1.631\% \\
&  &  &  &  & \$800,000 & \$200,000 & \$1,600,000 & \$1,829,489 & 1.261\% \\
&  &  &  &  & \$800,000 & \$0 & \$600,000 & \$2,984,529 & 1.045\% \\
\hline
Other &  &  &  & 2.926\% & Other &  &  &  & 3.215\% \\
\hline
\small \textbf{Game Value:} &&& \small \textbf{0.53} && \small \textbf{Winner:} &&& \small \textbf{P1}&\\
\hline
\end{tabular}
\caption{Mixed strategy output, when P1 is given an initial budget of \$800,000, P2 is given an initial budget of \$900,000, and the Weighted by State, Unfloored Utility Vector is used.}
\label{8v9table.5}
\end{figure}

\begin{figure}
\centering
\captionsetup{justification=raggedright}
\subfloat[P1: \$800,000 Initial Budget]{
\begin{tikzpicture}
\begin{axis}[
width=7.0cm,
ylabel={\tiny{Amount to Spend (\$1,000s)}},
xlabel={\tiny{Quarter}},
xmin=1, xmax=4,
ymin=-50, ymax=3000,
xtick={1,2,3,4},
ytick={0,1000,2000,3000},
legend pos=north west,
ymajorgrids=true,
grid style=dashed,
]
\addplot[draw=gr1]
coordinates {(1,800)(2,0)(3,600)(4,2529)};
\addplot[draw=gr2]
coordinates {(1,800)(2,200)(3,1600)(4,1374)};
\addplot[draw=gr3]
coordinates {(1,800)(2,0)(3,800)(4,2329)};
\addplot[draw=gr4]
coordinates {(1,800)(2,200)(3,1800)(4,1174)};
\addplot[draw=gr5]
coordinates {(1,800)(2,0)(3,1000)(4,2129)};
\addplot[draw=gr7]
coordinates {(1,800)(2,0)(3,1200)(4,1929)};
\addplot[draw=gr10]
coordinates {(1,800)(2,0)(3,1400)(4,1729)};
\addplot[draw=gr15]
coordinates {(1,800)(2,0)(3,1600)(4,1529)};
\addplot[draw=gr21]
coordinates {(1,800)(2,0)(3,1800)(4,1329)};
\addplot[draw=red]
coordinates {(1,800)(2,0)(3,2000)(4,1129)};
\end{axis}
\end{tikzpicture}
}
\qquad
\subfloat[P2: \$900,000 Initial Budget]{
\begin{tikzpicture}
\begin{axis}[
width=7.0cm,
ylabel={\tiny{Amount to Spend (\$1,000s)}},
xlabel={\tiny{Quarter}},
xmin=1, xmax=4,
ymin=-50, ymax=3000,
xtick={1,2,3,4},
ytick={0,1000,2000,3000},
legend pos=north west,
ymajorgrids=true,
grid style=dashed,
]
\addplot[draw=gr1]
coordinates {(1,800)(2,0)(3,600)(4,2985)};
\addplot[draw=gr1]
coordinates {(1,800)(2,200)(3,1600)(4,1829)};
\addplot[draw=gr2]
coordinates {(1,800)(2,0)(3,800)(4,2785)};
\addplot[draw=gr2]
coordinates {(1,800)(2,200)(3,1800)(4,1629)};
\addplot[draw=gr3]
coordinates {(1,800)(2,0)(3,1000)(4,2585)};
\addplot[draw=gr4]
coordinates {(1,800)(2,200)(3,2000)(4,1429)};
\addplot[draw=gr5]
coordinates {(1,800)(2,0)(3,1200)(4,2385)};
\addplot[draw=gr6]
coordinates {(1,800)(2,200)(3,2200)(4,1229)};
\addplot[draw=gr8]
coordinates {(1,800)(2,0)(3,1400)(4,2185)};
\addplot[draw=gr11]
coordinates {(1,800)(2,0)(3,1600)(4,1985)};
\addplot[draw=gr14]
coordinates {(1,800)(2,0)(3,1800)(4,1785)};
\addplot[draw=gr18]
coordinates {(1,800)(2,0)(3,2000)(4,1585)};
\addplot[draw=blue]
coordinates {(1,800)(2,0)(3,2200)(4,1385)};
\end{axis}
\end{tikzpicture}
}
\caption{Overlaid strategies for P1 and P2. Colored line indicates top strategy. Darker gray lines indicate heavily used strategies, while lighter gray lines indicated lightly used strategies.}
\label{8v9.5}
\end{figure}

\begin{figure}
\tiny
\begin{tabular}{ |p{1.0cm}p{1.0cm}p{1.0cm}p{2.0cm}|p{1.0cm}||p{1.0cm}p{1.0cm}p{1.0cm}p{2.0cm}|p{1.0cm}|}
\hline
\multicolumn{10}{|c|}{P1 \$800,000 vs. P2 \$1,000,000} \\
\hline
Player 1 & (P1) & & & & Player 2 & (P2) & & & \\
\hline
Q1 Strategy & Q2 Strategy & Q3 Strategy & Q4 Strategy  &  Play \% & Q1 Strategy & Q2 Strategy & Q3 Strategy & Q4 Strategy  &  Play \%\\
\hline
\$800,000 & \$0 & \$2,000,000 & \$1,129,445 & 29.439\% & \$1,000,000 & \$0 & \$2,400,000 & \$1,511,806 & 16.177\% \\
\$800,000 & \$0 & \$1,800,000 & \$1,329,445 & 20.974\% & \$1,000,000 & \$0 & \$2,200,000 & \$1,711,806 & 13.551\% \\
\$800,000 & \$0 & \$1,600,000 & \$1,529,445 & 14.694\% & \$1,000,000 & \$0 & \$2,000,000 & \$1,911,806 & 11.515\% \\
\$800,000 & \$0 & \$1,400,000 & \$1,729,445 & 10.461\% & \$1,000,000 & \$0 & \$1,800,000 & \$2,111,806 & 9.556\% \\
\$800,000 & \$0 & \$1,200,000 & \$1,929,445 & 7.257\% & \$1,000,000 & \$0 & \$1,600,000 & \$2,311,806 & 7.957\% \\
\$800,000 & \$0 & \$1,000,000 & \$2,129,445 & 5.197\% & \$1,000,000 & \$0 & \$1,400,000 & \$2,511,806 & 6.703\% \\
\$800,000 & \$200,000 & \$1,800,000 & \$1,174,405 & 3.567\% & \$1,000,000 & \$200,000 & \$2,200,000 & \$1,556,766 & 5.530\% \\
\$800,000 & \$0 & \$800,000 & \$2,329,445 & 2.557\% & \$1,000,000 & \$0 & \$1,200,000 & \$2,711,806 & 4.712\% \\
\$800,000 & \$200,000 & \$1,600,000 & \$1,374,405 & 1.742\% & \$1,000,000 & \$200,000 & \$2,000,000 & \$1,756,766 & 4.001\% \\
\$800,000 & \$0 & \$600,000 & \$2,529,445 & 1.277\% & \$1,000,000 & \$0 & \$1,000,000 & \$2,911,806 & 3.290\% \\
&  &  &  &  & \$1,000,000 & \$200,000 & \$1,800,000 & \$1,956,766 & 2.792\% \\
&  &  &  &  & \$1,000,000 & \$0 & \$800,000 & \$3,111,806 & 2.300\% \\
&  &  &  &  & \$1,000,000 & \$200,000 & \$1,600,000 & \$2,156,766 & 1.903\% \\
&  &  &  &  & \$1,000,000 & \$0 & \$600,000 & \$3,311,806 & 1.655\% \\
&  &  &  &  & \$1,000,000 & \$200,000 & \$1,400,000 & \$2,356,766 & 1.475\% \\
&  &  &  &  & \$1,000,000 & \$400,000 & \$2,200,000 & \$1,401,726 & 1.160\% \\
\hline
Other &  &  &  & 2.835\% & Other &  &  &  & 5.723\% \\
\hline
\small \textbf{Game Value:} &&& \small \textbf{0.82} && \small \textbf{Winner:} &&& \small \textbf{P1}&\\
\hline
\end{tabular}
\caption{Mixed strategy output, when P1 is given an initial budget of \$800,000, P2 is given an initial budget of \$1,000,000, and the Weighted by State, Unfloored Utility Vector is used.}
\label{8v10table.5}
\end{figure}

\begin{figure}
\centering
\captionsetup{justification=raggedright}
\subfloat[P1: \$800,000 Initial Budget]{
\begin{tikzpicture}
\begin{axis}[
width=7.0cm,
ylabel={\tiny{Amount to Spend (\$1,000s)}},
xlabel={\tiny{Quarter}},
xmin=1, xmax=4,
ymin=-50, ymax=4000,
xtick={1,2,3,4},
ytick={0,1000,2000,3000,4000},
legend pos=north west,
ymajorgrids=true,
grid style=dashed,
]
\addplot[draw=gr1]
coordinates {(1,800)(2,0)(3,600)(4,2529)};
\addplot[draw=gr2]
coordinates {(1,800)(2,200)(3,1600)(4,1374)};
\addplot[draw=gr3]
coordinates {(1,800)(2,0)(3,800)(4,2329)};
\addplot[draw=gr4]
coordinates {(1,800)(2,200)(3,1800)(4,1174)};
\addplot[draw=gr5]
coordinates {(1,800)(2,0)(3,1000)(4,2129)};
\addplot[draw=gr7]
coordinates {(1,800)(2,0)(3,1200)(4,1929)};
\addplot[draw=gr10]
coordinates {(1,800)(2,0)(3,1400)(4,1729)};
\addplot[draw=gr15]
coordinates {(1,800)(2,0)(3,1600)(4,1529)};
\addplot[draw=gr21]
coordinates {(1,800)(2,0)(3,1800)(4,1329)};
\addplot[draw=red]
coordinates {(1,800)(2,0)(3,2000)(4,1129)};
\end{axis}
\end{tikzpicture}
}
\qquad
\subfloat[P2: \$1,000,000 Initial Budget]{
\begin{tikzpicture}
\begin{axis}[
width=7.0cm,
ylabel={\tiny{Amount to Spend (\$1,000s)}},
xlabel={\tiny{Quarter}},
xmin=1, xmax=4,
ymin=-50, ymax=4000,
xtick={1,2,3,4},
ytick={0,1000,2000,3000,4000},
legend pos=north west,
ymajorgrids=true,
grid style=dashed,
]
\addplot[draw=gr1]
coordinates {(1,1000)(2,400)(3,2200)(4,1402)};
\addplot[draw=gr1]
coordinates {(1,1000)(2,200)(3,1400)(4,2357)};
\addplot[draw=gr2]
coordinates {(1,1000)(2,0)(3,600)(4,3312)};
\addplot[draw=gr2]
coordinates {(1,1000)(2,200)(3,1600)(4,2157)};
\addplot[draw=gr2]
coordinates {(1,1000)(2,0)(3,800)(4,3112)};
\addplot[draw=gr3]
coordinates {(1,1000)(2,200)(3,1800)(4,1957)};
\addplot[draw=gr3]
coordinates {(1,1000)(2,0)(3,1000)(4,2912)};
\addplot[draw=gr4]
coordinates {(1,1000)(2,200)(3,2000)(4,1757)};
\addplot[draw=gr5]
coordinates {(1,1000)(2,0)(3,1200)(4,2712)};
\addplot[draw=gr6]
coordinates {(1,1000)(2,200)(3,2200)(4,1557)};
\addplot[draw=gr7]
coordinates {(1,1000)(2,0)(3,1400)(4,2512)};
\addplot[draw=gr8]
coordinates {(1,1000)(2,0)(3,1600)(4,2312)};
\addplot[draw=gr10]
coordinates {(1,1000)(2,0)(3,1800)(4,2112)};
\addplot[draw=gr12]
coordinates {(1,1000)(2,0)(3,2000)(4,1912)};
\addplot[draw=gr14]
coordinates {(1,1000)(2,0)(3,2200)(4,1712)};
\addplot[draw=blue]
coordinates {(1,1000)(2,0)(3,2400)(4,1512)};
\end{axis}
\end{tikzpicture}
}
\caption{Overlaid strategies for P1 and P2. Colored line indicates top strategy. Darker gray lines indicate heavily used strategies, while lighter gray lines indicated lightly used strategies.}
\label{8v10.5}
\end{figure}

\begin{figure}
\tiny
\begin{tabular}{ |p{1.0cm}p{1.0cm}p{1.0cm}p{2.0cm}|p{1.0cm}||p{1.0cm}p{1.0cm}p{1.0cm}p{2.0cm}|p{1.0cm}|}
\hline
\multicolumn{10}{|c|}{P1 \$800,000 vs. P2 \$1,100,000} \\
\hline
Player 1 & (P1) & & & & Player 2 & (P2) & & & \\
\hline
Q1 Strategy & Q2 Strategy & Q3 Strategy & Q4 Strategy  &  Play \% & Q1 Strategy & Q2 Strategy & Q3 Strategy & Q4 Strategy  &  Play \%\\
\hline
\$800,000 & \$0 & \$2,000,000 & \$1,129,445 & 29.508\% & \$1,000,000 & \$0 & \$2,800,000 & \$1,566,890 & 13.093\% \\
\$800,000 & \$0 & \$1,800,000 & \$1,329,445 & 20.818\% & \$1,000,000 & \$0 & \$2,600,000 & \$1,766,890 & 11.256\% \\
\$800,000 & \$0 & \$1,600,000 & \$1,529,445 & 14.522\% & \$1,000,000 & \$0 & \$2,400,000 & \$1,966,890 & 9.844\% \\
\$800,000 & \$0 & \$1,400,000 & \$1,729,445 & 10.393\% & \$1,000,000 & \$0 & \$2,200,000 & \$2,166,890 & 8.573\% \\
\$800,000 & \$0 & \$1,200,000 & \$1,929,445 & 7.325\% & \$1,000,000 & \$0 & \$2,000,000 & \$2,366,890 & 7.500\% \\
\$800,000 & \$0 & \$1,000,000 & \$2,129,445 & 5.295\% & \$1,000,000 & \$0 & \$1,800,000 & \$2,566,890 & 6.480\% \\
\$800,000 & \$200,000 & \$1,800,000 & \$1,174,405 & 3.645\% & \$1,000,000 & \$200,000 & \$2,600,000 & \$1,611,850 & 5.776\% \\
\$800,000 & \$0 & \$800,000 & \$2,329,445 & 2.460\% & \$1,000,000 & \$0 & \$1,600,000 & \$2,766,890 & 5.047\% \\
\$800,000 & \$200,000 & \$1,600,000 & \$1,374,405 & 1.856\% & \$1,000,000 & \$200,000 & \$2,400,000 & \$1,811,850 & 4.239\% \\
\$800,000 & \$0 & \$600,000 & \$2,529,445 & 1.281\% & \$1,000,000 & \$0 & \$1,400,000 & \$2,966,890 & 3.764\% \\
&  &  &  &  & \$1,000,000 & \$200,000 & \$2,200,000 & \$2,011,850 & 3.194\% \\
&  &  &  &  & \$1,000,000 & \$0 & \$1,200,000 & \$3,166,890 & 2.677\% \\
&  &  &  &  & \$1,000,000 & \$200,000 & \$2,000,000 & \$2,211,850 & 2.431\% \\
&  &  &  &  & \$1,000,000 & \$0 & \$1,000,000 & \$3,366,890 & 2.092\% \\
&  &  &  &  & \$1,000,000 & \$200,000 & \$1,800,000 & \$2,411,850 & 1.791\% \\
&  &  &  &  & \$1,000,000 & \$0 & \$800,000 & \$3,566,890 & 1.641\% \\
&  &  &  &  & \$1,000,000 & \$200,000 & \$1,600,000 & \$2,611,850 & 1.439\% \\
&  &  &  &  & \$1,000,000 & \$400,000 & \$2,400,000 & \$1,656,810 & 1.232\% \\
&  &  &  &  & \$1,000,000 & \$0 & \$600,000 & \$3,766,890 & 1.092\% \\
\hline
Other &  &  &  & 2.897\% & Other &  &  &  & 6.839\% \\
\hline
\small \textbf{Game Value:} &&& \small \textbf{0.90} && \small \textbf{Winner:} &&& \small \textbf{P1}&\\
\hline
\end{tabular}
\caption{Mixed strategy output, when P1 is given an initial budget of \$800,000, P2 is given an initial budget of \$1,100,000, and the Weighted by State, Unfloored Utility Vector is used.}
\label{8v11table.5}
\end{figure}

\begin{figure}
\centering
\captionsetup{justification=raggedright}
\subfloat[P1: \$800,000 Initial Budget]{
\begin{tikzpicture}
\begin{axis}[
width=7.0cm,
ylabel={\tiny{Amount to Spend (\$1,000s)}},
xlabel={\tiny{Quarter}},
xmin=1, xmax=4,
ymin=-50, ymax=4000,
xtick={1,2,3,4},
ytick={0,1000,2000,3000,4000},
legend pos=north west,
ymajorgrids=true,
grid style=dashed,
]
\addplot[draw=gr1]
coordinates {(1,800)(2,0)(3,600)(4,2529)};
\addplot[draw=gr2]
coordinates {(1,800)(2,200)(3,1600)(4,1374)};
\addplot[draw=gr2]
coordinates {(1,800)(2,0)(3,800)(4,2329)};
\addplot[draw=gr4]
coordinates {(1,800)(2,200)(3,1800)(4,1174)};
\addplot[draw=gr5]
coordinates {(1,800)(2,0)(3,1000)(4,2129)};
\addplot[draw=gr7]
coordinates {(1,800)(2,0)(3,1200)(4,1929)};
\addplot[draw=gr10]
coordinates {(1,800)(2,0)(3,1400)(4,1729)};
\addplot[draw=gr15]
coordinates {(1,800)(2,0)(3,1600)(4,1529)};
\addplot[draw=gr21]
coordinates {(1,800)(2,0)(3,1800)(4,1329)};
\addplot[draw=red]
coordinates {(1,800)(2,0)(3,2000)(4,1129)};
\end{axis}
\end{tikzpicture}
}
\qquad
\subfloat[P2: \$1,100,000 Initial Budget]{
\begin{tikzpicture}
\begin{axis}[
width=7.0cm,
ylabel={\tiny{Amount to Spend (\$1,000s)}},
xlabel={\tiny{Quarter}},
xmin=1, xmax=4,
ymin=-50, ymax=4000,
xtick={1,2,3,4},
ytick={0,1000,2000,3000,4000},
legend pos=north west,
ymajorgrids=true,
grid style=dashed,
]
\addplot[draw=gr1]
coordinates {(1,1000)(2,0)(3,600)(4,3767)};
\addplot[draw=gr1]
coordinates {(1,1000)(2,400)(3,2400)(4,1657)};
\addplot[draw=gr1]
coordinates {(1,1000)(2,200)(3,1600)(4,2612)};
\addplot[draw=gr2]
coordinates {(1,1000)(2,0)(3,800)(4,3567)};
\addplot[draw=gr2]
coordinates {(1,1000)(2,200)(3,1800)(4,2412)};
\addplot[draw=gr2]
coordinates {(1,1000)(2,0)(3,1000)(4,3367)};
\addplot[draw=gr2]
coordinates {(1,1000)(2,200)(3,2000)(4,2212)};
\addplot[draw=gr3]
coordinates {(1,1000)(2,0)(3,1200)(4,3167)};
\addplot[draw=gr3]
coordinates {(1,1000)(2,200)(3,2200)(4,2012)};
\addplot[draw=gr4]
coordinates {(1,1000)(2,0)(3,1400)(4,2967)};
\addplot[draw=gr4]
coordinates {(1,1000)(2,200)(3,2400)(4,1812)};
\addplot[draw=gr5]
coordinates {(1,1000)(2,0)(3,1600)(4,2767)};
\addplot[draw=gr6]
coordinates {(1,1000)(2,200)(3,2600)(4,1612)};
\addplot[draw=gr6]
coordinates {(1,1000)(2,0)(3,1800)(4,2567)};
\addplot[draw=gr8]
coordinates {(1,1000)(2,0)(3,2000)(4,2367)};
\addplot[draw=gr9]
coordinates {(1,1000)(2,0)(3,2200)(4,2167)};
\addplot[draw=gr10]
coordinates {(1,1000)(2,0)(3,2400)(4,1967)};
\addplot[draw=gr11]
coordinates {(1,1000)(2,0)(3,2600)(4,1767)};
\addplot[draw=blue]
coordinates {(1,1000)(2,0)(3,2800)(4,1567)};
\end{axis}
\end{tikzpicture}
}
\caption{Overlaid strategies for P1 and P2. Colored line indicates top strategy. Darker gray lines indicate heavily used strategies, while lighter gray lines indicated lightly used strategies.}
\label{8v11.5}
\end{figure}

\begin{figure}
\tiny
\begin{tabular}{ |p{1.0cm}p{1.0cm}p{1.0cm}p{2.0cm}|p{1.0cm}||p{1.0cm}p{1.0cm}p{1.0cm}p{2.0cm}|p{1.0cm}|}
\hline
\multicolumn{10}{|c|}{P1 \$800,000 vs. P2 \$1,200,000} \\
\hline
Player 1 & (P1) & & & & Player 2 & (P2) & & & \\
\hline
Q1 Strategy & Q2 Strategy & Q3 Strategy & Q4 Strategy  &  Play \% & Q1 Strategy & Q2 Strategy & Q3 Strategy & Q4 Strategy  &  Play \%\\
\hline
\$800,000 & \$0 & \$2,000,000 & \$1,129,445 & 29.223\% & \$1,200,000 & \$0 & \$3,000,000 & \$1,694,167 & 9.734\% \\
\$800,000 & \$0 & \$1,800,000 & \$1,329,445 & 20.952\% & \$1,200,000 & \$0 & \$2,800,000 & \$1,894,167 & 8.767\% \\
\$800,000 & \$0 & \$1,600,000 & \$1,529,445 & 14.677\% & \$1,200,000 & \$0 & \$2,600,000 & \$2,094,167 & 7.889\% \\
\$800,000 & \$0 & \$1,400,000 & \$1,729,445 & 10.417\% & \$1,200,000 & \$0 & \$2,400,000 & \$2,294,167 & 7.119\% \\
\$800,000 & \$0 & \$1,200,000 & \$1,929,445 & 7.284\% & \$1,200,000 & \$0 & \$2,200,000 & \$2,494,167 & 6.457\% \\
\$800,000 & \$0 & \$1,000,000 & \$2,129,445 & 5.304\% & \$1,200,000 & \$0 & \$2,000,000 & \$2,694,167 & 5.817\% \\
\$800,000 & \$200,000 & \$1,800,000 & \$1,174,405 & 3.730\% & \$1,200,000 & \$200,000 & \$2,800,000 & \$1,739,127 & 5.282\% \\
\$800,000 & \$0 & \$800,000 & \$2,329,445 & 2.556\% & \$1,200,000 & \$0 & \$1,800,000 & \$2,894,167 & 4.914\% \\
\$800,000 & \$200,000 & \$1,600,000 & \$1,374,405 & 1.846\% & \$1,200,000 & \$200,000 & \$2,600,000 & \$1,939,127 & 4.378\% \\
\$800,000 & \$0 & \$600,000 & \$2,529,445 & 1.198\% & \$1,200,000 & \$0 & \$1,600,000 & \$3,094,167 & 3.859\% \\
&  &  &  &  & \$1,200,000 & \$200,000 & \$2,400,000 & \$2,139,127 & 3.458\% \\
&  &  &  &  & \$1,200,000 & \$0 & \$1,400,000 & \$3,294,167 & 3.163\% \\
&  &  &  &  & \$1,200,000 & \$200,000 & \$2,200,000 & \$2,339,127 & 2.836\% \\
&  &  &  &  & \$1,200,000 & \$0 & \$1,200,000 & \$3,494,167 & 2.596\% \\
&  &  &  &  & \$1,200,000 & \$200,000 & \$2,000,000 & \$2,539,127 & 2.241\% \\
&  &  &  &  & \$1,200,000 & \$0 & \$1,000,000 & \$3,694,167 & 2.098\% \\
&  &  &  &  & \$1,200,000 & \$200,000 & \$1,800,000 & \$2,739,127 & 1.946\% \\
&  &  &  &  & \$1,200,000 & \$400,000 & \$2,600,000 & \$1,784,087 & 1.739\% \\
&  &  &  &  & \$1,200,000 & \$0 & \$800,000 & \$3,894,167 & 1.514\% \\
&  &  &  &  & \$1,200,000 & \$200,000 & \$1,600,000 & \$2,939,127 & 1.372\% \\
&  &  &  &  & \$1,200,000 & \$400,000 & \$2,400,000 & \$1,984,087 & 1.292\% \\
&  &  &  &  & \$1,200,000 & \$0 & \$600,000 & \$4,094,167 & 1.153\% \\
\hline
Other &  &  &  & 2.813\% & Other &  &  &  & 10.376\% \\
\hline
\small \textbf{Game Value:} &&& \small \textbf{1.01} && \small \textbf{Winner:} &&& \small \textbf{P1}&\\
\hline
\end{tabular}
\caption{Mixed strategy output, when P1 is given an initial budget of \$800,000, P2 is given an initial budget of \$1,200,000, and the Weighted by State, Unfloored Utility Vector is used.}
\label{8v12table.5}
\end{figure}

\begin{figure}
\centering
\captionsetup{justification=raggedright}
\subfloat[P1: \$800,000 Initial Budget]{
\begin{tikzpicture}
\begin{axis}[
width=7.0cm,
ylabel={\tiny{Amount to Spend (\$1,000s)}},
xlabel={\tiny{Quarter}},
xmin=1, xmax=4,
ymin=-50, ymax=5000,
xtick={1,2,3,4},
ytick={0,1000,2000,3000,4000,5000},
legend pos=north west,
ymajorgrids=true,
grid style=dashed,
]
\addplot[draw=gr1]
coordinates {(1,800)(2,0)(3,600)(4,2529)};
\addplot[draw=gr2]
coordinates {(1,800)(2,200)(3,1600)(4,1374)};
\addplot[draw=gr3]
coordinates {(1,800)(2,0)(3,800)(4,2329)};
\addplot[draw=gr4]
coordinates {(1,800)(2,200)(3,1800)(4,1174)};
\addplot[draw=gr5]
coordinates {(1,800)(2,0)(3,1000)(4,2129)};
\addplot[draw=gr7]
coordinates {(1,800)(2,0)(3,1200)(4,1929)};
\addplot[draw=gr10]
coordinates {(1,800)(2,0)(3,1400)(4,1729)};
\addplot[draw=gr15]
coordinates {(1,800)(2,0)(3,1600)(4,1529)};
\addplot[draw=gr21]
coordinates {(1,800)(2,0)(3,1800)(4,1329)};
\addplot[draw=red]
coordinates {(1,800)(2,0)(3,2000)(4,1129)};
\end{axis}
\end{tikzpicture}
}
\qquad
\subfloat[P2: \$1,200,000 Initial Budget]{
\begin{tikzpicture}
\begin{axis}[
width=7.0cm,
ylabel={\tiny{Amount to Spend (\$1,000s)}},
xlabel={\tiny{Quarter}},
xmin=1, xmax=4,
ymin=-50, ymax=5000,
xtick={1,2,3,4},
ytick={0,1000,2000,3000,4000,5000},
legend pos=north west,
ymajorgrids=true,
grid style=dashed,
]
\addplot[draw=gr1]
coordinates {(1,1200)(2,0)(3,600)(4,4094)};
\addplot[draw=gr1]
coordinates {(1,1200)(2,400)(3,2400)(4,1984)};
\addplot[draw=gr1]
coordinates {(1,1200)(2,200)(3,1600)(4,2939)};
\addplot[draw=gr2]
coordinates {(1,1200)(2,0)(3,800)(4,3894)};
\addplot[draw=gr2]
coordinates {(1,1200)(2,400)(3,2600)(4,1784)};
\addplot[draw=gr2]
coordinates {(1,1200)(2,200)(3,1800)(4,2739)};
\addplot[draw=gr2]
coordinates {(1,1200)(2,0)(3,1000)(4,3694)};
\addplot[draw=gr2]
coordinates {(1,1200)(2,200)(3,2000)(4,2539)};
\addplot[draw=gr3]
coordinates {(1,1200)(2,0)(3,1200)(4,3494)};
\addplot[draw=gr3]
coordinates {(1,1200)(2,200)(3,2200)(4,2339)};
\addplot[draw=gr3]
coordinates {(1,1200)(2,0)(3,1400)(4,3294)};
\addplot[draw=gr3]
coordinates {(1,1200)(2,200)(3,2400)(4,2139)};
\addplot[draw=gr4]
coordinates {(1,1200)(2,0)(3,1600)(4,3094)};
\addplot[draw=gr4]
coordinates {(1,1200)(2,200)(3,2600)(4,1939)};
\addplot[draw=gr5]
coordinates {(1,1200)(2,0)(3,1800)(4,2894)};
\addplot[draw=gr5]
coordinates {(1,1200)(2,200)(3,2800)(4,1739)};
\addplot[draw=gr6]
coordinates {(1,1200)(2,0)(3,2000)(4,2694)};
\addplot[draw=gr6]
coordinates {(1,1200)(2,0)(3,2200)(4,2494)};
\addplot[draw=gr7]
coordinates {(1,1200)(2,0)(3,2400)(4,2294)};
\addplot[draw=gr8]
coordinates {(1,1200)(2,0)(3,2600)(4,2094)};
\addplot[draw=gr9]
coordinates {(1,1200)(2,0)(3,2800)(4,1894)};
\addplot[draw=blue]
coordinates {(1,1200)(2,0)(3,3000)(4,1694)};
\end{axis}
\end{tikzpicture}
}
\caption{Overlaid strategies for P1 and P2. Colored line indicates top strategy. Darker gray lines indicate heavily used strategies, while lighter gray lines indicated lightly used strategies.}
\label{8v12.5}
\end{figure}

\begin{figure}
\tiny
\begin{tabular}{ |p{1.0cm}p{1.0cm}p{1.0cm}p{2.0cm}|p{1.0cm}||p{1.0cm}p{1.0cm}p{1.0cm}p{2.0cm}|p{1.0cm}|}
\hline
\multicolumn{10}{|c|}{P1 \$800,000 vs. P2 \$1,300,000} \\
\hline
Player 1 & (P1) & & & & Player 2 & (P2) & & & \\
\hline
Q1 Strategy & Q2 Strategy & Q3 Strategy & Q4 Strategy  &  Play \% & Q1 Strategy & Q2 Strategy & Q3 Strategy & Q4 Strategy  &  Play \%\\
\hline
\$800,000 & \$0 & \$2,000,000 & \$1,129,445 & 29.183\% & \$1,200,000 & \$0 & \$3,400,000 & \$1,749,252 & 8.103\% \\
\$800,000 & \$0 & \$1,800,000 & \$1,329,445 & 20.896\% & \$1,200,000 & \$0 & \$3,200,000 & \$1,949,252 & 7.415\% \\
\$800,000 & \$0 & \$1,600,000 & \$1,529,445 & 14.778\% & \$1,200,000 & \$0 & \$3,000,000 & \$2,149,252 & 6.943\% \\
\$800,000 & \$0 & \$1,400,000 & \$1,729,445 & 10.455\% & \$1,200,000 & \$0 & \$2,800,000 & \$2,349,252 & 6.272\% \\
\$800,000 & \$0 & \$1,200,000 & \$1,929,445 & 7.367\% & \$1,200,000 & \$0 & \$2,600,000 & \$2,549,252 & 5.856\% \\
\$800,000 & \$0 & \$1,000,000 & \$2,129,445 & 5.243\% & \$1,200,000 & \$0 & \$2,400,000 & \$2,749,252 & 5.351\% \\
\$800,000 & \$200,000 & \$1,800,000 & \$1,174,405 & 3.587\% & \$1,200,000 & \$200,000 & \$3,200,000 & \$1,794,212 & 4.958\% \\
\$800,000 & \$0 & \$800,000 & \$2,329,445 & 2.511\% & \$1,200,000 & \$0 & \$2,200,000 & \$2,949,252 & 4.433\% \\
\$800,000 & \$200,000 & \$1,600,000 & \$1,374,405 & 1.866\% & \$1,200,000 & \$200,000 & \$3,000,000 & \$1,994,212 & 4.129\% \\
\$800,000 & \$0 & \$600,000 & \$2,529,445 & 1.242\% & \$1,200,000 & \$0 & \$2,000,000 & \$3,149,252 & 3.847\% \\
&  &  &  &  & \$1,200,000 & \$200,000 & \$2,800,000 & \$2,194,212 & 3.360\% \\
&  &  &  &  & \$1,200,000 & \$0 & \$1,800,000 & \$3,349,252 & 3.292\% \\
&  &  &  &  & \$1,200,000 & \$200,000 & \$2,600,000 & \$2,394,212 & 2.990\% \\
&  &  &  &  & \$1,200,000 & \$0 & \$1,600,000 & \$3,549,252 & 2.698\% \\
&  &  &  &  & \$1,200,000 & \$200,000 & \$2,400,000 & \$2,594,212 & 2.409\% \\
&  &  &  &  & \$1,200,000 & \$0 & \$1,400,000 & \$3,749,252 & 2.217\% \\
&  &  &  &  & \$1,200,000 & \$200,000 & \$2,200,000 & \$2,794,212 & 2.001\% \\
&  &  &  &  & \$1,200,000 & \$400,000 & \$3,000,000 & \$1,839,172 & 1.974\% \\
&  &  &  &  & \$1,200,000 & \$0 & \$1,200,000 & \$3,949,252 & 1.815\% \\
&  &  &  &  & \$1,200,000 & \$200,000 & \$2,000,000 & \$2,994,212 & 1.625\% \\
&  &  &  &  & \$1,200,000 & \$400,000 & \$2,800,000 & \$2,039,172 & 1.517\% \\
&  &  &  &  & \$1,200,000 & \$0 & \$1,000,000 & \$4,149,252 & 1.356\% \\
&  &  &  &  & \$1,200,000 & \$200,000 & \$1,800,000 & \$3,194,212 & 1.301\% \\
&  &  &  &  & \$1,200,000 & \$400,000 & \$2,600,000 & \$2,239,172 & 1.165\% \\
&  &  &  &  & \$1,200,000 & \$0 & \$800,000 & \$4,349,252 & 1.083\% \\
\hline
Other &  &  &  & 2.872\% & Other &  &  &  & 11.89\% \\
\hline
\small \textbf{Game Value:} &&& \small \textbf{1.00} && \small \textbf{Winner:} &&& \small \textbf{P1}&\\
\hline
\end{tabular}
\caption{Mixed strategy output, when P1 is given an initial budget of \$800,000, P2 is given an initial budget of \$1,300,000, and the Weighted by State, Unfloored Utility Vector is used.}
\label{8v13table.5}
\end{figure}

\begin{figure}
\centering
\captionsetup{justification=raggedright}
\subfloat[P1: \$800,000 Initial Budget]{
\begin{tikzpicture}
\begin{axis}[
width=7.0cm,
ylabel={\tiny{Amount to Spend (\$1,000s)}},
xlabel={\tiny{Quarter}},
xmin=1, xmax=4,
ymin=-50, ymax=5000,
xtick={1,2,3,4},
ytick={0,1000,2000,3000,4000,5000},
legend pos=north west,
ymajorgrids=true,
grid style=dashed,
]
\addplot[draw=gr1]
coordinates {(1,800)(2,0)(3,600)(4,2529)};
\addplot[draw=gr2]
coordinates {(1,800)(2,200)(3,1600)(4,1374)};
\addplot[draw=gr3]
coordinates {(1,800)(2,0)(3,800)(4,2329)};
\addplot[draw=gr4]
coordinates {(1,800)(2,200)(3,1800)(4,1174)};
\addplot[draw=gr5]
coordinates {(1,800)(2,0)(3,1000)(4,2129)};
\addplot[draw=gr7]
coordinates {(1,800)(2,0)(3,1200)(4,1929)};
\addplot[draw=gr10]
coordinates {(1,800)(2,0)(3,1400)(4,1729)};
\addplot[draw=gr15]
coordinates {(1,800)(2,0)(3,1600)(4,1529)};
\addplot[draw=gr21]
coordinates {(1,800)(2,0)(3,1800)(4,1329)};
\addplot[draw=red]
coordinates {(1,800)(2,0)(3,2000)(4,1129)};
\end{axis}
\end{tikzpicture}
}
\qquad
\subfloat[P2: \$1,300,000 Initial Budget]{
\begin{tikzpicture}
\begin{axis}[
width=7.0cm,
ylabel={\tiny{Amount to Spend (\$1,000s)}},
xlabel={\tiny{Quarter}},
xmin=1, xmax=4,
ymin=-50, ymax=5000,
xtick={1,2,3,4},
ytick={0,1000,2000,3000,4000,5000},
legend pos=north west,
ymajorgrids=true,
grid style=dashed,
]
\addplot[draw=gr1]
coordinates {(1,1200)(2,0)(3,800)(4,4349)};
\addplot[draw=gr1]
coordinates {(1,1200)(2,400)(3,2600)(4,2239)};
\addplot[draw=gr1]
coordinates {(1,1200)(2,200)(3,1800)(4,3194)};
\addplot[draw=gr1]
coordinates {(1,1200)(2,0)(3,1000)(4,4149)};
\addplot[draw=gr2]
coordinates {(1,1200)(2,400)(3,2800)(4,2039)};
\addplot[draw=gr2]
coordinates {(1,1200)(2,200)(3,2000)(4,2994)};
\addplot[draw=gr2]
coordinates {(1,1200)(2,0)(3,1200)(4,3949)};
\addplot[draw=gr2]
coordinates {(1,1200)(2,400)(3,3000)(4,1839)};
\addplot[draw=gr2]
coordinates {(1,1200)(2,200)(3,2200)(4,2794)};
\addplot[draw=gr2]
coordinates {(1,1200)(2,0)(3,1400)(4,3749)};
\addplot[draw=gr2]
coordinates {(1,1200)(2,200)(3,2400)(4,2594)};
\addplot[draw=gr3]
coordinates {(1,1200)(2,0)(3,1600)(4,3549)};
\addplot[draw=gr3]
coordinates {(1,1200)(2,200)(3,2600)(4,2394)};
\addplot[draw=gr3]
coordinates {(1,1200)(2,0)(3,1800)(4,3349)};
\addplot[draw=gr3]
coordinates {(1,1200)(2,200)(3,2800)(4,2194)};
\addplot[draw=gr4]
coordinates {(1,1200)(2,0)(3,2000)(4,3149)};
\addplot[draw=gr4]
coordinates {(1,1200)(2,200)(3,3000)(4,1994)};
\addplot[draw=gr4]
coordinates {(1,1200)(2,0)(3,2200)(4,2949)};
\addplot[draw=gr5]
coordinates {(1,1200)(2,200)(3,3200)(4,1794)};
\addplot[draw=gr5]
coordinates {(1,1200)(2,0)(3,2400)(4,2749)};
\addplot[draw=gr6]
coordinates {(1,1200)(2,0)(3,2600)(4,2549)};
\addplot[draw=gr6]
coordinates {(1,1200)(2,0)(3,2800)(4,2349)};
\addplot[draw=gr7]
coordinates {(1,1200)(2,0)(3,3000)(4,2149)};
\addplot[draw=gr7]
coordinates {(1,1200)(2,0)(3,3200)(4,1949)};
\addplot[draw=blue]
coordinates {(1,1200)(2,0)(3,3400)(4,1749)};
\end{axis}
\end{tikzpicture}
}
\caption{Overlaid strategies for P1 and P2. Colored line indicates top strategy. Darker gray lines indicate heavily used strategies, while lighter gray lines indicated lightly used strategies.}
\label{8v13.5}
\end{figure}

\begin{figure}
\tiny
\begin{tabular}{ |p{1.0cm}p{1.0cm}p{1.0cm}p{2.0cm}|p{1.0cm}||p{1.0cm}p{1.0cm}p{1.0cm}p{2.0cm}|p{1.0cm}|}
\hline
\multicolumn{10}{|c|}{P1 \$800,000 vs. P2 \$1,400,000} \\
\hline
Player 1 & (P1) & & & & Player 2 & (P2) & & & \\
\hline
Q1 Strategy & Q2 Strategy & Q3 Strategy & Q4 Strategy  &  Play \% & Q1 Strategy & Q2 Strategy & Q3 Strategy & Q4 Strategy  &  Play \%\\
\hline
\$800,000 & \$0 & \$2,000,000 & \$1,129,445 & 29.458\% & \$1,400,000 & \$0 & \$3,400,000 & \$2,076,528 & 6.407\% \\
\$800,000 & \$0 & \$1,800,000 & \$1,329,445 & 20.927\% & \$1,400,000 & \$0 & \$3,200,000 & \$2,276,528 & 5.917\% \\
\$800,000 & \$0 & \$1,600,000 & \$1,529,445 & 14.641\% & \$1,400,000 & \$0 & \$3,000,000 & \$2,476,528 & 5.480\% \\
\$800,000 & \$0 & \$1,400,000 & \$1,729,445 & 10.332\% & \$1,400,000 & \$0 & \$2,800,000 & \$2,676,528 & 5.257\% \\
\$800,000 & \$0 & \$1,200,000 & \$1,929,445 & 7.250\% & \$1,400,000 & \$0 & \$2,600,000 & \$2,876,528 & 4.841\% \\
\$800,000 & \$0 & \$1,000,000 & \$2,129,445 & 5.165\% & \$1,400,000 & \$200,000 & \$3,400,000 & \$1,921,488 & 4.538\% \\
\$800,000 & \$200,000 & \$1,800,000 & \$1,174,405 & 3.697\% & \$1,400,000 & \$0 & \$2,400,000 & \$3,076,528 & 4.291\% \\
\$800,000 & \$0 & \$800,000 & \$2,329,445 & 2.496\% & \$1,400,000 & \$200,000 & \$3,200,000 & \$2,121,488 & 4.084\% \\
\$800,000 & \$200,000 & \$1,600,000 & \$1,374,405 & 1.867\% & \$1,400,000 & \$0 & \$2,200,000 & \$3,276,528 & 3.742\% \\
\$800,000 & \$0 & \$600,000 & \$2,529,445 & 1.208\% & \$1,400,000 & \$200,000 & \$3,000,000 & \$2,321,488 & 3.522\% \\
&  &  &  &  & \$1,400,000 & \$0 & \$2,000,000 & \$3,476,528 & 3.206\% \\
&  &  &  &  & \$1,400,000 & \$200,000 & \$2,800,000 & \$2,521,488 & 3.082\% \\
&  &  &  &  & \$1,400,000 & \$0 & \$1,800,000 & \$3,676,528 & 2.916\% \\
&  &  &  &  & \$1,400,000 & \$200,000 & \$2,600,000 & \$2,721,488 & 2.664\% \\
&  &  &  &  & \$1,400,000 & \$0 & \$1,600,000 & \$3,876,528 & 2.616\% \\
&  &  &  &  & \$1,400,000 & \$200,000 & \$2,400,000 & \$2,921,488 & 2.438\% \\
&  &  &  &  & \$1,400,000 & \$400,000 & \$3,200,000 & \$1,966,448 & 2.245\% \\
&  &  &  &  & \$1,400,000 & \$0 & \$1,400,000 & \$4,076,528 & 2.086\% \\
&  &  &  &  & \$1,400,000 & \$200,000 & \$2,200,000 & \$3,121,488 & 1.991\% \\
&  &  &  &  & \$1,400,000 & \$400,000 & \$3,000,000 & \$2,166,448 & 1.759\% \\
&  &  &  &  & \$1,400,000 & \$0 & \$1,200,000 & \$4,276,528 & 1.734\% \\
&  &  &  &  & \$1,400,000 & \$200,000 & \$2,000,000 & \$3,321,488 & 1.563\% \\
&  &  &  &  & \$1,400,000 & \$400,000 & \$2,800,000 & \$2,366,448 & 1.519\% \\
&  &  &  &  & \$1,400,000 & \$0 & \$1,000,000 & \$4,476,528 & 1.438\% \\
&  &  &  &  & \$1,400,000 & \$200,000 & \$1,800,000 & \$3,521,488 & 1.328\% \\
&  &  &  &  & \$1,400,000 & \$400,000 & \$2,600,000 & \$2,566,448 & 1.235\% \\
&  &  &  &  & \$1,400,000 & \$0 & \$800,000 & \$4,676,528 & 1.148\% \\
&  &  &  &  & \$1,400,000 & \$200,000 & \$1,600,000 & \$3,721,488 & 1.111\% \\
\hline
Other &  &  &  & 2.959\% & Other &  &  &  & 15.842\% \\
\hline
\small \textbf{Game Value:} &&& \small \textbf{1.40} && \small \textbf{Winner:} &&& \small \textbf{P1}&\\
\hline
\end{tabular}
\caption{Mixed strategy output, when P1 is given an initial budget of \$800,000, P2 is given an initial budget of \$1,400,000, and the Weighted by State, Unfloored Utility Vector is used.}
\label{8v14table.5}
\end{figure}

\begin{figure}
\centering
\captionsetup{justification=raggedright}
\subfloat[P1: \$800,000 Initial Budget]{
\begin{tikzpicture}
\begin{axis}[
width=7.0cm,
ylabel={\tiny{Amount to Spend (\$1,000s)}},
xlabel={\tiny{Quarter}},
xmin=1, xmax=4,
ymin=-50, ymax=5000,
xtick={1,2,3,4},
ytick={0,1000,2000,3000,4000,5000},
legend pos=north west,
ymajorgrids=true,
grid style=dashed,
]
\addplot[draw=gr1]
coordinates {(1,800)(2,0)(3,600)(4,2529)};
\addplot[draw=gr2]
coordinates {(1,800)(2,200)(3,1600)(4,1374)};
\addplot[draw=gr2]
coordinates {(1,800)(2,0)(3,800)(4,2329)};
\addplot[draw=gr4]
coordinates {(1,800)(2,200)(3,1800)(4,1174)};
\addplot[draw=gr5]
coordinates {(1,800)(2,0)(3,1000)(4,2129)};
\addplot[draw=gr7]
coordinates {(1,800)(2,0)(3,1200)(4,1929)};
\addplot[draw=gr10]
coordinates {(1,800)(2,0)(3,1400)(4,1729)};
\addplot[draw=gr15]
coordinates {(1,800)(2,0)(3,1600)(4,1529)};
\addplot[draw=gr21]
coordinates {(1,800)(2,0)(3,1800)(4,1329)};
\addplot[draw=red]
coordinates {(1,800)(2,0)(3,2000)(4,1129)};
\end{axis}
\end{tikzpicture}
}
\qquad
\subfloat[P2: \$1,400,000 Initial Budget]{
\begin{tikzpicture}
\begin{axis}[
width=7.0cm,
ylabel={\tiny{Amount to Spend (\$1,000s)}},
xlabel={\tiny{Quarter}},
xmin=1, xmax=4,
ymin=-50, ymax=5000,
xtick={1,2,3,4},
ytick={0,1000,2000,3000,4000,5000},
legend pos=north west,
ymajorgrids=true,
grid style=dashed,
]
\addplot[draw=gr1]
coordinates {(1,1400)(2,200)(3,1600)(4,3721)};
\addplot[draw=gr1]
coordinates {(1,1400)(2,0)(3,800)(4,4677)};
\addplot[draw=gr1]
coordinates {(1,1400)(2,400)(3,2600)(4,2566)};
\addplot[draw=gr1]
coordinates {(1,1400)(2,200)(3,1800)(4,3521)};
\addplot[draw=gr1]
coordinates {(1,1400)(2,0)(3,1000)(4,4477)};
\addplot[draw=gr2]
coordinates {(1,1400)(2,400)(3,2800)(4,2366)};
\addplot[draw=gr2]
coordinates {(1,1400)(2,200)(3,2000)(4,3321)};
\addplot[draw=gr2]
coordinates {(1,1400)(2,0)(3,1200)(4,4277)};
\addplot[draw=gr2]
coordinates {(1,1400)(2,400)(3,3000)(4,2166)};
\addplot[draw=gr2]
coordinates {(1,1400)(2,200)(3,2200)(4,3121)};
\addplot[draw=gr2]
coordinates {(1,1400)(2,0)(3,1400)(4,4077)};
\addplot[draw=gr2]
coordinates {(1,1400)(2,400)(3,3200)(4,1966)};
\addplot[draw=gr2]
coordinates {(1,1400)(2,200)(3,2400)(4,2921)};
\addplot[draw=gr3]
coordinates {(1,1400)(2,0)(3,1600)(4,3877)};
\addplot[draw=gr3]
coordinates {(1,1400)(2,200)(3,2600)(4,2721)};
\addplot[draw=gr3]
coordinates {(1,1400)(2,0)(3,1800)(4,3677)};
\addplot[draw=gr3]
coordinates {(1,1400)(2,200)(3,2800)(4,2521)};
\addplot[draw=gr3]
coordinates {(1,1400)(2,0)(3,2000)(4,3477)};
\addplot[draw=gr4]
coordinates {(1,1400)(2,200)(3,3000)(4,2321)};
\addplot[draw=gr4]
coordinates {(1,1400)(2,0)(3,2200)(4,3277)};
\addplot[draw=gr4]
coordinates {(1,1400)(2,200)(3,3200)(4,2121)};
\addplot[draw=gr4]
coordinates {(1,1400)(2,0)(3,2400)(4,3077)};
\addplot[draw=gr5]
coordinates {(1,1400)(2,200)(3,3400)(4,1921)};
\addplot[draw=gr5]
coordinates {(1,1400)(2,0)(3,2600)(4,2877)};
\addplot[draw=gr5]
coordinates {(1,1400)(2,0)(3,2800)(4,2677)};
\addplot[draw=gr5]
coordinates {(1,1400)(2,0)(3,3000)(4,2477)};
\addplot[draw=gr6]
coordinates {(1,1400)(2,0)(3,3200)(4,2277)};
\addplot[draw=blue]
coordinates {(1,1400)(2,0)(3,3400)(4,2077)};
\end{axis}
\end{tikzpicture}
}
\caption{Overlaid strategies for P1 and P2. Colored line indicates top strategy. Darker gray lines indicate heavily used strategies, while lighter gray lines indicated lightly used strategies.}
\label{8v14.5}
\end{figure}

\begin{figure}
\tiny
\begin{tabular}{ |p{1.0cm}p{1.0cm}p{1.0cm}p{2.0cm}|p{1.0cm}||p{1.0cm}p{1.0cm}p{1.0cm}p{2.0cm}|p{1.0cm}|}
\hline
\multicolumn{10}{|c|}{P1 \$800,000 vs. P2 \$1,600,000} \\
\hline
Player 1 & (P1) & & & & Player 2 & (P2) & & & \\
\hline
Q1 Strategy & Q2 Strategy & Q3 Strategy & Q4 Strategy  &  Play \% & Q1 Strategy & Q2 Strategy & Q3 Strategy & Q4 Strategy  &  Play \%\\
\hline
\$800,000 & \$0 & \$2,000,000 & \$1,129,445 & 29.297\% & \$1,600,000 & \$0 & \$4,000,000 & \$2,258,890 & 4.255\% \\
\$800,000 & \$0 & \$1,800,000 & \$1,329,445 & 20.915\% & \$1,600,000 & \$0 & \$3,800,000 & \$2,458,890 & 4.107\% \\
\$800,000 & \$0 & \$1,600,000 & \$1,529,445 & 14.686\% & \$1,600,000 & \$0 & \$3,600,000 & \$2,658,890 & 3.942\% \\
\$800,000 & \$0 & \$1,400,000 & \$1,729,445 & 10.365\% & \$1,600,000 & \$0 & \$3,400,000 & \$2,858,890 & 3.823\% \\
\$800,000 & \$0 & \$1,200,000 & \$1,929,445 & 7.310\% & \$1,600,000 & \$0 & \$3,200,000 & \$3,058,890 & 3.620\% \\
\$800,000 & \$0 & \$1,000,000 & \$2,129,445 & 5.255\% & \$1,600,000 & \$0 & \$3,000,000 & \$3,258,890 & 3.469\% \\
\$800,000 & \$200,000 & \$1,800,000 & \$1,174,405 & 3.714\% & \$1,600,000 & \$200,000 & \$3,800,000 & \$2,303,850 & 3.269\% \\
\$800,000 & \$0 & \$800,000 & \$2,329,445 & 2.549\% & \$1,600,000 & \$0 & \$2,800,000 & \$3,458,890 & 3.190\% \\
\$800,000 & \$200,000 & \$1,600,000 & \$1,374,405 & 1.770\% & \$1,600,000 & \$200,000 & \$3,600,000 & \$2,503,850 & 3.021\% \\
\$800,000 & \$0 & \$600,000 & \$2,529,445 & 1.249\% & \$1,600,000 & \$0 & \$2,600,000 & \$3,658,890 & 2.934\% \\
&  &  &  &  & \$1,600,000 & \$200,000 & \$3,400,000 & \$2,703,850 & 2.855\% \\
&  &  &  &  & \$1,600,000 & \$0 & \$2,400,000 & \$3,858,890 & 2.615\% \\
&  &  &  &  & \$1,600,000 & \$0 & \$2,200,000 & \$4,058,890 & 2.541\% \\
&  &  &  &  & \$1,600,000 & \$200,000 & \$3,200,000 & \$2,903,850 & 2.511\% \\
&  &  &  &  & \$1,600,000 & \$200,000 & \$3,000,000 & \$3,103,850 & 2.349\% \\
&  &  &  &  & \$1,600,000 & \$0 & \$2,000,000 & \$4,258,890 & 2.217\% \\
&  &  &  &  & \$1,600,000 & \$200,000 & \$2,800,000 & \$3,303,850 & 2.192\% \\
&  &  &  &  & \$1,600,000 & \$400,000 & \$3,600,000 & \$2,348,810 & 2.134\% \\
&  &  &  &  & \$1,600,000 & \$0 & \$1,800,000 & \$4,458,890 & 1.952\% \\
&  &  &  &  & \$1,600,000 & \$200,000 & \$2,600,000 & \$3,503,850 & 1.900\% \\
&  &  &  &  & \$1,600,000 & \$400,000 & \$3,400,000 & \$2,548,810 & 1.882\% \\
&  &  &  &  & \$1,600,000 & \$0 & \$1,600,000 & \$4,658,890 & 1.720\% \\
&  &  &  &  & \$1,600,000 & \$200,000 & \$2,400,000 & \$3,703,850 & 1.633\% \\
&  &  &  &  & \$1,600,000 & \$400,000 & \$3,200,000 & \$2,748,810 & 1.556\% \\
&  &  &  &  & \$1,600,000 & \$200,000 & \$2,200,000 & \$3,903,850 & 1.447\% \\
&  &  &  &  & \$1,600,000 & \$0 & \$1,400,000 & \$4,858,890 & 1.440\% \\
&  &  &  &  & \$1,600,000 & \$400,000 & \$3,000,000 & \$2,948,810 & 1.378\% \\
&  &  &  &  & \$1,600,000 & \$0 & \$1,200,000 & \$5,058,890 & 1.330\% \\
&  &  &  &  & \$1,600,000 & \$200,000 & \$2,000,000 & \$4,103,850 & 1.289\% \\
&  &  &  &  & \$1,600,000 & \$400,000 & \$2,800,000 & \$3,148,810 & 1.160\% \\
&  &  &  &  & \$1,600,000 & \$0 & \$1,000,000 & \$5,258,890 & 1.155\% \\
&  &  &  &  & \$1,600,000 & \$200,000 & \$1,800,000 & \$4,303,850 & 1.087\% \\
&  &  &  &  & \$1,600,000 & \$400,000 & \$2,600,000 & \$3,348,810 & 1.061\% \\
&  &  &  &  & \$1,600,000 & \$600,000 & \$3,400,000 & \$2,393,770 & 1.009\% \\
\hline
Other &  &  &  & 2.89\% & Other &  &  &  & 21.957\% \\
\hline
\small \textbf{Game Value:} &&& \small \textbf{1.76} && \small \textbf{Winner:} &&& \small \textbf{P1}&\\
\hline
\end{tabular}
\caption{Mixed strategy output, when P1 is given an initial budget of \$800,000, P2 is given an initial budget of \$1,600,000, and the Weighted by State, Unfloored Utility Vector is used.}
\label{8v16table.5}
\end{figure}

\begin{figure}
\centering
\captionsetup{justification=raggedright}
\subfloat[P1: \$800,000 Initial Budget]{
\begin{tikzpicture}
\begin{axis}[
width=7.0cm,
ylabel={\tiny{Amount to Spend (\$1,000s)}},
xlabel={\tiny{Quarter}},
xmin=1, xmax=4,
ymin=-50, ymax=6000,
xtick={1,2,3,4},
ytick={0,1000,2000,3000,4000,5000,6000},
legend pos=north west,
ymajorgrids=true,
grid style=dashed,
]
\addplot[draw=gr1]
coordinates {(1,800)(2,0)(3,600)(4,2529)};
\addplot[draw=gr2]
coordinates {(1,800)(2,200)(3,1600)(4,1374)};
\addplot[draw=gr3]
coordinates {(1,800)(2,0)(3,800)(4,2329)};
\addplot[draw=gr4]
coordinates {(1,800)(2,200)(3,1800)(4,1174)};
\addplot[draw=gr5]
coordinates {(1,800)(2,0)(3,1000)(4,2129)};
\addplot[draw=gr7]
coordinates {(1,800)(2,0)(3,1200)(4,1929)};
\addplot[draw=gr10]
coordinates {(1,800)(2,0)(3,1400)(4,1729)};
\addplot[draw=gr15]
coordinates {(1,800)(2,0)(3,1600)(4,1529)};
\addplot[draw=gr21]
coordinates {(1,800)(2,0)(3,1800)(4,1329)};
\addplot[draw=red]
coordinates {(1,800)(2,0)(3,2000)(4,1129)};
\end{axis}
\end{tikzpicture}
}
\qquad
\subfloat[P2: \$1,600,000 Initial Budget]{
\begin{tikzpicture}
\begin{axis}[
width=7.0cm,
ylabel={\tiny{Amount to Spend (\$1,000s)}},
xlabel={\tiny{Quarter}},
xmin=1, xmax=4,
ymin=-50, ymax=6000,
xtick={1,2,3,4},
ytick={0,1000,2000,3000,4000,5000,6000},
legend pos=north west,
ymajorgrids=true,
grid style=dashed,
]
\addplot[draw=gr1]
coordinates {(1,1600)(2,600)(3,3400)(4,2394)};
\addplot[draw=gr1]
coordinates {(1,1600)(2,400)(3,2600)(4,3349)};
\addplot[draw=gr1]
coordinates {(1,1600)(2,200)(3,1800)(4,4304)};
\addplot[draw=gr1]
coordinates {(1,1600)(2,0)(3,1000)(4,5259)};
\addplot[draw=gr1]
coordinates {(1,1600)(2,400)(3,2800)(4,3149)};
\addplot[draw=gr1]
coordinates {(1,1600)(2,200)(3,2000)(4,4104)};
\addplot[draw=gr1]
coordinates {(1,1600)(2,0)(3,1200)(4,5059)};
\addplot[draw=gr1]
coordinates {(1,1600)(2,400)(3,3000)(4,2949)};
\addplot[draw=gr1]
coordinates {(1,1600)(2,0)(3,1400)(4,4859)};
\addplot[draw=gr1]
coordinates {(1,1600)(2,200)(3,2200)(4,3904)};
\addplot[draw=gr2]
coordinates {(1,1600)(2,400)(3,3200)(4,2749)};
\addplot[draw=gr2]
coordinates {(1,1600)(2,200)(3,2400)(4,3704)};
\addplot[draw=gr2]
coordinates {(1,1600)(2,0)(3,1600)(4,4659)};
\addplot[draw=gr2]
coordinates {(1,1600)(2,400)(3,3400)(4,2549)};
\addplot[draw=gr2]
coordinates {(1,1600)(2,200)(3,2600)(4,3504)};
\addplot[draw=gr2]
coordinates {(1,1600)(2,0)(3,1800)(4,4459)};
\addplot[draw=gr2]
coordinates {(1,1600)(2,400)(3,3600)(4,2349)};
\addplot[draw=gr2]
coordinates {(1,1600)(2,200)(3,2800)(4,3304)};
\addplot[draw=gr2]
coordinates {(1,1600)(2,0)(3,2000)(4,4259)};
\addplot[draw=gr2]
coordinates {(1,1600)(2,200)(3,3000)(4,3104)};
\addplot[draw=gr3]
coordinates {(1,1600)(2,200)(3,3200)(4,2904)};
\addplot[draw=gr3]
coordinates {(1,1600)(2,0)(3,2200)(4,4059)};
\addplot[draw=gr3]
coordinates {(1,1600)(2,0)(3,2400)(4,3859)};
\addplot[draw=gr3]
coordinates {(1,1600)(2,200)(3,3400)(4,2704)};
\addplot[draw=gr3]
coordinates {(1,1600)(2,0)(3,2600)(4,3659)};
\addplot[draw=gr3]
coordinates {(1,1600)(2,200)(3,3600)(4,2504)};
\addplot[draw=gr3]
coordinates {(1,1600)(2,0)(3,2800)(4,3459)};
\addplot[draw=gr3]
coordinates {(1,1600)(2,200)(3,3800)(4,2304)};
\addplot[draw=gr3]
coordinates {(1,1600)(2,0)(3,3000)(4,3259)};
\addplot[draw=gr4]
coordinates {(1,1600)(2,0)(3,3200)(4,3059)};
\addplot[draw=gr4]
coordinates {(1,1600)(2,0)(3,3400)(4,2859)};
\addplot[draw=gr4]
coordinates {(1,1600)(2,0)(3,3600)(4,2659)};
\addplot[draw=gr4]
coordinates {(1,1600)(2,0)(3,3800)(4,2459)};
\addplot[draw=blue]
coordinates {(1,1600)(2,0)(3,4000)(4,2259)};
\end{axis}
\end{tikzpicture}
}
\caption{Overlaid strategies for P1 and P2. Colored line indicates top strategy. Darker gray lines indicate heavily used strategies, while lighter gray lines indicated lightly used strategies.}
\label{8v16.5}
\end{figure}

\newpage
\subsection{Weighted by Sample, Unfloored Utility Vector}
The Weighted by Sample, Unfloored Utility Vector is most qualitatively similar to the Unweighted, Unfloored Utility Vector, in that it consists of 3 decreasing positive numbers followed by a negative number. It is defined by
\begin{equation}
[ 0.26559308, 0.242170061, 0.124087002, -0.419994208 ]\cdot 2\times 10^{-5}.
\end{equation}
Like the Unweighted, Unfloored vector, it rewards early spending, but punishes for spending in Q4. Also like the Unweighted, Unfloored vector, this vector has a conflicting interaction with the fundraising return function (which gives players more and more money to spend throughout the campaign) and the rule that mandates players must spend their entire remaining budget in Q4. 

However, unlike the Unweighted, Unfloored vector, the results of this vector were so strange that the code inputs and outputs were checked twice. At the lowest budget pairing (\textbf{Figure \ref{8v8.6}}), the budgets are optimized primarily in descending order of Q1 spend amounts --- in essence, optimal strategy is spending as much as possible in Q2, then minimizing amounts spend in Q1 and Q3, with seeming disregard for Q4. This behavior is different than that of the games in the Unweighted, Unfloored section, and most likely caused by the smaller ratios between values in the Weighted By Sample, Unfloored Utility Vector. The \$800,000 vs. \$900,000 game has similar behavior. 

When P2's buget is increased to \$1,000,000, however, the results change. Rather than a strategy consisting of upper and lower limits, P2's spending for \textit{all four quarters} is somewhere in the middle of its suboptimal strategies. At first glance, it may seem like the optimal strategy changed dramatically. However, the optimal strategy's Q2 spending is the same as the previous game - it's the suboptimal strategies that have changed. While the optimal strategy does fundamentally change in general, Q2 spending remains the same. 

The reason Q1 increases before Q2 is relatively simple --- Q1 has a higher utility than Q2 for spending money. It may seem at first glance, then that Q2 spending should increase before Q3. Remember, though, that players are punished heavily for spending in Q4. Therefore, P2 has a very high incentive to spend as much as possible in Q3 to decrease their total remaining budget. 

This type of strategy is employed for games when P2 has \$1,100,000 and \$1,200,000 as well. However, when P2 gets \$1,300,000 or \$1,400,000, the optimal strategy changes again! Presumably, the higher budgets allow for more freedom in P2's optimal strategy. 

When P2's budget increases to \$1,600,000, the optimal strategy changes yet again. However, examine the difference in percentages between P2's top two strategies in this game --- 0.127\%. Recall that the Monte-Carlo method was used to determine these percentages. How does one determine whether this change is really a result of a different game, or just a rounding error? In previous games, the P2 strategies followed a relatively intuitive path; however, in this game, there's been enough change to raise the suspicion that this is just an error. 

One way to determine whether this is just an error is to run the code again, with more simulations. However, this would be extremely time-intensive; even 100,000 iterations took an immense amount of time. The key to estimating the error lies in the first game using this utility vector - \$800,000 vs. \$800,000. If calculated directly, these players should have identical strategies. Therefore, a rough error bound may be calculated by taking the maximum of the differences in calculated percentages between identical strategies. In other words, if the strategies for P1 are respectively ordered 
\begin{align*}
p_{1,1}, p_{1,2}, ..., p_{1,k},
p_{2,1}, p_{2,2}, ..., p_{2,k},
\end{align*}
such that $p_{1,j} = p_{2,j} \forall j \in [1,k]\cap \mathbb{N}$, then the approximate error E can be calculated
\begin{equation}
E_{max} \approx max \mid \{p_{1,j} - p_{2,j} \} \forall j \in [1,k]\cap \mathbb{N}.
\end{equation}
In this game, $E_{max} \approx 0.312$. Therefore, it's very well possible that the optimal strategy in the \$800,000 vs. \$1,600,000 game is chosen over the top suboptimal strategy as a result of error, rather than performance. Note that this is also possible in some previous games as well. This possibility is examined most closely in this game, however, since each of the previous games' optimal strategies follow an intuitive pattern.

Lastly, note that P2 is again the winner in these games. However, the game values have the smallest absolute values of any games simulated in the report. What P1 gains in the final quarter of the game, P2 just barely makes up for in Q1, Q2, and Q3. The game values never break an absolute value of 0.5, making this easily the fairest game in the report. 

\begin{figure}
\tiny
\begin{tabular}{ |p{1.0cm}p{1.0cm}p{1.0cm}p{2.0cm}|p{1.0cm}||p{1.0cm}p{1.0cm}p{1.0cm}p{2.0cm}|p{1.0cm}|}
\hline
\multicolumn{10}{|c|}{P1 \$800,000 vs. P2 \$800,000} \\
\hline
Player 1 & (P1) & & & & Player 2 & (P2) & & & \\
\hline
Q1 Strategy & Q2 Strategy & Q3 Strategy & Q4 Strategy  &  Play \% & Q1 Strategy & Q2 Strategy & Q3 Strategy & Q4 Strategy  &  Play \%\\
\hline
\$0 & \$1,600,000 & \$1,200,000 & \$1,200,355 & 29.347\% & \$0 & \$1,600,000 & \$1,200,000 & \$1,200,355 & 29.462\% \\
\$0 & \$1,400,000 & \$1,400,000 & \$1,155,395 & 20.903\% & \$0 & \$1,400,000 & \$1,400,000 & \$1,155,395 & 20.591\% \\
\$200,000 & \$1,400,000 & \$1,200,000 & \$1,227,588 & 14.553\% & \$200,000 & \$1,400,000 & \$1,200,000 & \$1,227,588 & 14.670\% \\
\$200,000 & \$1,200,000 & \$1,400,000 & \$1,182,628 & 10.495\% & \$200,000 & \$1,200,000 & \$1,400,000 & \$1,182,628 & 10.513\% \\
\$400,000 & \$1,200,000 & \$1,200,000 & \$1,254,820 & 7.304\% & \$400,000 & \$1,200,000 & \$1,200,000 & \$1,254,820 & 7.296\% \\
\$200,000 & \$1,000,000 & \$1,600,000 & \$1,137,668 & 5.141\% & \$200,000 & \$1,000,000 & \$1,600,000 & \$1,137,668 & 5.182\% \\
\$400,000 & \$1,000,000 & \$1,400,000 & \$1,209,860 & 3.667\% & \$400,000 & \$1,000,000 & \$1,400,000 & \$1,209,860 & 3.630\% \\
\$600,000 & \$1,000,000 & \$1,200,000 & \$1,282,052 & 2.577\% & \$600,000 & \$1,000,000 & \$1,200,000 & \$1,282,052 & 2.610\% \\
\$400,000 & \$800,000 & \$1,600,000 & \$1,164,900 & 1.836\% & \$400,000 & \$800,000 & \$1,600,000 & \$1,164,900 & 1.826\% \\
\$600,000 & \$800,000 & \$1,400,000 & \$1,237,092 & 1.271\% & \$600,000 & \$800,000 & \$1,400,000 & \$1,237,092 & 1.299\% \\
\hline
Other &&&& 2.906\% & Other &&&& 2.921\% \\
\hline
\small \textbf{Game Value:} &&& \small \textbf{0.00} && \small \textbf{Winner:} &&& \small \textbf{Tie}&\\
\hline
\end{tabular}
\caption{Mixed strategy output, when P1 is given an initial budget of \$800,000, P2 is given an initial budget of \$800,000, and the Weighted by Sample, Unfloored Utility Vector is used.}
\label{8v8table.6}
\end{figure}

\begin{figure}
\centering
\captionsetup{justification=raggedright}
\subfloat[P1: \$800,000 Initial Budget]{
\begin{tikzpicture}
\begin{axis}[
width=7.0cm,
ylabel={\tiny{Amount to Spend (\$1,000s)}},
xlabel={\tiny{Quarter}},
xmin=1, xmax=4,
ymin=-50, ymax=2000,
xtick={1,2,3,4},
ytick={0,1000,2000},
legend pos=north west,
ymajorgrids=true,
grid style=dashed,
]
\addplot[draw=gr1]
coordinates {(1,600)(2,800)(3,1400)(4,1237)};
\addplot[draw=gr2]
coordinates {(1,400)(2,800)(3,1600)(4,1165)};
\addplot[draw=gr3]
coordinates {(1,600)(2,1000)(3,1200)(4,1282)};
\addplot[draw=gr4]
coordinates {(1,400)(2,1000)(3,1400)(4,1210)};
\addplot[draw=gr5]
coordinates {(1,200)(2,1000)(3,1600)(4,1138)};
\addplot[draw=gr7]
coordinates {(1,400)(2,1200)(3,1200)(4,1255)};
\addplot[draw=gr10]
coordinates {(1,200)(2,1200)(3,1400)(4,1183)};
\addplot[draw=gr15]
coordinates {(1,200)(2,1400)(3,1200)(4,1228)};
\addplot[draw=gr21]
coordinates {(1,0)(2,1400)(3,1400)(4,1155)};
\addplot[draw=red]
coordinates {(1,0)(2,1600)(3,1200)(4,1200)};
\end{axis}
\end{tikzpicture}
}
\qquad
\subfloat[P2: \$800,000 Initial Budget]{
\begin{tikzpicture}
\begin{axis}[
width=7.0cm,
ylabel={\tiny{Amount to Spend (\$1,000s)}},
xlabel={\tiny{Quarter}},
xmin=1, xmax=4,
ymin=-50, ymax=2000,
xtick={1,2,3,4},
ytick={0,1000,2000},
legend pos=north west,
ymajorgrids=true,
grid style=dashed,
]
\addplot[draw=gr1]
coordinates {(1,600)(2,800)(3,1400)(4,1237)};
\addplot[draw=gr2]
coordinates {(1,400)(2,800)(3,1600)(4,1165)};
\addplot[draw=gr3]
coordinates {(1,600)(2,1000)(3,1200)(4,1282)};
\addplot[draw=gr4]
coordinates {(1,400)(2,1000)(3,1400)(4,1210)};
\addplot[draw=gr5]
coordinates {(1,200)(2,1000)(3,1600)(4,1138)};
\addplot[draw=gr7]
coordinates {(1,400)(2,1200)(3,1200)(4,1255)};
\addplot[draw=gr11]
coordinates {(1,200)(2,1200)(3,1400)(4,1183)};
\addplot[draw=gr15]
coordinates {(1,200)(2,1400)(3,1200)(4,1228)};
\addplot[draw=gr21]
coordinates {(1,0)(2,1400)(3,1400)(4,1155)};
\addplot[draw=blue]
coordinates {(1,0)(2,1600)(3,1200)(4,1200)};
\end{axis}
\end{tikzpicture}
}
\caption{Overlaid strategies for P1 and P2. Colored line indicates top strategy. Darker gray lines indicate heavily used strategies, while lighter gray lines indicated lightly used strategies.}
\label{8v8.6}
\end{figure}

\begin{figure}
\tiny
\begin{tabular}{ |p{1.0cm}p{1.0cm}p{1.0cm}p{2.0cm}|p{1.0cm}||p{1.0cm}p{1.0cm}p{1.0cm}p{2.0cm}|p{1.0cm}|}
\hline
\multicolumn{10}{|c|}{P1 \$800,000 vs. P2 \$900,000} \\
\hline
Player 1 & (P1) & & & & Player 2 & (P2) & & & \\
\hline
Q1 Strategy & Q2 Strategy & Q3 Strategy & Q4 Strategy  &  Play \% & Q1 Strategy & Q2 Strategy & Q3 Strategy & Q4 Strategy  &  Play \%\\
\hline
\$0 & \$1,600,000 & \$1,200,000 & \$1,200,355 & 29.466\% & \$200,000 & \$1,600,000 & \$1,400,000 & \$1,327,632 & 22.922\% \\
\$0 & \$1,400,000 & \$1,400,000 & \$1,155,395 & 20.738\% & \$200,000 & \$1,400,000 & \$1,600,000 & \$1,282,672 & 17.625\% \\
\$200,000 & \$1,400,000 & \$1,200,000 & \$1,227,588 & 14.691\% & \$400,000 & \$1,400,000 & \$1,400,000 & \$1,354,865 & 13.598\% \\
\$200,000 & \$1,200,000 & \$1,400,000 & \$1,182,628 & 10.447\% & \$400,000 & \$1,200,000 & \$1,600,000 & \$1,309,905 & 10.613\% \\
\$400,000 & \$1,200,000 & \$1,200,000 & \$1,254,820 & 7.350\% & \$600,000 & \$1,200,000 & \$1,400,000 & \$1,382,097 & 8.142\% \\
\$200,000 & \$1,000,000 & \$1,600,000 & \$1,137,668 & 5.297\% & \$400,000 & \$1,000,000 & \$1,800,000 & \$1,264,945 & 6.264\% \\
\$400,000 & \$1,000,000 & \$1,400,000 & \$1,209,860 & 3.697\% & \$600,000 & \$1,000,000 & \$1,600,000 & \$1,337,137 & 4.806\% \\
\$600,000 & \$1,000,000 & \$1,200,000 & \$1,282,052 & 2.496\% & \$600,000 & \$800,000 & \$1,800,000 & \$1,292,177 & 3.770\% \\
\$400,000 & \$800,000 & \$1,600,000 & \$1,164,900 & 1.719\% & \$800,000 & \$800,000 & \$1,600,000 & \$1,364,369 & 2.827\% \\
\$600,000 & \$800,000 & \$1,400,000 & \$1,237,092 & 1.209\% & \$600,000 & \$600,000 & \$2,000,000 & \$1,247,217 & 2.172\% \\
&&&& & \$800,000 & \$600,000 & \$1,800,000 & \$1,319,409 & 1.662\% \\
&&&& & \$800,000 & \$400,000 & \$2,000,000 & \$1,274,449 & 1.268\% \\
&&&& & \$800,000 & \$200,000 & \$2,200,000 & \$1,229,489 & 1.048\% \\
\hline
Other &&&& 2.890\% & Other &&&& 3.283\% \\
\hline
\small \textbf{Game Value:} &&& \small \textbf{-0.11} && \small \textbf{Winner:} &&& \small \textbf{P2}&\\
\hline
\end{tabular}
\caption{Mixed strategy output, when P1 is given an initial budget of \$800,000, P2 is given an initial budget of \$900,000, and the Weighted by Sample, Unfloored Utility Vector is used.}
\label{8v9table.6}
\end{figure}

\begin{figure}
\centering
\captionsetup{justification=raggedright}
\subfloat[P1: \$800,000 Initial Budget]{
\begin{tikzpicture}
\begin{axis}[
width=7.0cm,
ylabel={\tiny{Amount to Spend (\$1,000s)}},
xlabel={\tiny{Quarter}},
xmin=1, xmax=4,
ymin=-50, ymax=3000,
xtick={1,2,3,4},
ytick={0,1000,2000,3000},
legend pos=north west,
ymajorgrids=true,
grid style=dashed,
]
\addplot[draw=gr1]
coordinates {(1,600)(2,800)(3,1400)(4,1237)};
\addplot[draw=gr2]
coordinates {(1,400)(2,800)(3,1600)(4,1165)};
\addplot[draw=gr2]
coordinates {(1,600)(2,1000)(3,1200)(4,1282)};
\addplot[draw=gr4]
coordinates {(1,400)(2,1000)(3,1400)(4,1210)};
\addplot[draw=gr5]
coordinates {(1,200)(2,1000)(3,1600)(4,1138)};
\addplot[draw=gr7]
coordinates {(1,400)(2,1200)(3,1200)(4,1255)};
\addplot[draw=gr10]
coordinates {(1,200)(2,1200)(3,1400)(4,1183)};
\addplot[draw=gr15]
coordinates {(1,200)(2,1400)(3,1200)(4,1228)};
\addplot[draw=gr21]
coordinates {(1,0)(2,1400)(3,1400)(4,1155)};
\addplot[draw=red]
coordinates {(1,0)(2,1600)(3,1200)(4,1200)};
\end{axis}
\end{tikzpicture}
}
\qquad
\subfloat[P2: \$900,000 Initial Budget]{
\begin{tikzpicture}
\begin{axis}[
width=7.0cm,
ylabel={\tiny{Amount to Spend (\$1,000s)}},
xlabel={\tiny{Quarter}},
xmin=1, xmax=4,
ymin=-50, ymax=3000,
xtick={1,2,3,4},
ytick={0,1000,2000,3000},
legend pos=north west,
ymajorgrids=true,
grid style=dashed,
]
\addplot[draw=gr1]
coordinates {(1,800)(2,200)(3,2200)(4,1229)};
\addplot[draw=gr1]
coordinates {(1,800)(2,400)(3,2000)(4,1274)};
\addplot[draw=gr2]
coordinates {(1,600)(2,600)(3,2000)(4,1247)};
\addplot[draw=gr2]
coordinates {(1,800)(2,600)(3,1800)(4,1319)};
\addplot[draw=gr3]
coordinates {(1,800)(2,800)(3,1600)(4,1364)};
\addplot[draw=gr4]
coordinates {(1,600)(2,800)(3,1800)(4,1292)};
\addplot[draw=gr5]
coordinates {(1,600)(2,1000)(3,1600)(4,1337)};
\addplot[draw=gr6]
coordinates {(1,400)(2,1000)(3,1800)(4,1265)};
\addplot[draw=gr8]
coordinates {(1,600)(2,1200)(3,1400)(4,1382)};
\addplot[draw=gr11]
coordinates {(1,400)(2,1200)(3,1600)(4,1310)};
\addplot[draw=gr14]
coordinates {(1,400)(2,1400)(3,1400)(4,1355)};
\addplot[draw=gr18]
coordinates {(1,200)(2,1400)(3,1600)(4,1283)};
\addplot[draw=blue]
coordinates {(1,200)(2,1600)(3,1400)(4,1328)};
\end{axis}
\end{tikzpicture}
}
\caption{Overlaid strategies for P1 and P2. Colored line indicates top strategy. Darker gray lines indicate heavily used strategies, while lighter gray lines indicated lightly used strategies.}
\label{8v9.6}
\end{figure}

\begin{figure}
\tiny
\begin{tabular}{ |p{1.0cm}p{1.0cm}p{1.0cm}p{2.0cm}|p{1.0cm}||p{1.0cm}p{1.0cm}p{1.0cm}p{2.0cm}|p{1.0cm}|}
\hline
\multicolumn{10}{|c|}{P1 \$800,000 vs. P2 \$1,000,000} \\
\hline
Player 1 & (P1) & & & & Player 2 & (P2) & & & \\
\hline
Q1 Strategy & Q2 Strategy & Q3 Strategy & Q4 Strategy  &  Play \% & Q1 Strategy & Q2 Strategy & Q3 Strategy & Q4 Strategy  &  Play \%\\
\hline
\$0 & \$1,600,000 & \$1,200,000 & \$1,200,355 & 29.238\% & \$400,000 & \$1,600,000 & \$1,600,000 & \$1,454,909 & 16.299\% \\
\$0 & \$1,400,000 & \$1,400,000 & \$1,155,395 & 20.783\% & \$600,000 & \$1,400,000 & \$1,600,000 & \$1,482,141 & 13.418\% \\
\$200,000 & \$1,400,000 & \$1,200,000 & \$1,227,588 & 14.763\% & \$600,000 & \$1,200,000 & \$1,800,000 & \$1,437,181 & 11.353\% \\
\$200,000 & \$1,200,000 & \$1,400,000 & \$1,182,628 & 10.452\% & \$800,000 & \$1,200,000 & \$1,600,000 & \$1,509,374 & 9.491\% \\
\$400,000 & \$1,200,000 & \$1,200,000 & \$1,254,820 & 7.322\% & \$800,000 & \$1,000,000 & \$1,800,000 & \$1,464,414 & 8.004\% \\
\$200,000 & \$1,000,000 & \$1,600,000 & \$1,137,668 & 5.141\% & \$1,000,000 & \$1,000,000 & \$1,600,000 & \$1,536,606 & 6.749\% \\
\$400,000 & \$1,000,000 & \$1,400,000 & \$1,209,860 & 3.625\% & \$800,000 & \$800,000 & \$2,000,000 & \$1,419,454 & 5.668\% \\
\$600,000 & \$1,000,000 & \$1,200,000 & \$1,282,052 & 2.562\% & \$1,000,000 & \$800,000 & \$1,800,000 & \$1,491,646 & 4.770\% \\
\$400,000 & \$800,000 & \$1,600,000 & \$1,164,900 & 1.829\% & \$1,000,000 & \$600,000 & \$2,000,000 & \$1,446,686 & 4.019\% \\
\$600,000 & \$800,000 & \$1,400,000 & \$1,237,092 & 1.327\% & \$1,000,000 & \$400,000 & \$2,200,000 & \$1,401,726 & 3.342\% \\
&&&& & \$0 & \$2,000,000 & \$1,400,000 & \$1,600,444 & 2.744\% \\
&&&& & \$0 & \$1,800,000 & \$1,600,000 & \$1,555,484 & 2.336\% \\
&&&& & \$200,000 & \$1,800,000 & \$1,400,000 & \$1,627,677 & 1.934\% \\
&&&& & \$0 & \$1,600,000 & \$1,800,000 & \$1,510,524 & 1.642\% \\
&&&& & \$200,000 & \$1,600,000 & \$1,600,000 & \$1,582,717 & 1.356\% \\
&&&& & \$400,000 & \$1,600,000 & \$1,400,000 & \$1,654,909 & 1.148\% \\
\hline
Other &&&& 2.958\% & Other &&&& 5.727\% \\
\hline
\small \textbf{Game Value:} &&& \small \textbf{-0.16} && \small \textbf{Winner:} &&& \small \textbf{P2}&\\
\hline
\end{tabular}
\caption{Mixed strategy output, when P1 is given an initial budget of \$800,000, P2 is given an initial budget of \$1,000,000, and the Weighted by Sample, Unfloored Utility Vector is used.}
\label{8v10table.6}
\end{figure}

\begin{figure}
\centering
\captionsetup{justification=raggedright}
\subfloat[P1: \$800,000 Initial Budget]{
\begin{tikzpicture}
\begin{axis}[
width=7.0cm,
ylabel={\tiny{Amount to Spend (\$1,000s)}},
xlabel={\tiny{Quarter}},
xmin=1, xmax=4,
ymin=-50, ymax=3000,
xtick={1,2,3,4},
ytick={0,1000,2000,3000},
legend pos=north west,
ymajorgrids=true,
grid style=dashed,
]
\addplot[draw=gr1]
coordinates {(1,600)(2,800)(3,1400)(4,1237)};
\addplot[draw=gr2]
coordinates {(1,400)(2,800)(3,1600)(4,1165)};
\addplot[draw=gr3]
coordinates {(1,600)(2,1000)(3,1200)(4,1282)};
\addplot[draw=gr4]
coordinates {(1,400)(2,1000)(3,1400)(4,1210)};
\addplot[draw=gr5]
coordinates {(1,200)(2,1000)(3,1600)(4,1138)};
\addplot[draw=gr7]
coordinates {(1,400)(2,1200)(3,1200)(4,1255)};
\addplot[draw=gr10]
coordinates {(1,200)(2,1200)(3,1400)(4,1183)};
\addplot[draw=gr15]
coordinates {(1,200)(2,1400)(3,1200)(4,1228)};
\addplot[draw=gr21]
coordinates {(1,0)(2,1400)(3,1400)(4,1155)};
\addplot[draw=red]
coordinates {(1,0)(2,1600)(3,1200)(4,1200)};
\end{axis}
\end{tikzpicture}
}
\qquad
\subfloat[P2: \$1,000,000 Initial Budget]{
\begin{tikzpicture}
\begin{axis}[
width=7.0cm,
ylabel={\tiny{Amount to Spend (\$1,000s)}},
xlabel={\tiny{Quarter}},
xmin=1, xmax=4,
ymin=-50, ymax=3000,
xtick={1,2,3,4},
ytick={0,1000,2000,3000},
legend pos=north west,
ymajorgrids=true,
grid style=dashed,
]
\addplot[draw=gr1]
coordinates {(1,200)(2,1600)(3,1600)(4,1583)};
\addplot[draw=gr1]
coordinates {(1,400)(2,1600)(3,1400)(4,1655)};
\addplot[draw=gr2]
coordinates {(1,0)(2,1600)(3,1800)(4,1511)};
\addplot[draw=gr2]
coordinates {(1,0)(2,1800)(3,1600)(4,1555)};
\addplot[draw=gr2]
coordinates {(1,200)(2,1800)(3,1400)(4,1628)};
\addplot[draw=gr3]
coordinates {(1,0)(2,2000)(3,1400)(4,1600)};
\addplot[draw=gr3]
coordinates {(1,1000)(2,400)(3,2200)(4,1402)};
\addplot[draw=gr4]
coordinates {(1,1000)(2,600)(3,2000)(4,1447)};
\addplot[draw=gr5]
coordinates {(1,1000)(2,800)(3,1800)(4,1492)};
\addplot[draw=gr6]
coordinates {(1,800)(2,800)(3,2000)(4,1419)};
\addplot[draw=gr7]
coordinates {(1,1000)(2,1000)(3,1600)(4,1537)};
\addplot[draw=gr8]
coordinates {(1,800)(2,1000)(3,1800)(4,1464)};
\addplot[draw=gr9]
coordinates {(1,800)(2,1200)(3,1600)(4,1509)};
\addplot[draw=gr11]
coordinates {(1,600)(2,1200)(3,1800)(4,1437)};
\addplot[draw=gr13]
coordinates {(1,600)(2,1400)(3,1600)(4,1482)};
\addplot[draw=blue]
coordinates {(1,400)(2,1600)(3,1600)(4,1455)};
\end{axis}
\end{tikzpicture}
}
\caption{Overlaid strategies for P1 and P2. Colored line indicates top strategy. Darker gray lines indicate heavily used strategies, while lighter gray lines indicated lightly used strategies.}
\label{8v10.6}
\end{figure}

\begin{figure}
\tiny
\begin{tabular}{ |p{1.0cm}p{1.0cm}p{1.0cm}p{2.0cm}|p{1.0cm}||p{1.0cm}p{1.0cm}p{1.0cm}p{2.0cm}|p{1.0cm}|}
\hline
\multicolumn{10}{|c|}{P1 \$800,000 vs. P2 \$1,100,000} \\
\hline
Player 1 & (P1) & & & & Player 2 & (P2) & & & \\
\hline
Q1 Strategy & Q2 Strategy & Q3 Strategy & Q4 Strategy  &  Play \% & Q1 Strategy & Q2 Strategy & Q3 Strategy & Q4 Strategy  &  Play \%\\
\hline
\$0 & \$1,600,000 & \$1,200,000 & \$1,200,355 & 28.970\% & \$800,000 & \$1,600,000 & \$1,600,000 & \$1,654,378 & 12.958\% \\
\$0 & \$1,400,000 & \$1,400,000 & \$1,155,395 & 20.864\% & \$800,000 & \$1,400,000 & \$1,800,000 & \$1,609,418 & 11.272\% \\
\$200,000 & \$1,400,000 & \$1,200,000 & \$1,227,588 & 14.714\% & \$1,000,000 & \$1,200,000 & \$1,800,000 & \$1,636,650 & 9.877\% \\
\$200,000 & \$1,200,000 & \$1,400,000 & \$1,182,628 & 10.518\% & \$1,000,000 & \$1,000,000 & \$2,000,000 & \$1,591,690 & 8.560\% \\
\$400,000 & \$1,200,000 & \$1,200,000 & \$1,254,820 & 7.462\% & \$0 & \$2,200,000 & \$1,600,000 & \$1,700,489 & 7.535\% \\
\$200,000 & \$1,000,000 & \$1,600,000 & \$1,137,668 & 5.224\% & \$0 & \$2,000,000 & \$1,800,000 & \$1,655,529 & 6.517\% \\
\$400,000 & \$1,000,000 & \$1,400,000 & \$1,209,860 & 3.651\% & \$200,000 & \$2,000,000 & \$1,600,000 & \$1,727,721 & 5.626\% \\
\$600,000 & \$1,000,000 & \$1,200,000 & \$1,282,052 & 2.595\% & \$0 & \$1,800,000 & \$2,000,000 & \$1,610,569 & 4.961\% \\
\$400,000 & \$800,000 & \$1,600,000 & \$1,164,900 & 1.860\% & \$200,000 & \$1,800,000 & \$1,800,000 & \$1,682,761 & 4.249\% \\
\$600,000 & \$800,000 & \$1,400,000 & \$1,237,092 & 1.232\% & \$400,000 & \$1,800,000 & \$1,600,000 & \$1,754,954 & 3.753\% \\
&&&& & \$0 & \$1,600,000 & \$2,200,000 & \$1,565,609 & 3.249\% \\
&&&& & \$200,000 & \$1,600,000 & \$2,000,000 & \$1,637,801 & 2.833\% \\
&&&& & \$400,000 & \$1,600,000 & \$1,800,000 & \$1,709,994 & 2.428\% \\
&&&& & \$0 & \$1,400,000 & \$2,400,000 & \$1,520,649 & 2.109\% \\
&&&& & \$600,000 & \$1,600,000 & \$1,600,000 & \$1,782,186 & 1.880\% \\
&&&& & \$200,000 & \$1,400,000 & \$2,200,000 & \$1,592,841 & 1.555\% \\
&&&& & \$800,000 & \$1,600,000 & \$1,400,000 & \$1,854,378 & 1.413\% \\
&&&& & \$400,000 & \$1,400,000 & \$2,000,000 & \$1,665,034 & 1.256\% \\
&&&& & \$600,000 & \$1,400,000 & \$1,800,000 & \$1,737,226 & 1.047\% \\
\hline
Other &&&& 2.910\% & Other &&&& 6.922\% \\
\hline
\small \textbf{Game Value:} &&& \small \textbf{-0.11} && \small \textbf{Winner:} &&& \small \textbf{P2}&\\
\hline
\end{tabular}
\caption{Mixed strategy output, when P1 is given an initial budget of \$800,000, P2 is given an initial budget of \$1,100,000, and the Weighted by Sample, Unfloored Utility Vector is used.}
\label{8v11table.6}
\end{figure}

\begin{figure}
\centering
\captionsetup{justification=raggedright}
\subfloat[P1: \$800,000 Initial Budget]{
\begin{tikzpicture}
\begin{axis}[
width=7.0cm,
ylabel={\tiny{Amount to Spend (\$1,000s)}},
xlabel={\tiny{Quarter}},
xmin=1, xmax=4,
ymin=-50, ymax=3000,
xtick={1,2,3,4},
ytick={0,1000,2000,3000},
legend pos=north west,
ymajorgrids=true,
grid style=dashed,
]
\addplot[draw=gr1]
coordinates {(1,600)(2,800)(3,1400)(4,1237)};
\addplot[draw=gr2]
coordinates {(1,400)(2,800)(3,1600)(4,1165)};
\addplot[draw=gr3]
coordinates {(1,600)(2,1000)(3,1200)(4,1282)};
\addplot[draw=gr4]
coordinates {(1,400)(2,1000)(3,1400)(4,1210)};
\addplot[draw=gr5]
coordinates {(1,200)(2,1000)(3,1600)(4,1138)};
\addplot[draw=gr7]
coordinates {(1,400)(2,1200)(3,1200)(4,1255)};
\addplot[draw=gr11]
coordinates {(1,200)(2,1200)(3,1400)(4,1183)};
\addplot[draw=gr15]
coordinates {(1,200)(2,1400)(3,1200)(4,1228)};
\addplot[draw=gr21]
coordinates {(1,0)(2,1400)(3,1400)(4,1155)};
\addplot[draw=red]
coordinates {(1,0)(2,1600)(3,1200)(4,1200)};
\end{axis}
\end{tikzpicture}
}
\qquad
\subfloat[P2: \$1,100,000 Initial Budget]{
\begin{tikzpicture}
\begin{axis}[
width=7.0cm,
ylabel={\tiny{Amount to Spend (\$1,000s)}},
xlabel={\tiny{Quarter}},
xmin=1, xmax=4,
ymin=-50, ymax=3000,
xtick={1,2,3,4},
ytick={0,1000,2000,3000},
legend pos=north west,
ymajorgrids=true,
grid style=dashed,
]
\addplot[draw=gr1]
coordinates {(1,600)(2,1400)(3,1800)(4,1737)};
\addplot[draw=gr1]
coordinates {(1,400)(2,1400)(3,2000)(4,1665)};
\addplot[draw=gr1]
coordinates {(1,800)(2,1600)(3,1400)(4,1854)};
\addplot[draw=gr2]
coordinates {(1,200)(2,1400)(3,2200)(4,1593)};
\addplot[draw=gr2]
coordinates {(1,600)(2,1600)(3,1600)(4,1782)};
\addplot[draw=gr2]
coordinates {(1,0)(2,1400)(3,2400)(4,1521)};
\addplot[draw=gr2]
coordinates {(1,400)(2,1600)(3,1800)(4,1710)};
\addplot[draw=gr3]
coordinates {(1,200)(2,1600)(3,2000)(4,1638)};
\addplot[draw=gr3]
coordinates {(1,0)(2,1600)(3,2200)(4,1566)};
\addplot[draw=gr4]
coordinates {(1,400)(2,1800)(3,1600)(4,1755)};
\addplot[draw=gr4]
coordinates {(1,200)(2,1800)(3,1800)(4,1683)};
\addplot[draw=gr5]
coordinates {(1,0)(2,1800)(3,2000)(4,1611)};
\addplot[draw=gr6]
coordinates {(1,200)(2,2000)(3,1600)(4,1728)};
\addplot[draw=gr7]
coordinates {(1,0)(2,2000)(3,1800)(4,1656)};
\addplot[draw=gr8]
coordinates {(1,0)(2,2200)(3,1600)(4,1700)};
\addplot[draw=gr9]
coordinates {(1,1000)(2,1000)(3,2000)(4,1592)};
\addplot[draw=gr10]
coordinates {(1,1000)(2,1200)(3,1800)(4,1637)};
\addplot[draw=gr11]
coordinates {(1,800)(2,1400)(3,1800)(4,1609)};
\addplot[draw=blue]
coordinates {(1,800)(2,1600)(3,1600)(4,1654)};
\end{axis}
\end{tikzpicture}
}
\caption{Overlaid strategies for P1 and P2. Colored line indicates top strategy. Darker gray lines indicate heavily used strategies, while lighter gray lines indicated lightly used strategies.}
\label{8v11.6}
\end{figure}

\begin{figure}
\tiny
\begin{tabular}{ |p{1.0cm}p{1.0cm}p{1.0cm}p{2.0cm}|p{1.0cm}||p{1.0cm}p{1.0cm}p{1.0cm}p{2.0cm}|p{1.0cm}|}
\hline
\multicolumn{10}{|c|}{P1 \$800,000 vs. P2 \$1,200,000} \\
\hline
Player 1 & (P1) & & & & Player 2 & (P2) & & & \\
\hline
Q1 Strategy & Q2 Strategy & Q3 Strategy & Q4 Strategy  &  Play \% & Q1 Strategy & Q2 Strategy & Q3 Strategy & Q4 Strategy  &  Play \%\\
\hline
\$0 & \$1,600,000 & \$1,200,000 & \$1,200,355 & 29.251\% & \$1,000,000 & \$1,600,000 & \$1,800,000 & \$1,781,655 & 9.784\% \\
\$0 & \$1,400,000 & \$1,400,000 & \$1,155,395 & 20.739\% & \$1,200,000 & \$1,200,000 & \$2,000,000 & \$1,763,927 & 8.798\% \\
\$200,000 & \$1,400,000 & \$1,200,000 & \$1,227,588 & 14.817\% & \$0 & \$2,400,000 & \$1,800,000 & \$1,800,533 & 7.833\% \\
\$200,000 & \$1,200,000 & \$1,400,000 & \$1,182,628 & 10.404\% & \$0 & \$2,200,000 & \$2,000,000 & \$1,755,573 & 7.184\% \\
\$400,000 & \$1,200,000 & \$1,200,000 & \$1,254,820 & 7.384\% & \$200,000 & \$2,200,000 & \$1,800,000 & \$1,827,766 & 6.482\% \\
\$200,000 & \$1,000,000 & \$1,600,000 & \$1,137,668 & 5.197\% & \$0 & \$2,000,000 & \$2,200,000 & \$1,710,613 & 5.851\% \\
\$400,000 & \$1,000,000 & \$1,400,000 & \$1,209,860 & 3.714\% & \$200,000 & \$2,000,000 & \$2,000,000 & \$1,782,806 & 5.359\% \\
\$600,000 & \$1,000,000 & \$1,200,000 & \$1,282,052 & 2.530\% & \$400,000 & \$2,000,000 & \$1,800,000 & \$1,854,998 & 4.894\% \\
\$400,000 & \$800,000 & \$1,600,000 & \$1,164,900 & 1.794\% & \$200,000 & \$1,800,000 & \$2,200,000 & \$1,737,846 & 4.189\% \\
\$600,000 & \$800,000 & \$1,400,000 & \$1,237,092 & 1.251\% & \$400,000 & \$1,800,000 & \$2,000,000 & \$1,810,038 & 3.904\% \\
&&&& & \$600,000 & \$1,800,000 & \$1,800,000 & \$1,882,230 & 3.463\% \\
&&&& & \$200,000 & \$1,600,000 & \$2,400,000 & \$1,692,886 & 3.163\% \\
&&&& & \$800,000 & \$1,800,000 & \$1,600,000 & \$1,954,423 & 2.866\% \\
&&&& & \$400,000 & \$1,600,000 & \$2,200,000 & \$1,765,078 & 2.604\% \\
&&&& & \$600,000 & \$1,600,000 & \$2,000,000 & \$1,837,270 & 2.285\% \\
&&&& & \$800,000 & \$1,600,000 & \$1,800,000 & \$1,909,463 & 2.074\% \\
&&&& & \$400,000 & \$1,400,000 & \$2,400,000 & \$1,720,118 & 1.839\% \\
&&&& & \$1,000,000 & \$1,600,000 & \$1,600,000 & \$1,981,655 & 1.661\% \\
&&&& & \$600,000 & \$1,400,000 & \$2,200,000 & \$1,792,310 & 1.564\% \\
&&&& & \$800,000 & \$1,400,000 & \$2,000,000 & \$1,864,503 & 1.455\% \\
&&&& & \$400,000 & \$1,200,000 & \$2,600,000 & \$1,675,158 & 1.226\% \\
&&&& & \$1,000,000 & \$1,400,000 & \$1,800,000 & \$1,936,695 & 1.168\% \\
&&&& & \$600,000 & \$1,200,000 & \$2,400,000 & \$1,747,350 & 1.019\% \\
\hline
Other &&&& 2.919\% & Other &&&& 9.335\% \\
\hline
\small \textbf{Game Value:} &&& \small \textbf{-0.13} && \small \textbf{Winner:} &&& \small \textbf{P2}&\\
\hline
\end{tabular}
\caption{Mixed strategy output, when P1 is given an initial budget of \$800,000, P2 is given an initial budget of \$1,200,000, and the Weighted by Sample, Unfloored Utility Vector is used.}
\label{8v12table.6}
\end{figure}

\begin{figure}
\centering
\captionsetup{justification=raggedright}
\subfloat[P1: \$800,000 Initial Budget]{
\begin{tikzpicture}
\begin{axis}[
width=7.0cm,
ylabel={\tiny{Amount to Spend (\$1,000s)}},
xlabel={\tiny{Quarter}},
xmin=1, xmax=4,
ymin=-50, ymax=3000,
xtick={1,2,3,4},
ytick={0,1000,2000,3000},
legend pos=north west,
ymajorgrids=true,
grid style=dashed,
]
\addplot[draw=gr1]
coordinates {(1,600)(2,800)(3,1400)(4,1237)};
\addplot[draw=gr2]
coordinates {(1,400)(2,800)(3,1600)(4,1165)};
\addplot[draw=gr3]
coordinates {(1,600)(2,1000)(3,1200)(4,1282)};
\addplot[draw=gr4]
coordinates {(1,400)(2,1000)(3,1400)(4,1210)};
\addplot[draw=gr5]
coordinates {(1,200)(2,1000)(3,1600)(4,1138)};
\addplot[draw=gr7]
coordinates {(1,400)(2,1200)(3,1200)(4,1255)};
\addplot[draw=gr10]
coordinates {(1,200)(2,1200)(3,1400)(4,1183)};
\addplot[draw=gr15]
coordinates {(1,200)(2,1400)(3,1200)(4,1228)};
\addplot[draw=gr21]
coordinates {(1,0)(2,1400)(3,1400)(4,1155)};
\addplot[draw=red]
coordinates {(1,0)(2,1600)(3,1200)(4,1200)};
\end{axis}
\end{tikzpicture}
}
\qquad
\subfloat[P2: \$1,200,000 Initial Budget]{
\begin{tikzpicture}
\begin{axis}[
width=7.0cm,
ylabel={\tiny{Amount to Spend (\$1,000s)}},
xlabel={\tiny{Quarter}},
xmin=1, xmax=4,
ymin=-50, ymax=3000,
xtick={1,2,3,4},
ytick={0,1000,2000,3000},
legend pos=north west,
ymajorgrids=true,
grid style=dashed,
]
\addplot[draw=gr1]
coordinates {(1,600)(2,1200)(3,2400)(4,1747)};
\addplot[draw=gr1]
coordinates {(1,1000)(2,1400)(3,1800)(4,1937)};
\addplot[draw=gr1]
coordinates {(1,400)(2,1200)(3,2600)(4,1675)};
\addplot[draw=gr1]
coordinates {(1,800)(2,1400)(3,2000)(4,1865)};
\addplot[draw=gr2]
coordinates {(1,600)(2,1400)(3,2200)(4,1792)};
\addplot[draw=gr2]
coordinates {(1,1000)(2,1600)(3,1600)(4,1982)};
\addplot[draw=gr2]
coordinates {(1,400)(2,1400)(3,2400)(4,1720)};
\addplot[draw=gr2]
coordinates {(1,800)(2,1600)(3,1800)(4,1909)};
\addplot[draw=gr2]
coordinates {(1,600)(2,1600)(3,2000)(4,1837)};
\addplot[draw=gr3]
coordinates {(1,400)(2,1600)(3,2200)(4,1765)};
\addplot[draw=gr3]
coordinates {(1,800)(2,1800)(3,1600)(4,1954)};
\addplot[draw=gr3]
coordinates {(1,200)(2,1600)(3,2400)(4,1693)};
\addplot[draw=gr3]
coordinates {(1,600)(2,1800)(3,1800)(4,1882)};
\addplot[draw=gr4]
coordinates {(1,400)(2,1800)(3,2000)(4,1810)};
\addplot[draw=gr4]
coordinates {(1,200)(2,1800)(3,2200)(4,1738)};
\addplot[draw=gr5]
coordinates {(1,400)(2,2000)(3,1800)(4,1855)};
\addplot[draw=gr5]
coordinates {(1,200)(2,2000)(3,2000)(4,1783)};
\addplot[draw=gr6]
coordinates {(1,0)(2,2000)(3,2200)(4,1711)};
\addplot[draw=gr6]
coordinates {(1,200)(2,2200)(3,1800)(4,1828)};
\addplot[draw=gr7]
coordinates {(1,0)(2,2200)(3,2000)(4,1756)};
\addplot[draw=gr8]
coordinates {(1,0)(2,2400)(3,1800)(4,1801)};
\addplot[draw=gr9]
coordinates {(1,1200)(2,1200)(3,2000)(4,1764)};
\addplot[draw=blue]
coordinates {(1,1000)(2,1600)(3,1800)(4,1782)};
\end{axis}
\end{tikzpicture}
}
\caption{Overlaid strategies for P1 and P2. Colored line indicates top strategy. Darker gray lines indicate heavily used strategies, while lighter gray lines indicated lightly used strategies.}
\label{8v12.6}
\end{figure}

\begin{figure}
\tiny
\begin{tabular}{ |p{1.0cm}p{1.0cm}p{1.0cm}p{2.0cm}|p{1.0cm}||p{1.0cm}p{1.0cm}p{1.0cm}p{2.0cm}|p{1.0cm}|}
\hline
\multicolumn{10}{|c|}{P1 \$800,000 vs. P2 \$1,300,000} \\
\hline
Player 1 & (P1) & & & & Player 2 & (P2) & & & \\
\hline
Q1 Strategy & Q2 Strategy & Q3 Strategy & Q4 Strategy  &  Play \% & Q1 Strategy & Q2 Strategy & Q3 Strategy & Q4 Strategy  &  Play \%\\
\hline
\$0 & \$1,600,000 & \$1,200,000 & \$1,200,355 & 29.504\% & \$0 & \$2,600,000 & \$2,000,000 & \$1,900,578 & 8.061\% \\
\$0 & \$1,400,000 & \$1,400,000 & \$1,155,395 & 20.814\% & \$200,000 & \$2,400,000 & \$2,000,000 & \$1,927,810 & 7.393\% \\
\$200,000 & \$1,400,000 & \$1,200,000 & \$1,227,588 & 14.680\% & \$200,000 & \$2,200,000 & \$2,200,000 & \$1,882,850 & 6.951\% \\
\$200,000 & \$1,200,000 & \$1,400,000 & \$1,182,628 & 10.272\% & \$400,000 & \$2,200,000 & \$2,000,000 & \$1,955,043 & 6.192\% \\
\$400,000 & \$1,200,000 & \$1,200,000 & \$1,254,820 & 7.476\% & \$600,000 & \$2,200,000 & \$1,800,000 & \$2,027,235 & 5.796\% \\
\$200,000 & \$1,000,000 & \$1,600,000 & \$1,137,668 & 5.136\% & \$400,000 & \$2,000,000 & \$2,200,000 & \$1,910,083 & 5.285\% \\
\$400,000 & \$1,000,000 & \$1,400,000 & \$1,209,860 & 3.629\% & \$600,000 & \$2,000,000 & \$2,000,000 & \$1,982,275 & 4.858\% \\
\$600,000 & \$1,000,000 & \$1,200,000 & \$1,282,052 & 2.605\% & \$800,000 & \$2,000,000 & \$1,800,000 & \$2,054,467 & 4.529\% \\
\$400,000 & \$800,000 & \$1,600,000 & \$1,164,900 & 1.801\% & \$400,000 & \$1,800,000 & \$2,400,000 & \$1,865,123 & 4.156\% \\
\$600,000 & \$800,000 & \$1,400,000 & \$1,237,092 & 1.221\% & \$600,000 & \$1,800,000 & \$2,200,000 & \$1,937,315 & 3.898\% \\
&&&& & \$800,000 & \$1,800,000 & \$2,000,000 & \$2,009,507 & 3.514\% \\
&&&& & \$400,000 & \$1,600,000 & \$2,600,000 & \$1,820,163 & 3.129\% \\
&&&& & \$1,000,000 & \$1,800,000 & \$1,800,000 & \$2,081,699 & 3.024\% \\
&&&& & \$600,000 & \$1,600,000 & \$2,400,000 & \$1,892,355 & 2.719\% \\
&&&& & \$800,000 & \$1,600,000 & \$2,200,000 & \$1,964,547 & 2.487\% \\
&&&& & \$1,000,000 & \$1,600,000 & \$2,000,000 & \$2,036,739 & 2.321\% \\
&&&& & \$600,000 & \$1,400,000 & \$2,600,000 & \$1,847,395 & 2.090\% \\
&&&& & \$800,000 & \$1,400,000 & \$2,400,000 & \$1,919,587 & 1.944\% \\
&&&& & \$1,000,000 & \$1,400,000 & \$2,200,000 & \$1,991,779 & 1.793\% \\
&&&& & \$600,000 & \$1,200,000 & \$2,800,000 & \$1,802,435 & 1.641\% \\
&&&& & \$1,200,000 & \$1,400,000 & \$2,000,000 & \$2,063,972 & 1.525\% \\
&&&& & \$800,000 & \$1,200,000 & \$2,600,000 & \$1,874,627 & 1.367\% \\
&&&& & \$1,000,000 & \$1,200,000 & \$2,400,000 & \$1,946,819 & 1.282\% \\
&&&& & \$1,200,000 & \$1,200,000 & \$2,200,000 & \$2,019,012 & 1.211\% \\
&&&& & \$800,000 & \$1,000,000 & \$2,800,000 & \$1,829,667 & 1.051\% \\
&&&& & \$1,000,000 & \$1,000,000 & \$2,600,000 & \$1,901,859 & 1.003\% \\
\hline
Other &&&& 2.862\% & Other &&&& 10.780\% \\
\hline
\small \textbf{Game Value:} &&& \small \textbf{-0.18} && \small \textbf{Winner:} &&& \small \textbf{P2}&\\
\hline
\end{tabular}
\caption{Mixed strategy output, when P1 is given an initial budget of \$800,000, P2 is given an initial budget of \$1,300,000, and the Weighted by Sample, Unfloored Utility Vector is used.}
\label{8v13table.6}
\end{figure}

\begin{figure}
\centering
\captionsetup{justification=raggedright}
\subfloat[P1: \$800,000 Initial Budget]{
\begin{tikzpicture}
\begin{axis}[
width=7.0cm,
ylabel={\tiny{Amount to Spend (\$1,000s)}},
xlabel={\tiny{Quarter}},
xmin=1, xmax=4,
ymin=-50, ymax=3000,
xtick={1,2,3,4},
ytick={0,1000,2000,3000},
legend pos=north west,
ymajorgrids=true,
grid style=dashed,
]
\addplot[draw=gr1]
coordinates {(1,600)(2,800)(3,1400)(4,1237)};
\addplot[draw=gr2]
coordinates {(1,400)(2,800)(3,1600)(4,1165)};
\addplot[draw=gr3]
coordinates {(1,600)(2,1000)(3,1200)(4,1282)};
\addplot[draw=gr4]
coordinates {(1,400)(2,1000)(3,1400)(4,1210)};
\addplot[draw=gr5]
coordinates {(1,200)(2,1000)(3,1600)(4,1138)};
\addplot[draw=gr7]
coordinates {(1,400)(2,1200)(3,1200)(4,1255)};
\addplot[draw=gr10]
coordinates {(1,200)(2,1200)(3,1400)(4,1183)};
\addplot[draw=gr15]
coordinates {(1,200)(2,1400)(3,1200)(4,1228)};
\addplot[draw=gr21]
coordinates {(1,0)(2,1400)(3,1400)(4,1155)};
\addplot[draw=red]
coordinates {(1,0)(2,1600)(3,1200)(4,1200)};
\end{axis}
\end{tikzpicture}
}
\qquad
\subfloat[P2: \$1,300,000 Initial Budget]{
\begin{tikzpicture}
\begin{axis}[
width=7.0cm,
ylabel={\tiny{Amount to Spend (\$1,000s)}},
xlabel={\tiny{Quarter}},
xmin=1, xmax=4,
ymin=-50, ymax=3000,
xtick={1,2,3,4},
ytick={0,1000,2000,3000},
legend pos=north west,
ymajorgrids=true,
grid style=dashed,
]
\addplot[draw=gr1]
coordinates {(1,1000)(2,1000)(3,2600)(4,1902)};
\addplot[draw=gr1]
coordinates {(1,800)(2,1000)(3,2800)(4,1830)};
\addplot[draw=gr1]
coordinates {(1,1200)(2,1200)(3,2200)(4,2019)};
\addplot[draw=gr1]
coordinates {(1,1000)(2,1200)(3,2400)(4,1947)};
\addplot[draw=gr1]
coordinates {(1,800)(2,1200)(3,2600)(4,1875)};
\addplot[draw=gr2]
coordinates {(1,1200)(2,1400)(3,2000)(4,2064)};
\addplot[draw=gr2]
coordinates {(1,600)(2,1200)(3,2800)(4,1802)};
\addplot[draw=gr2]
coordinates {(1,1000)(2,1400)(3,2200)(4,1992)};
\addplot[draw=gr2]
coordinates {(1,800)(2,1400)(3,2400)(4,1920)};
\addplot[draw=gr2]
coordinates {(1,600)(2,1400)(3,2600)(4,1847)};
\addplot[draw=gr2]
coordinates {(1,1000)(2,1600)(3,2000)(4,2037)};
\addplot[draw=gr2]
coordinates {(1,800)(2,1600)(3,2200)(4,1965)};
\addplot[draw=gr3]
coordinates {(1,600)(2,1600)(3,2400)(4,1892)};
\addplot[draw=gr3]
coordinates {(1,1000)(2,1800)(3,1800)(4,2082)};
\addplot[draw=gr3]
coordinates {(1,400)(2,1600)(3,2600)(4,1820)};
\addplot[draw=gr4]
coordinates {(1,800)(2,1800)(3,2000)(4,2010)};
\addplot[draw=gr4]
coordinates {(1,600)(2,1800)(3,2200)(4,1937)};
\addplot[draw=gr4]
coordinates {(1,400)(2,1800)(3,2400)(4,1865)};
\addplot[draw=gr5]
coordinates {(1,800)(2,2000)(3,1800)(4,2054)};
\addplot[draw=gr5]
coordinates {(1,600)(2,2000)(3,2000)(4,1982)};
\addplot[draw=gr5]
coordinates {(1,400)(2,2000)(3,2200)(4,1910)};
\addplot[draw=gr6]
coordinates {(1,600)(2,2200)(3,1800)(4,2027)};
\addplot[draw=gr6]
coordinates {(1,400)(2,2200)(3,2000)(4,1955)};
\addplot[draw=gr7]
coordinates {(1,200)(2,2200)(3,2200)(4,1883)};
\addplot[draw=gr7]
coordinates {(1,200)(2,2400)(3,2000)(4,1928)};
\addplot[draw=blue]
coordinates {(1,0)(2,2600)(3,2000)(4,1901)};
\end{axis}
\end{tikzpicture}
}
\caption{Overlaid strategies for P1 and P2. Colored line indicates top strategy. Darker gray lines indicate heavily used strategies, while lighter gray lines indicated lightly used strategies.}
\label{8v13.6}
\end{figure}

\begin{figure}
\tiny
\begin{tabular}{ |p{1.0cm}p{1.0cm}p{1.0cm}p{2.0cm}|p{1.0cm}||p{1.0cm}p{1.0cm}p{1.0cm}p{2.0cm}|p{1.0cm}|}
\hline
\multicolumn{10}{|c|}{P1 \$800,000 vs. P2 \$1,400,000} \\
\hline
Player 1 & (P1) & & & & Player 2 & (P2) & & & \\
\hline
Q1 Strategy & Q2 Strategy & Q3 Strategy & Q4 Strategy  &  Play \% & Q1 Strategy & Q2 Strategy & Q3 Strategy & Q4 Strategy  &  Play \%\\
\hline
\$0 & \$1,600,000 & \$1,200,000 & \$1,200,355 & 29.340\% & \$400,000 & \$2,400,000 & \$2,200,000 & \$2,055,087 & 6.408\% \\
\$0 & \$1,400,000 & \$1,400,000 & \$1,155,395 & 20.817\% & \$600,000 & \$2,400,000 & \$2,000,000 & \$2,127,279 & 5.896\% \\
\$200,000 & \$1,400,000 & \$1,200,000 & \$1,227,588 & 14.757\% & \$400,000 & \$2,200,000 & \$2,400,000 & \$2,010,127 & 5.540\% \\
\$200,000 & \$1,200,000 & \$1,400,000 & \$1,182,628 & 10.425\% & \$600,000 & \$2,200,000 & \$2,200,000 & \$2,082,319 & 5.272\% \\
\$400,000 & \$1,200,000 & \$1,200,000 & \$1,254,820 & 7.246\% & \$800,000 & \$2,200,000 & \$2,000,000 & \$2,154,512 & 4.981\% \\
\$200,000 & \$1,000,000 & \$1,600,000 & \$1,137,668 & 5.142\% & \$600,000 & \$2,000,000 & \$2,400,000 & \$2,037,359 & 4.634\% \\
\$400,000 & \$1,000,000 & \$1,400,000 & \$1,209,860 & 3.708\% & \$800,000 & \$2,000,000 & \$2,200,000 & \$2,109,552 & 4.280\% \\
\$600,000 & \$1,000,000 & \$1,200,000 & \$1,282,052 & 2.621\% & \$1,000,000 & \$2,000,000 & \$2,000,000 & \$2,181,744 & 3.972\% \\
\$400,000 & \$800,000 & \$1,600,000 & \$1,164,900 & 1.772\% & \$600,000 & \$1,800,000 & \$2,600,000 & \$1,992,399 & 3.805\% \\
\$600,000 & \$800,000 & \$1,400,000 & \$1,237,092 & 1.268\% & \$800,000 & \$1,800,000 & \$2,400,000 & \$2,064,592 & 3.489\% \\
&&&& & \$1,000,000 & \$1,800,000 & \$2,200,000 & \$2,136,784 & 3.304\% \\
&&&& & \$1,200,000 & \$1,800,000 & \$2,000,000 & \$2,208,976 & 3.067\% \\
&&&& & \$800,000 & \$1,600,000 & \$2,600,000 & \$2,019,632 & 2.890\% \\
&&&& & \$1,000,000 & \$1,600,000 & \$2,400,000 & \$2,091,824 & 2.689\% \\
&&&& & \$1,200,000 & \$1,600,000 & \$2,200,000 & \$2,164,016 & 2.453\% \\
&&&& & \$800,000 & \$1,400,000 & \$2,800,000 & \$1,974,672 & 2.412\% \\
&&&& & \$1,000,000 & \$1,400,000 & \$2,600,000 & \$2,046,864 & 2.287\% \\
&&&& & \$1,200,000 & \$1,400,000 & \$2,400,000 & \$2,119,056 & 2.118\% \\
&&&& & \$1,400,000 & \$1,400,000 & \$2,200,000 & \$2,191,248 & 1.913\% \\
&&&& & \$1,000,000 & \$1,200,000 & \$2,800,000 & \$2,001,904 & 1.746\% \\
&&&& & \$1,200,000 & \$1,200,000 & \$2,600,000 & \$2,074,096 & 1.715\% \\
&&&& & \$1,400,000 & \$1,200,000 & \$2,400,000 & \$2,146,288 & 1.629\% \\
&&&& & \$1,000,000 & \$1,000,000 & \$3,000,000 & \$1,956,944 & 1.507\% \\
&&&& & \$1,200,000 & \$1,000,000 & \$2,800,000 & \$2,029,136 & 1.459\% \\
&&&& & \$1,400,000 & \$1,000,000 & \$2,600,000 & \$2,101,328 & 1.251\% \\
&&&& & \$1,200,000 & \$800,000 & \$3,000,000 & \$1,984,176 & 1.233\% \\
&&&& & \$1,400,000 & \$800,000 & \$2,800,000 & \$2,056,368 & 1.195\% \\
&&&& & \$1,200,000 & \$600,000 & \$3,200,000 & \$1,939,216 & 1.071\% \\
&&&& & \$1,400,000 & \$600,000 & \$3,000,000 & \$2,011,408 & 1.068\% \\
\hline
Other &&&& 2.904\% & Other &&&& 14.716\% \\
\hline
\small \textbf{Game Value:} &&& \small \textbf{-0.27} && \small \textbf{Winner:} &&& \small \textbf{P2}&\\
\hline
\end{tabular}
\caption{Mixed strategy output, when P1 is given an initial budget of \$800,000, P2 is given an initial budget of \$1,400,000, and the Weighted by Sample, Unfloored Utility Vector is used.}
\label{8v14table.6}
\end{figure}

\begin{figure}
\centering
\captionsetup{justification=raggedright}
\subfloat[P1: \$800,000 Initial Budget]{
\begin{tikzpicture}
\begin{axis}[
width=7.0cm,
ylabel={\tiny{Amount to Spend (\$1,000s)}},
xlabel={\tiny{Quarter}},
xmin=1, xmax=4,
ymin=-50, ymax=4000,
xtick={1,2,3,4},
ytick={0,1000,2000,3000,4000},
legend pos=north west,
ymajorgrids=true,
grid style=dashed,
]
\addplot[draw=gr1]
coordinates {(1,600)(2,800)(3,1400)(4,1237)};
\addplot[draw=gr2]
coordinates {(1,400)(2,800)(3,1600)(4,1165)};
\addplot[draw=gr3]
coordinates {(1,600)(2,1000)(3,1200)(4,1282)};
\addplot[draw=gr4]
coordinates {(1,400)(2,1000)(3,1400)(4,1210)};
\addplot[draw=gr5]
coordinates {(1,200)(2,1000)(3,1600)(4,1138)};
\addplot[draw=gr7]
coordinates {(1,400)(2,1200)(3,1200)(4,1255)};
\addplot[draw=gr10]
coordinates {(1,200)(2,1200)(3,1400)(4,1183)};
\addplot[draw=gr15]
coordinates {(1,200)(2,1400)(3,1200)(4,1228)};
\addplot[draw=gr21]
coordinates {(1,0)(2,1400)(3,1400)(4,1155)};
\addplot[draw=red]
coordinates {(1,0)(2,1600)(3,1200)(4,1200)};
\end{axis}
\end{tikzpicture}
}
\qquad
\subfloat[P2: \$1,400,000 Initial Budget]{
\begin{tikzpicture}
\begin{axis}[
width=7.0cm,
ylabel={\tiny{Amount to Spend (\$1,000s)}},
xlabel={\tiny{Quarter}},
xmin=1, xmax=4,
ymin=-50, ymax=4000,
xtick={1,2,3,4},
ytick={0,1000,2000,3000,4000},
legend pos=north west,
ymajorgrids=true,
grid style=dashed,
]
\addplot[draw=gr1]
coordinates {(1,1400)(2,600)(3,3000)(4,2011)};
\addplot[draw=gr1]
coordinates {(1,1200)(2,600)(3,3200)(4,1939)};
\addplot[draw=gr1]
coordinates {(1,1400)(2,800)(3,2800)(4,2056)};
\addplot[draw=gr1]
coordinates {(1,1200)(2,800)(3,3000)(4,1984)};
\addplot[draw=gr1]
coordinates {(1,1400)(2,1000)(3,2600)(4,2101)};
\addplot[draw=gr1]
coordinates {(1,1200)(2,1000)(3,2800)(4,2029)};
\addplot[draw=gr2]
coordinates {(1,1000)(2,1000)(3,3000)(4,1957)};
\addplot[draw=gr2]
coordinates {(1,1400)(2,1200)(3,2400)(4,2146)};
\addplot[draw=gr2]
coordinates {(1,1200)(2,1200)(3,2600)(4,2074)};
\addplot[draw=gr2]
coordinates {(1,1000)(2,1200)(3,2800)(4,2002)};
\addplot[draw=gr2]
coordinates {(1,1400)(2,1400)(3,2200)(4,2191)};
\addplot[draw=gr2]
coordinates {(1,1200)(2,1400)(3,2400)(4,2119)};
\addplot[draw=gr2]
coordinates {(1,1000)(2,1400)(3,2600)(4,2047)};
\addplot[draw=gr2]
coordinates {(1,800)(2,1400)(3,2800)(4,1975)};
\addplot[draw=gr2]
coordinates {(1,1200)(2,1600)(3,2200)(4,2164)};
\addplot[draw=gr3]
coordinates {(1,1000)(2,1600)(3,2400)(4,2092)};
\addplot[draw=gr3]
coordinates {(1,800)(2,1600)(3,2600)(4,2020)};
\addplot[draw=gr3]
coordinates {(1,1200)(2,1800)(3,2000)(4,2209)};
\addplot[draw=gr3]
coordinates {(1,1000)(2,1800)(3,2200)(4,2137)};
\addplot[draw=gr3]
coordinates {(1,800)(2,1800)(3,2400)(4,2065)};
\addplot[draw=gr4]
coordinates {(1,600)(2,1800)(3,2600)(4,1992)};
\addplot[draw=gr4]
coordinates {(1,1000)(2,2000)(3,2000)(4,2182)};
\addplot[draw=gr4]
coordinates {(1,800)(2,2000)(3,2200)(4,2110)};
\addplot[draw=gr5]
coordinates {(1,600)(2,2000)(3,2400)(4,2037)};
\addplot[draw=gr5]
coordinates {(1,800)(2,2200)(3,2000)(4,2155)};
\addplot[draw=gr5]
coordinates {(1,600)(2,2200)(3,2200)(4,2082)};
\addplot[draw=gr6]
coordinates {(1,400)(2,2200)(3,2400)(4,2010)};
\addplot[draw=gr6]
coordinates {(1,600)(2,2400)(3,2000)(4,2127)};
\addplot[draw=blue]
coordinates {(1,400)(2,2400)(3,2200)(4,2055)};
\end{axis}
\end{tikzpicture}
}
\caption{Overlaid strategies for P1 and P2. Colored line indicates top strategy. Darker gray lines indicate heavily used strategies, while lighter gray lines indicated lightly used strategies.}
\label{8v14.6}
\end{figure}

\begin{figure}
\tiny
\begin{tabular}{ |p{1.0cm}p{1.0cm}p{1.0cm}p{2.0cm}|p{1.0cm}||p{1.0cm}p{1.0cm}p{1.0cm}p{2.0cm}|p{1.0cm}|}
\hline
\multicolumn{10}{|c|}{P1 \$800,000 vs. P2 \$1,600,000} \\
\hline
Player 1 & (P1) & & & & Player 2 & (P2) & & & \\
\hline
Q1 Strategy & Q2 Strategy & Q3 Strategy & Q4 Strategy  &  Play \% & Q1 Strategy & Q2 Strategy & Q3 Strategy & Q4 Strategy  &  Play \%\\
\hline
\$0 & \$1,600,000 & \$1,200,000 & \$1,200,355 & 29.681\% & \$800,000 & \$2,600,000 & \$2,400,000 & \$2,354,601 & 4.325\% \\
\$0 & \$1,400,000 & \$1,400,000 & \$1,155,395 & 20.696\% & \$1,000,000 & \$2,400,000 & \$2,400,000 & \$2,381,833 & 4.198\% \\
\$200,000 & \$1,400,000 & \$1,200,000 & \$1,227,588 & 14.725\% & \$1,000,000 & \$2,200,000 & \$2,600,000 & \$2,336,873 & 4.015\% \\
\$200,000 & \$1,200,000 & \$1,400,000 & \$1,182,628 & 10.256\% & \$1,200,000 & \$2,200,000 & \$2,400,000 & \$2,409,065 & 3.735\% \\
\$400,000 & \$1,200,000 & \$1,200,000 & \$1,254,820 & 7.366\% & \$1,200,000 & \$2,000,000 & \$2,600,000 & \$2,364,105 & 3.575\% \\
\$200,000 & \$1,000,000 & \$1,600,000 & \$1,137,668 & 5.175\% & \$1,400,000 & \$2,000,000 & \$2,400,000 & \$2,436,297 & 3.435\% \\
\$400,000 & \$1,000,000 & \$1,400,000 & \$1,209,860 & 3.721\% & \$1,200,000 & \$1,800,000 & \$2,800,000 & \$2,319,145 & 3.343\% \\
\$600,000 & \$1,000,000 & \$1,200,000 & \$1,282,052 & 2.609\% & \$1,400,000 & \$1,800,000 & \$2,600,000 & \$2,391,337 & 3.155\% \\
\$400,000 & \$800,000 & \$1,600,000 & \$1,164,900 & 1.764\% & \$1,400,000 & \$1,600,000 & \$2,800,000 & \$2,346,377 & 2.972\% \\
\$600,000 & \$800,000 & \$1,400,000 & \$1,237,092 & 1.176\% & \$1,400,000 & \$1,400,000 & \$3,000,000 & \$2,301,417 & 2.916\% \\
&&&& & \$1,600,000 & \$1,600,000 & \$2,600,000 & \$2,418,570 & 2.909\% \\
&&&& & \$1,600,000 & \$1,400,000 & \$2,800,000 & \$2,373,610 & 2.695\% \\
&&&& & \$1,600,000 & \$1,200,000 & \$3,000,000 & \$2,328,650 & 2.583\% \\
&&&& & \$1,600,000 & \$1,000,000 & \$3,200,000 & \$2,283,690 & 2.484\% \\
&&&& & \$0 & \$3,200,000 & \$2,400,000 & \$2,400,711 & 2.314\% \\
&&&& & \$0 & \$3,000,000 & \$2,600,000 & \$2,355,751 & 2.259\% \\
&&&& & \$200,000 & \$3,000,000 & \$2,400,000 & \$2,427,944 & 2.213\% \\
&&&& & \$400,000 & \$3,000,000 & \$2,200,000 & \$2,500,136 & 2.176\% \\
&&&& & \$0 & \$2,800,000 & \$2,800,000 & \$2,310,791 & 1.983\% \\
&&&& & \$200,000 & \$2,800,000 & \$2,600,000 & \$2,382,984 & 1.818\% \\
&&&& & \$0 & \$2,600,000 & \$3,000,000 & \$2,265,831 & 1.702\% \\
&&&& & \$400,000 & \$2,800,000 & \$2,400,000 & \$2,455,176 & 1.700\% \\
&&&& & \$600,000 & \$2,800,000 & \$2,200,000 & \$2,527,368 & 1.586\% \\
&&&& & \$200,000 & \$2,600,000 & \$2,800,000 & \$2,338,024 & 1.525\% \\
&&&& & \$400,000 & \$2,600,000 & \$2,600,000 & \$2,410,216 & 1.511\% \\
&&&& & \$600,000 & \$2,600,000 & \$2,400,000 & \$2,482,408 & 1.407\% \\
&&&& & \$200,000 & \$2,400,000 & \$3,000,000 & \$2,293,064 & 1.406\% \\
&&&& & \$800,000 & \$2,600,000 & \$2,200,000 & \$2,554,601 & 1.326\% \\
&&&& & \$400,000 & \$2,400,000 & \$2,800,000 & \$2,365,256 & 1.302\% \\
&&&& & \$600,000 & \$2,400,000 & \$2,600,000 & \$2,437,448 & 1.212\% \\
&&&& & \$200,000 & \$2,200,000 & \$3,200,000 & \$2,248,104 & 1.145\% \\
&&&& & \$800,000 & \$2,400,000 & \$2,400,000 & \$2,509,641 & 1.079\% \\
&&&& & \$400,000 & \$2,200,000 & \$3,000,000 & \$2,320,296 & 1.035\% \\
\hline
Other &&&& 2.831\% & Other &&&& 22.961\% \\
\hline
\small \textbf{Game Value:} &&& \small \textbf{ -0.36} && \small \textbf{Winner:} &&& \small \textbf{P2}&\\
\hline
\end{tabular}
\caption{Mixed strategy output, when P1 is given an initial budget of \$800,000, P2 is given an initial budget of \$1,600,000, and the Weighted by Sample, Unfloored Utility Vector is used.}
\label{8v16table.6}
\end{figure}

\begin{figure}
\centering
\captionsetup{justification=raggedright}
\subfloat[P1: \$800,000 Initial Budget]{
\begin{tikzpicture}
\begin{axis}[
width=7.0cm,
ylabel={\tiny{Amount to Spend (\$1,000s)}},
xlabel={\tiny{Quarter}},
xmin=1, xmax=4,
ymin=-50, ymax=4000,
xtick={1,2,3,4},
ytick={0,1000,2000,3000,4000},
legend pos=north west,
ymajorgrids=true,
grid style=dashed,
]
\addplot[draw=gr1]
coordinates {(1,600)(2,800)(3,1400)(4,1237)};
\addplot[draw=gr2]
coordinates {(1,400)(2,800)(3,1600)(4,1165)};
\addplot[draw=gr3]
coordinates {(1,600)(2,1000)(3,1200)(4,1282)};
\addplot[draw=gr4]
coordinates {(1,400)(2,1000)(3,1400)(4,1210)};
\addplot[draw=gr5]
coordinates {(1,200)(2,1000)(3,1600)(4,1138)};
\addplot[draw=gr7]
coordinates {(1,400)(2,1200)(3,1200)(4,1255)};
\addplot[draw=gr10]
coordinates {(1,200)(2,1200)(3,1400)(4,1183)};
\addplot[draw=gr15]
coordinates {(1,200)(2,1400)(3,1200)(4,1228)};
\addplot[draw=gr21]
coordinates {(1,0)(2,1400)(3,1400)(4,1155)};
\addplot[draw=red]
coordinates {(1,0)(2,1600)(3,1200)(4,1200)};
\end{axis}
\end{tikzpicture}
}
\qquad
\subfloat[P2: \$1,600,000 Initial Budget]{
\begin{tikzpicture}
\begin{axis}[
width=7.0cm,
ylabel={\tiny{Amount to Spend (\$1,000s)}},
xlabel={\tiny{Quarter}},
xmin=1, xmax=4,
ymin=-50, ymax=4000,
xtick={1,2,3,4},
ytick={0,1000,2000,3000,4000},
legend pos=north west,
ymajorgrids=true,
grid style=dashed,
]
\addplot[draw=gr1]
coordinates {(1,400)(2,2200)(3,3000)(4,2320)};
\addplot[draw=gr1]
coordinates {(1,800)(2,2400)(3,2400)(4,2510)};
\addplot[draw=gr1]
coordinates {(1,200)(2,2200)(3,3200)(4,2248)};
\addplot[draw=gr1]
coordinates {(1,600)(2,2400)(3,2600)(4,2437)};
\addplot[draw=gr1]
coordinates {(1,400)(2,2400)(3,2800)(4,2365)};
\addplot[draw=gr1]
coordinates {(1,800)(2,2600)(3,2200)(4,2555)};
\addplot[draw=gr1]
coordinates {(1,200)(2,2400)(3,3000)(4,2293)};
\addplot[draw=gr1]
coordinates {(1,600)(2,2600)(3,2400)(4,2482)};
\addplot[draw=gr2]
coordinates {(1,400)(2,2600)(3,2600)(4,2410)};
\addplot[draw=gr2]
coordinates {(1,200)(2,2600)(3,2800)(4,2338)};
\addplot[draw=gr2]
coordinates {(1,600)(2,2800)(3,2200)(4,2527)};
\addplot[draw=gr2]
coordinates {(1,400)(2,2800)(3,2400)(4,2455)};
\addplot[draw=gr2]
coordinates {(1,0)(2,2600)(3,3000)(4,2266)};
\addplot[draw=gr2]
coordinates {(1,200)(2,2800)(3,2600)(4,2383)};
\addplot[draw=gr2]
coordinates {(1,0)(2,2800)(3,2800)(4,2311)};
\addplot[draw=gr2]
coordinates {(1,400)(2,3000)(3,2200)(4,2500)};
\addplot[draw=gr2]
coordinates {(1,200)(2,3000)(3,2400)(4,2428)};
\addplot[draw=gr2]
coordinates {(1,0)(2,3000)(3,2600)(4,2356)};
\addplot[draw=gr2]
coordinates {(1,0)(2,3200)(3,2400)(4,2401)};
\addplot[draw=gr2]
coordinates {(1,1600)(2,1000)(3,3200)(4,2284)};
\addplot[draw=gr3]
coordinates {(1,1600)(2,1200)(3,3000)(4,2329)};
\addplot[draw=gr3]
coordinates {(1,1600)(2,1400)(3,2800)(4,2374)};
\addplot[draw=gr3]
coordinates {(1,1600)(2,1600)(3,2600)(4,2419)};
\addplot[draw=gr3]
coordinates {(1,1400)(2,1400)(3,3000)(4,2301)};
\addplot[draw=gr3]
coordinates {(1,1400)(2,1600)(3,2800)(4,2346)};
\addplot[draw=gr3]
coordinates {(1,1400)(2,1800)(3,2600)(4,2391)};
\addplot[draw=gr3]
coordinates {(1,1200)(2,1800)(3,2800)(4,2319)};
\addplot[draw=gr3]
coordinates {(1,1400)(2,2000)(3,2400)(4,2436)};
\addplot[draw=gr4]
coordinates {(1,1200)(2,2000)(3,2600)(4,2364)};
\addplot[draw=gr4]
coordinates {(1,1200)(2,2200)(3,2400)(4,2409)};
\addplot[draw=gr4]
coordinates {(1,1000)(2,2200)(3,2600)(4,2337)};
\addplot[draw=gr4]
coordinates {(1,1000)(2,2400)(3,2400)(4,2382)};
\addplot[draw=blue]
coordinates {(1,800)(2,2600)(3,2400)(4,2355)};
\end{axis}
\end{tikzpicture}
}
\caption{Overlaid strategies for P1 and P2. Colored line indicates top strategy. Darker gray lines indicate heavily used strategies, while lighter gray lines indicated lightly used strategies.}
\label{8v16.6}
\end{figure}



% ---------------------------------Critiques----------------------------------
\newpage
\section{Critiques}
\subsection{Fundraising Return Function}
The most presumptuous aspect of this model is, without question, the
fundraising return function. Fundraising in Senate campaigns is an
extremely complex equation, and in calculating a fundraising return, one
must consider incumbency status, party, state population, previous donors,
and a myriad of other parameters not touched in this model. Extensive
research has been done to attempt to predict fundraising in public office
campaigns\cite{incumbents}, and almost none of it is considered in this
model. This model's fundraising return function was used purely as a
mechanic to create a solvable game, and should not be cited as anything
more.

Additionally, the most recent elections were characterized by a paradigm
shift in methods of fundraising. With the rise of electronic advertising
and a swtich from over-the-phone to internet donations, the cost and
methods of fundraising are rapidly and unpredictably shifting
\cite{fundresearch1}. This creates a very short expiration date on any
function of advertising, and likely skews even the extremely inaccurate
numbers give in \textbf{Section 5.1}. On top of this, due to the reporting
requirements of the FEC, candidates are not required to be very specific or
standard in thier spending and fundraising reporting \cite{fecform}. This
creates a huge error in the fundraising return function, which could not
even limit the parameter of spending to purely advertising spending, but
rather all expenditures in a campaign.

Even under the invalid assumption that fundraising can be found strictly as
a function of past fundraising and past spending, this model's fundraising
return function is too simple. Parameters would almost definitely reach
much further back than the most recent quarter of spending and fundraising;
however, without much more time and research into the field, creating a
more nuanced function would create a much more computationally expensive
algorithm, with almost no guarantee of a more accurate answer. Therefore,
to simplify the model and to make as few faulty assumptions as possible,
the simplest fundraising return function was used.

\subsection{Utility Function}
The first problems with this model's utility function mirror those outlined
in the critique of the Fundraising Return Function, since one of its
parameters is a candidate's spending. In addition, a different year's data
was used for the utility function (2018) than for the fundraising return
function (2016). This was done to limit confounding variables present in
one year to one of the two functions. Obviously, however, confounding
variables exist for both years, and the decision to use different years for
each of these functions may have only exacerbated these errors.

A more glaring issue with this function, however, is the polling data used
to inform it. Data collected from 538 was neither uniform nor unbiased;
most data points corresponded to different polling agencies, and are almost
guaranteed to have employed different polling strategies. An ideal dataset
would have polls conducted by the same agencies employing the same polling
methods, and would be much more standardized in collection
methods. However, no such dataset is publicly available. The utility
function was made through grasping at straw polls, so to speak. The data
sets considered were the only ones even remotely close to the dataset
needed to make this equation more accurate, and even these can be described
as ``less than ideal'' at best.

Another issue with the utility function is the way in which the regression
lines were calculated. Specifically, independent regression lines were
calculated for each quarter and candidate. The idea behind this was to
isolate polling errors to the fiscal quarters in which they were
taken. However, as a result, the endpoints of these regression lines did
not necessarily match up, and so discrepancies between quarters were
ignored. Another method to calculate these regression lines would be, after
calculating endpoints of regression lines for each quarter individually, to
average discrepant endpoints and to set new regression lines between these
points. However, this could take large errors in one quarter and affect
polling numbers of its adjacent quarters. Picking between these methods is
like picking the better of two poor candidate choices. The first option was
picked in order to isolate errors to a single quarter.

\subsection{Monte-Carlo Method}
Solving the matrices was the most time-prohibitive element of this
model. It would have been ideal to solve each matrix directly, rather than
using the Monte-Carlo method. However, even small initial budgets created
huge strategy arrays, which in turn created immense matrices. These
matrices could not be solved in a reasonable amount of time; therefore,
Monte-Carlo had to be used.

The number of iterations for each matrix may also be criticized. Each
matrix was only run 100,000 times, which is relatively low for this sort of
problem. However, even running only one utility vector 100,000 times for
each set of budgets cost several days of uninterrupted
computing. Increasing this iteration size would have created weeks of
nonstop computing work. Efficiency of the model has huge room for
improvement.

\subsection{Error Analysis}
Because the Monte-Carlo method always produces an error, it's very useful to estimate this error. This can be done very roughly, as outlined in Section 6.5: \textbf{Weighted by Sample, Unfloored Utility Vector}, by calculating the maximum difference in percentages between identical strategies when the two players are given identical budgets. However, there are whole libraries and some careers dedicated entirely to error analysis. While the error analysis used is somewhat useful for this report, it can be calculated much more accurately using more sophisticated methods. This is yet another area in which the report could improve. 

% ---------------------------------Summary----------------------------------
\section{Summary}

While this model has many facets that can and should be improved upon, it's a
good starting point from which a more realistic U.S. Senate campaign model may
be built. Using data and knowledge provided, this model suggests that the most
effective time to spend money on a campaign is the final quarter of that
campaign, regardless of initial budget and opponent's initial budget. It also
suggests that whether a player has a higher initial budget than their opponent
is an extremely strong indicator that player will win the election. While some
sub optimal strategies allow for the lower-budget player to win, playing
optimally allows the higher-budget player to win in every case. Major areas for
improvement of the model lie in the fundraising return function, the utility
vector, efficiency of the testing code, and refinement of tests.


% ------------------------------References----------------------------------
\newpage
\begin{flushleft}
\begin{thebibliography}{9}
\begin{small}
\bibitem{realclearpolitics}
``Battle for the Senate 2018.'' RealClearPolitics.com, RealClearPolitics.
\bigskip

\bibitem{fec}
``Campaign Finance Database.'' FEC.gov.
\bigskip

\bibitem{gametheory}
Ferguson, Thomas S. ``Game Theory.'' World Scientific, 2017.
\bigskip

\bibitem{max}
Hokit, Max. Personal interview. 4 December 2018.
\bigskip

\bibitem{pareto}
Ingham, Sean. ``Pareto-Optimality.'' Encyclop{\ae}dia Britannica, Encyclop{\ae}dia Britannica, Inc., 27 Sept. 2017.
\bigskip

\bibitem{montecarlo}
Kenton, Will. ``Monte-Carlo Simulation.'' Investopedia, Investopedia, 6 Mar. 2019.
\bigskip

\bibitem{countonit}
Mendes, Anthony. Count On It. Fall Quarter 2018, California Polytechnic State University in San Luis Obispo, 2018.
\bigskip

\bibitem{incumbents}
Mixon, Franklin G., et al. ``Pivotal Power Brokers: Theory and Evidence on Political Fundraising.'' Public Choice, vol. 123, no. 3/4, 2005, pp. 477–493. JSTOR.
\bigskip

\bibitem{hopper}
Norwood, Arlisha R. ``Grace Hopper.'' WomensHistory.org, National Women's History Museum, 2017.
\bigskip

\bibitem{mediabias}
``Search and Learn the Bias of News Media.'' Media Bias/Fact Check, mediabiasfactcheck.com/.
\bigskip

\bibitem{fundresearch1}
Simpson, D., O'Shaughnessy, B., \& Schakowsky, J. (2016). Raising Money. In Winning Elections in the 21st Century (pp. 49-69). University Press of Kansas.
\bigskip

\bibitem{slope}
``SLOPE Function.'' Support.Office.com, Microsoft. Microsoft Excel function documentation.
\bigskip

\bibitem{fecform}
United States, Congress, ``Recording Receipts.'' Help for Candidates and Committees, FEC.
\bigskip

\bibitem{ballotpedia16}
``United States Senate Elections, 2016.'' Ballotpedia.org, Ballotpedia.
\bigskip

\bibitem{ballotpedia18}
``United States Senate Elections, 2018.'' Ballotpedia.org, Ballotpedia.
\bigskip

\bibitem{538}
``U.S. Senate Polls.'' FiveThirtyEight.com, FiveThirtyEight. 5 Nov. 2018.
\bigskip

\bibitem{jackie}
Wetzel, Jacqueline R. Personal interview. 16 October 2018.
\bigskip

\end{small}

\end{thebibliography}

\end{flushleft}
\noindent

\end{document}
